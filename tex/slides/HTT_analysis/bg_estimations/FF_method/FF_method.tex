\begin{frame}
\frametitle{The FF method}
%\beamercite{CMS-NOTE-2018-257}

\manip How many events contain misidentified \tauh? (fake taus)

\pause

\begin{center}
\includegraphics[width=\graphw,height=\graphh/11*10,keepaspectratio]{\PhDthesisdir/tex/slides/HTT_analysis/bg_estimations/FF_method/FF_ppe-tikz-partial.tex}
\end{center}
\end{frame}

\begin{frame}
\frametitle{Avoiding misidentification: Particles isolation -- qualitatively}
\begin{center}
\begin{tikzpicture}
\def\trackerrin{.100}
\def\trackerrout{1.185}
\def\trackercolor{ltcolorgray1}

\def\ECALrin{1.290}
\def\ECALrout{1.811}
\def\ECALcolor{ltcolorgreen1}

\def\HCALrin{1.812}
\def\HCALrout{2.854}
\def\HCALcolor{ltcoloryellow3}

\def\Solenrin{2.950}
\def\Solenrout{3.800}
\def\Solencolor{ltcolorgray2}

\def\ironryrina{3.850}
\def\ironryrouta{4.000}
\def\muonrina{4.020}
\def\muonrouta{4.400}
\def\ironryrinb{4.420}
\def\ironryroutb{4.880}
\def\muonrinb{4.905}
\def\muonroutb{5.285}
\def\ironryrinc{5.300}
\def\ironryroutc{5.960}
\def\muonrinc{5.975}
\def\muonroutc{6.355}
\def\ironryrind{6.375}
\def\ironryroutd{6.980}
\def\muonrind{7.000}
\def\muonroutd{7.380}
\def\muoncolor{ltcoloryellow1}
\def\ironrycolor{ltcolorred3}

\clip (-\graphw/2,-\graphh/2) rectangle (\graphw/2,\graphh/2);

\draw (0,0) coordinate (PV);

\foreach \rin/\rout/\color in {
\muonrind/\muonroutd/\muoncolor,
\ironryrind/\ironryroutd/\ironrycolor,
\muonrinc/\muonroutc/\muoncolor,
\ironryrinc/\ironryroutc/\ironrycolor,
\muonrinb/\muonroutb/\muoncolor,
\ironryrinb/\ironryroutb/\ironrycolor,
\muonrina/\muonrouta/\muoncolor,
%\ironryrina/\ironryrouta/\ironrycolor,
\Solenrin/\Solenrout/\Solencolor,
\HCALrin/\HCALrout/\HCALcolor,
\ECALrin/\ECALrout/\ECALcolor,
\trackerrin/\trackerrout/\trackercolor
}{
\fill (PV) [color = \color] circle (\rout);
\fill (PV) [color = white] circle (\rin);
}

\foreach \phiangle/\maxradius/\linecolor/\linethick in {
150/\HCALrin/ltcolorgreen/thin,
120/\HCALrin/ltcolorred/very thick,
122/\HCALrin/ltcolorred/thin,
117/\HCALrin/ltcolorred/thin,
0/\HCALrin/ltcolorblue/very thick,
5/\ECALrin/ltcolorblue/thin,
10/\HCALrin/ltcolorblue/thin,
-7/\ECALrin/ltcolorblue/thin,
-13/\HCALrin/ltcolorblue/thin,
8/\ECALrin/ltcolorblue/thin,
-2/\HCALrin/ltcolorblue/thin,
3/\ECALrin/ltcolorblue/thin}{
\draw [\linecolor, \linethick] (PV) --+ (\phiangle:\trackerrout);
\def\newmaxradius{\maxradius+.25}
\ifthenelse{\equal{\maxradius}{\ECALrin}}{\def\newmaxradius{\ECALrout}}{}
\ifthenelse{\equal{\maxradius}{\HCALrin}}{\def\newmaxradius{\HCALrout}}{}
\draw [\linecolor4, ultra thick] (PV)+(\phiangle:\maxradius)--+(\phiangle:\newmaxradius);
}

\end{tikzpicture}
\end{center}
\end{frame}

%\begin{frame}
%\frametitle{The FF method: isolation cuts}
%%\beamercite{CMS-NOTE-2018-257}
%
%\begin{minipage}[c]{.45\textwidth}
%For \tauh,
%\begin{itemize}
%\item iso = Tight MVA-iso
%\item anti-iso = VLoose \& not Tight % MVA-iso
%\end{itemize}
%
%\manip AR very pure in events with fake taus.
%\manip Impurities from actual \tauh\ decays are very low (less than few \%, can be up to \SI{20}{\%} close to \Zboson\ boson mass peak)
%\end{minipage}
%\hfill
%\begin{minipage}[c]{.45\textwidth}
%\vspace{\graphh}
%\end{minipage}
%\begin{minipage}[c]{.45\textwidth}
%\begin{center}
%\includegraphics[width=\linewidth,height=\graphh,keepaspectratio]{\PhDthesisdir/tex/slides/HTT_analysis/bg_estimations/FF_method/FF_ppe-tikz.tex}
%\end{center}
%\end{minipage}
%
%\end{frame}
%
%\begin{frame}
%\frametitle{The FF method: obtaining amount of fakes}
%%\beamercite{CMS-NOTE-2018-257}
%
%\begin{minipage}[c]{.45\textwidth}
%\manip One DR for each process giving fakes:\\$\Wboson+\text{jets}$, QCD multijets, $\quarkt\antiquarkt+\text{jets}$.
%
%\manip For each range of kinematic variables:\\
%\qquad $\pT$, $\eta$, nb. of hadrons, jet multiplicity...
%%TODO , $m_{\tau\tau}$\todo{?}.
%\begin{equation*}
%\boxed{n_{j\to\tauh} = n_\text{AR}\sum_i f_i \cdot \mathrm{FF}_i}
%\end{equation*}
%
%\manip $f_i$ and $\mathrm{FF}_i$ are then functions of the kinematic variables.
%\end{minipage}
%\hfill
%\begin{minipage}[c]{.45\textwidth}
%\vspace{\graphh}
%\end{minipage}
%\begin{minipage}[c]{.45\textwidth}
%\begin{center}
%\includegraphics[width=\linewidth,height=\graphh,keepaspectratio]{\PhDthesisdir/tex/slides/HTT_analysis/bg_estimations/FF_method/FF_ppe-tikz.tex}
%\end{center}
%\end{minipage}
%\end{frame}

\begin{frame}
\frametitle{The FF method: determination regions definitions}
%\beamercite{CMS-NOTE-2018-257}

\begin{block}{$\quarkt\antiquarkt+\text{jets}$}
Estimation from simulated samples, same selection as in SR.
\end{block}

\pause\vfill

\begin{block}{$\Wboson+\text{jets}$}
Same as SR, except:
\begin{itemize}
\item transverse mass $m_T^{(\ell)}>\SI{70}{GeV}$ ($m_T^{(\ell)}<\SI{50}{GeV}$ in the SR);
\item no \quarkb-jet (allowed in the SR).
\end{itemize}
\end{block}

\pause\vfill

\begin{block}{QCD multijets}
Same as SR, except:
\begin{itemize}
\item same signs for $L_1$ and $L_2$ electric charges (opposite signs in the SR).
\end{itemize}
\end{block}

\end{frame}

%\begin{frame}
%\frametitle{The FF method}
%%\beamercite{CMS-NOTE-2018-257}
%
%\manip Maybe the DR are not \SI{100}{\%} pure in terms of process-of-interest.
%
%\manip To ensure purity, slighlty change $\mathrm{FF}_i$ definition
%\begin{equation*}
%\mathrm{FF}_i = \frac{n_\text{iso}}{n_\text{anti-iso}}
%\rightsquigarrow
%\mathrm{FF}_i = \frac{n_\text{iso} - n_\text{iso}^\text{rest}}{n_\text{anti-iso} - n_\text{anti-iso}^\text{rest}}
%\end{equation*}
%\begin{center}
%{\small $n_x^\text{rest}$ = impurity of backgrounds other than from the process-of-interest in the DR, MC-driven.}
%\end{center}
%\end{frame}

\begin{frame}
\frametitle{The FF method}
%\beamercite{CMS-NOTE-2018-257}
\begin{center}
\includegraphics[width=\graphw,height=\graphh,keepaspectratio]{\PhDthesisdir/tex/slides/HTT_analysis/bg_estimations/FF_method/FF_ppe-tikz.tex}
\end{center}
\end{frame}