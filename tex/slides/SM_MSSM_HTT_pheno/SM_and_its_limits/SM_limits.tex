\begin{frame}
\frametitle{The Standard Model and its limits}
%pas de gravitation, déséquilibre matière/antimatière, matière noire, masses des neutrinos.
%Ce modèle ne rend cependant pas
%compte de nombreuses observations expérimentales, telles l’existence de la matière et
%de l’énergie noire et l’asymétrie entre matière et antimatière, et est généralement consi-
%déré comme une théorie effective de basse énergie d’une théorie plus fondamentale.
%De nombreuses théories au-delà du Modèle Standard ont été proposées mais aucune
%n’a encore été confirmée.

\begin{center}\Large
\begin{tikzpicture}[scale=2]
\draw (0,0) node {naturalness?};
\draw (1.5,1) node {matter vs antimatter?};
\draw (-2,.125) node {dark matter?};
\draw (-1.5,.75) node {dark energy?};
\draw (-1.25,-.875) node {gravity?};
\draw (2,-1.5) node {neutrinos masses?};
\end{tikzpicture}
\end{center}

\end{frame}

%\begin{frame}
%\frametitle{Naturalness and supersymmetry}
%\manip Higgs mass: $\displaystyle m_{\higgs}^{2}=2\mu ^{2}+\delta m_{\higgs}^{2} \simeq \SI{125}{GeV}$
%\begin{equation*}
%\delta m_{\higgs}^{2}\simeq {\frac {3}{4\pi ^{2}}}{\Bigl (}-\lambda _{\quarkt}^{2}+{\frac {g^{2}}{4}}+{\frac {g^{2}}{8\cos ^{2}\theta _{W}}}+\lambda {\Bigr )}\Lambda ^{2}
%\end{equation*}
%
%For instance, if one includes see-saw neutrinos into the Standard Model, then $\delta m_{\higgs}$ would blow up to near the see-saw scale, typically expected in the \SI{e13}{GeV} range. 
%
%\end{frame}
%
%\begin{frame}
%\frametitle{Naturalness and supersymmetry}
%\manip By supersymmetrizing the Standard Model, one arrives at a solution to the gauge hierarchy, or big hierarchy, problem in that supersymmetry guarantees cancellation of quadratic divergences to all orders in perturbation theory.
%
%\begin{fmffile}{Higgs_loop_fermion}\fmfstraight
\begin{fmfchar*}(30,20)
  \fmfleft{i}
  \fmfright{o}
  \fmf{dashes}{i,v1}
  \fmf{dashes}{v2,o}
  \fmf{fermion,left,tension=.3}{v1,v2,v1}
  \fmfdot{v1,v2}
  \fmflabel{ }{i}
\end{fmfchar*}
\end{fmffile}

%
%\begin{fmffile}{Higgs_loop_sfermion}\fmfstraight
\begin{fmfchar*}(50,30)
  \fmfleft{i}
  \fmfright{o}
  \fmf{dashes}{i,v,v,o}
\end{fmfchar*}
\end{fmffile}

%
%\end{frame}