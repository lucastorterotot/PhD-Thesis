\begin{frame}
\frametitle{Discriminant variable?}

\begin{minipage}[t]{.45\textwidth}
\manip \MET\ due to neutrinos.
\submanip No invariant mass!

\vspace{\baselineskip}

\begin{center}
\begin{fmffile}{H-tautau_mutau-nu_in_red}\fmfstraight
\begin{fmfchar*}(50,30)
  \fmfleft{h}
  \fmfright{nu1,upq,doq,muout,antinumu,nu2}
  \fmf{dashes, label=$\Hs,, \Hn,, \Ha$, l.side=left, tension=5}{h,v}
  \fmf{fermion, label=$\antitau$, l.side=left, tension=4}{t1d,v}
  \fmf{fermion, label=$\leptau$, l.side=left, tension=4}{v,t2d}
  \fmf{fermion, tension=3}{nu1,t1d}
  \fmf{boson, label=$\Wbosonplus$, l.side=left, tension=2}{t1d,W1d}
  \fmf{fermion}{doq,W1d,upq}
  \fmf{fermion, tension=3}{t2d,nu2}
  \fmf{boson, label=$\Wbosonminus$, tension=2}{t2d,W2d}
  \fmf{fermion}{antinumu,W2d,muout}
  \fmflabel{$\color{ltcolorred}\bm{\antinutau}$}{nu1}
  \fmflabel{$\color{ltcolorred}\bm{\nutau}$}{nu2}
  \fmflabel{$\muon$}{muout}
  \fmflabel{$\color{ltcolorred}\bm{\antinumu}$}{antinumu}
  \fmflabel{$\antiquarkd$}{doq}
  \fmflabel{$\quarku$}{upq}
  \fmfdot{v,t1d,t2d,W1d,W2d}
\end{fmfchar*}
\end{fmffile}

\end{center}

\end{minipage}
\hfill\pause
\begin{minipage}[t]{.45\textwidth}
\manip Consider

\vspace{-\baselineskip}

\begin{center}
\begin{fmffile}{H-tautau_mutau-for_mTtot}\fmfstraight
\begin{fmfchar*}(50,30)
  \fmfleft{h}
  \fmfright{La,NU,Lb}
  \fmf{dashes, label=$\Hs,, \Hn,, \Ha$, l.side=left,tension=2}{h,v}
  \fmf{fermion, label=$\antitau$, l.side=left}{t1d,v}
  \fmf{fermion, label=$\leptau$, l.side=left}{v,t2d}
  \fmf{phantom}{t1d,La}
  \fmf{phantom}{t2d,Lb}
  \fmf{phantom, tension=2}{t2d,m,t1d}
  \fmffreeze
  \fmf{plain}{La,t1d}
  \fmf{plain}{t2d,Lb}
  \fmf{plain}{m,NU}
  \fmflabel{$L_2$}{La}
  \fmflabel{$L_1$}{Lb}
  \fmflabel{$\sum\neutrino$}{NU}
  \fmfdot{v}
  \fmfblob{.2w}{m}
\end{fmfchar*}
\end{fmffile}

\end{center}

\vspace{-2\baselineskip}

where
\begin{itemize}
\item $L_1 = \mu$;
\item $L_2=\quarku\antiquarkd\to\tauh$;
\item $\sum\nu\simeq\MET$;
\end{itemize}
with respect to the left side.
\end{minipage}

\end{frame}

\begin{frame}
\frametitle{Discriminant variable: \mTtot}
\manip For $L_1$, $L_2$ and \MET\ system,
\submanip in the transverse plane (use \MET),
\submanip for $E_i \gg m_i$ (highly relativistic case),

deriving the \og invariant \fg{} mass would then lead to
\begin{center}
the \textbf{total transverse mass}, \mTtot
\end{center}
\begin{equation*}
\boxed{\mTtot = \sqrt{\mT^2(L_1,\MET) + \mT^2(L_2,\MET) + \mT^2(L_1,L_2)}}
\end{equation*}
\begin{minipage}[c]{.6\textwidth}
\begin{equation*}
\mT(1,2) = \sqrt{2p_T^{(1)} p_T^{(2)} (1-\cos\Delta\phi)}
\end{equation*}
\end{minipage}
\begin{minipage}[c]{.35\textwidth}
\begin{center}
\begin{tikzpicture}
\def\aangle{30}
\def\bangle{150}
\def\aRadius{1.75}
\def\bRadius{1.25}
\def\cRadius{.5}
\draw [-latex, ltcolorred] (0,0) --+ (\aangle:\aRadius) node [above] {$\vpT^{(1)}$};
\draw [-latex, ltcolorblue] (0,0) --+ (\bangle:\bRadius) node [above] {$\vpT^{(2)}$};
\draw (\aangle:\cRadius) arc (\aangle:{(\aangle+\bangle)/2}:\cRadius) node [above] {$\Delta\phi$}  arc ({(\aangle+\bangle)/2}:\bangle:\cRadius);
\end{tikzpicture}
\end{center}
\end{minipage}
\end{frame}
