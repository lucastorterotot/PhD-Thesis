\begin{frame}
\frametitle{The Fake Factor method}
\beamercite{CMS-NOTE-2019-170}

\manip How many events contain misidentified \tauh? (fake taus)

\pause

\begin{center}
\includegraphics[width=\graphw,height=\graphh/13*10,keepaspectratio]{\PhDthesisdir/plots_and_images/Fake_Factor/FF_ppe-tikz-partial.tex}
\end{center}
\end{frame}

\begin{frame}
\frametitle{Particles isolation -- qualitatively}
\begin{center}
\begin{tikzpicture}
\begin{scope}
\clip (-\graphw/2,-\graphh/2.1) rectangle (\graphw/2,\graphh/2.1);

\drawCMS

\printmuondeposit{7}{10}
\printmuon{7}{10}


\printjetdeposit{9}{-60}
\printmuondeposit{7}{-60}
\printjet{9}{-60}
\printmuonnolabel{7}{-60}

\printtauhdeposit{9}{150}
\printtauh{9}{150}

\printjetdeposit{9}{-150}
\printjetfakenolabel{9}{-150}

\printDR{0.4}{150}
\printDR{0.4}{-150}
\end{scope}

\draw (110:\ECALrout) node [text depth=0pt] (rt) {\textbf{real \tauh:}};
\draw (rt.east) node [right, text depth=0pt] {few tracks = isolated};
\draw (-125:1.1*\Solenrin) node (ft) {\textbf{${\text{jet}\to\tauh}$}};
\draw (ft) node [below] {\textbf{= \ftauh}};
\draw (ft.east) node [below right, text depth=0pt] {more tracks = not isolated};

\draw [ltcolorred, thick, -latex] (rt) to [out = -60, in = 45] (140:.9*\trackerrout);
\draw [ltcolorred, thick, -latex] (ft.east) to [out = 20, in = -50] (-140:.9*\trackerrout);

\end{tikzpicture}
\end{center}
\end{frame}

%\begin{frame}
%\frametitle{The FF method: isolation cuts}
%\beamercite{CMS-NOTE-2019-170}
%
%\begin{minipage}[c]{.45\textwidth}
%For \tauh,
%\begin{itemize}
%\item iso = Tight MVA-iso
%\item anti-iso = VLoose \& not Tight % MVA-iso
%\end{itemize}
%
%\manip AR very pure in events with fake taus.
%\manip Impurities from actual \tauh\ decays are very low (less than few \%, can be up to \SI{20}{\%} close to \Zboson\ boson mass peak)
%\end{minipage}
%\hfill
%\begin{minipage}[c]{.45\textwidth}
%\vspace{\graphh}
%\end{minipage}
%\begin{minipage}[c]{.45\textwidth}
%\begin{center}
%\includegraphics[width=\linewidth,height=\graphh,keepaspectratio]{\PhDthesisdir/plots_and_images/Fake_Factor/FF_ppe-tikz.tex}
%\end{center}
%\end{minipage}
%
%\end{frame}
%
%\begin{frame}
%\frametitle{The FF method: obtaining amount of fakes}
%\beamercite{CMS-NOTE-2019-170}
%
%\begin{minipage}[c]{.45\textwidth}
%\manip One DR for each process giving fakes:\\\Wjets, QCD multijet, $\ttbar+\text{jets}$.
%
%\manip For each range of kinematic variables:\\
%\qquad $\pT$, $\eta$, nb. of hadrons, jet multiplicity...
%%TODO , $m_{\tau\tau}$\todo{?}.
%\begin{equation*}
%\boxed{n_{j\to\tauh} = n_\text{AR}\sum_i f_i \cdot \mathrm{FF}_i}
%\end{equation*}
%
%\manip $f_i$ and $\mathrm{FF}_i$ are then functions of the kinematic variables.
%\end{minipage}
%\hfill
%\begin{minipage}[c]{.45\textwidth}
%\vspace{\graphh}
%\end{minipage}
%\begin{minipage}[c]{.45\textwidth}
%\begin{center}
%\includegraphics[width=\linewidth,height=\graphh,keepaspectratio]{\PhDthesisdir/plots_and_images/Fake_Factor/FF_ppe-tikz.tex}
%\end{center}
%\end{minipage}
%\end{frame}

\begin{frame}
\frametitle{The FF method: determination regions definitions}
\beamercite{CMS-NOTE-2019-170}

\begin{block}{\ttbar}
Estimation from simulated samples, same selection as in SR.
\end{block}

\pause\vfill

\begin{block}{\Wjets}
Same as SR, except:
\begin{itemize}
\item transverse mass $m_T^{(\ell)}>\SI{70}{\GeV}$ ($m_T^{(\ell)}<\SI{70}{\GeV}$ in the SR);
\item no \quarkb-jet (allowed in the SR).
\end{itemize}
\end{block}

\pause\vfill

\begin{block}{QCD multijet}
Same as SR, except:
\begin{itemize}
\item same signs for $L_1$ and $L_2$ electric charges (opposite signs in the SR).
\end{itemize}
\end{block}

\vfill
\end{frame}

%\begin{frame}
%\frametitle{The FF method}
%\beamercite{CMS-NOTE-2019-170}
%
%\manip Maybe the DR are not \SI{100}{\%} pure in terms of process-of-interest.
%
%\manip To ensure purity, slighlty change $\mathrm{FF}_i$ definition
%\begin{equation*}
%\mathrm{FF}_i = \frac{n_\text{iso}}{n_\text{anti-iso}}
%\rightsquigarrow
%\mathrm{FF}_i = \frac{n_\text{iso} - n_\text{iso}^\text{rest}}{n_\text{anti-iso} - n_\text{anti-iso}^\text{rest}}
%\end{equation*}
%\begin{center}
%{\small $n_x^\text{rest}$ = impurity of backgrounds other than from the process-of-interest in the DR, MC-driven.}
%\end{center}
%\end{frame}

\begin{frame}
\frametitle{The FF method}
\beamercite{CMS-NOTE-2019-170}
\vspace{-20pt}
\begin{center}
\includegraphics[width=\graphw,height=\graphh,keepaspectratio]{\PhDthesisdir/plots_and_images/Fake_Factor/FF_ppe-tikz.tex}
\end{center}
\vspace{-10pt}
\end{frame}