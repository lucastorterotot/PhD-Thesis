\subsection*{Training mass range high boundary}
\begin{frame}
\backupslinklabel{high_mass_boundary}
\frametitle{Training mass range high boundary}

\begin{minipage}[c]{.53\textwidth}
\manip We used $\higgsML\to\tau\tau$ events:
\submanip $\higgsML$ is SM Higgs (pdg ID 25) with a different mass,
\submanip \higgsML\ produced by gluon fusion,
\submanip set $\BR(\higgsML\to\tau\tau)=1$ to avoid non di-\tau\ events.

\manip SM particles well known (wrt. BSM particles).

\manip We produced samples with BSM particles too, \textbf{but}:
\submanip theoritical uncertainties (unknown particles !),
\submanip for a same mass point, \tau\ kinematics do not match with \higgsML\ samples !
\submanip couplings effect ?
\end{minipage}
\hfill
\begin{minipage}[c]{.45\textwidth}
\begin{center}
\begin{fmffile}{A-Zh-tautaubb}%\fmfstraight
\begin{fmfchar*}(40,40)
  \fmfleft{A}
  \fmfright{b1,b2,tau1,tau2}
  \fmf{dashes, tension=4}{A,v1}
  \fmf{dashes, label=$\higgs/\Higgs$, l.side=right, tension=1.5}{v1,v2}
  \fmf{boson, label=$\Zboson$, l.side=left, tension=1.5}{v1,v3}
  \fmf{fermion}{b1,v2,b2}
  \fmf{fermion}{tau1,v3,tau2}
  \fmflabel{$\HiggsA$}{A}
  \fmflabel{$\antiquarkb$}{b1}
  \fmflabel{$\quarkb$}{b2}
  \fmflabel{$\antileptau$}{tau1}
  \fmflabel{$\leptau$}{tau2}
  \fmfdot{v1,v2,v3}
\end{fmfchar*}
\end{fmffile}


\vspace{\baselineskip}

What to predict here?
\end{center}
\end{minipage}

\end{frame}