\begin{center}
\LARGE
\bf
\sffamily
Recherche de bosons de Higgs supplémentaires de haute masse se désintégrant en paire de taus dans l'expérience CMS au LHC à l'aide du \emph{machine learning}
\end{center}

\vspace{2\baselineskip}

Malgré plusieurs décennies de prédictions expérimentalement vérifiées,
le modèle standard (SM) souffre de lacunes
que des théories allant au-delà (BSM, \emph{Beyond SM})
tentent de combler.
L'une d'entre elles,
l'extension supersymétrique minimale du modèle standard (MSSM),
prédit l'existence de cinq bosons de Higgs dont trois neutres,
\higgs\ correspondant au boson découvert en 2012
et~\Higgs\ et~\HiggsA\ supplémentaires.
Leur phénoménologie au LHC
motive l'étude des événements avec une paire de leptons~\tau.
\par
Dans ce contexte,
les événements récoltés par la collaboration CMS
de 2016 à 2018
lors des collisions de protons du LHC
avec une énergie dans le centre de masse de \SI{13}{\TeV},
correspondant à une luminosité intégrée de \SI{137}{\femto\barn^{-1}},
sont exploités dans cette thèse.
\par
Les jets sont des objets physiques omniprésents lors des collisions au LHC.
Leur calibration à l'aide d'événements contenant un photon et un jet (\Gjet) est présentée ainsi que les résultats obtenus pour l'année 2018.
Cette étude %,
%également réalisée pour l'année 2017-UL,
est directement utilisée dans la calibration officielle de la collaboration CMS.
\par
Aucun excès significatif par rapport aux bruits de fond attendus n'est observé
dans l'analyse des événements avec une paire de \tau\ aux bosons~\Higgs\ et~\HiggsA.
Des limites d'exclusion sur le produit de la section efficace de production de~\Higgs\ et~\HiggsA\ avec leur rapport de branchement aux \tau\ sont données
en fonction de leur masse
pour les modes de production par fusion de gluon ou en association avec des quarks~\quarkb.
Ces limites sont comprises entre
\SI{10}{\pico\barn} à \SI{110}{\GeV}
et
\SI{e-3}{\pico\barn} à \SI{3.2}{\TeV}
pour la fusion de gluons
et entre
\SI{0.4}{\pico\barn} à \SI{110}{\GeV}
et
\SI{8e-4}{\pico\barn} à \SI{3.2}{\TeV}
en association avec des quarks~\quarkb.
Dans le scénario $M_{\higgs}^{125}$,
ces limites se traduisent en une région d'exclusion dans le plan $(m_{\HiggsA},\tan\beta)$.
Les valeurs de $m_{\HiggsA}$ inférieures à \SI{600}{\GeV} sont exclues.
Cette limite passe à
%\SI{1}{\TeV} pour $\tan\beta\gtrsim\num{10}$ et
\SI{2}{\TeV} pour $\tan\beta\gtrsim\num{50}$.
Dans le scénario $M_{\Higgs_1}^{125}(\text{CPV})$,
cette région est donnée dans le plan $(m_{\Higgspm},\tan\beta)$.
Les valeurs de $m_{\Higgspm}$ inférieures à \SI{400}{\GeV} sont exclues.
Lorsque
$\tan\beta\simeq\num{20}$,
l'exclusion s'étend jusqu'à
$m_{\Higgspm} \simeq \SI{1.4}{\TeV}$.
\par
La reconstruction de la masse d'une paire de \tau\
est cruciale pour les analyses
%en contenant dans l'état final, en particulier celles
où un boson~\Zboson\ ou de Higgs se désintègre en $\antitau\leptau$.
Les neutrinos issus des désintégrations des~\tau, invisibles dans le détecteur, compliquent cette tâche.
Le \emph{machine learning} apporte une solution.
Le modèle obtenu dans cette thèse permet
de reconstruire la masse d'une paire de~\tau\
de \SI{50}{\GeV} à \SI{800}{\GeV}
avec une résolution de
\SI{20}{\%} à \SI{50}{\GeV},
\SI{26}{\%} à \SI{250}{\GeV} et
\SI{22}{\%} à \SI{800}{\GeV}.
Ce modèle est
60 fois plus rapide
et propose une meilleure description du boson~\Zboson\
que l'algorithme \SVFIT\ actuellement utilisé au sein de la collaboration CMS.

\vfill

\noindent\textbf{\Large\sffamily Mots clés}\\
Higgs,
CMS,
Jets,
Calibration,
Tau,
Reconstruction de la masse,
Réseau de neurones,
Régression.

\vspace{2\baselineskip}