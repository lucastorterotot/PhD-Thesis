\begin{center}
\LARGE
\bf
\sffamily
Search for additional heavy Higgs bosons decaying to tau lepton pair in the CMS experiment at LHC with machine learning techniques
\end{center}

\vspace{2\baselineskip}

Despite decades of correct predictions,
physicists are convinced that the Standard Model (SM) does not show us the whole picture.
Among the various extensions going beyond the SM (BSM),
the Minimal Supersymmetric extension of the SM (MSSM)
predicts
two charged Higgs bosons, \Higgspm,
and three neutrals:
\higgs\ corresponding to the observed one discovered in 2012
and
\Higgs\ and \HiggsA\ being additionnals with respect to the SM.
The MSSM phenomenology motivates the focus on events containing a di-\tau\ pair.
The MSSM \HAtoTauTau\ analysis is thus the core of this thesis.
\par
In this thesis,
the search
for \Higgs\ and \HiggsA\ 
is performed
on the data collected
with the CMS detector from 2016 to 2018
on proton collisions
at a center-of-mass energy of \SI{13}{\TeV},
corresponding to an integrated luminosity of \SI{137}{\femto\barn^{-1}}.
\par
Jets are complex physics objets
obtained
in the proton collisions occuring at the CERN LHC.
Their calibration by
the study of events containing a photon and a jet (\Gjet)
and the corresponding results on 2018 data
are introduced.
These results %,
%as well as the ones obtained on 2017-UL data,
are used in the official CMS jet calibration.
\par
No significant deviation above the expected background is observed in the MSSM \HAtoTauTau\ analysis.
Model-independent limits are then set
on the product of the cross section and branching fraction
for the production
via gluon-fusion
or
in association with \quarkb~quarks.
These limits range
from
\SI{15}{\pico\barn} at \SI{110}{\GeV}
to
\SI{3e-4}{\pico\barn} at \SI{3.2}{\TeV}
for gluon-fusion
and from
\SI{1.2}{\pico\barn} at \SI{110}{\GeV}
to
\SI{3e-4}{\pico\barn} at \SI{3.2}{\TeV}
for \quarkb-associated production.
In the $M_{\higgs}^{125}$ scenario,
these limits translate into
an exclusion region in the $(m_{\HiggsA},\tan\beta)$ plane.
Values of $m_{\HiggsA}$ below \SI{600}{\GeV} are excluded
and this goes up to 
%\SI{1}{\TeV} for $\tan\beta\gtrsim\num{10}$ and
\SI{2}{\TeV} for $\tan\beta\gtrsim\num{50}$.
In the $M_{\Higgs_1}^{125}(\text{CPV})$ scenario,
the region is given in the $(m_{\Higgspm},\tan\beta)$ plane.
Values of $m_{\Higgspm}$ below \SI{400}{\GeV} are excluded
and this goes up to 
\SI{1.4}{\TeV} for $\tan\beta\simeq\num{20}$.
\par
To test any theory involving Higgs or \Zboson\ boson which are decaying to $\tau^+ \tau^-$,
the reconstruction of di-$\tau$ mass in a faster and more accurate way than the existing methods is crucial. 
However, it is an arduous task due to existence of neutrinos as decay product of each $\tau$ lepton which are invisible to detectors at LHC. 
Machine learning techniques bring a solution for this task. 
The reconstruction of the di-$\tau$ mass by a deep neural network (DNN)
is achieved in this thesis
from 
\SI{50}{\GeV} to \SI{800}{\GeV}
with a
\SI{20}{\%} resolution at \SI{50}{\GeV},
\SI{26}{\%} at \SI{250}{\GeV} and
\SI{22}{\%} at \SI{800}{\GeV}.
This DNN is
60 times faster
and
better at describing the \Zboson~boson
than the \SVFIT\ algorithm currently used in CMS.

\vfill

\noindent\textbf{\Large\sffamily Keywords}\\
Higgs,
CMS,
Jets,
Calibration,
Tau,
Mass reconstruction,
Neural Network,
Regression.

\vspace{2\baselineskip}