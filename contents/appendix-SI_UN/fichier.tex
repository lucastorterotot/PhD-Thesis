\chapter*{Systèmes d'unités}\label{appendix-SI_UN}
Le système d'unité utilisé dans cette thèse
est le système dit \og naturel \fg{} (UN),
différent du système international (SI) \cite{BIPM_SI_9}
plus largement utilisé.
À chaque grandeur physique correspond une dimension,
basée sur les sept dimensions et unités fondamentales du SI:
\begin{itemize}
\item le temps \dimT, en seconde;
\item la longueur \dimL, en mètre;
\item la masse \dimM, en kilogramme;
\item le courant électrique \dimI, en ampère;
\item la température \dimtemp, en kelvin;
\item la quantité de matière \dimN, en mole;
\item l'intensité lumineuse \dimJ, en candela.
\end{itemize}
La définition des sept unités fondamentales repose sur autant de constantes physiques dont la valeur est fixée par convention.
Ainsi,
la seconde est déterminée par la fréquence de la transition hyperfine du césium $\Delta \nu_{\ce{Cs}} = \SI{9192631770}{\hertz}$.
Le mètre peut alors être défini à partir de la vitesse de la lumière dans le vide $c=\SI{299792458}{\meter.\second^{-1}}$.
Puis,
le kilogramme est obtenu grâce à ces deux premières unités et la constante de Planck $h = \SI{6.62607015e-34}{\joule.\second}$.
Les grandeurs physiques usuellement rencontrées, exprimées avec ces unités, ont généralement des valeurs facilement manipulables.
Par exemple,
une personne peut mesurer \SI{1.7}{\meter} et peser \SI{70}{\kilo\gram}.
\par
Cependant, en physique des particules,
ces unités ne sont pas adaptées aux échelles des grandeurs physiques exploitées.
C'est pourquoi un autre système d'unités, dites naturelles, (UN) est utilisé dans lequel
$\hbar=c=1$.
Les dimensions fondamentales de masse, longueur et temps
sont remplacées par les dimensions de
$\hbar$, $c$ et d'énergie en \GeV,
avec $\hbar = \frac{h}{2\pi}$
et
$\SI{1}{\GeV} = \SI{e9}{\eV} = \num{e9} \times e \times \SI{1}{\volt}$,
$e$ étant la charge électrique élémentaire.
Ce changement de dimensions fondamentales permet toujours de décrire toutes les grandeurs physiques,
comme un changement de système de coordonnées permet toujours de décrire toutes les positions possibles.
\par
Dans le UN,
une masse s'exprime ainsi en \SI{}{\GeV.\clumunit^{-2}},
une longueur en \SI{}{\hbarunit.\clumunit.\GeV^{-1}},
une durée en \SI{}{\hbarunit.\GeV^{-1}},
etc.
Par abus de langage, comme $\hbar=c=1$,
les facteurs $\hbar$ et $c$ sont omis.
Une masse s'exprime alors en \SI{}{\GeV}, ce qui devrait pourtant correspondre à une énergie.
Par exemple,
l'électron possède une masse de
\SI{9.11e-31}{\kilo\gram},
soit
\SI{511}{\keV}.
Le tableau~\ref{tab-SI_UN} résume le passage entre
(\dimM, \dimL, \dimT), ($\hbar$, $c$, \SI{}{\GeV}) et ($\hbar=c=1$, \SI{}{\GeV}).
\begin{table}[h]
\centering
\def\clumunit{\text{$c$}}
\def\hbarunit{\text{$\hbar$}}
\begin{tabularx}{\textwidth}{lYYYl}
\toprule
Grandeur & \multicolumn{3}{c}{Dimensions: (\dimM, \dimL, \dimT), ($\hbar$, $c$, \SI{}{\GeV}), ($\hbar=c=1$, \SI{}{\GeV})} & Conversion $\text{SI}\rightarrow\text{UN}$\\
\midrule
Masse & \dimM & \SI{}{\GeV.\clumunit^{-2}} & \SI{}{\GeV} & $\SI{1}{\kilo\gram} = \SI{5.61e26}{\GeV}$ \\
Longueur & \dimL & \SI{}{\hbarunit.\clumunit.\GeV^{-1}} & \SI{}{\GeV^{-1}} & $\SI{1}{\meter} = \SI{5.07e15}{\GeV^{-1}}$ \\
Durée & \dimT & \SI{}{\hbarunit.\GeV^{-1}} & \SI{}{\GeV^{-1}} & $\SI{1}{\second} = \SI{1.52e24}{\GeV^{-1}}$ \\
Énergie & $\dimM \dimL^2 \dimT^{-2}$ & \SI{}{\GeV} & \SI{}{\GeV} & $\SI{1}{\joule} = \frac{\num{e-3}}{\num{1.6e-19}} \usp \SI{}{\GeV}$\\
Impulsion & $\dimM \dimL \dimT^{-1}$ & \SI{}{\GeV.\clumunit^{-1}} & \SI{}{\GeV} & $\SI{1}{\kilo\gram.\meter.\second^{-1}} = \SI{1.87e18}{\GeV}$ \\
\bottomrule
\end{tabularx}
\caption[Équivalence entre les systèmes d'unités international et naturel.]{Équivalence entre les systèmes d'unités international et naturel pour quelques grandeurs physiques.}
\label{tab-SI_UN}
\end{table}