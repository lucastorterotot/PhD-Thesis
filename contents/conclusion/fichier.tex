\chapter*{Conclusion et perspectives}\label{chapter-conclusion}
\addcontentsline{toc}{chapter}{Conclusion}

L'extension supersymétrique minimale du modèle standard (MSSM)
introduite au chapitre~\refChMSSM\
est une des théories visant à combler les lacunes du modèle standard (SM),
par exemple l'existence de la matière noire.
Dans le MSSM,
quatre bosons de Higgs sont présents en plus de celui prédit par le SM (\higgs),
deux chargés (\Higgspm)
et
deux neutres (\Higgs, \HiggsA).
La phénoménologie de ces derniers dans les collisionneurs tels que le LHC
favorise le canal de désintégration en paire de leptons~\tau\
(\HAtoTauTau)
pour la recherche d'un signal indiquant leur présence,
qui serait une preuve de l'existence d'une nouvelle physique
au-delà du SM (BSM).
\par
Les travaux réalisés lors de cette thèse
exploitent les données expérimentales
récoltées par le détecteur CMS
lors des collisions de protons du Run~II du LHC
à une énergie dans le centre de masse de \SI{13}{\TeV},
correspondant à une luminosité intégrée de \SI{137}{\femto\barn^{-1}}.
Le détecteur CMS a été présenté dans le chapitre~\refChLHCCMS.
Les particules issues des collisions y laissent des signaux lors de leur passage.
Ceux-ci sont utilisés pour reconstruire les particules correspondantes.
Les propriétés de ces dernières permettent d'étudier les processus physiques
ayant eu lieu lors de la collision initiale.
\par
La signature expérimentale
d'un parton issu de la collision, \ie\ un quark ou un gluon,
est un jet.
Il s'agit d'un objet physique complexe
composé de plusieurs particules.
De nombreuses analyses de la collaboration CMS exploitent les jets
et
les incertitudes liées à leur énergie sont généralement parmi les plus importantes.
La bonne caractérisation de ces objets physiques est donc cruciale.
Dans cette thèse,
les événements contenant un photon et un jet (\Gjet)
sont utilisés afin de calibrer les jets en énergie.
Les résultats obtenus lors de cette thèse pour les années 2018 et 2017-UL,
présentés dans le chapitre~\refChJERC,
sont exploités dans la calibration officielle de la collaboration.
\par
La recherche de bosons de Higgs supplémentaires de haute masse se désintégrant en paire de~\tau\
a été présentée dans le chapitre~\refChHTT.
L'estimation des bruits de fond dus aux processus du SM
est majoritairement obtenue à partir des données observées elles-mêmes.
Plusieurs catégories d'événements sont définies afin d'augmenter la sensibilité de l'analyse à \Higgs\ et \HiggsA.
Aucune déviation significative par rapport aux prédictions du SM n'est observée
dans
les distributions des variables discriminantes des différentes catégories.
Des limites hautes ont alors été déterminées par la méthode \CLS\ sur la section efficace de production de \Higgs\ et \HiggsA, multipliée par leur rapport de branchement à la désintégration en paire de leptons~\tau.
Une région d'exclusion du MSSM en faveur du SM,
déterminée dans le plan $(m_{\HiggsA},\tan\beta)$ pour le scénario $M_{\higgs}^{125}$,
permet d'exclure
$m_{\HiggsA} < \SI{600}{\GeV}$.
Dans le cas du scénario $M_{\Higgs_1}^{125}(\text{CPV})$,
une violation de $CP$ par les bosons de Higgs est possible.
Les états propres de masse sont
$\Higgs_1$, $\Higgs_2$ et $\Higgs_3$ au lieu de
\higgs, \Higgs\ et \HiggsA,
c'est pourquoi
la région d'exclusion est donnée dans le plan $(m_{\Higgspm},\tan\beta)$
et
$\Higgs_1$ doit jouer le rôle du boson découvert en 2012.
Les valeurs de $m_{\Higgspm}$ inférieures à \SI{400}{\GeV} sont exclues.
\par
Les neutrinos issus des désintégrations des leptons~\tau\
sont invisibles dans les détecteurs du LHC.
L'incertitude inhérente due à l'impossibilité de les mesurer
dégrade la précision et les performances des analyses.
Les travaux présentés dans le chapitre~\refChML\
ont permis d'obtenir un réseau de neurones profond (DNN)
capable de reconstruire la masse d'une particule se désintégrant en paire de leptons~\tau\
entre \SI{50}{\GeV} et \SI{800}{\GeV}
avec une précision de \num{20} à \SI{25}{\%}.
Une avancée majeure par rapport aux précédentes études est l'inclusion de l'empilement dans l'entraînement du DNN
qui peut alors être directement exploité dans les analyses de CMS.
\par
Malgré une meilleure résolution sur le signal de \Higgs\ et \HiggsA\ que \mTtot,
la variable discriminante actuellement utilisée dans l'analyse du chapitre~\refChHTT,
ce modèle ne permet pas d'étendre les limites d'exclusion obtenues.
Cela est dû aux contributions des fausses paires de leptons~\tau,
dont l'estimation de la masse par le DNN se retrouve dans la région de signal du MSSM.
Un bon estimateur de masse n'est donc pas forcément une bonne variable discriminante.
Cependant,
ce DNN permet de mieux
séparer le signal du boson~\Zboson\ de celui des processus \ttbar\
que \mTtot.
Il pourrait donc déjà bénéficier à d'autres analyses.
De plus,
une comparaison de ses estimations de masse à celles de l'algorithme \SVFIT\ utilisé par la collaboration CMS montre
une meilleure description du boson~\Zboson\ et une sensibilité au boson de Higgs du modèle standard
équivalente
dans certaines topologies d'événements,
tout en étant 60 fois plus rapide à fournir ces estimations.
Avec l'augmentation considérable de la quantité d'événements à traiter
dans les futures analyses de physique des particules,
notamment durant le Run~III du LHC et au HL-LHC,
la rapidité du DNN par rapport à \SVFIT\
est un atout majeur.
\par
L'entraînement du DNN est basé sur des événements
où un boson de Higgs \higgsML\ de masse variable est produit par fusion de gluons.
%Il s'agit du mode dominant de production de \higgs.
L'exploitation d'autres modes de production
pourrait améliorer les performances
du DNN sur les topologies correspondantes.
Il est de plus envisageable d'inclure
des événements correspondant expressément au boson~\Zboson\
et aux modes de production associés,
en plus de ceux pour lesquels $m_{\higgsML}\simeq m_{\Zboson}$,
pour augmenter encore le pouvoir discriminant du DNN
déjà plus important que celui de \mTtot\
dans cette région de l'espace des phases.
%\par
La prise en compte des fausses paires de leptons~\tau\
dans l'entraînement
n'est pas triviale
car il n'est pas possible de définir une valeur physique de masse à prédire
dans ce cas.
% ttbar case?
\par
L'utilisation du DNN issu des travaux de cette thèse
est déjà envisagée dans deux analyses de la collaboration CMS.
La première est l'analyse des événements avec une paire de leptons~\tau\
dans le cadre du NMSSM (\emph{Next to MSSM}),
théorie plus complexe que le MSSM dans laquelle existent sept bosons de Higgs au lieu de cinq.
La topologie des événements y est la même que dans l'analyse menée dans cette thèse,
l'implémentation du DNN est donc directe.
La seconde porte sur les événements avec deux bosons de Higgs du SM (\higgs),
l'un se désintégrant en paire de~\tau\
et l'autre en paire de quarks~\quarkb\
($\higgs\higgs\to\quarkb\antiquarkb\tau\tau$).
%Dans le cas \og non résonnant \fg,
%\ie\ où les deux bosons de Higgs sont produits par tout processus physique,
L'utilisation du DNN n'est pas directe à cause de la présence des deux jets de quarks~\quarkb,
situation inédite par rapport aux travaux menés dans cette thèse.
La prise en compte de ces jets dans les variables d'entrée du DNN n'est pas triviale
et
les études actuelles menées par le groupe en charge de cette analyse
tendent à montrer une dégradation des performances par rapport au simple cas $\higgsML\to\tau\tau$
pour lequel il est conçu.
Comme exprimé précédemment,
inclure des événements $\higgs\higgs\to\quarkb\antiquarkb\tau\tau$
dans l'entraînement du DNN pourrait remédier à cette dégradation.
Une piste prometteuse est d'étudier l'effet du boson de Higgs se désintégrant en $\quarkb\antiquarkb$ sur le recul de celui donnant la paire de~\tau.
\par
Dans l'optique de l'analyse \og résonnante \fg{} des événements
$\higgs\higgs\to\quarkb\antiquarkb\tau\tau$,
\ie\ lorsque les deux bosons \higgs\ sont eux-mêmes issus d'une seule particule initiale,
il est envisageable de créer un autre DNN
spécialement conçu pour prédire directement la masse de celle-ci.
Par rapport au DNN utilisé pour les événements $\higgsML\to\tau\tau$,
la liste des variables d'entrée pourrait être complétée
par des observables liées aux quarks~\quarkb\ et aux jets résultants.
\par
L'utilisation d'un DNN, ou plus généralement du \emph{machine learning}, afin d'estimer la masse d'une résonance se désintégrant en paire de leptons~\tau\
est ainsi très prometteuse.
Les analyses futures, en particulier lors du Run~III dès 2022,
pourraient bénéficier d'une meilleure résolution et d'une vitesse accrue par rapport aux outils actuels.
Ces avancées pourraient mener
à des mesures encore plus précises des paramètres du SM,
à mieux contraindre les modèles BSM tels que le MSSM et le NMSSM
voire
à l'observation d'une nouvelle physique.
De tels résultats permettront aux théoriciens de formuler de nouveaux modèles
plus complets
et plus à même de décrire l'Univers
que l'actuel SM,
qui seront eux-mêmes expérimentalement testés,
perpétuant ainsi la marche scientifique
ayant déjà mené avec succès à de nombreuses découvertes.