\section{Conclusion}\label{chapter-JERC-section-conclusion}
Ce chapitre a présenté la calibration en énergie des jets.
Elle permet d'obtenir des jets de la meilleure qualité possible pour les analyses de physique menées par l'ensemble de la collaboration CMS.
Il s'agit d'une approche factorisée de plusieurs corrections, chacune ayant pour but de corriger un effet en particulier.
\par L'obtention d'une de ces corrections pour les années 2018 et 2017-UL a fait partie de mon travail de thèse et a été développée plus en détails, ainsi que la phénoménologie des événements utilisés.
Il s'agit d'événements \Gjets\ dans lesquels l'équilibre entre le photon et un jet permet d'estimer l'échelle en énergie du jet connaissant celle du photon.
Pour l'année 2018, la réponse des jets dans ces événements est inférieure dans les données par rapport aux simulations
de
\SI{3}{\%} pour $\abs{\eta^\text{jet}} < \num{1.3}$
à
\SI{5}{\%} pour $\num{1.3} \leq \abs{\eta^\text{jet}} < \num{2.5}$.
L'incertitude absolue sur ces mesures est inférieure à \SI{0.3}{\%}.
\par La résolution en énergie des jets doit également être corrigée.
À l'aide d'une étude similaire, menée sur les mêmes événements et également détaillée dans ce chapitre, les facteurs d'échelle ont été obtenus lors de ma thèse pour les années 2018 et 2017-UL.
Pour l'année 2018, la résolution en énergie des jets dans ces événements est inférieure dans les données par rapport aux simulations
de
$(\num{6}\pm\num{3})\usp\%$ pour $\num{1.131} \leq \abs{\eta^\text{jet}} < \num{1.305}$
à
$(\num{60}\pm\num{9})\usp\%$ pour $\num{2.5} \leq \abs{\eta^\text{jet}} < \num{2.8}$.
\par Les jets sont omniprésents dans les collisions du LHC.
Ces travaux permettant la bonne caractérisation des jets sont donc essentiels pour réaliser des analyses de qualité.
Parmi elles se trouve celle présentée dans le chapitre~\refChHTT.