\section{Conclusion}\label{chapter-JERC-section-conclusion}
Ce chapitre a abordé le sujets des jets.
\par Dans un premier temps, leur mécanisme de formation a été décrit.
Leur apparition est due à l'interaction forte, d'abord dans un régime de haute énergie menant à une gerbe partonique, puis dans un régime de basse énergie donnant lieu au phénomène d'hadronisation.
\par Dans un second temps, la reconstruction des jets a été abordée.
Plusieurs algorithmes permettent de regrouper les particules individuellement reconstruites en jets.
La forme exacte des jets reconstruits dépend de la méthode utilisée.
Au sein de la collaboration CMS, l'algorithme de regroupement principalement utilisé est l'algorithme \og anti-\kT \fg.
Une liste de \og candidats \fg{} jets est ainsi obtenue.
Ces candidats doivent alors remplir des critères d'identification afin d'être effectivement considérés comme des jets.
\par La saveur des jets a brièvement été discutée.
En effet, selon le type de particule initiant le jet, ce dernier présente des caractéristiques variables.
Bien qu'il soit impossible de remonter à coup sûr à cette particule initiale, ces caractéristiques permettent de l'estimer.
\par Les jets ainsi reconstruits et identifiés sont des objets dits de \og haut niveau \fg\ qu'il est nécessaire de calibrer, comme tout autre objet reconstruit.
La procédure de calibration en énergie des jets utilisée dans la collaboration CMS a été présentée.
Elle permet d'obtenir des jets de la meilleure qualité possible pour les analyses de physique menées par l'ensemble de la collaboration.
Il s'agit d'une approche factorisée de plusieurs corrections, chacune ayant pour but de corriger un effet en particulier.
\par L'obtention d'une de ces corrections pour les années 2018 et 2017-UL a fait partie de mon travail de thèse et a été développée plus en détails, ainsi que la phénoménologie des événements utilisés.
Il s'agit d'événements \Gjets\ dans lesquels la balance entre le photon et un jet permet d'estimer l'échelle en énergie du jet connaissant celle du photon.
Pour l'année 2018, la réponse des jets dans ces événements est inférieure dans les données par rapport aux simulations
de
\SI{3}{\%} pour $\abs{\eta^\text{jet}} < \num{1.3}$
à
\SI{5}{\%} pour $\num{1.3} \leq \abs{\eta^\text{jet}} < \num{2.5}$.
L'incertitude absolue sur ces mesures est inférieure à \SI{0.3}{\%}.
\par La résolution en énergie des jets doit également être corrigée.
À l'aide d'une étude similaire menée sur les mêmes événements, également détaillée dans ce chapitre, les facteurs d'échelle ont été obtenus lors de ma thèse pour les années 2018 et 2017-UL.

Pour l'année 2018, la résolution en énergie des jets dans ces événements est inférieure dans les données par rapport aux simulations
de
$(\num{6}\pm\num{3})\usk\%$ pour $\num{1.131} \leq \abs{\eta^\text{jet}} < \num{1.305}$
à
$(\num{60}\pm\num{9})\usk\%$ pour $\num{2.5} \leq \abs{\eta^\text{jet}} < \num{2.8}$.
\par Les jets sont omniprésents dans les collisions du LHC.
Ces travaux permettent la bonne caractérisation des jets et sont donc essentiels pour réaliser des analyses de physique de qualité.
Parmi ces analyses se trouve celle des événements $\higgs\to\tau\tau$, sujet du chapitre~\ifref{chapter-HTT_analysis}{\ref{chapter-HTT_analysis}}{5}.
