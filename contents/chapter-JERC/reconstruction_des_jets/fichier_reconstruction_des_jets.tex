\section{Reconstruction des jets}\label{chapter-JERC-section-jets_reco}

q,g --> jet dans détecteur

\subsection{Algorithmes de reconstruction}\label{chapter-JERC-section-jets_reco-subsec-algo}

anti-\kT

M. Cacciari, et al. The anti-k t jet clustering algorithm. JHEP, 04 :063, 2008.
doi :10.1088/1126-6708/2008/04/063. 0802.1189.

\begin{equation}
d_{ij} = \min(\frac{1}{\pT^2_i}, \frac{1}{\pT^2_j}) \frac{\Delta R^2_{ij}}{R^2}
\end{equation}
voir le cours de GGrenier

produit des jets de forme régulière, plutôt conique

moins sensible aux perturbations dues aux partons spectateurs

regroupement autour des particules de plus haute énergie en utilisant les écarts angulaires

moins proche de l'évolution du parton shower

\subsection{Identification des jets dans CMS}\label{chapter-JERC-section-jets_reco-subsec-jetID}
quels critères?

\subsection{Saveur des jets}\label{chapter-JERC-section-jets_reco-subsec-flavor}
b-tagging

\begin{figure}
\centering
\begin{tikzpicture}[scale=1.5]
\def\trackerrin{.100}
\def\trackerrout{1.185}
\def\trackercolor{ltcolorgray1}

\def\ECALrin{1.290}
\def\ECALrout{1.811}
\def\ECALcolor{ltcolorgreen1}

\def\HCALrin{1.812}
\def\HCALrout{2.854}
\def\HCALcolor{ltcoloryellow3}

\def\Solenrin{2.950}
\def\Solenrout{3.800}
\def\Solencolor{ltcolorgray2}

\def\ironryrina{3.850}
\def\ironryrouta{4.000}
\def\muonrina{4.020}
\def\muonrouta{4.400}
\def\ironryrinb{4.420}
\def\ironryroutb{4.880}
\def\muonrinb{4.905}
\def\muonroutb{5.285}
\def\ironryrinc{5.300}
\def\ironryroutc{5.960}
\def\muonrinc{5.975}
\def\muonroutc{6.355}
\def\ironryrind{6.375}
\def\ironryroutd{6.980}
\def\muonrind{7.000}
\def\muonroutd{7.380}
\def\muoncolor{ltcoloryellow1}
\def\ironrycolor{ltcolorred2}

\def\printele#1{
\draw [thick, ltcolorred] (0,0) arc (#1-90:#1-90+27:3) coordinate (eledeposit);
\draw [ltcolorred] (#1-5:1.25) node {\ele};
}
\def\printmu#1{
\draw [thick, ltcolorblue] (0,0) arc (#1-90:#1-90+33:6) arc (#1-90+33:#1-90:-12) node{\mu};
\draw [ltcolorblue] (#1-7:1.5) node {\mu};
}

\def\printantiele#1{
\draw [thick, ltcolorred] (0,0) arc (#1-90:#1-90-27:-3) coordinate (eledeposit);
\draw [ltcolorred] (#1-7:1.5) node {\ele};
%\draw [ltcolorred4, ultra thick] (eledeposit)--+(#1+25:\ECALrout);
}
\def\printantimu#1{
\draw [thick, ltcolorblue] (0,0) arc (#1-90:#1-90-33:-6) arc (#1-90-33:#1-90:12);
\draw [ltcolorblue] (#1-7:1.5) node {\mu};
}

\def\printtauh#1{
\draw [thick, ltcolorgreen4] (0,0) arc (#1-90:#1-90+11:10) ;
\draw [thick, ltcolorgreen4] (0,0) arc (#1-90:#1-90+6:20) ;
\draw [thick, ltcolorgreen4] (0,0) arc (#1-90:#1-90-11:-10) ;
\draw [ltcolorgreen4] (#1-12:1.5) node {\tauh};
}
\def\printantitauh#1{
\draw [thick, ltcolorgreen4] (0,0) arc (#1-90:#1-90-11:-10) ;
\draw [thick, ltcolorgreen4] (0,0) arc (#1-90:#1-90-6:-20) ;
\draw [thick, ltcolorgreen4] (0,0) arc (#1-90:#1-90+11:10) ;
\draw [ltcolorgreen4] (#1-12:1.5) node {\tauh};
}

\def\printjetnolabel#1{
\draw [thick, ltcolororange] (0,0) arc (#1-90+10:#1-90+22+10:5) ;
\draw [thick, ltcolororange] (0,0) arc (#1-90+5:#1-90+12+5:10) ;
\draw [thick, ltcolororange] (0,0) arc (#1-90:#1-90-22:-5) ;
\draw [thick, ltcolororange] (0,0) arc (#1-90:#1-90+6:20) ;
\draw [thick, ltcolororange] (0,0) arc (#1-90+5:#1-90+8+5:10) ;
\draw [thick, ltcolororange] (0,0) arc (#1-90:#1-90-11:-10) ;
\draw [thick, ltcolororange] (0,0) arc (#1-90:#1-90+11:10) ;
}

\def\printjet#1{
\printjetnolabel{#1}
\draw [ltcolororange] (#1-25:.5) node {jet};
}

\def\printjetfake#1{
\printjet{#1}
\draw [thick, ltcolorgreen4] (0,0) arc (#1-90:#1-90+11:10) ;
\draw [thick, ltcolorgreen4] (0,0) arc (#1-90:#1-90+6:20) ;
\draw [thick, ltcolorgreen4] (0,0) arc (#1-90:#1-90-11:-10) ;
\draw [ltcolorgreen4] (#1-17:1.5) node {f.\tauh};
}

\def\printdeposit#1#2#3#4{
\fill [#1] (#2-2:#3) arc (#2-2:#2+2:#3) -- (#2+2:#4) arc (#2+2:#2-2:#4) ;
}

\def\printECALdeposit#1#2{\printdeposit{#1}{#2}{\ECALrin}{\ECALrout}}
\def\printHCALdeposit#1#2{\printdeposit{#1}{#2}{\HCALrin}{\HCALrout}}

\def\printtauhdeposit#1{
\printHCALdeposit{ltcoloryellow4}{#1+3}
\printHCALdeposit{ltcoloryellow4}{#1+5}
\printHCALdeposit{ltcoloryellow4}{#1-5}
}

\def\printjetdeposit#1{
\printHCALdeposit{ltcoloryellow4}{#1+3}
\printHCALdeposit{ltcoloryellow4}{#1+5}
\printHCALdeposit{ltcoloryellow4}{#1-5}
\printHCALdeposit{ltcoloryellow4}{#1+21}
\printHCALdeposit{ltcoloryellow4}{#1+11}
\printHCALdeposit{ltcoloryellow4}{#1-11}
}

\def\printMuChSigA#1#2{
\fill [red] (#1-7.5+20*#2:\muonrina) arc (#1-7.5+20*#2:#1+7.5+20*#2:\muonrina) -- (#1+7.5+20*#2:\muonrouta) arc (#1+7.5+20*#2:#1-7.5+20*#2:\muonrouta) ;
}
\def\printMuChSigB#1#2{
\fill [red] (#1-7.5+20*#2:\muonrinb) arc (#1-7.5+20*#2:#1+7.5+20*#2:\muonrinb) -- (#1+7.5+20*#2:\muonroutb) arc (#1+7.5+20*#2:#1-7.5+20*#2:\muonroutb) ;
}
\def\printMuChSigC#1#2{
\fill [red] (#1-7.5+20*#2:\muonrinc) arc (#1-7.5+20*#2:#1+7.5+20*#2:\muonrinc) -- (#1+7.5+20*#2:\muonroutc) arc (#1+7.5+20*#2:#1-7.5+20*#2:\muonroutc) ;
}
\def\printMuChSigD#1#2{
\fill [red] (#1-7.5+20*#2:\muonrind) arc (#1-7.5+20*#2:#1+7.5+20*#2:\muonrind) -- (#1+7.5+20*#2:\muonroutd) arc (#1+7.5+20*#2:#1-7.5+20*#2:\muonroutd) ;
}

\foreach\jetangle in {120,-145}{
\printbigjetnolabel{\jetangle}
\draw (\jetangle:2.5) node {jet} ;
}

\def\Bjetangle{30}
\def\Bhaddronangle{5}
\def\Bhaddronflight{.75}

\def\CurrentVertex{(\Bhaddronangle:\Bhaddronflight)}

\draw [thick, dotted] (0,0) -- \CurrentVertex ;

{
\def\jetcolor{ltcolorred}
\printjetnolabel{\Bjetangle}
\draw [\jetcolor] (\Bjetangle:2)+(1.5,.75) node {traces déplacées};
}
\begin{scope}
\clip circle (\Solenrin);
\printantimuonnolabel{\Bjetangle}
\end{scope}

\draw [\muoncolor] (\Bjetangle:4) +(0,{-2*\baselineskip}) node {lepton chargé};

\draw (\Bjetangle:4) + (-.125,-.5) circle (1.125);
\draw [-latex] (\Bjetangle:4) + (-.125,-1.625) --+ (-.125,-2) node [below] {jet de saveur lourde};

\draw (0,0) node [left] {PV} ;
\draw \CurrentVertex node [below right] {SV} ;

\fill (0,0) circle (2pt);
\fill \CurrentVertex circle (2pt);

\draw [dashed] \CurrentVertex --+ (\Bjetangle:-.8);

\draw [red, latex-latex] (0,0)--+ (-90+\Bjetangle:{\Bhaddronflight*sin(\Bjetangle-\Bhaddronangle)}) node [below right] {IP};
\end{tikzpicture}
\end{figure}