\section{Reconstruction des jets}\label{chapter-JERC-section-jets_reco}

q,g --> jet dans détecteur

\subsection{Algorithmes de reconstruction}\label{chapter-JERC-section-jets_reco-subsec-algo}

anti-\kT

M. Cacciari, et al. The anti-k t jet clustering algorithm. JHEP, 04 :063, 2008.
doi :10.1088/1126-6708/2008/04/063. 0802.1189.

\begin{equation}
d_{ij} = \min(\frac{1}{\pT^2_i}, \frac{1}{\pT^2_j}) \frac{\Delta R^2_{ij}}{R^2}
\end{equation}
voir le cours de GGrenier

produit des jets de forme régulière, plutôt conique

moins sensible aux perturbations dues aux partons spectateurs

regroupement autour des particules de plus haute énergie en utilisant les écarts angulaires

moins proche de l'évolution du parton shower

\subsection{Identification des jets dans CMS}\label{chapter-JERC-section-jets_reco-subsec-jetID}
quels critères?

\subsection{Saveur des jets}\label{chapter-JERC-section-jets_reco-subsec-flavor}
b-tagging
