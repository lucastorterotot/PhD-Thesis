\section{Introduction}\label{chapter-JERC-section-introduction}
Lors des collisions de protons, comme expliqué au chapitre~\refChLHCCMS,
ce sont en réalité leurs constituants respectifs qui interagissent,
\ie\ les quarks et les gluons, regroupés sous le terme de \og partons \fg.
Il est donc très probable que l'interaction forte intervienne.
De plus, les particules les plus lourdes issues de ces collisions, instables, peuvent émettre des quarks en se désintégrant.
De nombreux partons sont ainsi émis lors des collisions de protons au LHC.
Comme exposé dans le chapitre~\refChMSSM, les partons issus des collisions forment des jets, objets physiques de haut niveau dont la reconstruction est introduite au chapitre~\refChLHCCMS.
Ces jets sont omniprésents dans les analyses réalisées par la collaboration CMS,
leur caractérisation est donc un point essentiel.
Or, les jets sont des objets physiques composés de nombreuses particules.
Leur calibration en énergie est ainsi nécessaire afin d'obtenir une estimation de leur énergie réelle la plus fidèle possible.
Les étapes de cette calibration dans l'expérience CMS sont abordées dans la section~\ref{chapter-JERC-section-CMS}.
Lors de ma thèse, j'ai contribué à la détermination de cette calibration.
L'étape correspondante est discutée plus en détails.
Elle utilise des événements \Gjets, où un photon et au moins un jet sont présents.
La section~\ref{chapter-JERC-section-pheno-GJets} en présente la phénoménologie
et
la section~\ref{chapter-JERC-section-JES} aborde l'obtention de la calibration.
La correction de la résolution en énergie des jets à l'aide de ces mêmes événements est présentée dans la section~\ref{chapter-JERC-section-JER}.