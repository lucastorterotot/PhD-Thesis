\section{Introduction}\label{chapter-JERC-section-introduction}
Les quarks et les gluons, également nommés \og partons \fg, sont les particules portant une charge de couleur non nulle.
Ils sont donc sensibles à l'interaction forte, abordée dans le chapitre~\refChMSSM.
La constante de couplage de cette interaction, $\alpha_s$, décroît avec l'énergie.
Ainsi, aux hautes énergies, $\alpha_s \to 0$.
Ce phénomène est nommé \og liberté asymptotique \fg.
\par
L'échelle d'énergie atteinte au LHC, de l'ordre du \SI{}{\TeV}, permet de se placer dans ce régime de liberté asymptotique.
L'interaction forte peut alors être décrite à l'aide de calculs perturbatifs, comme $\alpha_s \to 0$.
Toutefois, cela n'est vrai qu'au moment de la collision initiale entre les protons.
En effet, les partons issus de cette collision, du fait de l'interaction forte à laquelle ils sont sensibles, vont émettre d'autres partons.
Par conservation, l'énergie portée par une de ces particules diminue au fur et à mesure de ces émissions et $\alpha_s$ augmente.
À des énergies de l'ordre de la centaine de \SI{}{\MeV}, il n'est plus possible de réaliser des calculs perturbatifs.
Le phénomène de confinement de couleur, présenté dans le chapitre~\refChMSSM, réapparaît alors et les partons sont confinés au sein de hadrons, dont la charge de couleur est nulle.
Ce processus est appelé \og hadronisation \fg.
Les partons produits lors des collisions de protons au LHC se manifestent ainsi sous la forme d'une gerbe de particules, un \og jet \fg.
\par
Lors des collisions de protons, comme expliqué au chapitre~\refChLHCCMS,
ce sont en réalité leurs constituants respectifs qui interagissent,
\ie\ les quarks et les gluons.
Il y a donc une forte probabilité pour que l'interaction forte intervienne.
De plus, les particules les plus lourdes issues de ces collisions, instables, peuvent émettre des quarks en se désintégrant.
De nombreux partons sont ainsi émis lors des collisions de protons au LHC.
Il en résulte une omniprésence des jets dans les analyses réalisées par la collaboration CMS.
Leur caractérisation est donc un point essentiel pour la collaboration.
Or, les jets sont des objets physiques composés de nombreuses particules.
Leur calibration en énergie est ainsi nécessaire afin d'obtenir une estimation de leur énergie réelle la plus fidèle possible.
\par
Le processus de formation des jets est décrit d'un point de vue théorique dans la section~\ref{chapter-JERC-section-jets}.
La méthode utilisée afin de les reconstruire est présentée dans la section~\ref{chapter-JERC-section-jets_reco}.
Ensuite, le principe de leur calibration en énergie dans l'expérience CMS est abordé dans la section~\ref{chapter-JERC-section-CMS}.
Une des étapes de cette calibration est discutée plus en détails.
Elle utilise des événements \Gjets, où un photon et au moins un jet sont présents.
La section~\ref{chapter-JERC-section-pheno-GJets} en présente la phénoménologie
et
la section~\ref{chapter-JERC-section-JES} aborde l'obtention de la calibration.
Enfin, la correction de la résolution en énergie des jets à l'aide de ces mêmes événements est présentée dans la section~\ref{chapter-JERC-section-JER}.