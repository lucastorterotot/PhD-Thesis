\section{Introduction}\label{chapter-JERC-section-introduction}
L'interaction forte est abordée dans le chapitre~\ifref{chapter-MS-MSSM}{\ref{chapter-MS-MSSM}}{sur le modèle standard}.
Les particules portant une charge de couleur non nulle y sont sensibles.
Il s'agit des quarks et des gluons, également nommés \og partons \fg.
La constante de couplage de l'interaction forte, $\alpha_s$, décroît  avec l'énergie.
Ainsi se produit à haute énergie le phénomène de \og liberté asymptotique \fg.
\par L'échelle d'énergie atteinte au LHC, de l'ordre du \SI{}{\TeV}, permet de réaliser des calculs perturbatifs car dans ce cas $\alpha_s \to 0$. Toutefois, cela n'est vrai qu'au moment de la collision initiale entre les protons.
En effet, les partons issus de cette collision, du fait de l'interaction forte à laquelle ils sont sensibles, vont radier d'autres partons.
Par conservation de l'énergie, l'énergie portée par une de ces particules diminue au fur et à mesure de ces radiations et $\alpha_s$ augmente.
Arrivé à des énergies de l'ordre de la centaine de \SI{}{\MeV}, il n'est plus possible de réaliser des calculs perturbatifs.
Le phénomène de confinement de couleur réapparaît alors et les partons sont ainsi confinés au sein de hadrons, dont la charge de couleur est nulle.
Ce processus est appelé \og hadronisation \fg.
Les partons produits lors des collisions de haute énergie se manifestent ainsi sous la forme d'un flux collimé de particules, un \og jet \fg.
\par Le LHC fait se collisionner des protons, aussi la collision a réellement lieu entre les constituants des protons, \ie\ les quarks et les gluons.
Lors des collisions de protons, il y a donc une forte probabilité d'interaction par interaction forte.
De plus, les particules lourdes, instables, peuvent émettre des quarks en se désintégrant.
De nombreux partons sont ainsi émis lors des collisions de protons au LHC.
Il en résulte une omniprésence des jets dans les analyses réalisées dans l'expérience CMS.
Leur caractérisation est donc un point essentiel pour la collaboration.
Or, les jets sont des objets physiques composés de nombreuses particules.
La calibration en énergie des jets est ainsi nécessaire afin d'obtenir une estimation de leur énergie \og vraie \fg{} la plus fidèle possible.
\par Le processus de formation des jets est décrit d'un point de vue théorique dans la section~\ref{chapter-JERC-section-jets}.
Puis, la méthode de reconstruction des jets dans le cadre de l'expérience CMS est présentée dans la section~\ref{chapter-JERC-section-jets_reco}.
Ensuite, le principe de la calibration en énergie des jets dans l'expérience CMS est abordé dans la section~\ref{chapter-JERC-section-CMS}.
Une des étapes de la calibration en énergie des jets est discutée plus en détails.
La section~\ref{chapter-JERC-section-pheno-GJets} présente la phénoménologie des événements utilisés et la section~\ref{chapter-JERC-section-JES} aborde la calibration en énergie des jets à l'aide de ces événements.
Enfin, la correction de la résolution en énergie des jets à l'aide de ces mêmes événements est présentée dans la section~\ref{chapter-JERC-section-JER}.