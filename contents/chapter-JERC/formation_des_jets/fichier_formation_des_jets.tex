\section{Formation des jets}\label{chapter-JERC-section-jets}
Lorsqu'une particule colorée, \ie\ un quark ou un gluon, est issue de la collision, cette particule possède une haute énergie et $g_s \ll 1$. Cette particule colorée radie, par interaction forte, d'autres particules colorées. Par conservation, l'énergie portée portée par chaque particule colorée ainsi obtenue diminue et par conséquence, $g_s$ augmente.
\par Tant que l'échelle d'énergie est suffisamment grande pour que $g_s \ll 1$, ce qui correspond à des énergies supérieures à la centaine de \SI{}{\MeV}, il est possible de réaliser des calculs perturbatifs. La radiation de particules colorées créé ce que s'appelle la \og gerbe partonique \fg, ce qui est le sujet de la prochaine section.
\par Au fur et à mesure des radiations, l'échelle en énergie diminue et en deçà d'une centaine de \SI{}{\MeV}, il n'est plus possible de réaliser des calculs perturbatifs car $g_s$ augmente. Des modèles paramétriques sont alors utilisés pour caractériser le phénomène de \og hadronisation \fg, sujet de la section suivante.

\subsection{Gerbe partonique}\label{chapter-JERC-section-jets-subsec-gerbe-partonique}
\fullcite{Unorthodox_Introduction_QCD}

\subsection{Hadronisation}\label{chapter-JERC-section-jets-subsec-hadronisation}

cordes de Lund\footcite{Andersson_parton_fragmentation}

agglomération hadronique\footcite{Winter_2004}
