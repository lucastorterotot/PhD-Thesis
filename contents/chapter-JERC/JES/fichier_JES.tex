\section{Correction résiduelle absolue en \pT\ avec les événements \Gjets}\label{chapter-JERC-section-JES}
L'obtention de la correction résiduelle absolue en \pT\ des jets
ainsi que la correction de leur résolution en énergie
avec les événements \Gjets\ a été un des mes travaux de thèse.
J'ai ainsi traité les données de l'année 2018 et les données \og 2017-UL \fg{} où UL signifie \emph{Ultra-Legacy}.
Il s'agit des données de l'année 2017 qui sont réinterprétées une fois les défauts du détecteurs mieux cernés, par exemple.
\subsection{Jeux de données et sélection des événements}\label{chapter-JERC-section-JES-subsec-evt_select}
%18 data
%     Run2018A-17Sep2018-v2
%     /EGamma/Run2018A-17Sep2018-v2/MINIAOD
%     13654.355526985
%     
%     Run2018B-17Sep2018-v1
%     /EGamma/Run2018B-17Sep2018-v1/MINIAOD
%     7057.825158567
%     
%     Run2018C-17Sep2018-v1
%     /EGamma/Run2018C-17Sep2018-v1/MINIAOD
%     6894.770971269
%     
%     Run2018D-PromptReco-v2
%     /EGamma/Run2018D-PromptReco-v2/MINIAOD
%     31066.589629726
%
%18 MC
%     GJet_Pt-15To6000_RunIIAutumn18MiniAOD-102X
%     /GJet_Pt-15To6000_TuneCP5-Flat_13TeV_pythia8/RunIIAutumn18MiniAOD-102X_upgrade2018_realistic_v15-v1/MINIAODSIM
%     283000.0
%
%17UL data
%     Run2017B-09Aug2019_UL2017-v1
%     /SinglePhoton/Run2017B-09Aug2019_UL2017-v1/MINIAOD
%     4793.961426839
%     
%     Run2017C-09Aug2019_UL2017-v1
%     /SinglePhoton/Run2017C-09Aug2019_UL2017-v1/MINIAOD
%     9631.214820913
%     
%     Run2017D-09Aug2019_UL2017-v1
%     /SinglePhoton/Run2017D-09Aug2019_UL2017-v1/MINIAOD
%     4247.682053046
%     
%     Run2017E-09Aug2019_UL2017-v1
%     /SinglePhoton/Run2017E-09Aug2019_UL2017-v1/MINIAOD
%     9313.642401775
%     
%     Run2017F-09Aug2019_UL2017-v1
%     /SinglePhoton/Run2017F-09Aug2019_UL2017-v1/MINIAOD
%     13539.378417564
%
%
%17UL MC
%     GJets_HT-40To100_RunIISummer19MiniAOD-106X
%     /GJets_HT-40To100_TuneCP5_13TeV-madgraphMLM-pythia8/RunIISummer19UL17MiniAOD-106X_mc2017_realistic_v6-v1/MINIAODSIM
%     18700.0
%     
%     GJets_HT-100To200_RunIISummer19MiniAOD-106X
%     /GJets_HT-100To200_TuneCP5_13TeV-madgraphMLM-pythia8/RunIISummer19UL17MiniAOD-4cores5k_106X_mc2017_realistic_v6-v1/MINIAODSIM
%     8640.0
%     
%     GJets_HT-200To400_RunIISummer19MiniAOD-106X
%     /GJets_HT-200To400_TuneCP5_13TeV-madgraphMLM-pythia8/RunIISummer19UL17MiniAOD-106X_mc2017_realistic_v6-v1/MINIAODSIM
%     2185.0
%     
%     GJets_HT-400To600_RunIISummer19MiniAOD-106X
%     /GJets_HT-400To600_TuneCP5_13TeV-madgraphMLM-pythia8/RunIISummer19UL17MiniAOD-106X_mc2017_realistic_v6-v1/MINIAODSIM
%     259.9
%     
%     GJets_HT-600ToInf_RunIISummer19MiniAOD-106X
%     /GJets_HT-600ToInf_TuneCP5_13TeV-madgraphMLM-pythia8/RunIISummer19UL17MiniAOD-106X_mc2017_realistic_v6-v1/MINIAODSIM
%     85.31 \pm 0.2564
Les jeux de données réelles utilisés pour 2018 et 2017-UL sont basés sur la présence d'un photon dans l'état final.
Plusieurs périodes sont considérées pour chacune de ces années, celles des collisions \proton\proton, dont la liste et les luminosités correspondantes sont présentés dans les tableaux~\ref{subtab-Runs_and_lumis_2018_GJet} et~\ref{subtab-Runs_and_lumis_2017UL_GJet}.
\begin{table}[h]
\centering
\subcaptionbox{Année 2018.\label{subtab-Runs_and_lumis_2018_GJet}}[.45\textwidth]
{\begin{tabular}{cc}
\toprule
Run & Luminosité (\SI{}{\femto\barn^{-1}})\\
\midrule
A & \num{13.65} \\
B & \num{7.06} \\
C & \num{6.89} \\
D & \num{31.07} \\
\midrule
Total & \num{58.67} \\
\bottomrule
\end{tabular}}
\subcaptionbox{Année 2017-UL.\label{subtab-Runs_and_lumis_2017UL_GJet}}[.45\textwidth]
{\begin{tabular}{cc}
\toprule
Run & Luminosité (\SI{}{\femto\barn^{-1}})\\
\midrule
B & \num{4.79} \\
C & \num{9.63} \\
D & \num{4.25} \\
E & \num{9.31} \\
F & \num{13.54} \\
\midrule
Total & \num{41.52} \\
\bottomrule
\end{tabular}}
\caption{Liste des périodes de prise de données considérées et luminosités correspondantes.}
\label{tab-Runs_and_lumis_2018_and_2017UL_GJet}
\end{table}
\par Les simulations utilisées contiennent des événements \Gjets\ de type $\quark\gluon\to\quark\photon$, comme ceux des figures~\ref{subfig-fgraph-gq_qGamma_S} et~\ref{subfig-fgraph-gq_qGamma_T}, et $\quark\quark\to\gluon\photon$, comme celui de la figure~\ref{subfig-fgraph-qq_gGamma}.
Pour l'année 2018, les événements sont générés en un seul jeu de données
à l'aide de \PYTHIA~8~\cite{pythia8.2}
avec les réglages CP5-Flat~\cite{tunes_2019}
et une énergie dans le centre de masse de \SI{13}{\TeV}.
Dans l'état final, un photon d'impulsion transverse comprise entre \num{15} et \SI{6000}{\GeV} est généré.
Pour l'année 2017-UL, les événements sont générés conjointement
à l'aide de \PYTHIA~8~\cite{pythia8.2}
avec les réglages CP5~\cite{tunes_2019}
et
\MADGRAPHc~\cite{madgraph5}
et une énergie dans le centre de masse de \SI{13}{\TeV}.
Dans l'état final, la somme scalaire des impulsions transverses des jets, notée HT, appartient à un intervalle, définissant ainsi cinq jeux de données.
Les sections efficaces des jeux de données d'événements simulés ainsi obtenus sont présentées dans le tableau~\ref{tab-MC_xsec_2018_and_2017UL_GJet}.
\begin{table}[h]
\centering
\begin{tabular}{clc}
\toprule
Année & Caractéristique & Section efficace (\SI{}{\pico\barn})\\
\midrule
2018 & $\pT^{\photon}\in [\num{15}, \num{6000}]\usp\SI{}{\GeV}$ & \num{283000.0}\\
2017-UL & $\text{HT} \in [\num{40}, \num{100}]\usp\SI{}{\GeV}$ & \num{18700.0} \\
2017-UL & $\text{HT} \in [\num{100}, \num{200}]\usp\SI{}{\GeV}$ & \num{8640.0} \\
2017-UL & $\text{HT} \in [\num{200}, \num{400}]\usp\SI{}{\GeV}$ & \num{2185.0} \\
2017-UL & $\text{HT} \in [\num{400}, \num{600}]\usp\SI{}{\GeV}$ & \num{259.9} \\
2017-UL & $\text{HT} > \SI{600}{\GeV}$ & \num{85.31} \\
\bottomrule
\end{tabular}
\caption{Sections efficaces des différents jeux de données \Gjets\ simulés.}
\label{tab-MC_xsec_2018_and_2017UL_GJet}
\end{table}
\par Une sélection plus fine des événements à considérer est réalisée lors de l'analyse elle-même.
En effet, les événements souhaités sont ceux contenant un photon avec un ou plusieurs jets;
un des bruits de fond principal provient d'événements multijets où un des jets est identifié à tort comme un photon.
Cette situation peut arriver lorsque ce jet contient de nombreux pions neutres, les \pionnull.
Les \pionnull\ se propagent sur des distances moyennes de \SI{26}{\nano\meter} puis se désintègrent dans \SI{99}{\%} des cas en deux photons~\cite{PDG_booklet_2018}.
Ces particules ne laissent donc aucune trace dans le trajectographe et un dépôt d'énergie dans le ECAL, tout comme un vrai photon issu de l'interaction initiale.
Un tel jet comporte ainsi une signature similaire à un photon d'un événement \Gjet\ autour duquel une activité hadronique existe.
Les topologies de ces deux types d'événements, semblables, sont représentées sur la figure~\ref{fig-Gamma_plus_jet_events_real_faked}.
\begin{figure}[h]
\centering
\subcaptionbox{Topologie d'un événement dijet, dont un jet contient de nombreux \pionnull.\label{subfig-Gamma_plus_jet_basic_event_dijet_faked}}[.45\textwidth]
{\includegraphics[width=.45\textwidth,height=.25\textheight,keepaspectratio]{\PhDthesisdir/tex/Event_displays/JERC/Dijets_pi0s.tex}}
\hfill
\subcaptionbox{Topologie d'un vrai événement \Gjet\ avec un peu d'activité hadronique autour du photon.\label{subfig-Gamma_plus_jet_basic_event_real}}[.45\textwidth]{\includegraphics[width=.45\textwidth,height=.25\textheight,keepaspectratio]{\PhDthesisdir/tex/Event_displays/JERC/Gamma_plus_jet_hadronic_noise.tex}}
\caption{Topologies d'événements \Gjet\ et dijets.}
\label{fig-Gamma_plus_jet_events_real_faked}
\end{figure}
\par Une sélection des photons est appliquée afin de réduire le bruit de fond multijets.
Pour cela, la collaboration CMS propose des critères d'identification des photons (lâche, moyen et strict) s'appuyant sur diverses propriétés du \og candidat \fg{} photon:
\begin{itemize}
\item $H/E$ est le rapport de l'énergie hadronique sur l'énergie électromagnétique associées à l'agglomérat d'énergie du photon.
Un photon est sensé déposer son énergie dans le ECAL et ne laisser aucun signal dans le HCAL.
Une faible valeur de $H/E$ est donc compatible avec un photon.
\item $\sigma_{i\eta i\eta}$ est l'étalement en $\eta$ du dépôt d'énergie dans le ECAL.
Cette observable est reliée à la forme de la gerbe électromagnétique, moins étalée pour un photon que pour un électron.
Une faible valeur de $\sigma_{i\eta i\eta}$ est donc compatible avec un photon.
\item $I_{CH}$ est l'isolation vis-à-vis des hadrons chargés.
Elle se définit comme le ratio entre la somme des impulsions transverses de tous les hadrons chargés situés à une distance $\Delta R$ du candidat photon dans le plan $(\eta,\phi)$ inférieure à \num{0.3} et l'impulsion transverse du candidat photon lui-même.
\item $I_{NH}$ est l'isolation vis-à-vis des hadrons neutres, analogue à $I_{CH}$.
\item $I_{\photon}$ est l'isolation vis-à-vis des photons autres que le candidat lui-même, analogue à $I_{CH}$.
\end{itemize}

\todo{R9}

\par Les variables d'isolation sont corrigées afin de prendre en compte l'empilement, on considère alors $I^\text{corr}$ au lieu de $I$, telle que
\begin{equation}
I^\text{corr} = \max\left( I - \rho \times \mathcal{E_A} , 0\right)
\end{equation}
où $\mathcal{E_A}$ est l'aire effective, \ie\ la fraction de l'espace $(\eta,\phi)$ correspondant à la zone d'isolation à corriger pour l'empilement. Les valeurs des aires effectives utilisées sont présentées dans le tableau~\ref{tab-CutBasedPhotonIdentificationRun2-effective_areas}.
Les coupures définissant les différents critères d'identification des photons ainsi que leurs efficacités d'identification et de réjection sont résumées dans le tableau~\ref{tab-CutBasedPhotonIdentificationRun2}.
\begin{table}[h]
\centering
\begin{tabular}{cccc}
\toprule
Région & Hadrons chargés & Hadrons neutres & Photons \\
\midrule
$\abs{\eta} \leq \num{1.0}$ & \num{0.0112} & \num{0.0668} & \num{0.1113} \\
$\num{1.0} < \abs{\eta} \leq \num{1.479}$ & \num{0.0108} & \num{0.1054} & \num{0.0953} \\
$\num{1.479} < \abs{\eta} \leq \num{2.0}$ & \num{0.0106} & \num{0.0786} & \num{0.0619} \\
$\num{2.0} < \abs{\eta} \leq \num{2.2}$ & \num{0.01002} & \num{0.0233} & \num{0.0837} \\
$\num{2.2} < \abs{\eta} \leq \num{2.3}$ & \num{0.0098} & \num{0.0078} & \num{0.1070} \\
$\num{2.3} < \abs{\eta} \leq \num{2.4}$ & \num{0.0089} & \num{0.0028} & \num{0.1212} \\
$\abs{\eta} > \num{2.4}$ & \num{0.0087} & \num{0.0137} & \num{0.1466} \\
\bottomrule
\end{tabular}
\caption[Aires effectives de correction de l'isolation du photon.]{Valeurs de l'aire effective $\mathcal{E_A}$ utilisée pour corriger la contribution de l'empilement aux isolations des photons vis-à-vis des autres particules.}
\label{tab-CutBasedPhotonIdentificationRun2-effective_areas}
\end{table}
\begin{table}[h]
\centering\small
\begin{tabularx}{\textwidth}{Xcccccc}
\toprule
Critère & \multicolumn{2}{c}{Lâche} & \multicolumn{2}{c}{Moyen} & \multicolumn{2}{c}{Strict} \\
\cmidrule(lr){2-3}\cmidrule(lr){4-5}\cmidrule(lr){6-7}
Région & Barillet & Bouchon & Barillet & Bouchon & Barillet & Bouchon\\
\midrule
Efficacité & \SI{90.08}{\%} & \SI{90.65}{\%} & \SI{80.29}{\%} & \SI{80.08}{\%} & \SI{70.24}{\%} & \SI{70.13}{\%} \\
Réjection & \SI{86.25}{\%} & \SI{76.72}{\%} & \SI{89.36}{\%} & \SI{81.85}{\%} & \SI{90.97}{\%} & \SI{84.55}{\%} \\
\midrule
$H/E$ & \num{0.04596} & \num{0.0590} & \num{0.02197} & \num{0.0326} & \num{0.02148} & \num{0.0321} \\
$\sigma_{i\eta i\eta}$ & \num{0.0106} & \num{0.0272} & \num{0.01015} & \num{0.0272} & \num{0.00996} & \num{0.0271} \\
$I_{CH}^\text{corr}$ & \num{1.694} & \num{2.089} & \num{1.141} & \num{1.051} & \num{0.65} & \num{0.517} \\
\multirow{3}{*}{$I_{NH}^\text{corr} \!\!\!\hphantom{I_{\photon}^\text{corr}} \left\lbrace \begin{matrix} \vphantom{0} \\ \vphantom{0} \\ \vphantom{0} \end{matrix} \right. $} & $\num{24.032}$& $\num{19.722}$& $\num{1.189}$& $\num{2.718}$& $\num{0.317}$& $\num{2.716}$\\
& $+ \num{0.01512} \, \pT$& $+ \num{0.011} \, \pT$& $+ \num{0.01512} \, \pT$& $+ \num{0.0117} \, \pT$& $+ \num{0.01512} \, \pT$& $+ \num{0.0117} \, \pT$ \\
& $+ \num{2.259} \pT^{\!2} \! / 10^5$ & $+ \num{2.3} \pT^{\!2} \! / 10^5$ & $+ \num{2.259} \pT^{\!2} \! / 10^5$ & $+ \num{2.3} \pT^{\!2} \! / 10^5$ & $+ \num{2.259} \pT^{\!2} \! / 10^5$ & $+ \num{2.3} \pT^{\!2} \! / 10^5$ \\
\multirow{2}{*}{$I_{\photon}^\text{corr} \!\!\!\hphantom{I_{NH}^\text{corr}} \left\lbrace \begin{matrix} \vphantom{0} \\ \vphantom{0} \end{matrix} \right. $} & $\num{2.876}$ & $\num{4.162}$ & $\num{2.08}$ & $\num{3.867}$ & $\num{2.044}$ & $\num{3.032}$ \\
& $+ \num{0.004017} \, \pT$ & $+ \num{0.0037} \, \pT$ & $+ \num{0.004017} \, \pT$ & $+ \num{0.0037} \, \pT$ & $+ \num{0.004017} \, \pT$ & $+ \num{0.0037} \, \pT$ \\
\bottomrule
\end{tabularx}
\caption[Coupures utilisées pour l'identification des photons.]{Valeurs maximales des observables considérées pour l'identification des photons selon le critère utilisé et la région du détecteur dans laquelle se trouve le candidat photon (barillet pour $\abs{eta} < \num{1.479}$, bouchon sinon).}
\label{tab-CutBasedPhotonIdentificationRun2}
\end{table}
\par Le critère d'identification des photons retenu dans l'analyse est le critère strict.
Seuls les photons situés dans le barillet sont utilisés car ils présentent la meilleure résolution.
La figure~\ref{fig-chapter-JERC-section-pheno-GJets-photon_resolution}, page~\pageref{fig-chapter-JERC-section-pheno-GJets-photon_resolution}, montre en effet que ces photons possède une résolution relative en énergie de l'ordre de \SI{1}{\%}, contre environ \SI{2.5}{\%} pour les photons des bouchons.
Une coupure sur leur pseudo-rapidité est donc appliquée, telle que $\abs{\eta} < \num{1.305}$.
\par \todo{Barrel photon study if done in service task, else small paragraph on this idea}
\par Les événements présentant un unique photon ainsi sélectionné sont retenus.
Avec ce photon doit être présent au moins un jet reconstruit à l'aide de l'algorithme anti-\kT~\cite{Cacciari_antikT} avec un paramètre $R=\num{0.4}$ et respectant les critères définis dans le tableau~\ref{tab-chapter-JERC-section-jets_reco-subsec-jetID-2018} pour les données de 2018 et ceux du tableau~\ref{tab-chapter-JERC-section-jets_reco-subsec-jetID-2017UL} pour les données de 2017-UL.
Ces critères permettent de rejeter les jets issus du bruit de fond avec une efficacité de \SI{99}{\%}.
\par Les jets ainsi sélectionnés sont calibrés en énergie en suivant la procédure décrite dans la section~\ref{chapter-JERC-section-CMS} jusqu'à la correction résiduelle relative en $\eta$ incluse. Ils sont alors triés par impulsion transverse décroissante.
Pour s'assurer d'une bonne balance dans le plan transverse entre le photon et le premier jet, \ie\ celui d'impulsion transverse la plus grande, seuls les événements proposant un écart angulaire entre le photon et le jet supérieur à \SI{2.8}{\rad} sont considérés dans la suite.
Le photon et le jet sont donc dos à dos dans le plan transverse, ce qui correspond à la situation illustrée figures~\ref{subfig-Gamma_plus_jet_basic_event}, \ref{subfig-Gamma_plus_two_jets} et~\ref{subfig-Gamma_plus_jet_basic_event_real}.
\par Si un second jet d'impulsion transverse supérieure à \SI{10}{\GeV} est présent, l'événement est rejeté si $\alpha>\num{0.3}$ où $\alpha$ est défini dans l'équation~\eqref{eq-chapter-JERC-definition_alpha}, page~\pageref{eq-chapter-JERC-definition_alpha}.
L'événement est également rejeté si un lepton (électron ou muon) isolé, en pratique hors des jets, est présent.
\par \todo{graph pureté?}
\par Dans le cas des données réelles, l'événement est sauvegardé si un chemin de déclenchement est activé\footnote{\todo{set a ref to the LHC chapter corresponding section when written}.}.
Seuls les événements dont le photon retenu correspond au photon du chemin de déclenchement sont retenus.
Il existe plusieurs chemins de déclenchement en fonction de l'impulsion du photon.
Certains de ces chemins proposent une quantité trop importante d'événements à sauvegarder et pourraient saturer la chaîne d'acquisition.
Pour éviter cette saturation, seule une fraction des événements passant un tel chemin de déclenchement sont effectivement sauvegardés.
Cette fraction est nommée \emph{prescale}.
Chaque chemin de déclenchement possède ainsi son \emph{prescale}.
Afin de ne pas introduire de biais dû à ces \emph{prescales} dans l'analyse, un intervalle d'impulsion transverse du photon retenu est défini pour chaque chemin de déclenchement utilisé.
Il est ainsi requis que le photon retenu soit le photon du chemin de déclenchement correspondant à l'intervalle dans lequel se trouve son impulsion transverse.
Les différents chemins de déclenchement, leurs \emph{prescales} et intervalles d'impulsion transverse sont présentés dans le tableau~\ref{tab-HLT_pT_precales_18_and_17UL}.
\begin{table}
\centering
\begin{tabular}{lccc}
\toprule
Chemin de déclenchement & $\pT^{\photon}$ (\SI{}{\GeV}) & \emph{Prescale} 2018 & \emph{Prescale} 2017-UL\\
\midrule
%\inlinepython{HLT_Photon33} & $[\num{40}, \num{60}[$ & \num{4.0115375867e-5} & \num{3.434860938821936e-4} \\
%\inlinepython{HLT_Photon50_R9Id90_HE10_IsoM} & $[\num{60}, \num{85}[$ & \num{3.9473720141e-3} & \num{7.40465688874224e-3} \\
%\inlinepython{HLT_Photon75_R9Id90_HE10_IsoM} & $[\num{85}, \num{105}[$ & \num{0.0156656382257} & \num{0.03195516364518142} \\
%\inlinepython{HLT_Photon90_R9Id90_HE10_IsoM} & $[\num{105}, \num{130}[$ & \num{0.0312899931745} & \num{0.0636323095006467} \\
%\inlinepython{HLT_Photon120_R9Id90_HE10_IsoM} & $[\num{130}, \num{175}[$ & \num{0.125036122867} & \num{0.18787162142132302} \\
%\inlinepython{HLT_Photon165_R9Id90_HE10_IsoM} & $[\num{175}, \num{230}[$ & \num{0.250030962458} & \num{0.6823580953102895} \\
%\inlinepython{HLT_Photon200} & $[\num{230}, +\infty$ & \num{1} & \num{1} \\
\inlinepython{HLT_Photon33} & $[\num{40}, \num{60}[$ & \num{4.01154e-5} & \num{3.43486e-4} \\
\inlinepython{HLT_Photon50_R9Id90_HE10_IsoM} & $[\num{60}, \num{85}[$ & \num{3.94737e-3} & \num{7.40466e-3} \\
\inlinepython{HLT_Photon75_R9Id90_HE10_IsoM} & $[\num{85}, \num{105}[$ & \num{0.0156656} & \num{0.0319552} \\
\inlinepython{HLT_Photon90_R9Id90_HE10_IsoM} & $[\num{105}, \num{130}[$ & \num{0.0312900} & \num{0.0636323} \\
\inlinepython{HLT_Photon120_R9Id90_HE10_IsoM} & $[\num{130}, \num{175}[$ & \num{0.125036} & \num{0.187872} \\
\inlinepython{HLT_Photon165_R9Id90_HE10_IsoM} & $[\num{175}, \num{230}[$ & \num{0.250031} & \num{0.682358} \\
\inlinepython{HLT_Photon200} & $[\num{230}, +\infty [$ & \num{1} & \num{1} \\
\bottomrule
\end{tabular}
\caption[Chemins de déclenchement.]{Chemins de déclenchement, intervalles d'impulsion transverse du photon et \emph{prescales} utilisés.}
\label{tab-HLT_pT_precales_18_and_17UL}
\end{table}
\par Par exemple, un photon d'impulsion transverse \SI{95}{\GeV} doit avoir déclenché le chemin nommé \inlinepython{HLT_Photon75_R9Id90_HE10_IsoM}.
Ce chemin de déclenchement requière un photon d'impulsion transverse minimale \SI{75}{\GeV}.
Ce même photon déclenche donc potentiellement le chemin nommé \inlinepython{HLT_Photon90_R9Id90_HE10_IsoM}.
Utiliser un écart minimal entre l'impulsion du photon et l'impulsion minimale requise au déclenchement du chemin permet de se placer au plateau d'efficacité maximale du chemin de déclenchement.
Des biais dus à la calibration du photon sont également évités grâce à cette méthode.

\subsection{Analyse}\label{chapter-JERC-section-JES-subsec-analyse}

\subsection{Incertitudes}\label{chapter-JERC-section-JES-subsec-uncertainties}

\subsection{Résultats}\label{chapter-JERC-section-JES-subsec-results}
