\chapter{Particules, interactions et phénoménologie}\label{chapter-MS-MSSM}

Ce chapitre présente le contexte dans lequel s'inscrit cette thèse.
Le modèle standard est le cadre théorique en place en physique des particules. Il permet de décrire les objets fondamentaux qui composent l'Univers, les particules, ainsi que leurs interactions.
\par Les particules du modèle standard sont présentées dans la section~\ref{chapter-MS-MSSM-section-SM_ptcs}.
Le formalisme mathématique permettant de décrire leur comportement, faisant apparaître les forces fondamentales, est introduit dans la section~\ref{chapter-MS-MSSM-section-formalisme}.
Le modèle standard ainsi construit propose une description de l'Univers à la fois précise et robuste.
\par Le boson de Higgs, dernière particule découverte à ce jour, a ainsi été postulé près de cinquante ans avant d'être observé. De nombreux succès, dont une présentation non exhaustive est proposée dans la section~\ref{chapter-MS-MSSM-section-succes_limites-subsec-succes}, couronnent ainsi le modèle standard.
Cependant, malgré plusieurs décennies de prédictions correctement vérifiées, certaines observations montrent que le modèle standard ne saurait prétendre au titre de \og théorie du tout \fg{}.
\par Ces limitations au modèle standard, dont certaines sont présentées dans la section~\ref{chapter-MS-MSSM-section-succes_limites-subsec-limites}, mènent à de nouveaux modèles dits \og au-delà du modèle standard \fg{}, dont il est question dans la section~\ref{chapter-MS-MSSM-section-BSM}. Parmi eux se trouvent des modèles dit \og à deux doublets de Higgs \fg, \ie\ avec un secteur de Higgs plus complexe, comme la supersymétrie.
\par Il existe plusieurs degrés de complexité dans ces nouveaux modèles, aussi seule l'extension supersymétrique minimale du modèle standard, ou MSSM, sera considérée pour l'analyse menée dans cette thèse.
Dans le cadre du MSSM, de nouvelles particules existent et la phénoménologie de ces particules, présentée dans la section~\ref{chapter-MS-MSSM-section-pheno_Higgs_MSSM}, motive le choix du type d'événements d'intérêt pour la recherche de cette nouvelle physique.


\section{Les particules du modèle standard}\label{chapter-MS-MSSM-section-SM_ptcs}
Une particule est considérée comme élémentaire si elle ne possède pas de sous-structure observée à ce jour. %ptc élémentaire = ? 10e-18 m
Celles du modèle standard sont présentées en figure~\ref{fig-MS-table}.
Leur masse, comme les autres grandeurs physiques de cette thèse, sont exprimées en unités naturelles,
différentes des unités du système international comme exposé dans l'annexe~\refApUNSI.
Ces particules peuvent être classées selon leurs propriétés.
La première d'entre elles, le \og spin \fg, est une observable quantique intrinsèque aux particules.
%Les particules de spin demi-entier sont les fermions, celles de spin entier les bosons.
\begin{figure}[h]
\centering
\includegraphics[width=\textwidth]{\PhDthesisdir/plots_and_images/Particles_tables/SM2018_FR.pdf}
\caption[Les particules élémentaires du modèle standard.]{Les particules élémentaires du modèle standard. Les quarks (en vert) et les leptons (en bleu) sont des fermions. Les bosons vecteurs (en orange) sont les médiateurs des forces fondamentales. Le boson de Higgs est quant à lui un boson scalaire (en jaune).}
\label{fig-MS-table}
\end{figure}

\subsection{Les fermions}\label{chapter-MS-MSSM-section-SM_ptcs-subsec-fermions}
Les fermions sont les particules élémentaires de spin demi-entier et suivent donc la statistique de Fermi-Dirac.
Ainsi, deux fermions ne peuvent pas occuper le même état quantique,
\ie\ avoir les mêmes nombres quantiques,
comme exposé par le principe d'exclusion de Pauli.
Le modèle standard comprend douze fermions constituant la matière, accompagnés de douze antifermions correspondants pour l'antimatière.
\par Les fermions peuvent se diviser d'une part en deux catégories, les quarks et les leptons, et d'autre part en trois catégories correspondant à trois \og générations \fg, comme illustré sur la figure~\ref{fig-MS-table}. La première génération (quarks~\quarku\ et~\quarkd, électron \electron\ et neutrino électronique \nuele) correspond aux particules les plus communes; les deuxièmes et troisièmes générations contiennent des particules analogues, plus massives et instables.
% générations supplémentaires et boson Z ?
% spin demi entier (stat Fermi-Dirac -> exclusion Pauli (expliquer) et relation anticommutation). Constituants de la matière, il y en a 12.

\subsubsection{Les quarks}\label{chapter-MS-MSSM-section-SM_ptcs-subsec-fermions-subsubsec-quarks}
Les quarks sont les fermions possédant une charge de couleur.
Il existe deux quarks par génération, un quark de type \emph{up} et un quark de type \emph{down}, formant un doublet d'isospin faible comme exposé en section~\ref{chapter-MS-MSSM-section-formalisme-subsec-EW}.
Il y a donc six quarks au total. Les quarks de type \emph{up} (\quarku, \quarkc\ et~\quarkt) portent une charge électrique $+\frac{2}{3}e$ avec $e$ la charge électrique élémentaire, les quarks de type \emph{down} (\quarkd, \quarks\ et~\quarkb) une charge $-\frac{1}{3}e$. Les antiquarks possèdent une charge électrique opposée ($-\frac{2}{3}e$ et $+\frac{1}{3}e$). Ces particules sont donc sensibles à l'interaction électromagnétique.
\par À l'instar de la charge électrique pour l'interaction électromagnétique, la \og couleur \fg rend les quarks sensibles à l'interaction forte. La charge de couleur peut prendre trois valeurs orthogonales, nommées par convention rouge, verte et bleue. Les particules portant une charge de couleur ne sont pas stables seules et se regroupent pour former des particules composites de charge de couleur nulle, ou de couleur \og blanche \fg. C'est ce que l'on appelle le phénomène de \og confinement de couleur \fg, décrit dans la section~\ref{chapter-MS-MSSM-section-formalisme-subsec-QCD-subsubsec-confinement}.
\par Les particules composées de quarks sont les hadrons. Ces particules sont de couleur blanche, ce qui peut être obtenu de deux manières:
\begin{itemize}
\item par association d'un quark rouge, un vert et un bleu; il s'agit d'un \og baryon \fg. Le proton (\quarku\quarku\quarkd) et le neutron (\quarku\quarkd\quarkd) sont deux exemples de baryons.
\item par association d'un quark et d'un antiquark; il s'agit d'un \og méson \fg. En effet, un antiquark porte une \og anticouleur \fg. Par exemple, un quark up (\quarku) rouge et un antiquark down (\antiquarkd) \og antirouge \fg{} forment un pion chargé \pionplus. Une combinaison \quarku\antiquarks\ est un kaon \Kaonplus.
\end{itemize}
\par Enfin, comme tous les fermions, les quarks sont également sensibles à l'interaction faible. Les quarks sont ainsi les seules particules sensibles à toutes les interactions fondamentales décrites par le modèle standard.
% fermions avec couleur = charge de la force forte, il y a en trois (orthogonales) + anticouleur. RGB. Weak isopsin doublet, trois doublets, 6 quarks. noms, charges.
%confinement, see other section. Particules de couleur \og blanche \fg{}. Hadrons. Mesons, Baryons. Protons et neutrons dans la matière.

\subsubsection{Les leptons}\label{chapter-MS-MSSM-section-SM_ptcs-subsec-fermions-subsubsec-leptons}
Les leptons sont les fermions ne possédant pas de charge de couleur. Ils sont donc insensibles à l'interaction forte. En revanche, ils sont tous sensibles à l'interaction faible.
Sur le même principe que pour les quarks, il y a un doublet d'isospin faible de deux leptons par génération, soit six leptons au total.
Les leptons d'isospin faible haut sont l'électron (\electron), le muon (\muon) et le tau (\leptau), ils portent une charge électrique $-e$ ($+e$ pour les antiparticules correspondantes). Les leptons d'isospin faible bas sont les neutrinos. Ces derniers ne portent pas de charge électrique et interagissent donc uniquement par interaction faible, ce qui en fait des particules difficiles à détecter.
% pas de couleur, insensibles interaction forte. trois générations, weak isospin doublet. electron-like + neutrino $\times3=6$ leptons. charges électriques. Neutrinos, faible masse. Uniquement force faible, \emph{elusive particles}. Difficile à détecter.


\subsection{Les bosons}\label{chapter-MS-MSSM-section-SM_ptcs-subsec-bosons}
Les bosons sont les particules élémentaires de spin entier.
Ils suivent la statistique de Bose-Einstein qui n'interdit pas la présence de plusieurs bosons dans le même état quantique, contrairement à la statistique de Fermi-Dirac.
\par
Les bosons de spin 1 sont les bosons de jauge, ou bosons vecteurs, et sont les médiateurs des interaction fondamentales.
Ainsi, le photon (\photon) est le boson vecteur de l'interaction électromagnétique. Il est de masse nulle et est électriquement neutre.
Les bosons \Wbosonplus, \Wbosonminus\ et \Zboson\ sont ceux de l'interaction faible. Le boson \Zboson\ est électriquement neutre et de masse $m_\Zboson=\SI{91.19}{\GeV}$, les bosons \Wboson\ portent une charge électrique de $\pm e$, ont une masse de $m_\Wboson=\SI{80.38}{\GeV}$ et n'interagissent qu'avec les particules de chiralité gauche et les antiparticules de chiralité droite.
La chiralité est définie dans la section~\ref{chapter-MS-MSSM-section-formalisme-subsec-EW}.
Enfin, huit gluons (\gluon) sont les médiateurs de l'interaction forte. Ils n'ont ni masse ni charge électrique, mais portent une charge de couleur et une charge d'anticouleur. Un gluon peut donc être chargé \og rouge et antibleu \fg{}. Si un tel gluon interagit avec un quark bleu, par conservation, ce quark devient rouge après interaction.
\par
Les bosons de spin 0 sont dits scalaires.
Le seul scalaire élémentaire observé à ce jour est le boson de Higgs, le modèle standard n'en prédit pas d'autre.
Comme exposé en section~\ref{chapter-MS-MSSM-section-formalisme-subsec-EW},
les forces électromagnétique et faible peuvent être unifiées en une seule force électrofaible.
La symétrie associée à la force électrofaible est spontanément brisée
par le mécanisme de Higgs,
présenté en section~\ref{chapter-MS-MSSM-section-formalisme-subsec-Higgs_mechanism}.
Les forces électromagnétique et faible
sont alors retrouvées,
les particules acquièrent une masse
et le boson de Higgs émerge de ce mécanisme.

%spin entier, 1 (bosons de jauge, bosons vecteurs, vecteurs de force) ou 0 (Higgs). Bosons de jauge = vecteurs des forces fondamentales. Lesquels pour lesquelles.
%\Wboson\ et ptcs chiralité gauche antiptcs droite.
%8 gluons (pourquoi)
%
%\Wboson\ et chiralité?
%
%Higgs : up-to-date mass + bib ref


\section{Formalisme théorique et interactions}\label{chapter-MS-MSSM-section-formalisme}
\begin{wrapfigure}{R}{7cm}
\centering
\begin{tikzpicture}
\draw [thick, domain=0:4,samples=100,smooth] coordinate (A) plot(\x, {\x/5*2+1-cos(\x/5*360)}) coordinate (B) ;
\fill (A) circle (1pt);
\fill (B) circle (1pt);
\draw (A) node [left] {$A$};
\draw (B) node [right] {$B$};
\draw [->] (-.7,-.2)--+(5.5,0) node [right] {$t$};
\draw [->] (-.7,-.2)--+(0,3.5) node [above] {$x$};

\draw (0,-.1)--+(0,-.2) node [below] {$0$};
\draw (4,-.1)--+(0,-.2) node [below] {$\tau$};
\end{tikzpicture}
\caption{Une particule se déplace au cours du temps d'un point $A$ à un point $B$ le long d'une dimension $x$.}
\label{fig-ptc_classique_ptA_to_pt_B}
\end{wrapfigure}
\subsection{Lagrangien et équation d'Euler-Lagrange}\label{chapter-MS-MSSM-section-formalisme-subsec-into_lagrangien}
%\paragraph{Introduction au lagrangien en mécanique Newtonienne}
Nous souhaitons ici trouver un moyen de décrire le comportement des particules, \ie\ leur évolution à travers le temps et l'espace.
Considérons, dans un premier temps, une particule de masse $m$, soumise à une force $F$, se déplaçant dans le temps le long d'une dimension $x$, d'un point $A$ à $t=0$ à un point $B$ à $t=\tau$, comme illustré sur la figure~\ref{fig-ptc_classique_ptA_to_pt_B}.
\par 
Comme cela est enseigné dès les premiers cours de physique, la trajectoire de cette particule peut être déterminée à l'aide du principe fondamental de la dynamique, ou seconde loi de Newton, qui s'exprime simplement dans ce cas sous la forme
\begin{equation}
m\dv[2]{x}{t} = F
\mend
\end{equation}
Nous obtenons alors la position de la particule à tout instant.
\begin{wrapfigure}{R}{7cm}
\centering
\begin{tikzpicture}
\draw [domain=0:4,samples=100,smooth] coordinate (A) plot(\x, {\x/5*2+1-cos(\x/5*360)}) coordinate (B) ;
\draw [dashed, domain=0:4,samples=10,smooth] plot(\x, {\x/5*2+1-cos(\x/5*360)+(rand)*\x*(4-\x)/5});
%\draw [dotted] (A) -- (B) ;
%\draw [dotted] (A) to[out=90, in=190] (B) ;
%\draw [dotted] (A) to[out=0, in=90] (B) ;
%\draw [dotted] (A) to[out=30, in=135] (B) ;
%\draw [dotted] (A) to[out=90, in=-90] (B) ;
\fill (A) circle (1pt);
\fill (B) circle (1pt);
\draw (A) node [left] {$A$};
\draw (B) node [right] {$B$};
\draw [->] (-.7,-.2)--+(5.5,0) node [right] {$t$};
\draw [->] (-.7,-.2)--+(0,3.5) node [above] {$x$};

\draw (0,-.1)--+(0,-.2) node [below] {$0$};
\draw (4,-.1)--+(0,-.2) node [below] {$\tau$};
\end{tikzpicture}
\caption{Variation de la trajectoire d'une particule se déplaçant au cours du temps d'un point $A$ à un point $B$.}
\label{fig-ptc_quantique_ptA_to_pt_B}
\end{wrapfigure}
\par Or, cette méthode ne permet pas de décrire le comportement des particules fondamentales. En effet, à leur échelle, la mécanique quantique prévaut et il n'est pas possible, lorsque l'on observe une particule à un point $A$ puis à un point $B$, de déterminer la trajectoire exacte suivie par cette particule.
La particule peut suivre la trajectoire déterminée avec la mécanique classique, \ie\ celle de la figure~\ref{fig-ptc_classique_ptA_to_pt_B}, comme toute autre trajectoire reliant $A$ à $B$, comme illustré sur la figure~\ref{fig-ptc_quantique_ptA_to_pt_B}.
\par Si le principe fondamental de la dynamique tel que formulé par Newton ne tient plus dans le contexte de la mécanique quantique, il existe un autre principe physique toujours en place, la conservation de l'énergie. Dans le cas de la particule précédemment décrit, il s'agit de la somme de son énergie cinétique $T$ et de son énergie potentielle $V$, \ie
\begin{equation}
E = T + V = \cste
\end{equation}
où $T$ dépend uniquement de la vitesse de la particule et $V$ uniquement de sa position. %Cette quantité est conservée au cours du temps.
Il en va ainsi de même pour les moyennes temporelles de ces grandeurs,
\begin{equation}
E = \average{E} = \average{T} + \average{V} = \cste
\end{equation}
avec
\begin{equation}
\average{T} = \frac{1}{\tau}\int_0^\tau T(\dot{x}(t)) \dd{t}
\msep
\average{V} = \frac{1}{\tau}\int_0^\tau V(x(t)) \dd{t}
\mend[,]
\end{equation}
où $\dot{x}=\dv{x}{t}$.
\par
Nous pouvons alors nous demander de quelle manière ces grandeurs sont modifiées lorsque la trajectoire suivie par la particule varie par rapport à la trajectoire déterminée par la mécanique Newtonienne. La variation de la valeur moyenne de l'énergie potentielle s'exprime
\begin{equation}
\fdv{\average{V}}{x(t')}
= \frac{1}{\tau}\int_0^\tau \fdv{V(x(t))}{x(t')} \dd{t}
= \frac{1}{\tau}\int_0^\tau \dv{V(x(t))}{x(t)}\delta(t-t') \dd{t}
= \frac{1}{\tau}\eval{\dv{V}{x(t)}}_{t=t'}
= -\frac{1}{\tau}F(x(t'))
\end{equation}
car la force $F$ est reliée au potentiel $V$ par $F = - \dv{V}{x}$.
De même, l'énergie cinétique moyenne varie selon
\begin{align}
\fdv{\average{T}}{x(t')}
&
= \frac{1}{\tau}\int_0^\tau \fdv{T(\dot{x}(t))}{x(t')} \dd{t}
=\frac{1}{\tau}\int_0^\tau \dv{T(\dot{x}(t))}{x(t)}\delta'(t-t') \dd{t}
=-\frac{1}{\tau}\int_0^\tau \delta(t-t') \dv{t}(\dv{T(\dot{x}(t))}{\dot{x}(t)}) \dd{t}
\nonumber
\\&
=-\frac{1}{\tau}\eval{\dv{t}(\dv{T(\dot{x}(t))}{\dot{x}(t)})}_{t=t'}
=-\frac{1}{\tau}m\eval{\dv[2]{x}{t}}_{t=t'}
\end{align}
car pour une particule de masse $m$, en mécanique newtonienne, $T = \frac{1}{2}m\left(\dv{x}{t}\right)^2$.
\par
Le long de la trajectoire classique, le principe fondamental de la dynamique est vérifié. Alors, les variations autour de la trajectoire classique sont reliées par
\begin{equation}
m\dv[2]{x}{t} = F
\Rightarrow
\fdv{\average{T}}{x(t')} = \fdv{\average{V}}{x(t')}
\Rightarrow
\fdv{x(t')}(\average{T}-\average{V}) = 0 \label{eq-fdv_x_T-V_is_0}
\mend
\end{equation}
Ainsi, la différence entre l'énergie cinétique et l'énergie potentielle du système étudié semble jouer un rôle particulier lorsque l'on s'intéresse aux différentes trajectoires possibles pour ce système. Définissons alors le lagrangien $L$ du système étudié comme
\begin{equation}
L = T-V
\mend
\end{equation}
L'intégrale au cours du temps du lagrangien est appelée action et est définie comme
\begin{equation}
S = \int_0^\tau\dd{t}L
\mend
\end{equation}
%Pour une particule quantique, toute trajectoire est possible pour aller d'un point $A$ à un point $B$. Afin de déterminer l'évolution au cours du temps d'une telle particule, il faut sommer toutes ces trajectoires, chacune étant associée à un facteur de phase
%\begin{equation}
%\exp(iS/\hbar)\mend
%\end{equation}
%Pour la plupart des trajectoires, ce facteur de phase donne des contributions qui s'annulent.
%\paragraph{Équation d'Euler-Lagrange}
Compte-tenu de l'équation~\eqref{eq-fdv_x_T-V_is_0}, l'action vérifie
\begin{equation}
\fdv{S}{x(t')}=0
\mend[,]
\end{equation}
ce qui est connu sous le nom de principe de moindre action. Or,
\begin{equation}
\fdv{S}{x(t')}
= \int_0^\tau\dd{t} \left[ \fdv{L}{x(t)}\delta(t-t') + \fdv{L}{\dot{x}(t)}\delta'(t-t')\right]
= \fdv{L}{x(t')} - \dv{t}\fdv{L}{\dot{x}(t')}
\mend[,]
\end{equation}
ce qui implique
\begin{equation}
\fdv{L}{x(t')} - \dv{t}\fdv{L}{\dot{x}(t')} = 0
\mend
\end{equation}
Cette équation est l'équation d'Euler-Lagrange et permet d'obtenir toutes les équations du mouvement du système, \ie\ de décrire son évolution au cours du temps.

\subsection{Lagrangien, champs et symétries}\label{chapter-MS-MSSM-section-formalisme-subsec-lagrangien_champs_symetries}
Le modèle standard décrit le comportement des particules fondamentales à l'aide de la théorie quantique des champs. Une particule est ainsi une excitation d'un champ quantique relativiste $\phi$ et il s'agit alors de décrire l'évolution de ces excitations.
\par Généralisons le raisonnement précédent à un espace à une dimension temporelle et trois dimensions spatiales.
À partir du lagrangien, il est possible de définir la densité lagrangienne \Lcal\ telle que
\begin{equation}
L = \int\dd[3]{x}\Lcal
\msep
S = \int\dd[4]{x}\Lcal
\end{equation}
où $x$ désigne la coordonnée dans l'espace de Minkowski, \ie\ l'espace-temps à quatre dimensions.
Considérons maintenant une densité lagrangienne dépendant d'un champ $\phi(x)$ et de ses dérivées $\partial_\mu\phi(x)$.
Alors,
\begin{equation}
S = \int\dd[4]{x}\Lcal(\phi(x),\partial_\mu\phi(x))
\end{equation}
et du principe de moindre action résultent les équations d'Euler-Lagrange pour cette densité lagrangienne,
\begin{equation}
\fdv{S}{\phi}
=
\pdv{\Lcal}{\phi} - \partial_\mu\pdv{\Lcal}{(\partial_\mu\phi)} = 0
\mend
\end{equation}
Il s'agit à présent de déterminer la densité lagrangienne \Lcal\ du modèle standard.
Par la suite, nous nommerons la densité lagrangienne \Lcal\ \og lagrangien \fg{} dans un souci de praticité.
%Nous cherchons donc à déterminer le lagrangien du modèle standard.
\par Un champ quantique peut subir une transformation de jauge locale. Une telle transformation doit laisser la physique inchangée, ainsi le lagrangien du modèle standard est construit pour être invariant sous les transformations de jauges locales du groupe de symétrie
\begin{equation}
SU(3)_C \otimes SU(2)_L \otimes U(1)_Y
\mend
\end{equation}
De cette construction résultent les interactions fondamentales, discutées ci-après.

\subsection{Interaction électromagnétique}\label{chapter-MS-MSSM-section-formalisme-subsec-QED}

\subsection{Interaction électrofaible}\label{chapter-MS-MSSM-section-formalisme-subsec-EW}

\subsection{Mécanisme de Higgs}\label{chapter-MS-MSSM-section-formalisme-subsec-Higgs_mechanism}

\subsection{Interaction forte}\label{chapter-MS-MSSM-section-formalisme-subsec-QCD}
\subsubsection{La couleur}\label{chapter-MS-MSSM-section-formalisme-subsec-QCD-subsubsec-couleur}
L'interaction forte est la troisième force fondamentale décrite par le modèle standard.
L'analogue de la charge électrique pour l'interaction électromagnétique est, dans le cas de l'interaction forte, la \og couleur \fg,
concept né de l'observation des baryons \Deltabaryonplusplus, \Deltabaryonminus, \Omegabaryonminus.
Dans le modèle des quarks, ces baryons sont composés comme
\begin{equation}
\Deltabaryonplusplus = (\quarku\quarku\quarku)
\msep
\Deltabaryonminus = (\quarkd\quarkd\quarkd)
\msep
\Omegabaryonminus = (\quarks\quarks\quarks)
\mend
\end{equation}
Or, ces baryons sont de spin $\frac{3}{2}$. Les quarks possédant un spin $\frac{1}{2}$, il faudrait alors que pour chacun de ces baryons, les trois quarks les composant aient leurs nombres quantiques égaux, ce qui va à l'encontre du principe de Pauli.
\par Il est possible de décrire ces baryons sans violer le principe d'exclusion de Pauli en introduisant un nouveau nombre quantique, la couleur. Les quarks portent ainsi une charge de couleur, pouvant prendre trois valeurs orthogonales que l'on nomme par convention rouge, vert et bleu. Les antiquarks portent une anticouleur. Il suffit alors que chaque quark porte une couleur différente, \ie
\begin{equation}
\Deltabaryonplusplus = ({\color{red}\quarku}{\color{green}\quarku}{\color{blue}\quarku})
\msep
\Deltabaryonminus = ({\color{red}\quarkd}{\color{green}\quarkd}{\color{blue}\quarkd})
\msep
\Omegabaryonminus = ({\color{red}\quarks}{\color{green}\quarks}{\color{blue}\quarks})
\mend
\end{equation}
\par Les baryons ainsi formés de trois quarks (un rouge, un vert et un bleu) portent une charge de couleur globale nulle, ils sont de couleur \og blanche \fg, comme cela est visible sur la figure~\ref{subfig-colors_for_baryons}. Dans le cas des antibaryons formés de trois antiquarks, sur la figure~\ref{subfig-colors_for_antibaryons}, c'est l'association des trois anticouleurs qui permet d'obtenir un baryon blanc.
Il est également possible de former une particule composite blanche par association d'un quark avec un antiquark portant l'anticouleur correspondante. Les trois combinaisons possibles sont illustrées sur la figure~\ref{subfig-colors_for_mesons}. Il s'agit alors de mésons.
\begin{figure}[h]
\centering
\subcaptionbox{Un baryon est constitué de trois quarks, un de chaque couleur.\label{subfig-colors_for_baryons}}[.3\textwidth]
{\begin{tikzpicture}
\def\rcircle{1.5}
\def\overlappct{.6}

\fill [ltcolorred2] (90:\overlappct*\rcircle) circle (\rcircle);
\fill [ltcolorblue2] (90-120:\overlappct*\rcircle) circle (\rcircle);
\fill [ltcolorgreen2] (90+120:\overlappct*\rcircle) circle (\rcircle);

\begin{scope}
\clip (90:\overlappct*\rcircle) circle (\rcircle);
\clip (90+120:\overlappct*\rcircle) circle (\rcircle);
\fill [ltcoloryellow2] (90:\overlappct*\rcircle) circle (\rcircle);
\end{scope}

\begin{scope}
\clip (90:\overlappct*\rcircle) circle (\rcircle);
\clip (90-120:\overlappct*\rcircle) circle (\rcircle);
\fill [ltcolormagenta2] (90:\overlappct*\rcircle) circle (\rcircle);
\end{scope}

\begin{scope}
\clip (90-120:\overlappct*\rcircle) circle (\rcircle);
\clip (90+120:\overlappct*\rcircle) circle (\rcircle);
\fill [ltcolorcyan2] (90-120:\overlappct*\rcircle) circle (\rcircle);
\end{scope}

\begin{scope}
\clip (90:\overlappct*\rcircle) circle (\rcircle);
\clip (90-120:\overlappct*\rcircle) circle (\rcircle);
\clip (90+120:\overlappct*\rcircle) circle (\rcircle);
\fill [white] (90-120:\overlappct*\rcircle) circle (\rcircle);
\end{scope}

\draw (90:\overlappct*\rcircle+.5*\rcircle) node {rouge} ;
\draw (90-120:\overlappct*\rcircle+.5*\rcircle) node [rotate=60] {bleu} ;
\draw (90+120:\overlappct*\rcircle+.5*\rcircle) node [rotate=-60] {vert} ;

%\draw (-90:\overlappct*\rcircle+0*\rcircle) node {antirouge} ;
%\draw (-90-120:\overlappct*\rcircle+0*\rcircle) node [rotate=60] {antibleu} ;
%\draw (-90+120:\overlappct*\rcircle+0*\rcircle) node [rotate=-60] {antivert} ;

\draw (0,0) node {blanc};
\end{tikzpicture}}
\hfill
\subcaptionbox{Un méson est constitué d'un quark et d'un antiquark de l'anticouleur correspondante.\label{subfig-colors_for_mesons}}[.3\textwidth]
{\begin{tikzpicture}
\def\rcircle{.6}
\def\overlappct{.9}
\def\Dypct{2.5}

\draw (30:\Dypct*\rcircle) coordinate (r);
\draw (150:\Dypct*\rcircle) coordinate (g);
\draw (-90:\Dypct*\rcircle) coordinate (b);

\fill [ltcolorred2] (r)+(-\overlappct*\rcircle/2,0) circle (\rcircle);
\fill [ltcolorgreen2] (g)+(-\overlappct*\rcircle/2,0) circle (\rcircle);
\fill [ltcolorblue2] (b)+(-\overlappct*\rcircle/2,0) circle (\rcircle);

\fill [ltcolorcyan2] (r)+(\overlappct*\rcircle/2,0) circle (\rcircle);
\fill [ltcolormagenta2] (g)+(\overlappct*\rcircle/2,0) circle (\rcircle);
\fill [ltcoloryellow2] (b)+(\overlappct*\rcircle/2,0) circle (\rcircle);

\foreach \pos in {r, g, b}{
\begin{scope}
\clip (\pos)+(-\overlappct*\rcircle/2,0) circle (\rcircle);
\clip (\pos)+(\overlappct*\rcircle/2,0) circle (\rcircle);
\fill [white] (\pos)+(\overlappct*\rcircle/2,0) circle (\rcircle);
\end{scope}
%\draw (\pos) node {blanc} ;
}

%\draw (r)+(-\overlappct*\rcircle/2,0) node {rouge} ;
%\draw (b)+(-\overlappct*\rcircle/2,0) node {bleu} ;
%\draw (g)+(-\overlappct*\rcircle/2,0) node {vert} ;
%
%\draw (r)+(\overlappct*\rcircle/2,0) node {antirouge} ;
%\draw (b)+(\overlappct*\rcircle/2,0) node {antibleu} ;
%\draw (g)+(\overlappct*\rcircle/2,0) node {antivert} ;

\def\rcircle{1.5}
\def\overlappct{.6}
\draw (-90:\overlappct*\rcircle)+(0, -\rcircle) coordinate (bottom);
\end{tikzpicture}}
\hfill
\subcaptionbox{Un antibaryon est constitué de trois antiquarks, un de chaque anticouleur.\label{subfig-colors_for_antibaryons}}[.3\textwidth]
{\begin{tikzpicture}
\def\rcircle{1.5}
\def\overlappct{.6}

\fill [ltcolorcyan2] (-90:\overlappct*\rcircle) circle (\rcircle);
\fill [ltcolormagenta2] (-90+120:\overlappct*\rcircle) circle (\rcircle);
\fill [ltcoloryellow2] (-90-120:\overlappct*\rcircle) circle (\rcircle);

\begin{scope}
\clip (-90:\overlappct*\rcircle) circle (\rcircle);
\clip (-90-120:\overlappct*\rcircle) circle (\rcircle);
\fill [ltcolorgreen2] (-90:\overlappct*\rcircle) circle (\rcircle);
\end{scope}

\begin{scope}
\clip (-90:\overlappct*\rcircle) circle (\rcircle);
\clip (-90+120:\overlappct*\rcircle) circle (\rcircle);
\fill [ltcolorblue2] (-90:\overlappct*\rcircle) circle (\rcircle);
\end{scope}

\begin{scope}
\clip (-90+120:\overlappct*\rcircle) circle (\rcircle);
\clip (-90-120:\overlappct*\rcircle) circle (\rcircle);
\fill [ltcolorred2] (-90+120:\overlappct*\rcircle) circle (\rcircle);
\end{scope}

\begin{scope}
\clip (-90:\overlappct*\rcircle) circle (\rcircle);
\clip (-90+120:\overlappct*\rcircle) circle (\rcircle);
\clip (-90-120:\overlappct*\rcircle) circle (\rcircle);
\fill [white] (-90+120:\overlappct*\rcircle) circle (\rcircle);
\end{scope}

\draw (-90:\overlappct*\rcircle+.5*\rcircle) node {antirouge} ;
\draw (-90+120:\overlappct*\rcircle+.5*\rcircle) node [rotate=-60] {antivert} ;
\draw (-90-120:\overlappct*\rcircle+.5*\rcircle) node [rotate=60] {antibleu} ;

%\draw (90:\overlappct*\rcircle+0*\rcircle) node {rouge} ;
%\draw (90+120:\overlappct*\rcircle+0*\rcircle) node [rotate=-60] {vert} ;
%\draw (90-120:\overlappct*\rcircle+0*\rcircle) node [rotate=60] {bleu} ;

\draw (0,0) node {blanc};

\input{\PhDthesisdir/contents/chapter-MS-MSSM/formalisme/QCD_diagrams_top_bottom.tex}
\end{tikzpicture}}
\caption[Combinaisons des couleurs des quarks dans les hadrons.]{Combinaisons des couleurs des quarks dans les hadrons. La couleur globale est toujours blanche, \ie\ que la charge de couleur globale est nulle.}
\label{fig-colors_for_hadrons}
\end{figure}
\par Les quarks et antiquarks se regroupent ainsi en particules composites, les hadrons (baryons et mésons), dont la neutralité de couleur est confirmée expérimentalement. Ce phénomène est connu sous le nom de \og confinement de couleur \fg{} et est abordé dans la section~\ref{chapter-MS-MSSM-section-formalisme-subsec-QCD-subsubsec-confinement}.
\subsubsection{Symétrie $SU(3)_C$}\label{chapter-MS-MSSM-section-formalisme-subsec-QCD-subsubsec-SU3C}
Afin de décrire l'interaction forte dans le même formalisme que les autre interactions fondamentales, il nous faut un groupe de symétrie. Étant donné qu'il existe trois dimensions de couleur (rouge, verte, bleue), la théorie quantique des champs associée à l'interaction forte se base sur le groupe $SU(3)_C$, où $C$ signifie \og couleur \fg.
\par Tout comme $SU(2)$, $SU(3)$ est un groupe non abélien. Il est possible de reprendre exactement les mêmes calculs que ceux de la section~\ref{chapter-MS-MSSM-section-formalisme-subsec-EW-SU2_general}, en procédant aux changements\footnote{La constante de couplage pour l'interaction forte est souvent notée $\alpha_s$. Nous utilisons ici la notation $g_s$ afin d'illustrer le rôle analogue avec celui $g_Y$ et $g_I$.}
\begin{equation}
\bm{\tau} \in \mathcal{M}_2(\mathbb{C})^3 \leftrightarrow \bm{\lambda} \in \mathcal{M}_3(\mathbb{C})^8
\msep
\bm{\alpha}\in\mathbb{R}^3 \leftrightarrow \bm{\theta}\in\mathbb{R}^8
\msep
g_I \leftrightarrow g_s
\msep
\bm{W}_\mu \leftrightarrow \bm{G}_\mu
\msep
\bm{W}_{\mu\nu} \leftrightarrow \bm{G}_{\mu\nu}
\label{eq-hapter-MS-MSSM-section-formalisme-subsec-QCD-subsubsec-SU3C-analogieSU2}
\end{equation}
où $\bm{\lambda}$ est un vecteur à huit composantes, chacune étant une matrice de Gell-Mann, définies dans l'annexe~\ifref{annexe-maths}{\ref{annexe-maths}}{A} et où $\bm{G}_\mu$ décrit donc huit gluons, bosons vecteurs de l'interaction forte.
\par Les gluons portent une couleur et une anticouleur. Lors de chaque interaction, la charge de couleur est conservée, ainsi un quark rouge interagissant avec un gluon bleu-antirouge devient un quark bleu. Le flux de couleur ainsi conservé dans cet exemple est représenté sur la figure~\ref{fig-fgraph-QCD_color_flux}.
\begin{figure}[h]
\centering
\vspace{\baselineskip}
\subcaptionbox{Diagramme de Feynman de l'interaction.\label{subfig-fgraph-qgqg}}[.3\textwidth]
{\begin{fmffile}{qgq}\fmfstraight
\begin{fmfchar*}(20,20)
  \fmfleft{i1,i2}
  \fmfright{o1}
  \fmf{fermion}{i2,v,o1}
  \fmf{gluon}{i1,v}
  \fmflabel{\gluon}{i1}
  \fmflabel{\quark}{i2}
  \fmflabel{\quark}{o1}
  \fmfdot{v}
\end{fmfchar*}
\end{fmffile}\vspace{\baselineskip}}
\hfill
\subcaptionbox{Représentation du flux de couleur conservé.\label{subfig-fgraph-qgq_colors}}[.3\textwidth]
{\input{\PhDthesisdir/tex/Feynman_diagrams/QCD/fgraph-qgq_colors.tex}\vspace{\baselineskip}}
\hfill
\subcaptionbox{Interprétation en utilisant les anticouleurs.\label{subfig-fgraph-qgq_colors_and_anticolors}}[.3\textwidth]
{\begin{fmffile}{qgq_colors_and_anticolors}\fmfstraight
\begin{fmfchar*}(20,20)
  \fmfleft{i1,i2}
  \fmfright{o1}
  \fmf{phantom}{i2,v,o1}
  \fmf{phantom}{i1,v}
  \fmflabel{\gluon}{i1}
  \fmflabel{\quark}{i2}
  \fmflabel{\quark}{o1}
  \fmffreeze
  \fmf{fermion, fore=green+blue}{i1,v}
  \fmf{fermion, fore=red}{i2,v}
  \fmf{fermion, fore=blue}{v,o1}
  \fmfi{fermion, fore=blue}{vpath (__i1,__v) shifted (thick*(2,0))}
  \fmfblob{.2w}{v}
\end{fmfchar*}
\end{fmffile}\vspace{\baselineskip}}

\caption[Interaction entre un quark et un gluon.]{Interaction entre un quark rouge et un gluon bleu-antirouge, donnant un quark bleu.}
\label{fig-fgraph-QCD_color_flux}
\end{figure}

\par Le terme non linéaire $\bm{G}_\mu\wedge\bm{G}_\nu$ dans l'expression de $\bm{G}_{\mu\nu}$\footnote{Obtenue à partir de l'analogie~\eqref{eq-hapter-MS-MSSM-section-formalisme-subsec-QCD-subsubsec-SU3C-analogieSU2} appliquée à l'équation~\eqref{eq-chapter-MS-MSSM-section-formalisme-subsec-EW-defWmunu}.} est lourd de conséquences.
Il permet le couplage entre trois et quatre gluons, comme cela est illustré sur la figure~\ref{fig-fgraph-QCD_3_et_4_gluons}, et donne à l'interaction forte toute sa singularité. En effet, ce terme est responsable de l'initiation de la gerbe partonique qui donne naissance aux jets, dont il est question au chapitre~\ifref{chapter-JERC}{\ref{chapter-JERC}}{sur la calibration en énergie des jets}, ainsi que du confinement de couleur.
\begin{figure}[h]
\centering
\vspace{\baselineskip}
\subcaptionbox{\label{subfig-fgraph-ggg}}[.45\textwidth]
{\begin{fmffile}{ggg}\fmfstraight
\begin{fmfchar*}(20,20)
  \fmfleft{i1}
  \fmfright{o1,o2}
  \fmf{gluon}{i1,v}
  \fmf{gluon}{o1,v}
  \fmf{gluon}{o2,v}
  \fmfdot{v}
  \fmflabel{\gluon}{i1}
  \fmflabel{\gluon}{o1}
  \fmflabel{\gluon}{o2}
\end{fmfchar*}
\end{fmffile}\vspace{\baselineskip}}
\hfill
\subcaptionbox{\label{subfig-fgraph-gggg}}[.45\textwidth]
{\begin{fmffile}{gggg}\fmfstraight
\begin{fmfchar*}(20,20)
  \fmfleft{i1,i2}
  \fmfright{o1,o2}
  \fmf{gluon}{i1,v}
  \fmf{gluon}{i2,v}
  \fmf{gluon}{o1,v}
  \fmf{gluon}{o2,v}
  \fmfdot{v}
  \fmflabel{\gluon}{i1}
  \fmflabel{\gluon}{i2}
  \fmflabel{\gluon}{o1}
  \fmflabel{\gluon}{o2}
\end{fmfchar*}
\end{fmffile}\vspace{\baselineskip}}

\caption{Diagrammes de Feynman correspondant à l'interaction entre trois et quatre gluons.}
\label{fig-fgraph-QCD_3_et_4_gluons}
\end{figure}
\subsubsection{Confinement de couleur et liberté asymptotique}\label{chapter-MS-MSSM-section-formalisme-subsec-QCD-subsubsec-confinement}
Le confinement de couleur force les quarks, particules colorées, à s'associer en formant des particules composites, les hadrons, états liés de charge globale de couleur nulle. Ce phénomène empirique peut s'expliquer par la variation, en fonction de l'échelle d'énergie, de la constante de couplage de l'interaction forte $g_s$, représentée sur la figure~\ref{fig-g_s_fct_energy}.
\begin{figure}[h]
\centering
\begin{tikzpicture}
\node[anchor=south west,inner sep=0] at (0,0) {\includegraphics[width=10cm]{\PhDthesisdir/contents/chapter-MS-MSSM/formalisme/QCD_value_fct_Q.png}};
\fill [white] (1,0) rectangle (10, .65);
\fill [white] (1.15,0) rectangle (0,6);
\fill [white] (1.5,.85) rectangle (7.2,1.3);

\draw (1.2,.5) node {\small \num{1}} ;
\draw (3.8,.5) node {\small \num{10}} ;
\draw (6.45,.5) node {\small \num{100}} ;
\draw (9.1,.5) node {\small \num{1000}} ;

\draw (5,.25) node {$k$ (\SI{}{\GeV})} ;

\draw (.8,1.5) node {\small \num{0.1}} ;
\draw (.8,3.15) node {\small \num{0.2}} ;
\draw (.8,4.7) node {\small \num{0.3}} ;

\draw (.5,5.8) node {$g_s(k)$} ;

\draw (1.6, 1.1) node [right] {\small $\equiv$ QCD $g_s(m_{\Zboson}) = \num{0.1181}\pm\num{0.0011}$};
\end{tikzpicture}
\caption[Mesure de $g_s$ en fonction de l'échelle d'énergie.]{Mesures de $g_s$ en fonction de l'échelle d'énergie $k$ (points) et prédiction théorique (courbe)~\cite{PDG_booklet_2018}. Le degré des calculs perturbatifs de QCD utilisés pour extraire $g_s$ est indiqué entre parenthèses (NLO: \emph{next-to-leading order}, \ie\ jusqu'à l'ordre suivant le premier degré non nul; NNLO: un ordre de plus que NLO; etc.).}
\label{fig-g_s_fct_energy}
\end{figure}
\par Aux basses énergie, $g_s$ diverge.
Ainsi, séparer et isoler des particules colorées mène à une énergie potentielle de couleur suffisamment grande pour créer des paires quark-antiquark. Ce processus se poursuit alors jusqu'à ce qu'il ne reste plus que des particules blanches. Lorsqu'un quark est issu d'une collision en physique des particules, ce processus se réalise et s'appelle \emph{hadronisation}. Il s'agit d'une étape de la formation des jets, flux collimé de particules caractéristique de la production de quarks.
\par De plus, à cause de la valeur élevée de $g_s$ aux basses énergies, il n'est pas possible de réaliser des calculs perturbatifs pourtant usuels en théorie quantique des champs.
D'autres techniques sont toutefois utilisées, comme la méthode de QCD sur réseau. Son principe est de discrétiser l'espace-temps en en un réseau de points. Bien que cette méthode requière d'importantes capacités de calcul et beaucoup de temps, elle permet d'obtenir avec succès les masses des hadrons comme cela se voit sur la figure~\ref{fig-lattice_QCD_masses} pour les hadrons légers.
\begin{figure}[h]
\centering
\includegraphics[width=.5\textwidth]{\PhDthesisdir/contents/chapter-MS-MSSM/formalisme/light_hadrons_masses.png}
\caption{Spectre de masse des hadrons légers. Les lignes horizontales ainsi que les zones grisées sont les valeurs expérimentales et les largeurs de désintégration. Les résultats issus de~\cite{ab_initio_hadron_masses} en utilisant des calculs de QCD sur réseau sont représentés par des cercles, avec les erreurs associées. Seules les masses des hadrons \pion, \Kaon\ et \Xibaryon\ sont sans barre d'erreur, elles sont utilisées pour fixer des paramètres libres du modèle.}
\label{fig-lattice_QCD_masses}
\end{figure}
\par La valeur de $g_s$ à une échelle d'énergie $k$ est reliée à la valeur de $g_s$ à une échelle d'énergie $\mu$ par la relation
\begin{equation}
g_s(k) = \frac{g_s(\mu)}{1+ \frac{11n_c-2n_f}{12\pi} g_s(\mu)\ln(\frac{k^2}{\mu^2})}
\end{equation}
avec $n_c$ le nombre de couleurs et $n_f$ le nombre de saveurs de quarks, \ie\ $n_c=3$ et $n_f=6$~\cite{salam2010elements}.
Cette relation peut ainsi se réécrire
\begin{equation}
g_s (k) =
\frac{6\pi}{21 \ln(\frac{k}{\Lambda_\text{QCD}})}
\msep
\Lambda_\text{QCD} = 218\pm\SI{24}{\MeV}
\mend[,]
\end{equation}
avec $\Lambda_\text{QCD}$ l'échelle d'énergie à laquelle $g_s$ diverge.
Il ressort que $g_s$ décroît lorsque l'échelle d'énergie augmente.
Cette diminution de $g_s$ aux hautes énergies est la \og liberté asymptotique \fg, régime où les particules colorées ne sont plus confinées et peuvent se propager comme des particules libres. Aux LHC, les énergies de collision permettent d'atteindre ce régime.
%the coupling decreases logarithmically, a phenomenon known as asymptotic freedom (the discovery of which was awarded with the Nobel Prize in Physics in 2004).



\newpage
\section{Succès et limites du modèle standard}
\subsection{Succès}

\subsection{Limites}
\paragraph{Gravitation}
\paragraph{Masse des neutrinos}
\paragraph{Matière noire}
bullet cluster!
\paragraph{Énergie noire}
\paragraph{Asymétrie matière-antimatière}

\subsection{Région \og BSM \fg{}}\label{chapter-HTT_analysis-section-categorisation-BSM}
Les catégories BSM, introduites dans la référence~\cite{CMS-PAS-HIG-17-020}, sont construites dans le but de chercher une résonance correspondant à un boson de Higgs lourd.
Les coupures présentées dans cette section sont appliquées en plus des coupures permettant de séparer les régions SM et BSM.
\par
Une première catégorisation est basée sur la présence de jets issus de quarks~\quarkb.
Deux catégories sont ainsi définies:
\begin{itemize}
\item \CATnobtag: $\Nbtag =0$;
\item \CATbtag: $\Nbtag\geq1$.
\end{itemize}
Dans le cas des canaux \mu\tauh, \ele\tauh\ et \ele\mu, chacune de ces deux catégories est à nouveau subdivisée.
\paragraph{Canaux \mu\tauh\ et \ele\tauh}
Dans ces deux canaux, la masse transverse de $L_1$ (le muon ou l'électron, notés $\ell$) définie par
\begin{equation}
\mT^{(\ell)} = \sqrt{2 \, \pT^{(\ell)} \, \MET \, (1-\cos\Delta\phi)} \label{eq-mT_def-ell}
\end{equation}
avec $\Delta\phi = \phi^{(\ell)} - \phi^{(\MET)}$
est utilisée afin de définir deux catégories:
\begin{itemize}
\item \CATtightmt: $\mT^{(\ell)} < \SI{40}{\GeV}$;
\item \CATloosemt: $\SI{40}{\GeV} \leq \mT^{(\ell)} < \SI{70}{\GeV}$;
\end{itemize}
la limite haute sur \mT\ pour la catégorie \CATloosemt\ étant appliquée afin de s'assurer que la région de signal soit orthogonale à la région de détermination (DR) des facteurs de faux des événements $\Wboson+\text{jets}$.
Les facteurs de faux sont abordés dans la section~\ref{chapter-HTT_analysis-section-bg_estimation-FF_method}.
La majorité des événements de signal, en particulier pour \Higgs\ et \HiggsA\ de basse masse, se trouve dans la catégorie \CATtightmt.
La catégorie \CATloosemt\ permet quant à elle d'augmenter l'acceptance du signal pour $m_{\Higgs,\HiggsA} > \SI{700}{\GeV}$.
La figure~\ref{subfig-chapter-HTT_analysis-section-categorisation-BSM-subcats-mT} illustre ces coupures sur $\mT^{(\ell)}$ dans le cas du canal \ele\tauh\ pour l'année 2018.
\begin{wrapfigure}{R}{.45\textwidth}
\centering

\begin{tikzpicture}
%% base
\draw [->] (0,0)--(2,0) node [right] {$\bvec_x$};
\draw [->] (0,0)--(0,2) node [above] {$\bvec_y$};

\def\xMET{-1.5}
\def\yMET{-1.25}
\def\xELE{2}
\def\yELE{3}
\def\xMU{.5}
\def\yMU{-2}

\draw [thick, -latex, ltcolorred] (0,0) -- (\xMET,\yMET) coordinate (vMET);
\draw [ltcolorred] (vMET) node [below] {\vMET};
\draw [thick, -latex, ltcolorblue] (0,0) -- (\xELE,\yELE) coordinate (vE);
\draw [ltcolorblue] (vE) node [right] {$\vpT (\ele)$};
\draw [thick, -latex, ltcolorblue] (0,0) -- (\xMU,\yMU) coordinate (vM);
\draw [ltcolorblue] (vM) node [right] {$\vpT (\mu)$};

\draw [dashed, -latex] ({-\xELE+-\xMU}, {-\yELE+-\yMU}) -- ({1.25*(\xELE+\xMU)}, {1.25*(\yELE+\yMU)}) node [above] {$\vec{\zeta}$};

\draw [thick, ltcolorgreen, -latex] (0,0) -- ({\xELE+\xMU}, {\yELE+\yMU}) coordinate (vVIS);
\draw [ltcolorgreen] (vVIS) node [below right] {$p_\zeta^\text{vis}\hat{\zeta}$};

\draw [thick, -latex, ltcolorred4] (0,0) --+ ({180+acos((\xELE+\xMU)/(((\xELE+\xMU)*(\xELE+\xMU)+(\yELE+\yMU)*(\yELE+\yMU))^(0.5)))}:{(\xMET*\xMET+\yMET*\yMET)^(0.5)*cos(180-acos((\xELE+\xMU)/(((\xELE+\xMU)*(\xELE+\xMU)+(\yELE+\yMU)*(\yELE+\yMU))^(0.5)))-acos((\xMET)/((\xMET*\xMET+\yMET*\yMET)^(0.5))))})  coordinate (vMETzeta) ;
\draw [ltcolorred4] (vMETzeta) node [above] {$p_\zeta^\text{miss}\hat{\zeta}$};

\draw [dotted] (vMET) -- (vMETzeta);
\draw [dotted] (vE) -- (vVIS);
\draw [dotted] (vM) -- (vVIS);

\end{tikzpicture}
\caption{Illustration de la définition de $\hat{\zeta}$~\cite{Jang_thesis}. Le plan de ce schéma est le plan transverse.}
\label{fig-zeta_illustration}
\end{wrapfigure}
\paragraph{Canal \ele\mu}
La variable \Dzeta\ est définie selon
\begin{equation}
\Dzeta = p_\zeta^\text{miss} - \num{0.85} p_\zeta^\text{vis}
\label{eq-Dzeta_def}
\end{equation}
avec
\begin{equation}
p_\zeta^\text{miss} = \vMET \cdot \hat{\zeta}
\msep
p_\zeta^\text{vis} = \left( \vpT^{\ele} + \vpT^{\mu} \right) \cdot \hat{\zeta}
\end{equation}
où $\hat{\zeta}$ est la direction bisectionnelle entre l'électron et le muon dans le plan transverse~\cite{Jang_thesis}, comme illustré sur la figure~\ref{fig-zeta_illustration}.
Trois catégories sont alors définies:
\begin{itemize}
\item \CATlowdz: $-\SI{35}{\GeV} \leq \Dzeta < -\SI{10}{\GeV}$;
\item \CATmediumdz: $-\SI{10}{\GeV} \leq \Dzeta < \SI{30}{\GeV}$;
\item \CAThighdz: $\SI{30}{\GeV} \leq \Dzeta$;
\end{itemize}
la limite basse sur \Dzeta\ pour la catégorie \CATlowdz\ étant appliquée afin de s'assurer que la région de signal soit orthogonale à la région de contrôle (CR) du bruit de fond \ttbar.
Ces trois catégories permettent d'obtenir diverses pureté de signal et fractions de bruit de fond \ttbar.
La majorité des événements de signal se trouve dans la catégorie \CATmediumdz.
La figure~\ref{subfig-chapter-HTT_analysis-section-categorisation-BSM-subcats-Dz} illustre ces coupures sur \Dzeta.
\begin{figure}[h]
\centering

\subcaptionbox{Catégorisation basée sur $\mT^{(\ell)}$.\label{subfig-chapter-HTT_analysis-section-categorisation-BSM-subcats-mT}}[.475\textwidth]
{\plotHTTcontrolCATmt{2018}{fully_classic}{et}}
\hfill
\subcaptionbox{Catégorisation basée sur \Dzeta.\label{subfig-chapter-HTT_analysis-section-categorisation-BSM-subcats-Dz}}[.475\textwidth]
{\plotHTTcontrolCATdz{2018}{fully_classic}{em}}

\caption[Illustrations des catégorisations basées sur $\mT^{(\ell)}$ et \Dzeta]{Illustrations des catégorisations basées sur $\mT^{(\ell)}$ et \Dzeta, respectivement sur les événements des canaux \ele\tauh\ et \ele\mu\ de l'année 2018.}
\label{fig-chapter-HTT_analysis-section-categorisation-BSM-subcats}
\end{figure}
\paragraph{Récapitulatif}
Les catégories BSM correspondant à la région de signal (SR), \ie\ en dehors des régions de détermination (DR) et de contrôle (CR), sont résumées sur la figure~\ref{fig-chapter-HTT_analysis-section-categorisation-BSM-cats_recap} pour les quatre canaux considérés.
\begin{figure}[h]
\centering
\begin{tikzpicture}
\foreach \dx in {0, 6}{
    
    \draw (\dx, -3) node {\small\CATlowdz\vphantom{Àq}};
    \draw (\dx+2, -3) node {\small\CATmediumdz\vphantom{Àq}};
    \draw (\dx+4, -3) node {\small\CAThighdz\vphantom{Àq}};
    
    \draw [thick] (\dx-.925,-3-.425) rectangle  + (1.85,.85); 
    \draw [thick] (\dx+2-.925,-3-.425) rectangle  + (1.85,.85); 
    \draw [thick] (\dx+4-.925,-3-.425) rectangle  + (1.85,.85); 
    
    \foreach \ddx/\cat in {0/\CATtightmt, 3/\CATloosemt}{
        \foreach \ddy in {-1, -2}{
            
            \draw (\dx+\ddx+.5, \ddy) node {\small\cat\vphantom{Àq}};
            \draw (\dx+\ddx+.5, \ddy) node {\small\cat\vphantom{Àq}};
            
            \draw [thick] (\dx+\ddx-.925,\ddy-.425) rectangle  + (2.85,.85); 
            \draw [thick] (\dx+\ddx-.925,\ddy-.425) rectangle  + (2.85,.85); 
            
        }
    }
    
    \draw [thick] (\dx-.925,-.425) rectangle  + (5.85,.85); 
}

\draw (-1, 0) node (a) [left] {$\Higgs\to\tau\tau\to\tauh\tauh$};
\draw (a.west) + (0, -1) node [right] {$\Higgs\to\tau\tau\to\mu\tauh$};
\draw (a.west) + (0, -2) node [right] {$\Higgs\to\tau\tau\to\ele\tauh$};
\draw (a.west) + (0, -3) node [right] {$\Higgs\to\tau\tau\to\ele\mu$};

\draw (2, 1) node {\CATnobtag\vphantom{Àq}};
\draw (8, 1) node {\CATbtag\vphantom{Àq}};

\draw [very thick] (-1.25, .5) -- (11.25, .5) ;
\draw [very thick] (5, 1.5) -- (5, -3.75) ;

\end{tikzpicture}
\caption{Catégories BSM pour les quatre canaux considérés.}
\label{fig-chapter-HTT_analysis-section-categorisation-BSM-cats_recap}
\end{figure}

\section{Phénoménologie des bosons de Higgs du MSSM}\label{chapter-MS-MSSM-section-pheno_Higgs_MSSM}
5 bosons, 3 neutres.

At leading order, 2 free parameters: $(m_{\HiggsA}, \tan\beta)$.

\begin{table}[H]
\centering
\begin{tabular}{rccc}
\toprule
Couplage avec & \higgs & \Higgs & \HiggsA \\
\midrule
Bosons vecteurs & $\sim\sin(\beta-\alpha)$ & $\sim\cos(\beta-\alpha)$ & $0$\\
Fermions hauts & $\displaystyle \sim\frac{\cos\alpha}{\sin\beta}$ & $\displaystyle \sim\frac{\sin\alpha}{\sin\beta}$ & $\displaystyle \sim\frac{1}{\tan\beta}$ \\
Fermions bas & $\displaystyle \sim\frac{\sin\alpha}{\cos\beta}$ & $\displaystyle \sim\frac{\cos\alpha}{\cos\beta}$ & $\sim\tan\beta$ \\
\bottomrule
\end{tabular}
\caption{Valeurs de $y_{ij}^\text{MSSM,X}/y_{ij}^\text{SM,\higgs}$ pour $X\in\set{\higgs,\Higgs,\HiggsA}$.}
\end{table}

Decoupling limit $m_{\HiggsA}\gg m_{\Zboson}$: $\alpha\to\beta-\frac{\pi}{2}$.

\begin{table}[H]
\centering
\begin{tabular}{rccc}
\toprule
Couplage avec & \higgs & \Higgs & \HiggsA \\
\midrule
Bosons vecteurs & $\sim1$ & $\sim0$ & $0$\\
Fermions hauts & $\sim1$ & $\displaystyle \sim\frac{1}{\tan\beta}$ & $\displaystyle \sim\frac{1}{\tan\beta}$ \\
Fermions bas & $\sim1$ & $\sim\tan\beta$ & $\sim\tan\beta$ \\
\bottomrule
\end{tabular}
\caption{Valeurs de $y_{ij}^\text{MSSM,X}/y_{ij}^\text{SM,\higgs}$ pour $X\in\set{\higgs,\Higgs,\HiggsA}$ \emph{in the decoupling limit}.}
\end{table}

Consequences:
\begin{itemize}
\item SM-like behaviour of \higgs;
\item vanishing $\Higgs\to VV$ coupling;
\item $\Higgs\to\fermion\antifermion$ couplings similar to those for \HiggsA.
\end{itemize}

With increasing $\tan\beta$:
\begin{itemize}
\item enhanced $\Higgs/\HiggsA \to \tau\tau$ decay;
\item enhanced \quarkb-associated production of \Higgs\ and \HiggsA.
\end{itemize}

\subsection{Production de bosons de Higgs}\label{chapter-MS-MSSM-section-pheno_Higgs_MSSM-subsec-production}

\begin{figure}[h]
\centering
\vspace{\baselineskip}
\subcaptionbox{Production par fusion de gluons.\label{subfig-fgraph-gg_loop_h}}[.3\textwidth]
{\input{\PhDthesisdir/tex/Feynman_diagrams/H_production/fgraph-gg_loop_h.tex}\vspace{\baselineskip}}
\hfill
\subcaptionbox{Production par fusion de bosons vecteurs en voie $t$.\label{subfig-fgraph-Higgs_VBF_t}}[.3\textwidth]
{\input{\PhDthesisdir/tex/Feynman_diagrams/H_production/fgraph-Higgs_VBF_t.tex}\vspace{\baselineskip}}
\hfill
\subcaptionbox{Production par fusion de bosons vecteurs en voie $u$.\label{subfig-fgraph-Higgs_VBF_u}}[.3\textwidth]
{\begin{fmffile}{Higgs_VBF_u}\fmfstraight
\begin{fmfchar*}(30,20)
  \fmfleft{qa,qb}
  \fmfright{qc,h,qd}
  \fmf{fermion}{v1,qc}
  \fmf{fermion}{v2,qd}
  \fmf{phantom,tension=2}{qa,v1}
  \fmf{phantom,tension=2}{qb,v2}
  \fmffreeze
  \fmf{plain}{vv2,v2}
  \fmf{plain}{vv1,v1}
  \fmf{fermion}{qa,vv2}
  \fmf{fermion}{qb,vv1}
  \fmf{boson, label=$\Wbosonmp,, \Zboson$, l.side=right, tension=3}{v1,v3}
  \fmf{boson, label=$\Wbosonpm,, \Zboson$, l.side=left, tension=3}{v2,v3}
  \fmf{dashes}{v3,h}
  \fmfdot{v1,v2,v3}
  \fmflabel{\quark}{qa}
  \fmflabel{\quark}{qb}
  \fmflabel{\quark}{qc}
  \fmflabel{\quark}{qd}
  \fmflabel{\higgs}{h}
\end{fmfchar*}
\end{fmffile}
\vspace{\baselineskip}}
\caption[Production de boson de Higgs par fusion de gluons et de bosons vecteurs.]{Diagrammes de Feynman de production de boson de Higgs dans le cadre du modèle standard par fusion de gluons (\gluon\gluon\higgs) et fusion de bosons vecteurs (VBF).}
\label{fig-fgraph-Higgs_prod_ggh_VBF}
\end{figure}

\begin{figure}[h]
\centering
\vspace{\baselineskip}
\subcaptionbox{Production en association avec un boson \Wboson.\label{subfig-fgraph-Higgs_VH_W}}[.3\textwidth]
{\input{\PhDthesisdir/tex/Feynman_diagrams/H_production/fgraph-Higgs_VH_W.tex}\vspace{\baselineskip}}
\hfill
\subcaptionbox{Production en association avec un boson \Zboson.\label{subfig-fgraph-Higgs_VH_Z}}[.3\textwidth]
{\begin{fmffile}{Higgs_VH_Z}\fmfstraight
\begin{fmfchar*}(42,25)
  \fmfleft{i1,i2}
  \fmfright{o1,o2}
  \fmf{fermion}{i1,v1,i2}
  \fmf{boson, label=${\Zboson}^*$, l.side=left}{v1,v2}
  \fmf{boson}{v2,o1}
  \fmf{dashes}{v2,o2}
  \fmfdot{v1,v2}
  \fmflabel{\quark}{i1}
  \fmflabel{\antiquark}{i2}
  \fmflabel{\Zboson}{o1}
  \fmflabel{\higgs}{o2}
\end{fmfchar*}
\end{fmffile}
\vspace{\baselineskip}}
\hfill
\subcaptionbox{Production par fusion de gluons associée à un boson \Zboson.\label{subfig-fgraph-Higgs_gg_loop_Zh}}[.3\textwidth]
{\begin{fmffile}{Higgs_gg_loop_Zh}\fmfstraight
\begin{fmfchar*}(42,25)
  \fmfleft{g1,g2}
  \fmfright{Z,h}
  \fmf{gluon}{g1,g1loop}
  \fmf{gluon}{g2,g2loop}
  \fmf{fermion}{g1loop,Zloop,hloop,g2loop,g1loop}
  \fmf{dashes}{hloop,h}
  \fmf{boson}{Zloop,Z}
  \fmfdot{g1loop,Zloop,hloop,g2loop}
  \fmflabel{\gluon}{g1}
  \fmflabel{\gluon}{g2}
  \fmflabel{\higgs}{h}
  \fmflabel{\Zboson}{Z}
\end{fmfchar*}
\end{fmffile}
\vspace{\baselineskip}}
\caption{Diagrammes de Feynman de production de boson de Higgs dans le cadre du modèle standard en association avec un boson.}
\label{fig-fgraph-Higgs_prod_VH_ggZh}
\end{figure}

\begin{figure}[h]
\centering
\vspace{\baselineskip}
\subcaptionbox{\label{subfig-fgraph-Higgs_with_b_gg_g_bbh}}[.45\textwidth]
{\begin{fmffile}{Higgs_with_b_gg_g_bbh}\fmfstraight
\begin{fmfchar*}(30,20)
  \fmfleft{i1,i2}
  \fmfright{o1,o2,o3}
  \fmf{gluon}{i1,v1}
  \fmf{gluon}{i2,v1}
  \fmf{gluon}{v1,v2}
  \fmf{phantom}{o1,v2,o3}
  \fmffreeze
  \fmf{fermion}{o1,v2,v3,o3}
  \fmffreeze
  \fmf{dashes}{v3,o2}
  \fmfdot{v1,v2,v3}
  \fmflabel{\gluon}{i1}
  \fmflabel{\gluon}{i2}
  \fmflabel{\antiquarkb}{o1}
  \fmflabel{\quarkb}{o3}
  \fmflabel{\higgs}{o2}
\end{fmfchar*}
\end{fmffile}
\vspace{\baselineskip}}
\qquad
\subcaptionbox{\label{subfig-fgraph-Higgs_with_b_qq_g_bbh}}[.45\textwidth]
{\begin{fmffile}{Higgs_with_b_qq_g_bbh}\fmfstraight
\begin{fmfchar*}(42,25)
  \fmfleft{i1,i2}
  \fmfright{o1,o2,o3}
  \fmf{fermion}{i1,v1,i2}
  \fmf{gluon}{v1,v2}
  \fmf{phantom}{o1,v2,o3}
  \fmffreeze
  \fmf{fermion}{o1,v2,v3,o3}
  \fmffreeze
  \fmf{dashes}{v3,o2}
  \fmfdot{v1,v2,v3}
  \fmflabel{\quark}{i1}
  \fmflabel{\antiquark}{i2}
  \fmflabel{\antiquarkb}{o1}
  \fmflabel{\quarkb}{o3}
  \fmflabel{\higgs}{o2}
\end{fmfchar*}
\end{fmffile}
\vspace{\baselineskip}}

\vspace{2\baselineskip}
\subcaptionbox{\label{subfig-fgraph-Higgs_with_b_gg_hbb}}[.45\textwidth]
{\input{\PhDthesisdir/tex/Feynman_diagrams/H_production/fgraph-Higgs_with_b_gg_hbb.tex}\vspace{\baselineskip}}
\qquad
\subcaptionbox{\label{subfig-fgraph-Higgs_with_b_bg_b_bh}}[.45\textwidth]
{\begin{fmffile}{Higgs_with_b_bg_b_bh}\fmfstraight
\begin{fmfchar*}(42,25)
  \fmfleft{i1,i2}
  \fmfright{o1,o2}
  \fmf{fermion}{i2,v1,v2,o1}
  \fmf{gluon}{i1,v1}
  \fmf{dashes}{v2,o2}
  \fmfdot{v1,v2}
  \fmflabel{\gluon}{i1}
  \fmflabel{\quarkb}{i2}
  \fmflabel{\quarkb}{o1}
  \fmflabel{\higgs}{o2}
\end{fmfchar*}
\end{fmffile}
\vspace{\baselineskip}}

\caption{Diagrammes de Feynman de production de boson de Higgs dans le cadre du modèle standard en association avec un quark \quarkb.}
\label{fig-fgraph-Higgs_prod_with_b}
\end{figure}

\begin{fmffile}{bb_hHA}\fmfstraight
\begin{fmfchar*}(42,25)
  \fmfleft{b1,b2}
  \fmfright{h}
  \fmf{fermion, label=$b$, l.side=left}{b1,bh}
  \fmf{fermion, label=$\bar{b}$, l.side=left}{bh,b2}
  \fmf{dashes, label=$\Hs,, \Hn,, \Ha$, l.side=left}{bh,h}
  \fmfdot{bh}
\end{fmfchar*}
\end{fmffile}


\begin{fmffile}{bg_b_bhHA}\fmfstraight
\begin{fmfchar*}(42,25)
  \fmfleft{i1,i2}
  \fmfright{o1,o2}
  \fmf{fermion}{i2,v1,v2,o1}
  \fmf{gluon}{i1,v1}
  \fmf{dashes, label=$\Hs,, \Hn,, \Ha$, l.side=left}{v2,o2}
  \fmfdot{v1,v2}
  \fmflabel{\gluon}{i1}
  \fmflabel{\quarkb}{i2}
  \fmflabel{\quarkb}{o1}
\end{fmfchar*}
\end{fmffile}

\begin{fmffile}{gg_hHAbb}\fmfstraight
\begin{fmfchar*}(50,25)
  \fmfleft{g1,g2}
  \fmfright{b1,h,b2}
  \fmf{gluon, label=$g$, l.side=left, l.d=2mm}{g1,g1b1}
  \fmf{gluon, label=$g$, l.d=3mm}{g2,g2b2}
  \fmf{fermion, label=$b$, l.side=left}{g1b1,b1}
  \fmf{fermion, label=$\bar{b}$, l.side=left}{b2,g2b2}
  \fmffreeze
  \fmf{fermion}{g2b2,bh,g1b1}
  \fmffreeze
  \fmf{dashes, label=$\Hs,, \Hn,, \Ha$, l.side=left}{bh,h}
  \fmfdot{g1b1,g2b2,bh}
\end{fmfchar*}
\end{fmffile}


\begin{fmffile}{gg_loop_hHA}\fmfstraight
\begin{fmfchar*}(30,20)
  \fmfleft{g1,fi,g2}
  \fmfright{fo1,h,fo2}
  \fmf{gluon}{g1,g1loop}
  \fmf{gluon}{g2,g2loop}
  \fmf{phantom, tension=.6}{g1loop,fo1}
  \fmf{phantom, tension=.6}{g2loop,fo2}
  \fmffreeze
  \fmf{fermion}{g1loop,hloop,g2loop,g1loop}
  \fmf{fermion}{g2loop,g1loop}
  \fmf{dashes, label=$\Hs,, \Hn,, \Ha$, l.side=left, tension=1.75}{hloop,h}
  \fmfdot{g1loop,hloop,g2loop}
  \fmffreeze
  \fmf{phantom}{g1loop,fakev1}
  \fmf{phantom}{g2loop,fakev1}
  \fmffreeze
%  \fmf{phantom,tension=1.5}{hloop,fakev2}
%  \fmf{phantom, label=$t,,\bar{t}$, l.side=left}{fakev1,fakev2}
%  \fmf{phantom, label=$b,,\bar{b}$, l.side=right}{fakev1,fakev2}
  \fmflabel{\gluon}{g1}
  \fmflabel{\gluon}{g2}
\end{fmfchar*}
\end{fmffile}



\subsection{Désintégration de bosons de Higgs}\label{chapter-MS-MSSM-section-pheno_Higgs_MSSM-subsec-desintegration_Higgs}

\begin{fmffile}{H-tautau_small}\fmfstraight
\begin{fmfchar*}(40,30)
  \fmfleft{h}
  \fmfright{tau1,tau2}
  \fmf{dashes, label=$\Hs,, \Hn,, \Ha$, l.side=left}{h,v}
  \fmf{fermion, label=$\tau^+$, l.side=left}{tau1,v}
  \fmf{fermion, label=$\tau^-$, l.side=left}{v,tau2}
  \fmfdot{v}
\end{fmfchar*}
\end{fmffile}

\subsection{Désintégration des leptons tau}\label{chapter-MS-MSSM-section-pheno_Higgs_MSSM-subsec-desintegration_lepton_tau}
\emph{The branching fractions for decays into five or more charged hadrons are negligible. The lifetime of the tau lepton amounts to \SI{290}{fs}, corresponding to $c\tau = \SI{87}{\micro m}$.}

\begin{fmffile}{tau_to_ele}%\fmfstraight
\begin{fmfchar*}(30,20)
  \fmfleft{taui}
  \fmfright{l1,l2,nuout}
  \fmf{fermion, label=$\leptau$, l.side=left, tension=2}{taui,v1}
  \fmf{fermion}{v1,nuout}
  \fmf{phantom}{v1,l1}
  \fmffreeze
  \fmflabel{$\nutau$}{nuout}
  \fmf{boson, label=$\Wbosonminus$, l.side=right, tension=2}{v1,v2}
  \fmf{fermion}{l2,v2,l1}
  \fmflabel{$\electron$}{l1}
  \fmflabel{$\antinuele$}{l2}
  \fmfdot{v1,v2}
\end{fmfchar*}
\end{fmffile}


\begin{fmffile}{tau_to_mu}%\fmfstraight
\begin{fmfchar*}(30,20)
  \fmfleft{taui}
  \fmfright{l1,l2,nuout}
  \fmf{fermion, tension=2}{taui,v1}
  \fmf{fermion}{v1,nuout}
  \fmf{phantom}{v1,l1}
  \fmffreeze
  \fmflabel{$\nutau$}{nuout}
  \fmf{boson, label=$\Wbosonminus$, l.side=right, tension=2}{v1,v2}
  \fmf{fermion}{l2,v2,l1}
  \fmflabel{$\muon$}{l1}
  \fmflabel{$\antinumu$}{l2}
  \fmflabel{\leptau}{taui}
  \fmfdot{v1,v2}
\end{fmfchar*}
\end{fmffile}


\begin{fmffile}{tau_to_tauh_qqbar}%\fmfstraight
\begin{fmfchar*}(30,20)
  \fmfleft{taui}
  \fmfright{l1,l2,l3,f1,f2,f3,nuout}
  \fmf{fermion, tension=2}{taui,v1}
  \fmf{fermion}{v1,nuout}
  \fmf{phantom}{v1,l1}
  \fmffreeze
  \fmflabel{$\nutau$}{nuout}
  \fmf{boson, label=$\Wbosonminus$, l.side=right, tension=2}{v1,v2}
  \fmf{phantom}{v2,td1,l1}
  \fmf{phantom}{v2,td2,l2}
  \fmf{phantom}{v2,td3,l3}
  \fmffreeze
  \fmf{fermion}{l3,v2,l1}
  \fmflabel{\antiquark}{l3}
  \fmflabel{\quark}{l1}
  \fmflabel{\leptau}{taui}
  \fmfdot{v1,v2}
\end{fmfchar*}
\end{fmffile}


\input{\PhDthesisdir/tex/Feynman_diagrams/tau_decays/fgraph-tau_to_tauh-1prong.tex}

\begin{fmffile}{tau_to_tauh-3prongs}%\fmfstraight
\begin{fmfchar*}(30,20)
  \fmfleft{taui}
  \fmfright{l1,l2,l3,f1,f2,f3,nuout}
  \fmf{fermion, label=$\leptau$, l.side=left, tension=2}{taui,v1}
  \fmf{fermion}{v1,nuout}
  \fmf{phantom}{v1,l1}
  \fmffreeze
  \fmflabel{$\nutau$}{nuout}
  \fmf{boson, label=$\Wbosonminus$, l.side=right, tension=2}{v1,v2}
  \fmf{phantom}{v2,td1,l1}
  \fmf{phantom}{v2,td2,l2}
  \fmf{phantom}{v2,td3,l3}
  \fmffreeze
  \fmf{plain}{td3,v2,td1}
  \fmfblob{.15w}{td2}
  \fmf{plain}{td2,l1}
  \fmf{plain}{td2,l2}
  \fmf{plain}{td2,l3}
  \fmflabel{$\hadron^-$}{l1}
  \fmflabel{$\hadron^-$}{l2}
  \fmflabel{$\hadron^+$}{l3}
  \fmfdot{v1,v2}
\end{fmfchar*}
\end{fmffile}


\input{\PhDthesisdir/tex/Feynman_diagrams/tau_decays/fgraph-tau_to_ell.tex}

\begin{fmffile}{tau_to_tauh_small_beamer}%\fmfstraight
\begin{fmfchar*}(20,10)
  \fmfleft{taui}
  \fmfright{l1,l2,l3,f1,f2,f3,nuout}
  \fmf{fermion, tension=2}{taui,v1}
  \fmf{fermion}{v1,nuout}
  \fmf{phantom}{v1,l1}
  \fmffreeze
  \fmflabel{$\nutau$}{nuout}
  \fmf{boson, label=$\Wbosonminus$, l.side=right, tension=2}{v1,v2}
  \fmf{phantom}{v2,td1,l1}
  \fmf{phantom}{v2,td2,l2}
  \fmf{phantom}{v2,td3,l3}
  \fmffreeze
  \fmf{plain}{td3,v2,td1}
  \fmfblob{.15w}{td2}
  \fmf{plain}{td2,l2}
  \fmflabel{\tauhm}{l2}
  \fmflabel{\leptau}{taui}
  \fmfdot{v1,v2}
\end{fmfchar*}
\end{fmffile}



\section{Conclusion}\label{chapter-JERC-section-conclusion}
