\section{Au-delà du modèle standard}\label{chapter-MS-MSSM-section-BSM}
\subsection{La supersymétrie}\label{chapter-MS-MSSM-section-BSM-subsec-SUSY}
prometteur

candidat pour la matière noire

superpartenaires

sfermions et bosinos

Higgs mass quadratic divergence = naturalness
p. 233 of~\cite{Higgs_hunter_guide}
\subsection{Modèles à deux doublets de Higgs}\label{chapter-MS-MSSM-section-BSM-subsec-dbl_H_dbl}

\fullcite{Higgs_hunter_guide}

\fullcite{Higgs_hunter_guide_errata}

Dans le cadre du modèle standard, le secteur du Higgs comporte un doublet de Higgs complexe, défini par~\eqref{eq-chapter-MS-MSSM-section-formalisme-subsec-Higgs_mechanism-SM_Higgs_doublet}.
Il existe alors un seul boson de Higgs, \higgs, boson scalaire neutre observé en 2012~\cite{ATLAS_Higgs_discovery,CMS_Higgs_discovery,CMS_Higgs_discovery_2013,ATLAS-CMS-Higgs_combined_1,ATLAS-CMS-Higgs_combined_2}.


% eq-chapter-MS-MSSM-section-formalisme-subsec-Higgs_mechanism-thetaW GLASHOW1961579

éviter FCNC:
\begin{itemize}
\item soit les masses des bosons de Higgs sont élevée, de l'ordre du \SI{}{\TeV}, supprimant ainsi suffisamment le FCNC pour rester dans les limites observées;
\item soit [11 from \cite{Higgs_hunter_guide} = Glashow and Weinberg Phys. Rev. D15 (1977) 1958] tous les fermions portant une même charge électrique ne sont couplés qu'à un seul doublet de Higgs au plus.
\end{itemize}

Les modèles à deux doublets de Higgs
\begin{itemize}
\item sont une extension du modèle standard ajoutant une nouvelle physique;
\item apportent le moins de nouveaux paramètres arbitraires possibles;
\item satisfont $m_{\Wboson} \simeq m_{\Zboson} \cos\theta_W$;
\item ne proposent pas de changement de saveur par courant neutre à l'arbre;
\end{itemize}
Ils sont nécessaires dans les modèles supersymétriques \og basse énergie \fg{}.


\begin{multline}
V(\phi_1,\phi_2)
= \lambda_1 (\phi_1^\dagger\phi_1-v_1^2)
+ \lambda_2 (\phi_2^\dagger\phi_2-v_2^2)
\\
+ \lambda_3 \left[ (\phi_1^\dagger\phi_1-v_1^2) + (\phi_2^\dagger\phi_2-v_2^2) \right]^2
+ \lambda_4 \left[ (\phi_1^\dagger\phi_1)(\phi_2^\dagger\phi_2) - (\phi_1^\dagger\phi_2)(\phi_2^\dagger\phi_1) \right]
\\
+ \lambda_5 \left[ \Re(\phi_1^\dagger\phi_2) -v_1v_2\cos\xi \right]^2
+ \lambda_6 \left[ \Im(\phi_1^\dagger\phi_2) -v_1v_2\sin\xi \right]^2
\\
+ \lambda_7 \left[ \Re(\phi_1^\dagger\phi_2) -v_1v_2\cos\xi \right]\left[ \Im(\phi_1^\dagger\phi_2) -v_1v_2\sin\xi \right]
\end{multline}
où le dernier terme peut être éliminé en redéfinissant les phases des champs scalaires~\cite{Higgs_hunter_guide_errata}.

$\lambda_i\in\mathbb{R}_+$

$\lambda_5=\lambda_6$ for SUSY

$CP$ violation si $\sin\xi\neq0$

\begin{equation}
\tan\beta = \frac{v_2}{v_1}
\end{equation}
avec $0\leq\beta\leq\pi/2$.

\begin{equation}
v^2 = v_1^2+v_2^2
\end{equation}

\begin{align}
&
\text{Higgs chargés:}
&
\Higgspm &= - \phi_1^\pm \sin\beta + \phi_2^\pm \cos\beta
\msep&
m_{\Higgspm} &= \lambda_4 v
\mend[,]
\\
&
\text{Higgs pseudo-scalaire:}
&
\HiggsA &= \sqrt{2}\left(-\Im(\phi_1^0)\sin\beta+\Im(\phi_2^0)\cos\beta\right)
\msep&
m_{\HiggsA} &= \lambda_6 v
\mend[,]
\end{align}
ainsi que deux bosons de Higgs neutres scalaires dont les champs quantiques sont mélangés par la matrice
\begin{equation}
\mathcal{M} = \begin{pmatrix}
4v_1^2 (\lambda_1+\lambda_3) + v_2^2\lambda_5 & (4\lambda_3+\lambda_5)v_1v_2 \\
(4\lambda_3+\lambda_5)v_1v_2 & 4v_2^2 (\lambda_2+\lambda_3) + v_1^2\lambda_5
\end{pmatrix}
\mend
\end{equation}
Ces deux bosons de Higgs sont
\begin{align}
\higgs &= \sqrt{2}\left(-\Re(\phi_1^0-v_1)\sin\alpha+\Re(\phi_2^0-v_2)\cos\alpha\right)
\mend[,]
\\
\Higgs &= \sqrt{2}\left(\Re(\phi_1^0-v_1)\cos\alpha+\Re(\phi_2^0-v_2)\sin\alpha\right)
\mend[,]
\end{align}
où l'angle de mixage $\alpha$ s'obtient par
\begin{equation}
\sin 2\alpha = \frac{2\mathcal{M}_{12}}{\sqrt{(\mathcal{M}_{11}-\mathcal{M}_{22})^2+4\mathcal{M}_{12}^2}}
\msep
\cos 2\alpha = \frac{\mathcal{M}_{11}-\mathcal{M}_{22}}{\sqrt{(\mathcal{M}_{11}-\mathcal{M}_{22})^2+4\mathcal{M}_{12}^2}}
\end{equation}
avec $-\pi/2\leq\alpha\leq0$
et
dont les masses s'expriment, en considérant $m_{\higgs} \leq m_{\Higgs}$,
\begin{equation}
m_{\higgs,\Higgs}^2 = \frac{1}{2} \left( \mathcal{M}_{11} + \mathcal{M}_{22} \mp \sqrt{(\mathcal{M}_{11}-\mathcal{M}_{22})^2+4\mathcal{M}_{12}^2} \right)
\mend
\end{equation}

6 paramètres libres:
\begin{itemize}
\item les masses des bosons de Higgs: $m_{\higgs}$, $m_{\Higgs}$, $m_{\HiggsA}$, $m_{\Higgspm}$;
\item $\tan\beta$;
\item $\alpha$ l'angle de mixage des Higgs.
\end{itemize}

Modèle de type~I  = quarks et leptons ne sont pas couplés à $\phi_1$, mais le sont à $\phi_2$.
Modèle de type~II = down/up-type fermions : $\phi_2$ pour up-type

\begin{table}[H]
\centering
\begin{tabular}{rccc}
\toprule
Couplage avec & \higgs & \Higgs & \HiggsA \\
\midrule
Bosons vecteurs & $\sim\sin(\beta-\alpha)$ & $\sim\cos(\beta-\alpha)$ & $0$\\
Fermions hauts & $\displaystyle \sim\frac{\cos\alpha}{\sin\beta}$ & $\displaystyle \sim\frac{\sin\alpha}{\sin\beta}$ & $\sim\cot\beta$ \\
Fermions bas & $\displaystyle \sim\frac{-\sin\alpha}{\cos\beta}$ & $\displaystyle \sim\frac{\cos\alpha}{\cos\beta}$ & $\sim\tan\beta$ \\
\bottomrule
\end{tabular}
\caption[Couplages des bosons de Higgs neutres.]{Couplages des bosons de Higgs neutres des modèles de type~II par rapport aux couplages du boson de Higgs du modèle standard~\cite{Higgs_hunter_guide}.}
\end{table}


\subsection{L'extension supersymétrique minimale du modèle standard ou MSSM}\label{chapter-MS-MSSM-section-BSM-subsec-MSSM}
MSSM = modèle à deux doublets de Higgs


\begin{equation}
\Higgs_u
=
\begin{pmatrix}
{\phi_1^0}^* \\ -\phi_1^-
\end{pmatrix}
\msep
\Higgs_d
=
\begin{pmatrix}
\phi_2^+ \\ \phi_2^0
\end{pmatrix}
\mend
\label{eq-chapter-MS-MSSM-section-BSM-Higgs_doublets}
\end{equation}
\begin{equation}
\average{\Higgs_u} = \frac{1}{\sqrt{2}} \begin{pmatrix}v_1 \\ 0 \end{pmatrix}
\msep
\average{\Higgs_d} = \frac{1}{\sqrt{2}} \begin{pmatrix} 0\\ v_2 \end{pmatrix}
\mend
\end{equation}


\begin{align}
V(\Higgs_u,\Higgs_d)
&
=
\mu_u^2\Higgs_u^\dagger\Higgs_u
+
\mu_d^2\Higgs_d^\dagger\Higgs_d
-
\mu^2 (\Higgs_u\wedge\Higgs_d+\text{h.c.})
\\
&\hphantom{=}
+
\frac{g_I^2+g_Y^2}{8} (\Higgs_d^\dagger\Higgs_d-\Higgs_u^\dagger\Higgs_u)
+
\frac{g_I^2}{2} (\Higgs_d^\dagger\Higgs_u)^2
\mend
\end{align}

\begin{subequations}
\begin{align}
\lambda_2 &= \lambda_1\\
\lambda_3 &= \frac{1}{8} (g_I^2+g_Y^2)-\lambda_1\\
\lambda_4 &= 2\lambda_1 - \frac{1}{2} g_Y^2\\
\lambda_5 &= \lambda_6 = 2\lambda_1 - \frac{1}{2}(g_I^2+g_Y^2)\\
\mu_u^2 &= 2 \lambda_1 v_2^2 - \frac{1}{2} m_{\Zboson}^2\\
\mu_d^2 &= 2 \lambda_1 v_1^2 - \frac{1}{2} m_{\Zboson}^2\\
\mu^2 &= -\frac{1}{2}v_1v_2(g_I^2+g_Y^2-4\lambda_1)
\end{align}
\end{subequations}


\begin{align}
m_{\HiggsA}^2 &= \mu^2 (\tan\beta+\cot\beta) = \frac{2\mu^2}{\sin 2\beta}
\mend[,]\\
m_{\Higgspm}^2 &= m_{\HiggsA}^2+m_{\Wboson}^2
\mend[,]\\
m_{\higgs,\Higgs}^2 &= \frac{1}{2} \left( m_{\HiggsA}^2+m_{\Zboson}^2 \mp \sqrt{(m_{\HiggsA}^2+m_{\Zboson}^2)^2-4m_{\HiggsA}^2m_{\Zboson}^2\cos^2 2\beta} \right)
\mend
\end{align}

\begin{equation}
\cos 2\alpha = - \frac{m_{\HiggsA}^2-m_{\Zboson}^2}{m_{\Higgs}^2-m_{\higgs}^2} \cos 2\beta
\msep
\sin 2\alpha = - \frac{m_{\Higgs}^2+m_{\higgs}^2}{m_{\Higgs}^2-m_{\higgs}^2} \sin 2\beta
\end{equation}
\begin{equation}
\tan 2\alpha = \frac{m_{\HiggsA}^2+m_{\Zboson}^2}{m_{\HiggsA}^2-m_{\Zboson}^2} \tan 2\beta
\end{equation}

\begin{equation}
m_{\Wboson} = \frac{1}{2} v g_I
\msep
m_{\Zboson} = \frac{\mu_d^2\mu_u^2 \, \tan^2\beta}{\tan^2\beta-1}
\mend
\end{equation}


