\section{Au-delà du modèle standard}\label{chapter-MS-MSSM-section-BSM}
Le modèle standard souffre ainsi de lacunes malgré ses prédictions précises.
Des modèles sont développés afin de combler ces lacunes, ils sont dits \og au-delà \fg{} du modèle standard, ou BSM (\emph{Beyond Standard Model}).
Nous développerons ici le cas de l'extension supersymétrique minimale du modèle standard, ou MSSM, cas particulier d'un modèle de supersymétrie, ou SUSY.
\subsection{La supersymétrie}\label{chapter-MS-MSSM-section-BSM-subsec-SUSY}
La supersymétrie (SUSY)~\cite{MARTIN_1998} introduit une nouvelle symétrie entre fermions et bosons.
Ces deux types de particules ne sont plus indépendants, ce sont des saveurs, ou manifestations, d'un champ quantique plus complexe.
À chaque particule du modèle standard correspond alors une nouvelle particule du fait de cette symétrie, nommée \og superpartenaire \fg.
Les fermions du modèle standard ont des superpartenaires de spin entier, \ie\ des bosons, les \og sfermions \fg,
tandis que
les bosons du modèle standard ont des superpartenaires de spin demi-entier, \ie\ des fermions, les \og bosinos \fg.
Chaque particule et son superpartenaire ont les mêmes nombres quantiques à l'exception de leurs spins.
\par De nouvelles interactions sont potentiellement possibles avec la SUSY dans lesquelles les nombres baryonique $B$ et leptonique $L$ ne sont pas conservés et $B-L$ non plus.
Or, ce type d'interactions rendent le proton instable, ce qui n'est pas observé expérimentalement.
Une nouvelle symétrie est ainsi introduite afin de supprimer ces interactions violant la conservation de $B-L$, la parité $R$.
L'opérateur de parité $R$ est ainsi défini comme
\begin{equation}
P_R = (-1)^{3(B-L)-2s}
\end{equation}
où $s$ correspond au spin de la particule.
Les particules du modèle standard possèdent une parité $R$ égale à $1$, leurs superpartenaires une parité $R$ égale à $-1$.
La conservation de cette nouvelle parité permet non seulement de garder le proton stable, mais rend également stable la particule supersymétrique de plus basse masse, notée LSP (\emph{Lightest Supersymmetric Particle}).
\par La SUSY est un des modèles BSM les plus prometteurs.
Ce type de modèle permet en effet de résoudre, s'il est confirmé expérimentalement, de nombreuses lacunes du modèle standard.
Les trois forces fondamentales décrites par le modèle standard pourraient être unifiées grâce à ce modèle.
Comme nous l'avons vu dans la section~\ref{chapter-MS-MSSM-section-formalisme}, les forces électromagnétique et faible sont déjà unifiées.
Toutefois, la force électrofaible et la force forte ne semblent pas s'unifier à haute énergie.
Or, les interactions avec les superpartenaires introduits par la SUSY modifient le comportement des constantes de couplages des trois forces fondamentales de manière à les unifier à haute énergie.
La SUSY propose également un candidat pour la matière noire dans le cas où la LSP est de charge électrique nulle, potentiellement un neutralino ou un sneutrino.
De plus, la SUSY permet de résoudre le problème de l'ajustement fin.
La divergence quadratique de la masse du Higgs est naturellement supprimée par les diagrammes à boucles des superpartenaires dont les contributions ont des signes opposées à celles des particules, les fermions ayant des contributions positives et les bosons des contributions négatives~\cite{Higgs_hunter_guide}.
\par Cependant, l'ajout d'un superpartenaire fermionique au boson de Higgs du modèle standard, unique, apporte une anomalie chirale au modèle.
En d'autres termes, un courant chiral n'est plus conservé et peut alors générer des états de basse masse.
De tels états n'ayant pas été observés expérimentalement, il doit exister une suppression d'un tel mécanisme.
De plus, la suppression du changement de saveur par courant neutre, ou FCNC (\emph{Flavor-Changing Neutral Currents}), n'est pas garantie.
Or, le FCNC n'est pas observé expérimentalement non plus.
Afin d'éviter de telles incohérences avec les observations, la SUSY nécessite l'introduction d'un second doublet de Higgs et est donc un cas particulier de modèle à deux doublets de Higgs.
\subsection{Modèles à deux doublets de Higgs}\label{chapter-MS-MSSM-section-BSM-subsec-dbl_H_dbl}
%Dans le cadre du modèle standard, le secteur du Higgs comporte un doublet de Higgs complexe, défini par~\eqref{eq-chapter-MS-MSSM-section-formalisme-subsec-Higgs_mechanism-SM_Higgs_doublet}.
%Il existe alors un seul boson de Higgs, \higgs, boson scalaire neutre observé en 2012~\cite{ATLAS_Higgs_discovery,CMS_Higgs_discovery,CMS_Higgs_discovery_2013,ATLAS-CMS-Higgs_combined_1,ATLAS-CMS-Higgs_combined_2}.
\par Les modèles à deux doublets de Higgs (2HDM) introduisent un second doublet de Higgs.
Ainsi, au lieu d'avoir uniquement le doublet $\phi$ défini par~\eqref{eq-chapter-MS-MSSM-section-formalisme-subsec-Higgs_mechanism-SM_Higgs_doublet}, il existe $\phi_1$ et $\phi_2$.
Le potentiel de Higgs~\eqref{eq-chapter-MS-MSSM-section-formalisme-subsec-Higgs_mechanism-SM_Higgs_potential} du modèle standard est remplacé par
\begin{multline}
V(\phi_1,\phi_2)
= \lambda_1 (\phi_1^\dagger\phi_1-v_1^2)
+ \lambda_2 (\phi_2^\dagger\phi_2-v_2^2)
\\
+ \lambda_3 \left[ (\phi_1^\dagger\phi_1-v_1^2) + (\phi_2^\dagger\phi_2-v_2^2) \right]^2
+ \lambda_4 \left[ (\phi_1^\dagger\phi_1)(\phi_2^\dagger\phi_2) - (\phi_1^\dagger\phi_2)(\phi_2^\dagger\phi_1) \right]
\\
+ \lambda_5 \left[ \Re(\phi_1^\dagger\phi_2) -v_1v_2\cos\xi \right]^2
+ \lambda_6 \left[ \Im(\phi_1^\dagger\phi_2) -v_1v_2\sin\xi \right]^2
\\
+ \lambda_7 \left[ \Re(\phi_1^\dagger\phi_2) -v_1v_2\cos\xi \right]\left[ \Im(\phi_1^\dagger\phi_2) -v_1v_2\sin\xi \right]
\label{eq-chapter-MS-MSSM-section-BSM-subsec-dbl_H_dbl-Higgs_potential}
\end{multline}
où $v_1$ et $v_2$ sont les minimums de potentiel des deux doublets.
Le dernier terme peut être éliminé en redéfinissant les phases des champs scalaires~\cite{Higgs_hunter_guide,Higgs_hunter_guide_errata}.
Les paramètres $\lambda_i$ sont réels et dans le cas de la SUSY, $\lambda_5=\lambda_6$.
Dans le cas $\sin\xi\neq0$, le secteur de Higgs du modèle viole la symétrie $CP$.
\par Il est possible de définir, à ce stade, une variable importante dans la suite, le rapport des condensats des doublets de Higgs dans le vide,
\begin{equation}
\tan\beta = \frac{\average{\phi_2}_0}{\average{\phi_1}_0} = \frac{v_2}{v_1}
\end{equation}
avec $0\leq\beta\leq\pi/2$.
Il est aussi possible de définir
\begin{equation}
v^2 = v_1^2+v_2^2
\end{equation}
afin d'alléger les notations dans la suite.
\par De ce formalisme découle l'existence de cinq bosons de Higgs massifs,
\begin{align}
&
\text{deux Higgs chargés:}
&
\Higgspm &= - \phi_1^\pm \sin\beta + \phi_2^\pm \cos\beta
\msep&
m_{\Higgspm} &= \lambda_4 v
\mend[,]
\\
&
\text{un Higgs pseudo-scalaire:}
&
\HiggsA &= \sqrt{2}\left(-\Im(\phi_1^0)\sin\beta+\Im(\phi_2^0)\cos\beta\right)
\msep&
m_{\HiggsA} &= \lambda_6 v
\mend[,]
\end{align}
ainsi que deux bosons de Higgs scalaires neutres dont les champs quantiques sont mélangés par la matrice
\begin{equation}
\mathcal{M} = \begin{pmatrix}
4v_1^2 (\lambda_1+\lambda_3) + v_2^2\lambda_5 & (4\lambda_3+\lambda_5)v_1v_2 \\
(4\lambda_3+\lambda_5)v_1v_2 & 4v_2^2 (\lambda_2+\lambda_3) + v_1^2\lambda_5
\end{pmatrix}
\mend
\end{equation}
Ces deux bosons de Higgs sont
\begin{align}
\higgs &= \sqrt{2}\left(-\Re(\phi_1^0-v_1)\sin\alpha+\Re(\phi_2^0-v_2)\cos\alpha\right)
\mend[,]
\\
\Higgs &= \sqrt{2}\left(\Re(\phi_1^0-v_1)\cos\alpha+\Re(\phi_2^0-v_2)\sin\alpha\right)
\mend[,]
\end{align}
où l'angle de mixage $\alpha$ s'obtient par
\begin{equation}
\sin 2\alpha = \frac{2\mathcal{M}_{12}}{\sqrt{(\mathcal{M}_{11}-\mathcal{M}_{22})^2+4\mathcal{M}_{12}^2}}
\msep
\cos 2\alpha = \frac{\mathcal{M}_{11}-\mathcal{M}_{22}}{\sqrt{(\mathcal{M}_{11}-\mathcal{M}_{22})^2+4\mathcal{M}_{12}^2}}
\end{equation}
avec $-\pi/2\leq\alpha\leq0$
et
dont les masses s'expriment, en considérant $m_{\higgs} \leq m_{\Higgs}$,
\begin{equation}
m_{\higgs,\Higgs}^2 = \frac{1}{2} \left( \mathcal{M}_{11} + \mathcal{M}_{22} \mp \sqrt{(\mathcal{M}_{11}-\mathcal{M}_{22})^2+4\mathcal{M}_{12}^2} \right)
\end{equation}
où l'on considère que \higgs\ est le boson de Higgs observé expérimentalement en 2012~\cite{ATLAS_Higgs_discovery,CMS_Higgs_discovery,CMS_Higgs_discovery_2013,ATLAS-CMS-Higgs_combined_1,ATLAS-CMS-Higgs_combined_2} et que les quatre bosons de Higgs supplémentaires sont plus massifs, ce qui semble être favorisé par les observations actuelles~\cite{ATLAS-CMS-Higgs_combined_1,ATLAS-CMS-Higgs_combined_2,CMS-MSSM-HTT_2014}.
\par Le 2HDM ainsi construit possède 6 paramètres libres:
\begin{itemize}
\item $m_{\higgs}$, $m_{\Higgs}$, $m_{\HiggsA}$, $m_{\Higgspm}$ les masses des bosons de Higgs;
\item $\tan\beta$ le rapport des condensats des doublets de Higgs dans le vide;
\item $\alpha$ l'angle de mixage des Higgs.
\end{itemize}
\par Afin d'être compatible avec l'absence d'observation de FCNC~\cite{Higgs_hunter_guide},
\begin{itemize}
\item soit les masses des bosons de Higgs sont élevées, de l'ordre du \SI{}{\TeV}, supprimant ainsi suffisamment le FCNC pour rester dans les limites observées;
\item soit tous les fermions portant une même charge électrique ne sont couplés qu'à un seul doublet de Higgs au plus.
\end{itemize}
Or, la masse du Higgs du modèle standard n'est pas de l'ordre du \SI{}{\TeV}; la seconde option est donc celle à suivre.
\par Dans le cas des modèles de type~II\footnote{Dans les modèle de type~I, les fermions ne sont pas couplés à $\phi_1$, mais le sont à $\phi_2$.}, les fermions d'isospin faible bas sont ainsi couplés à $\phi_1$ et ceux d'isospin faible haut à $\phi_2$.
Les intensités de ces couplages, par rapport aux couplages du boson de Higgs du modèle standard à ces mêmes particules, sont présentés dans le tableau~\ref{tab-Higgs_couplings_2HDM}. En particulier, le couplage du boson de Higgs du modèle standard, \higgs, est modifié et une mesure précise de ses couplages aux autres particules est un test de ce type de modèles.
\begin{table}[H]
\centering
\begin{tabular}{rccc}
\toprule
Couplage avec & \higgs & \Higgs & \HiggsA \\
\midrule
Bosons vecteurs & $\sin(\beta-\alpha)$ & $\cos(\beta-\alpha)$ & $0$\\
Fermions hauts & $\displaystyle \frac{\cos\alpha}{\sin\beta}$ & $\displaystyle \frac{\sin\alpha}{\sin\beta}$ & $\cot\beta$ \\
Fermions bas & $\displaystyle \frac{-\sin\alpha}{\cos\beta}$ & $\displaystyle \frac{\cos\alpha}{\cos\beta}$ & $\tan\beta$ \\
\bottomrule
\end{tabular}
\caption[Couplages des bosons de Higgs neutres.]{Couplages des bosons de Higgs neutres des modèles de type~II par rapport aux couplages du boson de Higgs du modèle standard~\cite{Higgs_hunter_guide}.}
\label{tab-Higgs_couplings_2HDM}
\end{table}
\par Les modèles à deux doublets de Higgs sont donc une extension du modèle standard ajoutant une nouvelle physique, par exemple l'existence de nouveaux bosons de Higgs.
Ces modèles, par rapport à d'autres possibilités explorées, apportent le moins de nouveaux paramètres arbitraires possibles, ce qui est un critère important dans l'élaboration d'une nouvelle théorie.
Enfin, ils doivent être introduits dans les modèles supersymétriques pour que ceux-ci respectent les observations expérimentales.
Nous avons à présent le formalisme nécessaire pour discuter de l'extension supersymétrique minimale du modèle standard.
\subsection{L'extension supersymétrique minimale du modèle standard}\label{chapter-MS-MSSM-section-BSM-subsec-MSSM}
L'extension supersymétrique minimale du modèle standard ou MSSM
est un modèle supersymétrique et donc, à ce titre, un cas particulier de modèle à deux doublets de Higgs.
Il s'agit du modèle le plus simple permettant d'introduire la supersymétrie tout en étant compatible avec les observations expérimentales à ce jour.
Dans le MSSM, les deux doublets de Higgs s'expriment en fonction de $\phi_1$ et $\phi_2$ introduits dans la section traitant de la supersymétrie comme
\begin{equation}
\Higgs_u
=
\begin{pmatrix}
{\phi_1^0}^* \\ -\phi_1^-
\end{pmatrix}
\msep
\Higgs_d
=
\begin{pmatrix}
\phi_2^+ \\ \phi_2^0
\end{pmatrix}
\mend
\label{eq-chapter-MS-MSSM-section-BSM-Higgs_doublets}
\end{equation}
Leurs condensats dans le vide sont alors
\begin{equation}
\average{\Higgs_u}_0 = \frac{1}{\sqrt{2}} \begin{pmatrix}v_1 \\ 0 \end{pmatrix}
\msep
\average{\Higgs_d}_0 = \frac{1}{\sqrt{2}} \begin{pmatrix} 0\\ v_2 \end{pmatrix}
\mend
\end{equation}
L'expression du potentiel de Higgs~\eqref{eq-chapter-MS-MSSM-section-BSM-subsec-dbl_H_dbl-Higgs_potential} se simplifie en
\begin{align}
V(\Higgs_u,\Higgs_d)
&
=
\mu_u^2\Higgs_u^\dagger\Higgs_u
+
\mu_d^2\Higgs_d^\dagger\Higgs_d
-
\mu^2 (\Higgs_u\wedge\Higgs_d+\text{h.c.})
\nonumber\\
&\hphantom{=}
+
\frac{g_I^2+g_Y^2}{8} (\Higgs_d^\dagger\Higgs_d-\Higgs_u^\dagger\Higgs_u)
+
\frac{g_I^2}{2} (\Higgs_d^\dagger\Higgs_u)^2
\mend
\end{align}
en posant
\vspace{-.5\baselineskip}\par\noindent
\begin{subequations}
\noindent
\begin{minipage}[c]{.49\textwidth}
\begin{align}
\lambda_2 &= \lambda_1 \mend[,]\\
\lambda_3 &= \frac{1}{8} (g_I^2+g_Y^2)-\lambda_1 \mend[,]\\
\lambda_4 &= 2\lambda_1 - \frac{1}{2} g_Y^2 \mend[,]\\
\lambda_5 &= \lambda_6 = 2\lambda_1 - \frac{1}{2}(g_I^2+g_Y^2) \mend[,]
\end{align}
\end{minipage}
\hfill
\begin{minipage}[c]{.49\textwidth}
\begin{align}
\mu_u^2 &= 2 \lambda_1 v_2^2 - \frac{1}{2} m_{\Zboson}^2 \mend[,]\\
\mu_d^2 &= 2 \lambda_1 v_1^2 - \frac{1}{2} m_{\Zboson}^2 \mend[,]\\
\mu^2 &= -\frac{1}{2}v_1v_2(g_I^2+g_Y^2-4\lambda_1)
\mend
\end{align}
\end{minipage}
\end{subequations}
\vspace{.5\baselineskip}\par\noindent
Alors, les masses des bosons de Higgs s'expriment
\begin{align}
m_{\HiggsA}^2 &= \mu^2 (\tan\beta+\cot\beta) = \frac{2\mu^2}{\sin 2\beta}
\mend[,]\\
m_{\Higgspm}^2 &= m_{\HiggsA}^2+m_{\Wboson}^2
\mend[,]\\
m_{\higgs,\Higgs}^2 &= \frac{1}{2} \left( m_{\HiggsA}^2+m_{\Zboson}^2 \mp \sqrt{(m_{\HiggsA}^2+m_{\Zboson}^2)^2-4m_{\HiggsA}^2m_{\Zboson}^2\cos^2 2\beta} \right)
\mend[,]
\end{align}
et l'angle de mixage des Higgs scalaires neutres vérifie
\begin{equation}
\cos 2\alpha = - \frac{m_{\HiggsA}^2-m_{\Zboson}^2}{m_{\Higgs}^2-m_{\higgs}^2} \cos 2\beta
\msep
\sin 2\alpha = - \frac{m_{\Higgs}^2+m_{\higgs}^2}{m_{\Higgs}^2-m_{\higgs}^2} \sin 2\beta
\mend
\end{equation}
Enfin, les masses des bosons de l'interaction faible sont à présent
\begin{equation}
m_{\Wboson} = \frac{1}{2} v g_I
\msep
m_{\Zboson} = \frac{\mu_d^2\mu_u^2 \, \tan^2\beta}{\tan^2\beta-1}
\mend
\end{equation}
\par Les particules du MSSM et leur superpartenaires sont résumés dans le tableau~\ref{tab-ptcs_and_superpartners}.
Un test expérimental est possible par la recherche d'un signal correspondant aux bosons de Higgs supplémentaire, ce qui est un des sujets de cette thèse.
L'étude de la phénoménologie de ces bosons de Higgs, présentée ci-après, nous permet de déterminer les conditions favorables à la recherche d'un tel signal.
\begin{table}
\centering
\begin{tabular}{cccccccc}
\toprule
\multicolumn{4}{c}{Particules} & \multicolumn{4}{c}{Superpartenaires}\\
\cmidrule(lr){1-4}\cmidrule(lr){5-8}
Type & Spin & Particules & Symboles & Type & Spin & Particules & Symboles \\
\midrule
\multirow{2}{*}{Fermions} & \multirow{2}{*}{$\dfrac{1}{2}$} & quarks & \quark &
\multirow{2}{*}{Sfermions} & \multirow{2}{*}{$0$} & squarks & \squark \\
 &  & leptons & $\ell$ &
 &  & sleptons & $\tilde{\ell}$ \\
\cmidrule(lr){1-8}
\multirow{5}{*}{Bosons} & \multirow{4}{*}{$1$} & gluon & \gluon &
\multirow{5}{*}{Bosinos} & \multirow{5}{*}{$\dfrac{1}{2}$} & gluino & \gluino \\
 & & bosons \Wbosonpm & \Wbosonplus, \Wbosonminus &
 & & winos & \sWbosonplus, \sWbosonminus \\
 & & photon & \photon &
 & & photino & \photino \\
 & & boson \Zboson & \Zboson &
 & & zino & \sZboson \\
 & $0$ & Higgs & \higgs, \Higgs, \HiggsA, \Higgspm &
 & & Higgsinos & $\tilde{h}$, $\tilde{H}$, $\tilde{A}$, $\tilde{H}^\pm$  \\
\bottomrule
\end{tabular}
\caption[Particules et leurs superpartenaires.]{Particules et leurs superpartenaires. La présence de plusieurs bosons de Higgs est justifiée par la nécessité d'un second doublet de Higgs. Ce formalisme est décrit dans la section~\ref{chapter-MS-MSSM-section-BSM-subsec-dbl_H_dbl}.}
\label{tab-ptcs_and_superpartners}
\end{table}