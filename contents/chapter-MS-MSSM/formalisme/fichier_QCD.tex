\subsection{Interaction forte}\label{chapter-MS-MSSM-section-formalisme-subsec-QCD}
\emph{Observation des baryons \Deltabaryonplusplus, \Deltabaryonminus, \Omegabaryonminus, tous de spin $3/2$. Interprétés dans le modèle des quarks comme}
\begin{equation}
\Deltabaryonplusplus = (\quarku\quarku\quarku)
\msep
\Deltabaryonminus = (\quarkd\quarkd\quarkd)
\msep
\Omegabaryonminus = (\quarks\quarks\quarks)
\mend
\end{equation}
\emph{À cause du principe de Pauli, pour que ces résonances existent, il faut postuler l'existence d'un nombre quantique supplémentaire, la couleur.}
