\subsection{Mécanisme de Higgs}\label{chapter-MS-MSSM-section-formalisme-subsec-Higgs_mechanism}
Afin de briser la symétrie électrofaible, introduisons un champ complexe, scalaire, massif, le \emph{champ de Higgs}, noté $\phi$. Il s'agit d'un champ à quatre composantes, qu'il est possible d'écrire sous la forme d'un doublet d'isospin faible,
\begin{equation}
\phi
=
\begin{pmatrix}
\phi^+ \\ \phi^0
\end{pmatrix}
=
\frac{1}{\sqrt{2}}
\begin{pmatrix}
\phi_3 + \im \phi_4 \\ \phi_1 + \im \phi_2
\end{pmatrix}
\mend
\end{equation}
\par Le champ de Higgs a pour hypercharge $Y=+1$ et pour isospin $I=\frac{1}{2}$. Ainsi, il se transforme selon, respectivement sous $U(1)_Y$ et $SU(2)_L$,
\begin{equation}
\begin{pmatrix}
\phi^+ \\ \phi^0
\end{pmatrix}
\to
\begin{pmatrix}
\eexp{\im\frac{\beta}{2}} & 0 \\ 0 & \eexp{\im\frac{\beta}{2}}
\end{pmatrix}
\begin{pmatrix}
\phi^+ \\ \phi^0
\end{pmatrix}
\msep
\begin{pmatrix}
\phi^+ \\ \phi^0
\end{pmatrix}
\to
\eexp{\frac{\im}{2}\bm{\tau}\cdot\bm{\alpha}}
\begin{pmatrix}
\phi^+ \\ \phi^0
\end{pmatrix}
\end{equation}
\par La dérivée covariante définie par l'équation~\eqref{eq-definition_Dmu_EW} agit donc sur le champ de Higgs selon
\begin{equation}
D_\mu \phi = \left[ \partial_\mu - \frac{\im}{2}g_W \bm{\tau}\cdot\bm{W}_\mu - \frac{\im}{2}g'B_\mu \right] \phi
\mend[,]
\end{equation}
et ce champ de Higgs apporte les termes $\Lcal_{\higgs}$ au lagrangien du modèle standard, où
\begin{equation}
\Lcal_{\higgs} = (D^\mu\phi)^\dagger(D_\mu\phi) - V(\phi)
\end{equation}
avec
\begin{equation}
V(\phi)
= \mu^2\phi^\dagger\phi + \lambda (\phi^\dagger\phi)^2
\mend
\end{equation}