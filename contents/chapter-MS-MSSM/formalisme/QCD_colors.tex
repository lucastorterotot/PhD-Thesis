\begin{tikzpicture}
\def\rcircle{1.5}
\def\overlappct{.6}

\fill [ltcolorred2] (90:\overlappct*\rcircle) circle (\rcircle);
\fill [ltcolorblue2] (90-120:\overlappct*\rcircle) circle (\rcircle);
\fill [ltcolorgreen2] (90+120:\overlappct*\rcircle) circle (\rcircle);

\begin{scope}
\clip (90:\overlappct*\rcircle) circle (\rcircle);
\clip (90+120:\overlappct*\rcircle) circle (\rcircle);
\fill [ltcoloryellow2] (90:\overlappct*\rcircle) circle (\rcircle);
\end{scope}

\begin{scope}
\clip (90:\overlappct*\rcircle) circle (\rcircle);
\clip (90-120:\overlappct*\rcircle) circle (\rcircle);
\fill [ltcolormagenta2] (90:\overlappct*\rcircle) circle (\rcircle);
\end{scope}

\begin{scope}
\clip (90-120:\overlappct*\rcircle) circle (\rcircle);
\clip (90+120:\overlappct*\rcircle) circle (\rcircle);
\fill [ltcolorcyan2] (90-120:\overlappct*\rcircle) circle (\rcircle);
\end{scope}

\begin{scope}
\clip (90:\overlappct*\rcircle) circle (\rcircle);
\clip (90-120:\overlappct*\rcircle) circle (\rcircle);
\clip (90+120:\overlappct*\rcircle) circle (\rcircle);
\fill [white] (90-120:\overlappct*\rcircle) circle (\rcircle);
\end{scope}

\draw (90:\overlappct*\rcircle+.5*\rcircle) node {rouge} ;
\draw (90-120:\overlappct*\rcircle+.5*\rcircle) node [rotate=60] {bleu} ;
\draw (90+120:\overlappct*\rcircle+.5*\rcircle) node [rotate=-60] {vert} ;

%\draw (-90:\overlappct*\rcircle+0*\rcircle) node {antirouge} ;
%\draw (-90-120:\overlappct*\rcircle+0*\rcircle) node [rotate=60] {antibleu} ;
%\draw (-90+120:\overlappct*\rcircle+0*\rcircle) node [rotate=-60] {antivert} ;

\draw (0,0) node {blanc};

\def\rcircle{1.5}
\def\overlappct{.6}

\draw (30:\overlappct*\rcircle)+(0, \rcircle) coordinate (top);
\draw (-150:\overlappct*\rcircle)+(0, -\rcircle) coordinate (bottom);
\end{tikzpicture}