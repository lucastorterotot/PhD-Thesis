\section{Formalisme théorique et interactions}
\begin{wrapfigure}{R}{7cm}
\centering
\begin{tikzpicture}
\draw [thick, domain=0:4,samples=100,smooth] coordinate (A) plot(\x, {\x/5*2+1-cos(\x/5*360)}) coordinate (B) ;
\fill (A) circle (1pt);
\fill (B) circle (1pt);
\draw (A) node [left] {$A$};
\draw (B) node [right] {$B$};
\draw [->] (-.7,-.2)--+(5.5,0) node [right] {$t$};
\draw [->] (-.7,-.2)--+(0,3.5) node [above] {$x$};

\draw (0,-.1)--+(0,-.2) node [below] {$0$};
\draw (4,-.1)--+(0,-.2) node [below] {$\tau$};
\end{tikzpicture}
\caption{Une particule se déplace au cours du temps d'un point $A$ à un point $B$ le long d'une dimension $x$.}
\label{fig-ptc_classique_ptA_to_pt_B}
\end{wrapfigure}
\subsection{Lagrangien et équation d'Euler-Lagrange}
%\paragraph{Introduction au lagrangien en mécanique Newtonienne}
Nous souhaitons ici trouver un moyen de décrire le comportement des particules, \ie\ leur évolution à travers le temps et l'espace.
Considérons, dans un premier temps, une particule de masse $m$, soumise à une force $F$, se déplaçant dans le temps le long d'une dimension $x$, d'un point $A$ à $t=0$ à un point $B$ à $t=\tau$, comme illustré sur la figure~\ref{fig-ptc_classique_ptA_to_pt_B}.
\par 
Comme cela est enseigné dès les premiers cours de physique, la trajectoire de cette particule peut être déterminée à l'aide du principe fondamental de la dynamique, ou seconde loi de Newton, qui s'exprime simplement dans ce cas sous la forme
\begin{equation}
m\dv[2]{x}{t} = F
\mend
\end{equation}
Nous obtenons alors la position de la particule à tout instant.
\begin{wrapfigure}{R}{7cm}
\centering
\begin{tikzpicture}
\draw [domain=0:4,samples=100,smooth] coordinate (A) plot(\x, {\x/5*2+1-cos(\x/5*360)}) coordinate (B) ;
\draw [dashed, domain=0:4,samples=10,smooth] plot(\x, {\x/5*2+1-cos(\x/5*360)+(rand)*\x*(4-\x)/5});
%\draw [dotted] (A) -- (B) ;
%\draw [dotted] (A) to[out=90, in=190] (B) ;
%\draw [dotted] (A) to[out=0, in=90] (B) ;
%\draw [dotted] (A) to[out=30, in=135] (B) ;
%\draw [dotted] (A) to[out=90, in=-90] (B) ;
\fill (A) circle (1pt);
\fill (B) circle (1pt);
\draw (A) node [left] {$A$};
\draw (B) node [right] {$B$};
\draw [->] (-.7,-.2)--+(5.5,0) node [right] {$t$};
\draw [->] (-.7,-.2)--+(0,3.5) node [above] {$x$};

\draw (0,-.1)--+(0,-.2) node [below] {$0$};
\draw (4,-.1)--+(0,-.2) node [below] {$\tau$};
\end{tikzpicture}
\caption{Variation de la trajectoire d'une particule se déplaçant au cours du temps d'un point $A$ à un point $B$.}
\label{fig-ptc_quantique_ptA_to_pt_B}
\end{wrapfigure}
\par Or, cette méthode ne permet pas de décrire le comportement des particules fondamentales. En effet, à leur échelle, la mécanique quantique prévaut et il n'est pas possible, lorsque l'on observe une particule à un point $A$ puis à un point $B$, de déterminer la trajectoire exacte suivie par cette particule.
La particule peut suivre la trajectoire déterminée avec la mécanique classique, \ie\ celle de la figure~\ref{fig-ptc_classique_ptA_to_pt_B}, comme toute autre trajectoire reliant $A$ à $B$, comme illustré sur la figure~\ref{fig-ptc_quantique_ptA_to_pt_B}.
\par Si le principe fondamental de la dynamique tel que formulé par Newton ne tient plus dans le contexte de la mécanique quantique, il existe un autre principe physique toujours en place, la conservation de l'énergie. Dans le cas de la particule précédemment décrit, il s'agit de la somme de son énergie cinétique $T$ et de son énergie potentielle $V$, \ie
\begin{equation}
E = T + V = \cste
\end{equation}
où $T$ dépend uniquement de la vitesse de la particule et $V$ uniquement de sa position. %Cette quantité est conservée au cours du temps.
Il en va ainsi de même pour les moyennes temporelles de ces grandeurs,
\begin{equation}
E = \average{E} = \average{T} + \average{V} = \cste
\end{equation}
avec
\begin{equation}
\average{T} = \frac{1}{\tau}\int_0^\tau T(\dot{x}(t)) \dd{t}
\msep
\average{V} = \frac{1}{\tau}\int_0^\tau V(x(t)) \dd{t}
\mend[,]
\end{equation}
où $\dot{x}=\dv{x}{t}$.
\par
Nous pouvons alors nous demander de quelle manière ces grandeurs sont modifiées lorsque la trajectoire suivie par la particule varie par rapport à la trajectoire déterminée par la mécanique Newtonienne. La variation de la valeur moyenne de l'énergie potentielle s'exprime
\begin{equation}
\fdv{\average{V}}{x(t')}
= \frac{1}{\tau}\int_0^\tau \fdv{V(x(t))}{x(t')} \dd{t}
= \frac{1}{\tau}\int_0^\tau \dv{V(x(t))}{x(t)}\delta(t-t') \dd{t}
= \frac{1}{\tau}\eval{\dv{V}{x(t)}}_{t=t'}
= -\frac{1}{\tau}F(x(t'))
\end{equation}
car la force $F$ est reliée au potentiel $V$ par $F = - \dv{V}{x}$.
De même, l'énergie cinétique moyenne varie selon
\begin{align}
\fdv{\average{T}}{x(t')}
&
= \frac{1}{\tau}\int_0^\tau \fdv{T(\dot{x}(t))}{x(t')} \dd{t}
=\frac{1}{\tau}\int_0^\tau \dv{T(\dot{x}(t))}{x(t)}\delta'(t-t') \dd{t}
=-\frac{1}{\tau}\int_0^\tau \delta(t-t') \dv{t}(\dv{T(\dot{x}(t))}{\dot{x}(t)}) \dd{t}
\nonumber
\\&
=-\frac{1}{\tau}\eval{\dv{t}(\dv{T(\dot{x}(t))}{\dot{x}(t)})}_{t=t'}
=-\frac{1}{\tau}m\eval{\dv[2]{x}{t}}_{t=t'}
\end{align}
car pour une particule de masse $m$, en mécanique newtonienne, $T = \frac{1}{2}m\left(\dv{x}{t}\right)^2$.
\par
Le long de la trajectoire classique, le principe fondamental de la dynamique est vérifié. Alors, les variations autour de la trajectoire classique sont reliées par
\begin{equation}
m\dv[2]{x}{t} = F
\Rightarrow
\fdv{\average{T}}{x(t')} = \fdv{\average{V}}{x(t')}
\Rightarrow
\fdv{x(t')}(\average{T}-\average{V}) = 0 \label{eq-fdv_x_T-V_is_0}
\mend
\end{equation}
Ainsi, la différence entre l'énergie cinétique et l'énergie potentielle du système étudié semble jouer un rôle particulier lorsque l'on s'intéresse aux différentes trajectoires possibles pour ce système. Définissons alors le lagrangien $L$ du système étudié comme
\begin{equation}
L = T-V
\mend
\end{equation}
L'intégrale au cours du temps du lagrangien est appelée action et est définie comme
\begin{equation}
S = \int_0^\tau\dd{t}L
\mend
\end{equation}
%Pour une particule quantique, toute trajectoire est possible pour aller d'un point $A$ à un point $B$. Afin de déterminer l'évolution au cours du temps d'une telle particule, il faut sommer toutes ces trajectoires, chacune étant associée à un facteur de phase
%\begin{equation}
%\exp(iS/\hbar)\mend
%\end{equation}
%Pour la plupart des trajectoires, ce facteur de phase donne des contributions qui s'annulent.
%\paragraph{Équation d'Euler-Lagrange}
Compte-tenu de l'équation~\eqref{eq-fdv_x_T-V_is_0}, l'action vérifie
\begin{equation}
\fdv{S}{x(t')}=0
\mend[,]
\end{equation}
ce qui est connu sous le nom de principe de moindre action. Or,
\begin{equation}
\fdv{S}{x(t')}
= \int_0^\tau\dd{t} \left[ \fdv{L}{x(t)}\delta(t-t') + \fdv{L}{\dot{x}(t)}\delta'(t-t')\right]
= \fdv{L}{x(t')} - \dv{t}\fdv{L}{\dot{x}(t')}
\mend[,]
\end{equation}
ce qui implique
\begin{equation}
\fdv{L}{x(t')} - \dv{t}\fdv{L}{\dot{x}(t')} = 0
\mend
\end{equation}
Cette équation est l'équation d'Euler-Lagrange et permet d'obtenir toutes les équations du mouvement du système, \ie\ de décrire son évolution au cours du temps.

\subsection{Lagrangien, champs et symétries}
Le modèle standard décrit le comportement des particules fondamentales à l'aide de la théorie quantique des champs. Une particule est ainsi une excitation d'un champ quantique relativiste $\phi$ et il s'agit alors de décrire l'évolution de ces excitations.
\par Généralisons le raisonnement précédent à un espace à une dimension temporelle et trois dimensions spatiales.
À partir du lagrangien, il est possible de définir la densité lagrangienne \Lcal\ telle que
\begin{equation}
L = \int\dd[3]{x}\Lcal
\msep
S = \int\dd[4]{x}\Lcal
\end{equation}
où $x$ désigne la coordonnée dans l'espace de Minkowski, \ie\ l'espace-temps à quatre dimensions.
Considérons maintenant une densité lagrangienne dépendant d'un champ $\phi(x)$ et de ses dérivées $\partial_\mu\phi(x)$.
Alors,
\begin{equation}
S = \int\dd[4]{x}\Lcal(\phi(x),\partial_\mu\phi(x))
\end{equation}
et du principe de moindre action résultent les équations d'Euler-Lagrange pour cette densité lagrangienne,
\begin{equation}
\fdv{S}{\phi}
=
\pdv{\Lcal}{\phi} - \partial_\mu\pdv{\Lcal}{(\partial_\mu\phi)} = 0
\mend
\end{equation}
Il s'agit à présent de déterminer la densité lagrangienne \Lcal\ du modèle standard.
Par la suite, nous nommerons la densité lagrangienne \Lcal\ \og lagrangien \fg{} dans un souci de praticité.
%Nous cherchons donc à déterminer le lagrangien du modèle standard.
\par Un champ quantique peut subir une transformation de jauge locale. Une telle transformation doit laisser la physique inchangée, ainsi le lagrangien du modèle standard est construit pour être invariant sous les transformations de jauges locales du groupe de symétrie
\begin{equation}
SU(3)_C \otimes SU(2)_L \otimes U(1)_Y
\mend
\end{equation}
De cette construction résultent les interactions fondamentales, discutées ci-après.

\subsection{Interaction électromagnétique}
\subsection{Interaction électrofaible}
\subsection{Mécanisme de Higgs}
\subsection{Interaction forte}
