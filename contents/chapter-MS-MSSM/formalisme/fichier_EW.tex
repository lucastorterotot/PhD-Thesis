\subsection{Interaction électrofaible}\label{chapter-MS-MSSM-section-formalisme-subsec-EW}
Le modèle standard décrit les interactions électromagnétiques et faible comme deux facettes d'une seule et même interaction qui les unifie, l'interaction électrofaible, notée \og EW \fg{} pour \emph{electroweak}.
\todo{brisure symétrie et Higgs avec réf}
\par Nous avons vu précédemment que l'interaction électromagnétique repose sur l'invariance de jauge sous les transformations locale du groupe $U(1)_{em}$.
Dans le cas de l'interaction électrofaible, ce groupe de symétrie est $SU(2)_L \times U(1)_Y$, où $L$ signifie \og left \fg{} car l'interaction faible ne couple que les fermions de chiralité\footnote{\todo{cf...}} gauche et les anti-fermions de chiralité droite et où $Y$ est l'\emph{hypercharge}.
\par L'interaction électrofaible introduit en effet plusieurs nouveaux nombres quantiques. Le premier est l'\emph{isospin faible}, $I$. Les fermions de chiralité gauche sont rassemblés en doublet d'isospin faible $I=\frac{1}{2}$, les fermions de chiralité droite en singlets d'isospin faible $I=0$. Ces derniers sont ainsi invariants sous les transformations de $SU(2)_L$.
Le second est l'\emph{hypercharge} $Y$, reliée à $Q$ la charge électrique et à $I_3$ la projection de l'isospin faible par la relation de Gell-Mann--Nishijima,
\begin{equation}
Y = 2(Q-I_3)
\mend
\end{equation}
\par Une différence notable entre $SU(2)$ et $U(1)$ est que $SU(2)$ est un groupe \emph{non abélien}. Cela signifie deux transformations successives $a$ et $b$ de ce groupe ne donnent pas le même résultat selon que l'on applique $a$ puis $b$ ou $b$ puis $a$.

\begin{table}[h]
\centering
\begin{tabular}{cccc}
\toprule
Champ & $\nuele$ & $\ele_L$ & $\ele_R$\\
\midrule
$Y$ & $-1$ & $-1$ & $-2$ \\
$I$ & $\frac{1}{2}$ & $\frac{1}{2}$ & $0$ \\
$I_3$ & $\frac{1}{2}$ & $-\frac{1}{2}$ & $0$ \\
\midrule
$Q$ & $0$ & $-1$ & $-1$\\
\bottomrule
\end{tabular}
\end{table}