\subsection{Interaction électrofaible}\label{chapter-MS-MSSM-section-formalisme-subsec-EW}
Le modèle standard décrit les interactions électromagnétique et faible comme deux facettes d'une seule et même interaction qui les unifie, l'interaction électrofaible, notée \og EW \fg{} pour \emph{electroweak}.
Celle-ci a été développée à partir des travaux de \citeauthor{Glashow_EW}~\cite{Glashow_EW}, \citeauthor{Salam_EW}~\cite{Salam_EW} et \citeauthor{Weinberg_leptons_model}~\cite{Weinberg_leptons_model}, récompensés par le prix Nobel de physique en 1979.
\par Une des raisons de l'unification de ces deux forces provient du calcul de la section efficace de production de paire $\Wbosonplus\Wbosonminus$.
Le lien entre la section efficace d'un processus physique et le nombre d'événements de ce processus est introduit dans le chapitre~\refChLHCCMS.
La section efficace ne peut pas être infinie, ce ne saurait correspondre à la réalité physique.
Or, sans considérer chacun des diagrammes de la figure~\ref{fig-fgraph-ff_WW} pour ce calcul,
la section efficace de production de paire $\Wbosonplus\Wbosonminus$ est infinie.
La prise en compte des interactions électromagnétique, figure~\ref{subfig-fgraph-ff_WW1},
et faible, figures~\ref{subfig-fgraph-ff_WW2} et~\ref{subfig-fgraph-ff_WW3},
pousse à les unifier.
\begin{figure}[h]
\centering
\vspace{\baselineskip}
\subcaptionbox{\label{subfig-fgraph-ff_WW1}}[.3\textwidth]
{\begin{fmffile}{ee_gamma_WW}\fmfstraight
\begin{fmfchar*}(20,20)
  \fmfleft{i1,i2}
  \fmfright{o1,o2}
  \fmf{fermion}{i1,v1,i2}
  \fmf{photon, label=\photon{}}{v1,v2}
  \fmf{boson}{o1,v2,o2}
  \fmflabel{\antiele}{i2}
  \fmflabel{\electron}{i1}
  \fmflabel{\Wbosonplus}{o2}
  \fmflabel{\Wbosonminus}{o1}
  \fmfdot{v1,v2}
\end{fmfchar*}
\end{fmffile}\vspace{\baselineskip}}
\hfill
\subcaptionbox{\label{subfig-fgraph-ff_WW2}}[.3\textwidth]
{\begin{fmffile}{ee_nu_WW}\fmfstraight
\begin{fmfchar*}(20,20)
  \fmfleft{i1,i2}
  \fmfright{o1,o2}
  \fmf{fermion}{i1,v1}
  \fmf{fermion}{v2,i2}
  \fmf{fermion, label=\nuele{}}{v1,v2}
  \fmf{boson}{v1,o1}
  \fmf{boson}{v2,o2}
  \fmflabel{\antiele}{i2}
  \fmflabel{\electron}{i1}
  \fmflabel{\Wbosonplus}{o2}
  \fmflabel{\Wbosonminus}{o1}
  \fmfdot{v1,v2}
\end{fmfchar*}
\end{fmffile}\vspace{\baselineskip}}
\hfill
\subcaptionbox{\label{subfig-fgraph-ff_WW3}}[.3\textwidth]
{\begin{fmffile}{ee_Z_WW}\fmfstraight
\begin{fmfchar*}(20,20)
  \fmfleft{i1,i2}
  \fmfright{o1,o2}
  \fmf{fermion}{i1,v1,i2}
  \fmf{photon, label=\Zboson{}}{v1,v2}
  \fmf{boson}{o1,v2,o2}
  \fmflabel{\antiele}{i2}
  \fmflabel{\electron}{i1}
  \fmflabel{\Wbosonplus}{o2}
  \fmflabel{\Wbosonminus}{o1}
  \fmfdot{v1,v2}
\end{fmfchar*}
\end{fmffile}\vspace{\baselineskip}}

\caption{Diagrammes de Feynman de production de paire $\Wbosonplus\Wbosonminus$ à l'arbre.}
\label{fig-fgraph-ff_WW}
\end{figure}
\par L'interaction électromagnétique repose sur l'invariance de jauge sous les transformations locales du groupe $U(1)_{em}$.
Dans le cas de l'interaction électrofaible, ce groupe de symétrie est $SU(2)_L \times U(1)_Y$.
\par
Dans un premier temps, seul le cas de $SU(2)_L$ avec les leptons est traité et permet de soulever toute la richesse de ce groupe par rapport à $U(1)$.
Ensuite, $SU(2)_L \times U(1)_Y$ est abordé, toujours avec les leptons.
Puis, les traitement des quarks est présenté, donnant alors une description de l'interaction électrofaible.
\subsubsection{Symétrie $SU(2)_L$ et chiralité}\label{chapter-MS-MSSM-section-formalisme-subsec-EW-SU2_L}
Une des propriétés les plus importantes de l'interaction faible est de violer la symétrie de parité, notée $P$.
La symétrie $P$ consiste à remplacer une des trois coordonnées spatiales par son opposé, comme le fait un miroir.
La violation de $P$ par l'interaction faible, observée expérimentalement avec les désintégrations $\beta$~\cite{Wu_P_violation}, peut être expliquée en considérant que les couplages aux bosons \Wboson\ ne se font qu'avec certains états de chiralité.
\par
La chiralité est une propriété des particules.
Pour un fermion décrit par un spineur $\psi$,
elle peut être droite (\emph{right}) ou gauche (\emph{left}).
La composante de chiralité gauche $\psi_L$ s'obtient à l'aide du projecteur chiral $\gamma^5$ selon
\begin{equation}
\psi_L = \frac{1}{2}(1-\gamma^5)\psi
\mend[,]
\label{eq-projo_L}
\end{equation}
celle de chiralité gauche $\psi_R$ selon
\begin{equation}
\psi_R = \frac{1}{2}(1+\gamma^5)\psi
\mend
\label{eq-projo_R}
\end{equation}
Pour les antiparticules décrites par $\bar{\psi}=\psi^\dagger \gamma^0$,
\begin{equation}
\overline{\psi_L}
= (\psi_L)^\dagger \gamma^0
= \left(\frac{1}{2}(1-\gamma^5)\psi\right)^\dagger\gamma^0
= \psi^\dagger \frac{1}{2} (1-\gamma^5) \gamma^0
= \psi^\dagger \gamma^0 \frac{1}{2} (1+\gamma^5)
= \bar{\psi}_R
\mend[,]
\end{equation}
et de même, $\overline{\psi_R} = \bar{\psi}_L$.
\par
Les expériences montrent que les bosons \Wboson\ ne sont couplés qu'aux fermions de chiralité gauche et aux antifermions de chiralité droite, ce qui correspond à introduire un facteur $\gamma^\mu(1-\gamma^5)$ aux termes de couplage correspondants dans le lagrangien.
Dans la notation $SU(2)_L$, $L$ signifie ainsi \emph{left}.
Les couplages étant différents selon l'état de parité, la symétrie $P$ est violée.
Les couplages du boson \Zboson\ ne sont pas purement en $(1-\gamma^5)$, il est donc quand même couplé aux fermions de chiralité droite et aux antifermions de chiralité gauche.
Ce comportement du \Zboson\ peut être expliqué dans le cadre de la force électrofaible, la force faible à elle seule ne permet pas d'en rendre compte.
\par La violation de $P$ par l'interaction faible a également pour conséquence la violation de $C$, la symétrie de charge.
En effet, $C$ change une particule de chiralité gauche en une antiparticule de même chiralité, dont les couplages aux bosons de l'interaction faible ne sont pas les mêmes.
En revanche, la symétrie $CP$ change une particule de chiralité gauche en une antiparticule de chiralité droite, ce qui semble être conservé par l'interaction faible.
L'étude des désintégrations des kaons a toutefois montré que l'interaction faible viole également la symétrie $CP$~\cite{Fitch_Cronin_CP_violation}, ce qui a pu être expliqué théoriquement en postulant l'existence d'une troisième génération de quarks~\cite{CKM_KM} observée depuis.
\par L'introduction de la symétrie $SU(2)_L$ amène un nouveau nombre quantique, l'\og isospin faible \fg, noté $I$. Il se comporte mathématiquement comme le spin des particules, d'où son nom. % \emph{iso}spin.
Les fermions de chiralité gauche sont rassemblés en doublets d'isospin faible $I=\frac{1}{2}$, les fermions de chiralité droite en singlets d'isospin faible $I=0$. Ces derniers sont ainsi invariants sous les transformations de $SU(2)_L$, ce qui se traduit physiquement par une insensibilité à l'interaction faible.
\par
Les fermions peuvent être de chiralité droite ou gauche.
Dans le cadre actuel du modèle standard, les neutrinos existent toutefois uniquement avec une chiralité gauche.
Il n'y a à ce jour aucune raison pour les neutrinos de chiralité droite de ne pas exister.
Cependant, ils n'interagissent pas, par construction, avec la matière dans le cadre du modèle standard.
Ainsi, il est possible de les retirer du modèle tout en conservant une description cohérente du comportement des particules.
Cela mène aux représentations du tableau~\ref{tab-chapter-MS-MSSM-section-formalisme-subsec-EW-rzpt_fermions_chiralite_isospin}.
\begin{table}[h]
\centering
\begin{tabular}{ccccc}
\toprule
$I$ & Quarks gauches & Quarks droits & Leptons gauches & Leptons droits\\
\midrule
$\frac{1}{2}$ & $\displaystyle \begin{pmatrix} \quarku_i \\ \quarkd_i \end{pmatrix}_L$ & - & $\displaystyle \begin{pmatrix} \neutrino_i \\ \ell_i \end{pmatrix}_L$ & - \\
$0$ & - & $\quarku_{i,R}$, $\quarkd_{i,R}$ & - & $\ell_{i,R}$\\
\bottomrule
\end{tabular}
\caption[Représentation des fermions selon leur chiralité et leur isospin faible.]{Représentation des fermions selon leur chiralité et leur isospin faible. L'indice $i\in\set{1,2,3}$ correspond à la génération des particules. Ainsi, les symboles $\quarku_i$, $\quarkd_i$, $\ell_i$ et $\neutrino_i$ correspondent, respectivement, aux quarks d'isospin faible haut (\quarku, \quarkc, \quarkt), d'isospin faible bas (\quarkd, \quarks, \quarkb), aux leptons chargés (\ele, \mu, \tau) et aux neutrinos (\nuele, \numu, \nutau).}
\label{tab-chapter-MS-MSSM-section-formalisme-subsec-EW-rzpt_fermions_chiralite_isospin}
\end{table}
\subsubsection{Symétrie $SU(2)$ et interactions entre bosons}\label{chapter-MS-MSSM-section-formalisme-subsec-EW-SU2_general}
Afin d'alléger les notations, le cas plus général d'un groupe de symétrie $SU(2)$ est traité. Pour étendre les résultats à $SU(2)_L$, il suffit de se souvenir que les couplages ont uniquement lieu entre fermions de chiralité gauche et antifermions de chiralité droite.
La méthode reste la même que pour l'électromagnétisme.
Sous une transformation de $SU(2)$, les spineurs se transforment selon
\begin{equation}
\psi \to \eexp{\frac{\im}{2}\bm{\tau}\cdot\bm{\alpha}(x)}\psi
\msep
\bar{\psi} \to \bar{\psi}\eexp{-\frac{\im}{2}\bm{\tau}\cdot\bm{\alpha}(x)}
\label{eq-chapter-MS-MSSM-section-formalisme-subsec-EW-tr_jauge-spineur}
\end{equation}
où $\bm{\alpha}\in\mathbb{R}^3$ et $\bm{\tau}$ est un vecteur dont les composantes $\tau_i$ sont les générateurs de $SU(2)$.
Ces générateurs sont des matrices $2\times2$ s'identifiant aux matrices de Pauli $\sigma_i$ définies dans l'annexe~\refApMath.
Toutefois, ces générateurs agissent dans le cas de $SU(2)_L$ sur les doublets d'isospin faible alors que les matrices de Pauli agissent sur le spin d'un fermion.
Afin d'éviter les confusions, la notation $\tau$ est utilisée.
L'équation~\eqref{eq-chapter-MS-MSSM-section-formalisme-subsec-EW-tr_jauge-spineur} est l'analogue directe de l'équation~\eqref{eq-chapter-MS-MSSM-section-formalisme-subsec-QED-tr_jauge-spineur}.
\par Afin de simplifier les calculs qui suivent, seules les transformations infinitésimales sont considérées. En effet, $SU(2)$ est un groupe non abélien. Cela signifie que deux transformations successives $a$ et $b$ de ce groupe ne donnent pas le même résultat selon que soient appliquées $a$ puis $b$ ou $b$ puis $a$, \ie\ $ab-ba\neq0$.
%Ainsi, des termes supplémentaires apparaissent, ou plutôt ne se simplifient pas entre eux.
Les transformations précédentes sous leurs formes infinitésimales, \ie\ au premier ordre en $\bm{\alpha}$, s'expriment
\begin{equation}
\psi \to \left(1+\frac{\im}{2}\bm{\tau}\cdot\bm{\alpha}(x)\right)\psi
\msep
\bar{\psi} \to \bar{\psi}\left(1-\frac{\im}{2}\bm{\tau}\cdot\bm{\alpha}(x)\right)
\mend
\end{equation}
Les termes du lagrangien du fermion libre, introduit dans l'équation~\eqref{eq-lagrangien_fermion_libre_partial}, se transforment alors comme
\begin{equation}
-m\bar{\psi}\psi
\to
-m\bar{\psi}\left(1-\frac{\im}{2}\bm{\tau}\cdot\bm{\alpha}(x)\right)\left(1+\frac{\im}{2}\bm{\tau}\cdot\bm{\alpha}(x)\right)\psi
=-m\bar{\psi}\psi+\Order{\alpha^2}
\end{equation}
et
\begin{align}
\im\bar{\psi} \gamma^\mu\partial_\mu \psi
&
\to
\im \bar{\psi}\left(1-\frac{\im}{2}\bm{\tau}\cdot\bm{\alpha}(x)\right)\gamma^\mu\partial_\mu \left(\left(1+\frac{\im}{2}\bm{\tau}\cdot\bm{\alpha}(x)\right)\psi\right)
\nonumber\\&\hphantom{\to}
=
\im \bar{\psi}\gamma^\mu\partial_\mu \psi
-
\bar{\psi}\frac{1}{2}\bm{\tau}\cdot\gamma^\mu\partial_\mu\bm{\alpha}(x) \psi+\Order{\alpha^2}
\mend[,]
\label{eq-chapter-MS-MSSM-section-formalisme-subsec-EW-tr_jauge-terme_cinetique}
\end{align}
ce qui fait apparaître, sur le même principe qu'avec l'interaction électromagnétique, un terme supplémentaire brisant l'invariance de jauge du lagrangien.
Une nouvelle dérivée covariante peut être définie afin de rétablir l'invariance de jauge,
\begin{equation}
D_\mu  = \partial_\mu  - \frac{\im}{2} g_I \bm{\tau}\cdot\bm{W}_\mu(x)
\mend[,]
\end{equation}
où l'on introduit $g_I$ la constante de couplage d'isospin faible, ainsi que trois champs de jauge vectoriels $W^i_\mu(x)$, $i\in\set{1,2,3}$ se transformant tels que
\begin{equation}
\bm{W}_\mu \to \bm{W}_\mu + \frac{1}{g_I}\partial_\mu\bm{\alpha} - (\bm{\alpha}\wedge\bm{W}_\mu)
\mend
\end{equation}
Dans ce cas, le lagrangien du fermion libre se réécrit sous la forme
\begin{align}
\Lcal_{\text{fermion libre}}'
&
= \im \bar{\psi} \gamma^\mu D_\mu \psi - m \bar{\psi}\psi
= \im \bar{\psi} \gamma^\mu \partial_\mu \psi + \bar{\psi} \gamma^\mu \frac{1}{2} g_I \bm{\tau}\cdot\bm{W}_\mu \psi - m \bar{\psi}\psi
\nonumber
\\&
=
\Lcal_{\text{fermion libre}}
+
\bar{\psi} \gamma^\mu \frac{1}{2} g_I \bm{\tau}\cdot\bm{W}_\mu \psi
\mend
\end{align}
Ainsi, le terme supplémentaire du lagrangien se transforme tel que
\begin{align}
\bar{\psi} \gamma^\mu \frac{1}{2} g_I \bm{\tau}\cdot\bm{W}_\mu \psi
\to
{}&{}
\bar{\psi}\left(1-\frac{\im}{2}\bm{\tau}\cdot\bm{\alpha}(x)\right)
\gamma^\mu \frac{g_I}{2} \bm{\tau}\cdot
\left( \bm{W}_\mu + \frac{1}{g_I}\partial_\mu\bm{\alpha} - (\bm{\alpha}\wedge\bm{W}_\mu) \right)
\left(1+\frac{\im}{2}\bm{\tau}\cdot\bm{\alpha}(x)\right)\psi
\nonumber\\&
=
\bar{\psi} \gamma^\mu \frac{g_I}{2} \bm{\tau}\cdot\bm{W}_\mu \psi
-
\bar{\psi} \frac{\im}{2}\bm{\tau}\cdot\bm{\alpha}(x) \gamma^\mu \frac{g_I}{2} \bm{\tau}\cdot\bm{W}_\mu \psi
+
\bar{\psi} \gamma^\mu \frac{g_I}{2} \bm{\tau}\cdot\bm{W}_\mu \frac{\im}{2}\bm{\tau}\cdot\bm{\alpha}(x) \psi
\nonumber\\&\hphantom{=}
+
\bar{\psi} \gamma^\mu \frac{1}{2} \bm{\tau}\cdot \partial_\mu\bm{\alpha} \psi
-
\bar{\psi} \gamma^\mu \frac{g_I}{2} \bm{\tau}\cdot (\bm{\alpha}\wedge\bm{W}_\mu) \psi
+\Order{\alpha^2}
\mend
\label{eq-chapter-MS-MSSM-section-formalisme-subsec-EW-tr_jauge-terme_suppl}
\end{align}
Or,
\begin{equation}
(\bm{\tau}\cdot\bm{a})(\bm{\tau}\cdot\bm{b})
=
(\bm{a}\cdot\bm{b}) + \im \bm{\tau} \cdot (\bm{a}\wedge\bm{b})
\Leftrightarrow
\bm{\tau} \cdot (\bm{a}\wedge\bm{b})
=
\im [(\bm{a}\cdot\bm{b}) - (\bm{\tau}\cdot\bm{a})(\bm{\tau}\cdot\bm{b})]
\mend
\end{equation}
Ainsi,
\begin{align}
\bm{\tau}\cdot (\bm{\alpha}\wedge\bm{W}_\mu)
&
= \frac{1}{2}  \left[ \bm{\tau}\cdot (\bm{\alpha}\wedge\bm{W}_\mu) - \bm{\tau}\cdot (\bm{W}_\mu\wedge\bm{\alpha}) \right]
\nonumber\\
&
= \frac{\im}{2}  \left[ [(\bm{\alpha}\cdot\bm{W}_\mu) - (\bm{\tau}\cdot\bm{\alpha})(\bm{\tau}\cdot\bm{W}_\mu)] - [(\bm{W}_\mu\cdot\bm{\alpha}) - (\bm{\tau}\cdot\bm{W}_\mu)(\bm{\tau}\cdot\bm{\alpha})] \right]
\nonumber\\
&
= \frac{\im}{2}  \left[  (\bm{\tau}\cdot\bm{W}_\mu)(\bm{\tau}\cdot\bm{\alpha}) - (\bm{\tau}\cdot\bm{\alpha})(\bm{\tau}\cdot\bm{W}_\mu) \right]
\label{eq-chapter-MS-MSSM-section-formalisme-subsec-EW-tr_jauge-simplification_trick}
\mend
\end{align}
En combinant les équations~\eqref{eq-chapter-MS-MSSM-section-formalisme-subsec-EW-tr_jauge-terme_suppl} et~\eqref{eq-chapter-MS-MSSM-section-formalisme-subsec-EW-tr_jauge-simplification_trick}, il vient
\begin{equation}
\bar{\psi} \gamma^\mu \frac{1}{2} g_I \bm{\tau}\cdot\bm{W}_\mu \psi
\to
\bar{\psi} \gamma^\mu \frac{g_I}{2} \bm{\tau}\cdot\bm{W}_\mu \psi
+
\bar{\psi} \gamma^\mu \frac{1}{2} \bm{\tau}\cdot \partial_\mu\bm{\alpha} \psi
+\Order{\alpha^2}
\mend[,]
\end{equation}
où le dernier terme obtenu compense exactement le terme brisant l'invariance de jauge dans l'équation~\eqref{eq-chapter-MS-MSSM-section-formalisme-subsec-EW-tr_jauge-terme_cinetique}.
\par À ce stade, l'analogie avec l'électromagnétisme pousse à introduire $\bm{W}_{\mu\nu}$ l'analogue à $F_{\mu\nu}$ tel que
$\bm{W}_{\mu\nu} = \partial_\mu \bm{W}_\nu - \partial_\nu\bm{W}_\mu$. Or, les invariances de jauge imposées mènent à utiliser une définition légèrement différente,
\begin{equation}
\bm{W}_{\mu\nu} = \partial_\mu \bm{W}_\nu - \partial_\nu\bm{W}_\mu + g_I(\bm{W}_\mu \wedge \bm{W}_\nu)
\mend
\label{eq-chapter-MS-MSSM-section-formalisme-subsec-EW-defWmunu}
\end{equation}
Le lagrangien pour $SU(2)$ s'écrit alors
\begin{equation}
\Lcal_{SU(2)} = \bar{\psi}(\im \slashed{D} - m) \psi - \frac{1}{4} \bm{W}_{\mu\nu} \cdot \bm{W}^{\mu\nu}
\mend
\end{equation}
\begin{wrapfigure}{R}{6.25cm}
\centering
\begin{fmffile}{WWW}\fmfstraight
\begin{fmfchar*}(20,20)
  \fmfleft{i1}
  \fmfright{o1,o2}
  \fmf{boson}{i1,v}
  \fmf{boson}{o1,v}
  \fmf{boson}{o2,v}
  \fmfdot{v}
  \fmflabel{ }{i1}
  \fmflabel{ }{o1}
  \fmflabel{ }{o2}
\end{fmfchar*}
\end{fmffile}
\caption{Diagramme de Feynman correspondant à l'interaction entre trois bosons.}
\label{fig-fgraph-WWW}
\end{wrapfigure}
\par Une différence notable et importante vis-à-vis de $\Lcal_{QCD}$ est la non linéarité de $\bm{W}_{\mu\nu}$ par rapport à $\bm{W}_\mu$ et $\bm{W}_\nu$. Cette composante non linéaire ouvre la porte aux interactions directes entre les champs $W^i_\mu$, \ie\ entre les bosons, ce qui était impossible avec QED. De nouveaux types de vertex, comme celui de la figure~\ref{fig-fgraph-WWW}, sont donc possibles dans une théorie de jauge avec une symétrie locale $SU(2)$.

\begin{wraptable}{R}{6.25cm}
\centering
\begin{tabular}{cccc}
\toprule
Champ & $\nuele$ & $\ele_L$ & $\ele_R$\\
\midrule
$Y$ & $-1$ & $-1$ & $-2$ \\
$I$ & $\frac{1}{2}$ & $\frac{1}{2}$ & $0$ \\
$I_3$ & $\frac{1}{2}$ & $-\frac{1}{2}$ & $0$ \\
\midrule
$Q$ & $0$ & $-1$ & $-1$\\
\bottomrule
\end{tabular}
\caption{Valeurs des hypercharges, isospins faibles et charges électriques pour les leptons.}
\label{tab-Y_I_I3_Q-leptons}
\end{wraptable}
\subsubsection{Symétrie $SU(2)_L \times U(1)_Y$ et unification électrofaible}\label{chapter-MS-MSSM-section-formalisme-subsec-EW-U1}
Dans la notation $U(1)_Y$, $Y$ est l'\og hypercharge \fg, reliée à $Q$ la charge électrique et à $I_3$ la projection de l'isospin faible par la relation de Gell-Mann--Nishijima,
\begin{equation}
Q = I_3 + \frac{Y}{2}
\mend[,]
\end{equation}
dont les résultats pour les différents leptons sont présentés dans le tableau~\ref{tab-Y_I_I3_Q-leptons}.
\par Les raisonnements réalisés précédemment peuvent ici être mis à profit. En effet, le cas de $U(1)_{em}$ est traité dans la section~\ref{chapter-MS-MSSM-section-formalisme-subsec-QED}. Il est possible d'obtenir directement les mêmes résultats pour $U(1)_Y$ en procédant à l'analogie $U(1)_{em}\leftrightarrow U(1)_Y$, avec
\begin{equation}
A_\mu \leftrightarrow B_\mu
\msep
F_{\mu\nu} \leftrightarrow F^{(B)}_{\mu\nu}
\msep
e \leftrightarrow g_Y
\msep
Q \leftrightarrow \frac{1}{2}Y
\mend
\end{equation}
De plus, sachant que $SU(2)_L$ couple les fermions de chiralité gauche et les antifermions de chiralité droite, les résultats pour $SU(2)$ sont directement utilisables en ajoutant les projections décrites par les équations~\eqref{eq-projo_L} et~\eqref{eq-projo_R}.
\par La dérivée covariante pour $SU(2)_L \times U(1)_Y$ est ainsi
\begin{equation}
D_\mu = \partial_\mu - \im g_II \bm{\tau}\cdot\bm{W}_\mu - \frac{\im}{2}g_YYB_\mu
\mend[,]
\label{eq-definition_Dmu_EW}
\end{equation}
pouvant agir sur un doublet d'isospin faible, noté $L$, ou un singlet d'isospin faible, noté $R$, selon
\begin{align}
D_\mu L &= \left[ \partial_\mu - \frac{\im}{2}g_I \bm{\tau}\cdot\bm{W}_\mu + \frac{\im}{2}g_YB_\mu \right] L
\mend[,]
\\
D_\mu R &= \left[ \partial_\mu + \im g_YB_\mu \right] R
\mend[,]
\end{align}
compte-tenu des différentes valeurs de $Y$ et $I$ données dans le tableau~\ref{tab-Y_I_I3_Q-leptons}.
\par Le lagrangien invariant sous $SU(2)_L\times U(1)_Y$ de l'interaction électrofaible s'écrit alors
\begin{equation}
\Lcal_{EW}
=
\im \bar{\psi} \slashed{D} \psi - \frac{1}{4} \bm{W}_{\mu\nu} \cdot \bm{W}^{\mu\nu} - \frac{1}{4} F^{(B)}_{\mu\nu} \cdot F^{(B)\mu\nu}
\mend
\label{eq-chapter-MS-MSSM-section-formalisme-subsec-EW-LSU2LU1Y}
\end{equation}
Attention toutefois, les champs $B_\mu$ et $W^i_\mu$ ne correspondent pas, respectivement, au photon et aux bosons \Wbosonpm\ et \Zboson.
Ces quatre bosons sont en fait des combinaisons de ces quatre champs, ce qui est décrit dans la section~\ref{chapter-MS-MSSM-section-formalisme-subsec-Higgs_mechanism-subsubsec-bosons}.
\subsubsection{Interaction électrofaible pour les quarks}\label{chapter-MS-MSSM-section-formalisme-subsec-EW-quarks}
Le lagrangien électrofaible ainsi construit pour les leptons pourrait facilement être réutilisé dans le cas des quarks, $\psi$ étant un champ décrivant un fermion. Cependant, le lagrangien de l'équation~\eqref{eq-chapter-MS-MSSM-section-formalisme-subsec-EW-LSU2LU1Y} ne couple entre eux que des fermions de même génération. Or, il a été observé expérimentalement que l'interaction faible peut également coupler des quarks de générations différentes.
\par Un mécanisme rendant possible de tels couplages a été introduit par Cabibbo, Kobayashi et Maskawa~\cite{CMK_C1,CMK_C2,CKM_KM}. Le principe est de faire, pour les quarks, la distinction entre les états propres de masse, \ie\ ceux que l'on observe, et les états propres de l'interaction faible. Ces deux ensembles d'états propres diffèrent ainsi pour les quarks d'isospin faible bas et sont reliés entre eux par la matrice CKM $\mathcal{M}_{CKM}$, matrice $3\times3$ unitaire complexe,
\begin{equation}
\begin{pmatrix}
\quarkd' \\ \quarks' \\ \quarkb'
\end{pmatrix}
=
\begin{pmatrix}
V_{\quarku\quarkd} & V_{\quarku\quarks} & V_{\quarku\quarkb} \\
V_{\quarkc\quarkd} & V_{\quarkc\quarks} & V_{\quarkc\quarkb} \\
V_{\quarkt\quarkd} & V_{\quarkt\quarks} & V_{\quarkt\quarkb}
\end{pmatrix}
\begin{pmatrix}
\quarkd \\ \quarks \\ \quarkb
\end{pmatrix}
\mend[,]
\end{equation}
où $\quarkd'$, $\quarks'$ et $\quarkb'$ sont les états propres de l'interaction faible et $\quarkd$, $\quarks$ et $\quarkb$ ceux de masse. L'élément de matrice $V_{ij}$ ou son conjugué $V_{ij}^*$ est ainsi un facteur appliqué au vertex pour le calcul de la section efficace des processus impliquant des quarks et l'interaction faible. Ces coefficients ne sont pas prédits par le modèle standard et sont donc mesurés expérimentalement. Les valeurs de leurs modules sont les suivantes~\cite{PDG_booklet_2020}
\begin{equation}
\abs{\mathcal{M}_{CKM}} = 
\begin{pmatrix}
\num{0.97370}\pm\num{0.00014} & \num{0.2245}\pm\num{0.0008} & \num{0.00382}\pm\num{0.00024} \\
\num{0.221}\pm\num{0.004} & \num{0.987}\pm\num{0.011} & \num{0.0410}\pm\num{0.0014} \\
\num{0.0080}\pm\num{0.0003} & \num{0.0388}\pm\num{0.0011} & \num{1.013}\pm\num{0.030} 
\end{pmatrix}
\mend
\end{equation}
Cette matrice a une structure très prononcée, presque diagonale. Le couplage entre les quarks de générations différentes est faible, ce qui se traduit expérimentalement par des durées de vie de certains hadrons contenant des quarks de deuxième et troisième génération suffisamment longs pour qu'ils se propagent sur quelques millimètres, voire quelques mètres.
\par Le modèle ainsi construit décrit l'interaction électrofaible pour tous les fermions.
Cependant, il n'y a aucun terme de masse dans le lagrangien de l'équation~\eqref{eq-chapter-MS-MSSM-section-formalisme-subsec-EW-LSU2LU1Y}. En effet, un terme de masse pour les fermions serait de la forme
\begin{equation}
-m\bar{\psi}\psi
=
-m (\bar{\psi}_R+\bar{\psi}_L)(\psi_R+\psi_L)
=
-m (
\bar{\psi}_R \psi_R
+\bar{\psi}_R \psi_L
+\bar{\psi}_L \psi_R
+\bar{\psi}_L \psi_L
)
=
-m (\bar{\psi}_R \psi_L + \bar{\psi}_L \psi_R)
\mend
\end{equation}
Or, ce terme n'est pas invariant sous $SU(2)_L\times U(1)_Y$.
Pour les champs $W^i_\mu$ et $B_\mu$, des termes de masse violeraient également la symétrie de jauge.
Dès lors, il semble difficile pour un tel lagrangien de décrire les forces électromagnétique et faible.
\par En réalité, ce lagrangien décrit l'interaction \og électrofaible \fg.
Les interactions électromagnétique et faible résultent d'un mécanisme de brisure spontanée de symétrie, qui se trouve dans ce cas être le mécanisme de Higgs.
La section suivante montre comment l'introduction du champ de Higgs amène cette brisure de symétrie et comment sont obtenus des fermions massifs, le photon et les bosons \Wbosonpm\ et \Zboson.
