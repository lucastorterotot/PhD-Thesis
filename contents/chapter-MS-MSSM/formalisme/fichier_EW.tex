\subsection{Interaction électrofaible}\label{chapter-MS-MSSM-section-formalisme-subsec-EW}
L'interaction électrofaible unifie les interactions électromagnétique et faible.
Dans un premier temps, décrivons l'interaction faible en procédant de même que pour l'interaction électromagnétique.
Nous verrons ensuite pourquoi unifier ces deux forces et comment y parvenir.

\subsubsection{Interaction faible}\label{chapter-MS-MSSM-section-formalisme-subsec-EW-subsubsec-weak}
Nous avons vu précédemment que l'interaction électromagnétique repose sur l'invariance de jauge locale sous les transformations du groupe $U(1)_{em}$
Dans le cas de l'interaction faible, ce groupe de symétrie est $SU(2)_L$ où $L$ signifie \og left \fg{} car l'interaction faible ne couple que les fermions de chiralité\footnote{\todo{cf...}} gauche et les anti-fermions de chiralité droite.

\subsubsection{Unification avec l'interaction électromagnétique}\label{chapter-MS-MSSM-section-formalisme-subsec-EW-subsubsec-unification}
$SU(2)_L \times U(1)_\gamma$
$\gamma$ est l'\emph{hypercharge}.

\begin{equation}
Q = I_3 + \frac{Y}{2}
\mend
\end{equation}

\begin{table}
\centering
\begin{tabular}{cccc}
\toprule
Champ & $\nuele$ & $\ele_L$ & $\ele_R$\\
\midrule
$Y$ & $-1$ & $-1$ & $-2$ \\
$I$ & $\frac{1}{2}$ & $\frac{1}{2}$ & $0$ \\
$I_3$ & $\frac{1}{2}$ & $-\frac{1}{2}$ & $0$ \\
\midrule
$Q$ & $0$ & $-1$ & $-1$\\
\bottomrule
\end{tabular}
\end{table}