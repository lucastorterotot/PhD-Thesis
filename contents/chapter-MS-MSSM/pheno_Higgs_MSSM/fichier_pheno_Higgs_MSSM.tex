\section{Phénoménologie des bosons de Higgs du MSSM}\label{chapter-MS-MSSM-section-pheno_Higgs_MSSM}
Pour concevoir une analyse de physique des particules à même de tester le MSSM, il nous faut dans un premier temps déterminer la manifestation du MSSM à observer.
Comme cela a été développé dans la section précédente, le MSSM implique l'existence de quatre bosons de Higgs supplémentaires, en particulier deux nouveaux bosons de Higgs neutres, \Higgs\ et \HiggsA.
Si de tels bosons existent, un signal leur correspondant doit pouvoir être observé.
Dans la suite, nous nous concentrons donc sur la phénoménologie des trois bosons de Higgs neutres.
\par Nous avons vu qu'au premier ordre, les masses des bosons de Higgs s'expriment en fonction de deux paramètres uniquement, $m_{\HiggsA}$ et $\tan\beta$.
Les couplages des trois bosons de Higgs neutres du MSSM aux autres particules, par rapport aux couplages du boson de Higgs du modèle standard, sont présentés dans le tableau~\ref{tab-Higgs_couplings_2HDM} en fonction de $\alpha$ et $\beta$.
Or, $\alpha$ et $\beta$ sont reliés par les équations~\eqref{eq-Higgs_mixing_angle_MSSM}, donnant
\begin{equation}
\tan 2\alpha = \frac{m_{\HiggsA}^2+m_{\Zboson}^2}{m_{\HiggsA}^2-m_{\Zboson}^2} \tan 2\beta
\mend
\label{eq-Higgs_mixing_angle_MSSM_tan}
\end{equation}
\begin{wraptable}{R}{8cm}
\centering
\begin{tabular}{rccc}
\toprule
Couplage avec & \higgs & \Higgs & \HiggsA \\
\midrule
Bosons vecteurs & $1$ & $0$ & $0$\\
Fermions hauts & $1$ & $-\cot\beta$ & $\cot\beta$ \\
Fermions bas & $1$ & $\tan\beta$ & $\tan\beta$ \\
\bottomrule
\end{tabular}
\caption[Couplages des bosons de Higgs neutres dans la limite découplée.]{Couplages des bosons de Higgs neutres dans la limite découplée du MSSM par rapport aux couplages du boson de Higgs du modèle standard.}
\label{tab-Higgs_couplings_MSSM_decoupling}
\end{wraptable}
\par Les observations expérimentales semblent favoriser $m_{\HiggsA}\gg m_{\Zboson}$~\cite{ATLAS-CMS-Higgs_combined_1,ATLAS-CMS-Higgs_combined_2,CMS-MSSM-HTT_2014}.
Cette situation correspond à la limite découplée, dans laquelle
\begin{equation}
\lim_{m_{\HiggsA}\gg m_{\Zboson}} \tan 2\alpha = \tan 2\beta
\end{equation}
d'après ~\eqref{eq-Higgs_mixing_angle_MSSM_tan}.
Alors, dans la limite découplée, $\alpha \sim \beta$ ou $\alpha\sim\beta\pm\frac{\pi}{2}$.
Or, $\beta\geq0$ et $\alpha\leq0$.
Il ne reste donc plus que la possibilité $\alpha\sim\beta-\frac{\pi}{2}$.
Dans la limite découplée, les couplages du tableau~\ref{tab-Higgs_couplings_2HDM} deviennent alors ceux du tableau~\ref{tab-Higgs_couplings_MSSM_decoupling}.
\par Les couplages ainsi obtenus dans le tableau~\ref{tab-Higgs_couplings_MSSM_decoupling} présentent trois caractéristiques d'intérêt:
\begin{itemize}
\item le boson de Higgs le plus léger, \higgs, se comporte exactement comme le boson de Higgs du modèle standard, ce qui le rend tout à fait cohérent avec les observations actuelles;
\item les bosons de Higgs neutres massifs \Higgs\ et \HiggsA\ ne présentent aucun couplage aux bosons vecteurs, par exemple la désintégration $\HiggsA\to\Zboson\Zboson$ est impossible mais $\HiggsA\to\Zboson\higgs$ est possible;
\item les bosons de Higgs neutres massifs \Higgs\ et \HiggsA\ sont couplés de manière similaire aux fermions.
\end{itemize}
\par De plus, $\tan\beta$ est contraint par~\cite{Ridolfi-SUSY}
\begin{equation}
1 < \tan\beta \lesssim \frac{m_{\quarkt}}{m_{\quarkb}} \simeq \num{42} \mend
\end{equation}
Or, lorsque $\tan\beta$ augmente, les couplages de \Higgs\ et \HiggsA\ aux fermions d'isospin bas sont augmentés et leurs couplages aux fermions d'isospin haut supprimés.
La production des bosons de Higgs neutres supplémentaires, tout comme leurs désintégrations, s'en trouvent donc intrinsèquement liées à la présence de fermions d'isospin bas.
\subsection{Production de bosons de Higgs}\label{chapter-MS-MSSM-section-pheno_Higgs_MSSM-subsec-production}

\begin{figure}[h]
\centering
\vspace{\baselineskip}
\subcaptionbox{Production par fusion de gluons.\label{subfig-fgraph-gg_loop_h}}[.3\textwidth]
{\begin{fmffile}{gg_loop_h}\fmfstraight
\begin{fmfchar*}(30,20)
  \fmfleft{g1,fi,g2}
  \fmfright{fo1,h,fo2}
  \fmf{gluon}{g1,g1loop}
  \fmf{gluon}{g2,g2loop}
  \fmf{phantom, tension=.6}{g1loop,fo1}
  \fmf{phantom, tension=.6}{g2loop,fo2}
  \fmffreeze
  \fmf{fermion}{g1loop,hloop,g2loop,g1loop}
  \fmf{fermion}{g2loop,g1loop}
  \fmf{dashes, tension=1.75}{hloop,h}
  \fmfdot{g1loop,hloop,g2loop}
  \fmffreeze
  \fmf{phantom}{g1loop,fakev1}
  \fmf{phantom}{g2loop,fakev1}
  \fmffreeze
%  \fmf{phantom,tension=1.5}{hloop,fakev2}
%  \fmf{phantom, label=$t,,\bar{t}$, l.side=left}{fakev1,fakev2}
%  \fmf{phantom, label=$b,,\bar{b}$, l.side=right}{fakev1,fakev2}
  \fmflabel{\gluon}{g1}
  \fmflabel{\gluon}{g2}
  \fmflabel{\higgs}{h}
\end{fmfchar*}
\end{fmffile}
\vspace{\baselineskip}}
\hfill
\subcaptionbox{Production par fusion de bosons vecteurs en voie $t$.\label{subfig-fgraph-Higgs_VBF_t}}[.3\textwidth]
{\begin{fmffile}{Higgs_VBF_t}\fmfstraight
\begin{fmfchar*}(30,20)
  \fmfleft{qa,qb}
  \fmfright{qc,h,qd}
  \fmf{fermion}{qa,v1,qc}
  \fmf{fermion}{qb,v2,qd}
  \fmf{phantom}{qa,v1}
  \fmf{phantom}{qb,v2}
  \fmffreeze
  \fmf{boson, label=$\Wbosonmp,, \Zboson$, l.side=right, tension=3}{v1,v3}
  \fmf{boson, label=$\Wbosonpm,, \Zboson$, l.side=left, tension=3}{v2,v3}
  \fmf{dashes}{v3,h}
  \fmfdot{v1,v2,v3}
  \fmflabel{\quark}{qa}
  \fmflabel{\quark}{qb}
  \fmflabel{\quark}{qc}
  \fmflabel{\quark}{qd}
  \fmflabel{\higgs}{h}
\end{fmfchar*}
\end{fmffile}
\vspace{\baselineskip}}
\hfill
\subcaptionbox{Production par fusion de bosons vecteurs en voie $u$.\label{subfig-fgraph-Higgs_VBF_u}}[.3\textwidth]
{\begin{fmffile}{Higgs_VBF_u}\fmfstraight
\begin{fmfchar*}(42,25)
  \fmfleft{qa,qb}
  \fmfright{qc,h,qd}
  \fmf{fermion}{v1,qc}
  \fmf{fermion}{v2,qd}
  \fmf{phantom,tension=2}{qa,v1}
  \fmf{phantom,tension=2}{qb,v2}
  \fmffreeze
  \fmf{plain}{vv2,v2}
  \fmf{plain}{vv1,v1}
  \fmf{fermion}{qa,vv2}
  \fmf{fermion}{qb,vv1}
  \fmf{boson, label=$\Wbosonpm,, \Zboson$, l.side=right, tension=3}{v1,v3}
  \fmf{boson, label=$\Wbosonpm,, \Zboson$, l.side=left, tension=3}{v2,v3}
  \fmf{dashes}{v3,h}
  \fmfdot{v1,v2,v3}
  \fmflabel{\quark}{qa}
  \fmflabel{\quark}{qb}
  \fmflabel{\quark}{qc}
  \fmflabel{\quark}{qd}
  \fmflabel{\higgs}{h}
\end{fmfchar*}
\end{fmffile}
\vspace{\baselineskip}}
\caption[Production de boson de Higgs par fusion de gluons et de bosons vecteurs.]{Diagrammes de Feynman de production de boson de Higgs dans le cadre du modèle standard par fusion de gluons (\gluon\gluon\higgs) et fusion de bosons vecteurs (VBF).}
\label{fig-fgraph-Higgs_prod_ggh_VBF}
\end{figure}

\begin{figure}[h]
\centering
\vspace{\baselineskip}
\subcaptionbox{Production en association avec un boson \Wboson.\label{subfig-fgraph-Higgs_VH_W}}[.3\textwidth]
{\begin{fmffile}{Higgs_VH_W}\fmfstraight
\begin{fmfchar*}(30,20)
  \fmfleft{i1,i2}
  \fmfright{o1,o2}
  \fmf{fermion}{i1,v1,i2}
  \fmf{boson, label=${\Wbosonpm}^*$, l.side=left}{v1,v2}
  \fmf{boson}{v2,o1}
  \fmf{dashes}{v2,o2}
  \fmfdot{v1,v2}
  \fmflabel{\quark}{i1}
  \fmflabel{\antiquark}{i2}
  \fmflabel{\Wbosonpm}{o1}
  \fmflabel{\higgs}{o2}
\end{fmfchar*}
\end{fmffile}
\vspace{\baselineskip}}
\hfill
\subcaptionbox{Production en association avec un boson \Zboson.\label{subfig-fgraph-Higgs_VH_Z}}[.3\textwidth]
{\begin{fmffile}{Higgs_VH_Z}\fmfstraight
\begin{fmfchar*}(42,25)
  \fmfleft{i1,i2}
  \fmfright{o1,o2}
  \fmf{fermion}{i1,v1,i2}
  \fmf{boson, label=${\Zboson}^*$, l.side=left}{v1,v2}
  \fmf{boson}{v2,o1}
  \fmf{dashes}{v2,o2}
  \fmfdot{v1,v2}
  \fmflabel{\quark}{i1}
  \fmflabel{\antiquark}{i2}
  \fmflabel{\Zboson}{o1}
  \fmflabel{\higgs}{o2}
\end{fmfchar*}
\end{fmffile}
\vspace{\baselineskip}}
\hfill
\subcaptionbox{Production par fusion de gluons associée à un boson \Zboson.\label{subfig-fgraph-Higgs_gg_loop_Zh}}[.3\textwidth]
{\begin{fmffile}{Higgs_gg_loop_Zh}\fmfstraight
\begin{fmfchar*}(42,25)
  \fmfleft{g1,g2}
  \fmfright{Z,h}
  \fmf{gluon}{g1,g1loop}
  \fmf{gluon}{g2,g2loop}
  \fmf{fermion}{g1loop,Zloop,hloop,g2loop,g1loop}
  \fmf{dashes}{hloop,h}
  \fmf{boson}{Zloop,Z}
  \fmfdot{g1loop,Zloop,hloop,g2loop}
  \fmflabel{\gluon}{g1}
  \fmflabel{\gluon}{g2}
  \fmflabel{\higgs}{h}
  \fmflabel{\Zboson}{Z}
\end{fmfchar*}
\end{fmffile}
\vspace{\baselineskip}}
\caption[Production de boson de Higgs en association avec un boson.]{Diagrammes de Feynman de production de boson de Higgs dans le cadre du modèle standard en association avec un boson.}
\label{fig-fgraph-Higgs_prod_VH_ggZh}
\end{figure}

\begin{figure}[h]
\centering
\vspace{\baselineskip}
\subcaptionbox{\label{subfig-fgraph-Higgs_with_b_gg_g_bbh}}[.45\textwidth]
{\begin{fmffile}{Higgs_with_b_gg_g_bbh}\fmfstraight
\begin{fmfchar*}(42,25)
  \fmfleft{i1,i2}
  \fmfright{o1,o2,o3}
  \fmf{gluon}{i1,v1}
  \fmf{gluon}{i2,v1}
  \fmf{gluon}{v1,v2}
  \fmf{phantom}{o1,v2,o3}
  \fmffreeze
  \fmf{fermion}{o1,v2,v3,o3}
  \fmffreeze
  \fmf{dashes}{v3,o2}
  \fmfdot{v1,v2,v3}
  \fmflabel{\gluon}{i1}
  \fmflabel{\gluon}{i2}
  \fmflabel{\antiquarkb}{o1}
  \fmflabel{\quarkb}{o3}
  \fmflabel{\higgs}{o2}
\end{fmfchar*}
\end{fmffile}
\vspace{\baselineskip}}
\hfill
\subcaptionbox{\label{subfig-fgraph-Higgs_with_b_qq_g_bbh}}[.45\textwidth]
{\begin{fmffile}{Higgs_with_b_qq_g_bbh}\fmfstraight
\begin{fmfchar*}(30,20)
  \fmfleft{i1,i2}
  \fmfright{o1,o2,o3}
  \fmf{fermion}{i1,v1,i2}
  \fmf{gluon}{v1,v2}
  \fmf{phantom}{o1,v2,o3}
  \fmffreeze
  \fmf{fermion}{o1,v2,v3,o3}
  \fmffreeze
  \fmf{dashes}{v3,o2}
  \fmfdot{v1,v2,v3}
  \fmflabel{\quark}{i1}
  \fmflabel{\antiquark}{i2}
  \fmflabel{\antiquarkb}{o1}
  \fmflabel{\quarkb}{o3}
  \fmflabel{\higgs}{o2}
\end{fmfchar*}
\end{fmffile}
\vspace{\baselineskip}}

\vspace{2\baselineskip}
\subcaptionbox{\label{subfig-fgraph-Higgs_with_b_gg_hbb}}[.45\textwidth]
{\begin{fmffile}{gg_hbb}\fmfstraight
\begin{fmfchar*}(30,20)
  \fmfleft{g1,g2}
  \fmfright{b1,h,b2}
  \fmf{gluon}{g1,g1b1}
  \fmf{gluon}{g2,g2b2}
  \fmf{fermion}{g1b1,b1}
  \fmf{fermion}{b2,g2b2}
  \fmffreeze
  \fmf{fermion}{g2b2,bh,g1b1}
  \fmffreeze
  \fmf{dashes}{bh,h}
  \fmfdot{g1b1,g2b2,bh}
  \fmflabel{\gluon}{g1}
  \fmflabel{\gluon}{g2}
  \fmflabel{\quarkb}{b1}
  \fmflabel{\antiquarkb}{b2}
  \fmflabel{\higgs}{h}
\end{fmfchar*}
\end{fmffile}
\vspace{\baselineskip}}
\hfill
\subcaptionbox{\label{subfig-fgraph-Higgs_with_b_bg_b_bh}}[.45\textwidth]
{\begin{fmffile}{Higgs_with_b_bg_b_bh}\fmfstraight
\begin{fmfchar*}(30,20)
  \fmfleft{i1,i2}
  \fmfright{o1,o2}
  \fmf{fermion}{i2,v1,v2,o1}
  \fmf{gluon}{i1,v1}
  \fmf{dashes}{v2,o2}
  \fmfdot{v1,v2}
  \fmflabel{\gluon}{i1}
  \fmflabel{\quarkb}{i2}
  \fmflabel{\quarkb}{o1}
  \fmflabel{\higgs}{o2}
\end{fmfchar*}
\end{fmffile}
\vspace{\baselineskip}}

\caption[Production de boson de Higgs en association avec un quark \quarkb.]{Diagrammes de Feynman de production de boson de Higgs dans le cadre du modèle standard en association avec un quark \quarkb.}
\label{fig-fgraph-Higgs_prod_with_b}
\end{figure}

\begin{fmffile}{bb_hHA}\fmfstraight
\begin{fmfchar*}(30,20)
  \fmfleft{b1,b2}
  \fmfright{h}
  \fmf{fermion}{b1,bh}
  \fmf{fermion}{bh,b2}
  \fmf{dashes, label=$\Hs,, \Hn,, \Ha$, l.side=left}{bh,h}
  \fmfdot{bh}
  \fmflabel{\quarkb}{b1}
  \fmflabel{\antiquarkb}{b2}
\end{fmfchar*}
\end{fmffile}


\begin{fmffile}{bg_b_bhHA}\fmfstraight
\begin{fmfchar*}(42,25)
  \fmfleft{i1,i2}
  \fmfright{o1,o2}
  \fmf{fermion}{i2,v1,v2,o1}
  \fmf{gluon}{i1,v1}
  \fmf{dashes, label=$\Hs,, \Hn,, \Ha$, l.side=left}{v2,o2}
  \fmfdot{v1,v2}
  \fmflabel{\gluon}{i1}
  \fmflabel{\quarkb}{i2}
  \fmflabel{\quarkb}{o1}
\end{fmfchar*}
\end{fmffile}

\begin{fmffile}{gg_hHAbb}\fmfstraight
\begin{fmfchar*}(30,20)
  \fmfleft{g1,g2}
  \fmfright{b1,h,b2}
  \fmf{gluon}{g1,g1b1}
  \fmf{gluon}{g2,g2b2}
  \fmf{fermion}{g1b1,b1}
  \fmf{fermion}{b2,g2b2}
  \fmffreeze
  \fmf{fermion}{g2b2,bh,g1b1}
  \fmffreeze
  \fmf{dashes, label=$\Hs,, \Hn,, \Ha$, l.side=left}{bh,h}
  \fmfdot{g1b1,g2b2,bh}
  \fmflabel{\gluon}{g1}
  \fmflabel{\gluon}{g2}
  \fmflabel{\quarkb}{b1}
  \fmflabel{\antiquarkb}{b2}
\end{fmfchar*}
\end{fmffile}


\begin{fmffile}{gg_loop_hHA}\fmfstraight
\begin{fmfchar*}(42,25)
  \fmfleft{g1,fi,g2}
  \fmfright{fo1,h,fo2}
  \fmf{gluon}{g1,g1loop}
  \fmf{gluon}{g2,g2loop}
  \fmf{phantom, tension=.6}{g1loop,fo1}
  \fmf{phantom, tension=.6}{g2loop,fo2}
  \fmffreeze
  \fmf{fermion}{g1loop,hloop,g2loop,g1loop}
  \fmf{fermion}{g2loop,g1loop}
  \fmf{dashes, label=$\Hs,, \Hn,, \Ha$, l.side=left, tension=1.75}{hloop,h}
  \fmfdot{g1loop,hloop,g2loop}
  \fmffreeze
  \fmf{phantom}{g1loop,fakev1}
  \fmf{phantom}{g2loop,fakev1}
  \fmffreeze
%  \fmf{phantom,tension=1.5}{hloop,fakev2}
%  \fmf{phantom, label=$t,,\bar{t}$, l.side=left}{fakev1,fakev2}
%  \fmf{phantom, label=$b,,\bar{b}$, l.side=right}{fakev1,fakev2}
  \fmflabel{\gluon}{g1}
  \fmflabel{\gluon}{g2}
\end{fmfchar*}
\end{fmffile}



\subsection{Désintégration de bosons de Higgs}\label{chapter-MS-MSSM-section-pheno_Higgs_MSSM-subsec-desintegration_Higgs}

\begin{fmffile}{H-tautau_small}\fmfstraight
\begin{fmfchar*}(40,30)
  \fmfleft{h}
  \fmfright{tau1,tau2}
  \fmf{dashes, label=$\Hs,, \Hn,, \Ha$, l.side=left}{h,v}
  \fmf{fermion, label=$\tau^+$, l.side=left}{tau1,v}
  \fmf{fermion, label=$\tau^-$, l.side=left}{v,tau2}
  \fmfdot{v}
\end{fmfchar*}
\end{fmffile}

\subsection{Désintégration des leptons tau}\label{chapter-MS-MSSM-section-pheno_Higgs_MSSM-subsec-desintegration_lepton_tau}
\emph{The branching fractions for decays into five or more charged hadrons are negligible. The lifetime of the tau lepton amounts to \SI{290}{fs}, corresponding to $c\tau = \SI{87}{\micro m}$.}

\begin{fmffile}{tau_to_ele_small_beamer}%\fmfstraight
\begin{fmfchar*}(20,10)
  \fmfleft{taui}
  \fmfright{l1,l2,l3,f1,f2,f3,nuout}
  \fmf{fermion, tension=2}{taui,v1}
  \fmf{fermion}{v1,nuout}
  \fmf{phantom}{v1,l1}
  \fmffreeze
  \fmflabel{$\nutau$}{nuout}
  \fmf{boson, label=$\Wbosonminus$, l.side=right, tension=2}{v1,v2}
  \fmf{phantom}{v2,td1,l1}
  \fmf{phantom}{v2,td2,l2}
  \fmf{phantom}{v2,td3,l3}
  \fmffreeze
  \fmf{fermion}{l1,v2,l3}
  \fmflabel{\antinuele}{l1}
  \fmflabel{\electron}{l3}
  \fmflabel{\leptau}{taui}
  \fmfdot{v1,v2}
\end{fmfchar*}
\end{fmffile}


\begin{fmffile}{tau_to_mu}%\fmfstraight
\begin{fmfchar*}(30,20)
  \fmfleft{taui}
  \fmfright{l1,l2,nuout}
  \fmf{fermion, tension=2}{taui,v1}
  \fmf{fermion}{v1,nuout}
  \fmf{phantom}{v1,l1}
  \fmffreeze
  \fmflabel{$\nutau$}{nuout}
  \fmf{boson, label=$\Wbosonminus$, l.side=right, tension=2}{v1,v2}
  \fmf{fermion}{l2,v2,l1}
  \fmflabel{$\muon$}{l1}
  \fmflabel{$\antinumu$}{l2}
  \fmflabel{\leptau}{taui}
  \fmfdot{v1,v2}
\end{fmfchar*}
\end{fmffile}


\begin{fmffile}{tau_to_tauh_qqbar}%\fmfstraight
\begin{fmfchar*}(30,20)
  \fmfleft{taui}
  \fmfright{l1,l2,l3,f1,f2,f3,nuout}
  \fmf{fermion, tension=2}{taui,v1}
  \fmf{fermion}{v1,nuout}
  \fmf{phantom}{v1,l1}
  \fmffreeze
  \fmflabel{$\nutau$}{nuout}
  \fmf{boson, label=$\Wbosonminus$, l.side=right, tension=2}{v1,v2}
  \fmf{phantom}{v2,td1,l1}
  \fmf{phantom}{v2,td2,l2}
  \fmf{phantom}{v2,td3,l3}
  \fmffreeze
  \fmf{fermion}{l1,v2,l3}
  \fmflabel{\antiquark}{l1}
  \fmflabel{\quark}{l3}
  \fmflabel{\leptau}{taui}
  \fmfdot{v1,v2}
\end{fmfchar*}
\end{fmffile}


\begin{fmffile}{tau_to_tauh-1prong}%\fmfstraight
\begin{fmfchar*}(30,20)
  \fmfleft{taui}
  \fmfright{l1,l2,l3,f1,f2,f3,nuout}
  \fmf{fermion, label=$\leptau$, l.side=left, tension=2}{taui,v1}
  \fmf{fermion}{v1,nuout}
  \fmf{phantom}{v1,l1}
  \fmffreeze
  \fmflabel{$\nutau$}{nuout}
  \fmf{boson, label=$\Wbosonminus$, l.side=right, tension=2}{v1,v2}
  \fmf{phantom}{v2,td1,l1}
  \fmf{phantom}{v2,td2,l2}
  \fmf{phantom}{v2,td3,l3}
  \fmffreeze
  \fmf{plain}{td3,v2,td1}
  \fmfblob{.15w}{td2}
  \fmf{plain}{td2,l2}
  \fmflabel{$\hadron^-$}{l2}
  \fmfdot{v1,v2}
\end{fmfchar*}
\end{fmffile}


\begin{fmffile}{tau_to_tauh-3prongs}%\fmfstraight
\begin{fmfchar*}(30,20)
  \fmfleft{taui}
  \fmfright{l1,l2,l3,f1,f2,f3,nuout}
  \fmf{fermion, label=$\leptau$, l.side=left, tension=2}{taui,v1}
  \fmf{fermion}{v1,nuout}
  \fmf{phantom}{v1,l1}
  \fmffreeze
  \fmflabel{$\nutau$}{nuout}
  \fmf{boson, label=$\Wbosonminus$, l.side=right, tension=2}{v1,v2}
  \fmf{phantom}{v2,td1,l1}
  \fmf{phantom}{v2,td2,l2}
  \fmf{phantom}{v2,td3,l3}
  \fmffreeze
  \fmf{plain}{td3,v2,td1}
  \fmfblob{.15w}{td2}
  \fmf{plain}{td2,l1}
  \fmf{plain}{td2,l2}
  \fmf{plain}{td2,l3}
  \fmflabel{$\hadron^-$}{l1}
  \fmflabel{$\hadron^-$}{l2}
  \fmflabel{$\hadron^+$}{l3}
  \fmfdot{v1,v2}
\end{fmfchar*}
\end{fmffile}


\begin{fmffile}{tau_to_ell}%\fmfstraight
\begin{fmfchar*}(30,20)
  \fmfleft{taui}
  \fmfright{l1,l2,l3,f1,f2,f3,nuout}
  \fmf{fermion, tension=2}{taui,v1}
  \fmf{fermion}{v1,nuout}
  \fmf{phantom}{v1,l1}
  \fmffreeze
  \fmflabel{$\nutau$}{nuout}
  \fmf{boson, label=$\Wbosonminus$, l.side=right, tension=2}{v1,v2}
  \fmf{phantom}{v2,td1,l1}
  \fmf{phantom}{v2,td2,l2}
  \fmf{phantom}{v2,td3,l3}
  \fmffreeze
  \fmf{fermion}{l1,v2,l3}
  \fmflabel{$\antineutrino_\ell$}{l1}
  \fmflabel{$\ell$}{l3}
  \fmflabel{\leptau}{taui}
  \fmfdot{v1,v2}
\end{fmfchar*}
\end{fmffile}


\begin{fmffile}{tau_to_tauh_small_beamer}%\fmfstraight
\begin{fmfchar*}(20,10)
  \fmfleft{taui}
  \fmfright{l1,l2,l3,f1,f2,f3,nuout}
  \fmf{fermion, tension=2}{taui,v1}
  \fmf{fermion}{v1,nuout}
  \fmf{phantom}{v1,l1}
  \fmffreeze
  \fmflabel{$\nutau$}{nuout}
  \fmf{boson, label=$\Wbosonminus$, l.side=right, tension=2}{v1,v2}
  \fmf{phantom}{v2,td1,l1}
  \fmf{phantom}{v2,td2,l2}
  \fmf{phantom}{v2,td3,l3}
  \fmffreeze
  \fmf{plain}{td3,v2,td1}
  \fmfblob{.15w}{td2}
  \fmf{plain}{td2,l2}
  \fmflabel{\tauhm}{l2}
  \fmflabel{\leptau}{taui}
  \fmfdot{v1,v2}
\end{fmfchar*}
\end{fmffile}

