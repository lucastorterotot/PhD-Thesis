\section{Conclusion}\label{chapter-MS-MSSM-section-conclusion}
Les constituants fondamentaux de la matière usuelle sont les fermions de la première génération, \ie\ les quarks~\quarku\ et~\quarkd\ pouvant former les protons (\quarku\quarku\quarkd) et les neutrons (\quarku\quarkd\quarkd), l'électron et le neutrino électronique, dont la présence se manifeste lors des désintégrations radioactives $\beta$.
Cependant, la liste des particules fondamentales est bien plus longue.
Il existe ainsi trois générations de fermions, portant leur nombre à douze.
Chacun de ces fermions est accompagné d'un antifermion correspondant.
\par Aux fermions s'ajoutent les bosons.
Les bosons de jauge sont les vecteurs des forces fondamentales, il s'agit du photon, des bosons \Wbosonplus, \Wbosonminus\ et \Zboson\ et des gluons.
Le boson de Higgs, quant à lui, est scalaire.
\par Le modèle standard, reposant sur la théorie quantique des champs, permet de décrire le comportement de ces particules.
L'invariance de jauge locale sous les transformations du groupe $SU(3)_C \times SU(2)_L \times U(1)_Y$ fait émerger naturellement les forces électrofaible et forte.
Le mécanisme de brisure spontanée de symétrie introduit le champ de Higgs et donne une masse aux particules.
\par Ce modèle a permis de prédire l'existence de particules comme les bosons \Wboson\ et \Zboson, le quark top ou encore le boson de Higgs de nombreuses années avant leurs observations.
Malgré ces prouesses et les décennies de prédictions correctement vérifiées, la communauté scientifique sait pertinemment que le modèle standard n'est qu'une étape vers une théorie plus complète.
En effet, le modèle standard ne permet pas d'expliquer certains faits expérimentaux comme la présence de la matière noire.
D'autres phénomènes comme l'ajustement fin laissent à croire qu'il s'agit d'une théorie effective à basse énergie.
\par De nombreuses extensions au modèle standard sont alors proposées, comme la supersymétrie.
Les modèles supersymétriques introduisent les \og sparticules \fg, partenaires supersymétriques des particules.
Dans sa version la plus simple, l'extension supersymétrique minimale du modèle standard ou MSSM, il n'y a pas un mais cinq bosons de Higgs, ainsi que leurs partenaires.
La recherche d'un signal associé à ces bosons de Higgs supplémentaires est un test expérimental du MSSM.
\par Dans le cas où ces bosons de Higgs supplémentaires sont de haute masse, le canal de désintégration en paire de leptons~\tau\ est le plus prometteur.
Les leptons~\tau\ se désintègrent eux-mêmes en électron, muon ou tau hadronique.
Il existe donc six états finaux différents.
Le chapitre~\refChHTT\ présente une analyse expérimentale menée pour la recherche de bosons de Higgs supplémentaires de haute masse se désintégrant en paire de leptons~\tau.
\par Grâce à la collaboration CMS (\emph{Compact Muon Solenoid}) et au détecteur du même nom installé au LHC (\emph{Large Hadron Collider}) de l'organisation européenne pour la recherche nucléaire ou CERN (Conseil Européen pour la Recherche Nucléaire), les conditions expérimentales sont réunies pour procéder à la
recherche de bosons de Higgs supplémentaires de haute masse se désintégrant en paire de leptons~\tau.
Le chapitre~\refChLHCCMS\ présente ce dispositif expérimental.