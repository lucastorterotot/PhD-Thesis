\section{Succès et limites du modèle standard}\label{chapter-MS-MSSM-section-succes_limites}
\subsection{Succès}\label{chapter-MS-MSSM-section-succes_limites-subsec-succes}
Le modèle standard a été conçu il y a plus de 50~ans et a évolué avec les nouvelles observations expérimentales.
Ses succès sont nombreux, notamment de par son pouvoir prédictif.
L'existence de bosons massifs \Wboson\ et \Zboson\ est ainsi prédite dès la fin des années 60~\cite{Weinberg_leptons_model}, leur observation est réalisée en 1983~\cite{Wboson_discovery1,Wboson_discovery2,Zboson_discovery1,Zboson_discovery2,Wboson_discovery3}.
De même, le quark top prédit en 1973~\cite{CKM_KM} est observé 22~ans plus tard~\cite{top_discovery1,top_discovery2}.
Dernier tour de force en date, le boson de Higgs, prédit en 1964~\cite{Englert_Brout,Higgs_1,Higgs_2,Guralnik_Hagen_Kibble} et observé en 2012~\cite{ATLAS_Higgs_discovery,CMS_Higgs_discovery}.
\par Le lagrangien du modèle standard s'écrit à ce jour sous la forme
\begin{equation}
\Lcal_{SM} = 
\underbrace{- \frac{1}{4} \mathcal{F}_{\mu\nu}\mathcal{F}^{\mu\nu} \vphantom{\frac{1}{4}}}_{\substack{\text{bosons vecteurs\vphantom{YHg}}\\\text{libres\vphantom{(lHg)}}}}
+ \underbrace{\im \bar{\psi} \slashed{D} \psi \vphantom{\frac{1}{4}}}_{\substack{\text{fermions et\vphantom{YHg}}\\\text{interactions\vphantom{(lHg)}}}}
+ \underbrace{(D^\mu\phi)^\dagger(D_\mu\phi) - V(\phi) \vphantom{\frac{1}{4}}}_{\substack{\text{boson de Higgs et\vphantom{YHg}}\\\text{interactions avec les autres bosons\vphantom{(lHg)}}}}
+ \underbrace{(y \bar{\psi}\phi\psi + \text{h.c.}) \vphantom{\frac{1}{4}}}_{\substack{\text{termes de Yukawa\vphantom{YHg}}\\\text{(interactions fermions-Higgs)\vphantom{(lHg)}}}}
\end{equation}
où \og $\text{h.c.}$ \fg{} signifie conjugué hermitien et
\begin{align}
\mathcal{F}_{\mu\nu}\mathcal{F}^{\mu\nu} &= \bm{G}_{\mu\nu} \cdot \bm{G}^{\mu\nu} + \bm{W}_{\mu\nu} \cdot \bm{W}^{\mu\nu} + F^{(B)}_{\mu\nu} F^{(B)\mu\nu}
\mend[,]\\
\slashed{D} = \gamma^\mu D_\mu &= \gamma^\mu \left[ \partial_\mu - \im g_I I \bm{\tau}\cdot\bm{W}_\mu - \im g_Y \frac{Y}{2} B_\mu - \im g_s \frac{C}{2} \bm{\lambda}\cdot\bm{G}_\mu \right]
\mend[,]\\
V(\phi) &= \mu^2\phi^\dagger\phi + \frac{\lambda^2}{2} (\phi^\dagger\phi)^2
\mend
\end{align}

\subsection{Limites}\label{chapter-MS-MSSM-section-succes_limites-subsec-limites}
% Issues just with this model in this state
\paragraph{Nombre de paramètres libres}
\begin{table}[h]
\centering
\begin{tabular}{lcrl}
\toprule
Grandeur & Symbole & \multicolumn{2}{c}{Valeur} \\
\midrule
Masse du quark up & $m_{\quarku}$ & $\num{2.2}^{+\num{0.5}}_{-\num{0.4}}$ & \SI{}{\MeV} \\
Masse du quark down & $m_{\quarkd}$ & $\num{4.7}^{+\num{0.5}}_{-\num{0.3}}$ & \SI{}{\MeV} \\
Masse du quark strange & $m_{\quarks}$ & $\num{95}^{+\num{9}}_{-\num{3}}$ & \SI{}{\MeV} \\
Masse du quark charm & $m_{\quarkc}$ & $\num{1.275}^{+\num{0.025}}_{-\num{0.035}}$ & \SI{}{\GeV} \\
Masse du quark bottom & $m_{\quarkb}$ & $\num{4.18}^{+\num{0.04}}_{-\num{0.03}}$ & \SI{}{\GeV} \\
Masse du quark top & $m_{\quarkt}$ & $\num{173.0}\pm\num{0.4}$ & \SI{}{\GeV} \\
Masse de l'électron & $m_{\ele}$ & $\num{0.5109989461}\pm\num{0.0000000031}$ & \SI{}{\MeV} \\
Masse du muon & $m_{\mu}$ & $\num{105.6583745}\pm\num{0.0000024}$ & \SI{}{\MeV} \\
Masse du tau & $m_{\tau}$ & $\num{1776.86}\pm\num{0.12}$ & \SI{}{\MeV} \\
Angle de mixage CKM I-II & $\theta_{12}$ & $\num{13.01}\pm\num{0.03}$ & \SI{}{\degree} \\
Angle de mixage CKM II-III & $\theta_{23}$ & $\num{2.35}\pm\num{0.09}$ & \SI{}{\degree} \\
Angle de mixage CKM I-III & $\theta_{13}$ & $\num{0.20}\pm\num{0.04}$ & \SI{}{\degree} \\
Phase de violation CP CKM & $\delta_{\text{CKM}}$ & $\num{70}\pm\num{3}$ & \SI{}{\degree} \\
Constante de couplage $U(1)_Y$ & $g_Y$ & $\num{0.34970}\pm\num{0.00019}$ & \\
Constante de couplage $SU(2)_L$ & $g_I$ & $\num{0.65295}\pm\num{0.00012}$ & \\
Constante de couplage $SU(3)_C$ & $g_s$ & $\num{0.1182}\pm\num{0.00012}$ & \\
Angle QCD & $\theta_{\text{QCD}}$ & $<\num{e-10}$ & \\
Condensat du champ de Higgs & $v$ & $\num{246}\pm\num{6e-5}$ & \SI{}{\GeV} \\
Masse du boson de Higgs & $m_{\higgs}$ & $\num{125.18}\pm\num{0.16}$ & \SI{}{\GeV} \\
\bottomrule
\end{tabular}
\caption{Valeurs expérimentales des 19 paramètres libres du modèle standard~\cite{PDG_booklet_2018}.}
\end{table}

\emph{It is interesting to note that 15 out of the 19 (the 9 Yukawa fermion mass terms, the Higgs mass, the Higgs potential v.e.v., and the four CKM values) are related to the Higgs boson. In other words, most of our ignorance in the Standard Model is related to the Higgs.}

\emph{the canonical form of the Standard Model includes massless neutrinos. We know that neutrinos must have mass, and also that they oscillate (turn into one another), which means that their mass eigenstates do not coincide with their eigenstates with respect to the weak interaction. Thus, another mixing matrix must be involved, which is called the Pontecorvo-Maki-Nakagawa-Sakata (PMNS) matrix. So we end up with three neutrino masses $m_1$, $m_2$ and $m_3$, and the three angles $\theta_{12}$, $\theta_{23}$ and $\theta_{13}$ (not to be confused with the CKM angles above) plus the CP-violating phase $\delta_{\text{PMNS}}$ of the PMNS matrix.}

\emph{So this is potentially as many as 26 parameters in the Standard Model that need to be determined by experiment. This is quite a long way away from the “holy grail” of theoretical physics, a theory that combines all four interactions, all the particle content, and which preferably has no free parameters whatsoever. Nonetheless the theory, and the level of our understanding of Nature’s fundamental building blocks that it represents, is a remarkable intellectual achievement of our era.}
\paragraph{Ajustement fin}
% Direct lacks of the model
\paragraph{Nombre de générations}
\paragraph{Masse des neutrinos}and right-handed neutrinos
% Beyond, gravity
\paragraph{Gravitation}
\paragraph{Matière noire}
bullet cluster!\cite{Clowe_2006}
\paragraph{Énergie noire}
% Just why are we here?!
\paragraph{Asymétrie matière-antimatière}
