\chapter{Diagrammes de Feynman}\label{annexe-fmf}

Il s'agit de représentations graphiques des interactions entre particules. Dans ce manuscrit, le temps s'écoule de gauche à droite sur un diagramme de Feynman, ainsi l'état initial se trouve à gauche, et l'état final à droite.

La propagation d'une particule est modélisée par un trait et une interaction a lieu au niveau des connexions entre ces traits, nommés \og vertex \fg.


Cas de l'interaction électromagnétique.
\begin{equation}
\bar{\psi}\gamma^\mu eQA_\mu \psi
=
\bar{\psi}_a \,\, [\gamma^\mu]_{ab} \, eQ \,\, A_\mu \,\, \psi_b
\label{eq-QCD_interaction_developped}
\end{equation}
\begin{figure}[h]
\centering
\vspace{\baselineskip}
\begin{fmffile}{ff_Gamma1-large_with_annotations}\fmfstraight
\begin{fmfchar*}(40,40)
  \fmfleft{i1,i2}
  \fmfright{o1}
  \fmf{fermion, label=$\psi_b$, l.side=right}{i1,v}
  \fmf{fermion, label=$\bar{\psi}_a$}{v,i2}
  \fmf{photon, label=$A_\mu$}{v,o1}
  \fmflabel{\fermion}{i1}
  \fmflabel{\antifermion}{i2}
  \fmflabel{\photon}{o1}
  \fmflabel{$eQ\gamma^\mu_{ab}\longrightarrow$}{v}
  \fmfdot{v}
\end{fmfchar*}
\end{fmffile}
\vspace{\baselineskip}
\caption[Diagramme de Feynman issu du terme~\eqref{eq-QCD_interaction_developped}.]{Diagramme de Feynman issu du terme du lagrangien du modèle standard de l'équation~\eqref{eq-QCD_interaction_developped}. Un fermion \fermion\ et un antifermion \antifermion\ sont présent dans l'état initial et sont décrits par un champ fermionique $\psi$ et son adjoint $\bar{\psi}$. Le champ $\psi$ interagit par ses composantes $a$ et $b$ avec le champ vectoriel $A_\mu$ au vertex, avec un couplage donné par $eQ\gamma^\mu_{ab}$. Il en résulte un photon \photon, décrit par ce champ $A_\mu$.}
\label{fig-fgraph-ff_Gamma1-large-annexeB}
\end{figure}