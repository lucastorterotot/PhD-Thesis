\chapter{Jeux de données -- $\Higgs\to\tau\tau$}\label{annexe-datasets-HTT}
L'analyse est basée sur les données à $\sqrt{s}=\SI{13}{\TeV}$ collectées en 2016, 2017 et 2018 par l'expérience CMS, correspondant à une luminosité intégrée de $\num{35.9}+\num{41.5}+\SI{59.7}{\femto\barn^{-1}}$.
Seuls les événements certifiés par la collaboration CMS sont considérés. Cette sélection est renseignée dans les fichiers \texttt{JSON} du tableau~\ref{tab-annexe-datasets-HTT-JSON_files}.
Les jeux de données utilisés pour chacun des états finaux considérés, ainsi que leurs gammes de \emph{runs} et luminosités intégrées respectives, sont donnés dans les tableaux~\ref{tab-annexe-datasets-HTT-2016_data}, \ref{tab-annexe-datasets-HTT-2017_data} et~\ref{tab-annexe-datasets-HTT-2018_data}.
\par
La modélisation du signal issu du modèle standard,
\ie\ $\higgs\to\tau\tau$ avec $m_{\higgs}=\SI{125}{\GeV}$,
est obtenue avec les jeux de données simulées correspondent aux modes de production du boson de Higgs
$\gluon\gluon\higgs$,
VBF
et
VH ($\Wboson\higgs$, $\Zboson\higgs$ et $\gluon\gluon\Zboson\higgs$).
Les listes de ces jeux de données simulées utilisés pour les trois années analysées sont données dans les tableaux~\ref{tab-annexe-datasets-HTT-2016_MC_signal_SM}, \ref{tab-annexe-datasets-HTT-2017_MC_signal_SM} et~\ref{tab-annexe-datasets-HTT-2018_MC_signal_SM}.
\par
La modélisation du signal issu du MSSM,
\ie\ $\Phi\to\tau\tau$ avec $\Phi=\Higgs,\HiggsA$,
est obtenue avec les jeux de données
$\gluon\gluon\to\Phi\to\tau\tau$ simulé avec \PYTHIA~8.1~\cite{pythia8.1}
et
$\gluon\gluon\to\quarkb\quarkb\Phi\to\tau\tau$ simulé avec \AMCATNLO~\cite{amcatnlo} et \PYTHIA\ pour l'hadronisation.
Les listes de ces jeux de données simulées utilisés pour les trois années analysées sont données dans les tableaux~\ref{tab-annexe-datasets-HTT-2016_MC_signal_MSSM}, \ref{tab-annexe-datasets-HTT-2017_MC_signal_MSSM} et~\ref{tab-annexe-datasets-HTT-2018_MC_signal_MSSM}.
\par
Les jeux de données simulées utilisés afin de modéliser le bruit de fond sont listés dans les tableaux~\ref{tab-annexe-datasets-HTT-2016_MC_backgrounds}, \ref{tab-annexe-datasets-HTT-2017_MC_backgrounds} et~\ref{tab-annexe-datasets-HTT-2018_MC_backgrounds}.
Les différents processus sont regroupés comme suit:

\vspace{.5\baselineskip}

\begin{minipage}[t]{.3\textwidth}
\begin{itemize}
\item $\Zboson\to\tau\tau$, $\Zboson\to\ell\ell$:
\begin{itemize}
\item $\Zboson \to LL$,
\item $\Zboson+\text{1 jet}$,
\item $\Zboson+\text{2 jets}$,
\item $\Zboson+\text{3 jets}$,
\item $\Zboson+\text{4 jets}$,
\item EWK $\Zboson\to LL$,
\item EWK $\Zboson\to \nu\nu$;
\end{itemize}
\item \ttbar;
\end{itemize}
\end{minipage}
\hfill
\begin{minipage}[t]{.3\textwidth}
\begin{itemize}
\item \Wjets:
\begin{itemize}
\item $\Wboson+\text{jets}$,
\item $\Wboson+\text{1 jet}$,
\item $\Wboson+\text{2 jets}$,
\item $\Wboson+\text{3 jets}$,
\item $\Wboson+\text{4 jets}$,
\item EWK \Wbosonminus,
\item EWK \Wbosonplus,
\item \Wboson\photon;
\end{itemize}
\end{itemize}
\end{minipage}
\hfill
\begin{minipage}[t]{.3\textwidth}
\begin{itemize}
\item Diboson:
\begin{itemize}
\item \emph{Single top},
%\item WW,
%\item WZ,
%\item ZZ,
\item VVTo2L2Nu,
\item WZTo2L2Q,
\item WZTo3LNu,
\item ZZTo2L2Q,
\item ZZTo4L.
\end{itemize}
\end{itemize}
\end{minipage}

\vspace{.5\baselineskip}

\par
Les jeux de données encapsulées (\emph{embedded}) sont listés dans les tableaux~\ref{tab-annexe-datasets-HTT-2016_embedded}, \ref{tab-annexe-datasets-HTT-2017_embedded} et~\ref{tab-annexe-datasets-HTT-2018_embedded}.
Ces jeux de données sont utilisés dans une estimation du bruit de fond contenant des paires de leptons tau à partir des données elles-mêmes.


\begin{table}[p]
\centering
\begin{tabular}{cl}
\toprule
Année & Fichier de certification \texttt{JSON}\\
\midrule
2016 & \inlinecode{bash}{Cert_271036-284044_13TeV_ReReco_07Aug2017_Collisions16_JSON.txt} \\
2017 & \inlinecode{bash}{Cert_294927-306462_13TeV_EOY2017ReReco_Collisions17_JSON_v1.txt} \\
\multirow{2}{*}{2018} & \inlinecode{bash}{Cert_314472-325175_13TeV_17SeptEarlyReReco} \\
 & \quad\inlinecode{bash}{2018ABC_PromptEraD_Collisions18_JSON.txt} \\
\bottomrule
\end{tabular}
\caption{Fichiers de certification \texttt{JSON}.}
\label{tab-annexe-datasets-HTT-JSON_files}
\end{table}

\begin{table}[p]
\centering
\begin{tabular}{clcc}
\toprule
Canal & Jeu de données & Gamme de \emph{run} & \Lumi\ (\SI{}{\femto\barn^{-1}})\\
\midrule
\tauh\tauh & \inlinecode{bash}{/Tau/Run2016B-17Jul2018_ver2-v1/MINIAOD} & $\num{272007}-\num{275376}$ & $\num{5.788}$ \\
\tauh\tauh & \inlinecode{bash}{/Tau/Run2016C-17Jul2018-v1/MINIAOD} & $\num{275657}-\num{276283}$ & $\num{2.573}$ \\
\tauh\tauh & \inlinecode{bash}{/Tau/Run2016D-17Jul2018-v1/MINIAOD} & $\num{276315}-\num{276811}$ & $\num{4.248}$ \\
\tauh\tauh & \inlinecode{bash}{/Tau/Run2016E-17Jul2018-v1/MINIAOD} & $\num{276831}-\num{277420}$ & $\num{4.009}$ \\
\tauh\tauh & \inlinecode{bash}{/Tau/Run2016F-17Jul2018-v1/MINIAOD} & $\num{277772}-\num{278808}$ & $\num{3.102}$ \\
\tauh\tauh & \inlinecode{bash}{/Tau/Run2016G-17Jul2018-v1/MINIAOD} & $\num{278820}-\num{280385}$ & $\num{7.540}$ \\
\tauh\tauh & \inlinecode{bash}{/Tau/Run2016H-17Jul2018-v1/MINIAOD} & $\num{280919}-\num{284044}$ & $\num{8.606}$ \\
\midrule
\mu\tauh & \inlinecode{bash}{/SingleMuon/Run2016B-17Jul2018_ver2-v1/MINIAOD} & $\num{272007}-\num{275376}$ & $\num{5.788}$ \\
\mu\tauh & \inlinecode{bash}{/SingleMuon/Run2016C-17Jul2018-v1/MINIAOD} & $\num{275657}-\num{276283}$ & $\num{2.573}$ \\
\mu\tauh & \inlinecode{bash}{/SingleMuon/Run2016D-17Jul2018-v1/MINIAOD} & $\num{276315}-\num{276811}$ & $\num{4.248}$ \\
\mu\tauh & \inlinecode{bash}{/SingleMuon/Run2016E-17Jul2018-v1/MINIAOD} & $\num{276831}-\num{277420}$ & $\num{4.009}$ \\
\mu\tauh & \inlinecode{bash}{/SingleMuon/Run2016F-17Jul2018-v1/MINIAOD} & $\num{277772}-\num{278808}$ & $\num{3.102}$ \\
\mu\tauh & \inlinecode{bash}{/SingleMuon/Run2016G-17Jul2018-v1/MINIAOD} & $\num{278820}-\num{280385}$ & $\num{7.540}$ \\
\mu\tauh & \inlinecode{bash}{/SingleMuon/Run2016H-17Jul2018-v1/MINIAOD} & $\num{280919}-\num{284044}$ & $\num{8.606}$ \\
\midrule
\ele\tauh & \inlinecode{bash}{/SingleElectron/Run2016B-17Jul2018_ver2-v1/MINIAOD}\hspace{-4.7pt} & $\num{272007}-\num{275376}$ & $\num{5.788}$ \\
\ele\tauh & \inlinecode{bash}{/SingleElectron/Run2016C-17Jul2018-v1/MINIAOD} & $\num{275657}-\num{276283}$ & $\num{2.573}$ \\
\ele\tauh & \inlinecode{bash}{/SingleElectron/Run2016D-17Jul2018-v1/MINIAOD} & $\num{276315}-\num{276811}$ & $\num{4.248}$ \\
\ele\tauh & \inlinecode{bash}{/SingleElectron/Run2016E-17Jul2018-v1/MINIAOD} & $\num{276831}-\num{277420}$ & $\num{4.009}$ \\
\ele\tauh & \inlinecode{bash}{/SingleElectron/Run2016F-17Jul2018-v1/MINIAOD} & $\num{277772}-\num{278808}$ & $\num{3.102}$ \\
\ele\tauh & \inlinecode{bash}{/SingleElectron/Run2016G-17Jul2018-v1/MINIAOD} & $\num{278820}-\num{280385}$ & $\num{7.540}$ \\
\ele\tauh & \inlinecode{bash}{/SingleElectron/Run2016H-17Jul2018-v1/MINIAOD} & $\num{280919}-\num{284044}$ & $\num{8.606}$ \\
\midrule
\ele\mu & \inlinecode{bash}{/MuonEG/Run2016B-17Jul2018_ver2-v1/MINIAOD} & $\num{272007}-\num{275376}$ & $\num{5.788}$ \\
\ele\mu & \inlinecode{bash}{/MuonEG/Run2016C-17Jul2018-v1/MINIAOD} & $\num{275657}-\num{276283}$ & $\num{2.573}$ \\
\ele\mu & \inlinecode{bash}{/MuonEG/Run2016D-17Jul2018-v1/MINIAOD} & $\num{276315}-\num{276811}$ & $\num{4.248}$ \\
\ele\mu & \inlinecode{bash}{/MuonEG/Run2016E-17Jul2018-v1/MINIAOD} & $\num{276831}-\num{277420}$ & $\num{4.009}$ \\
\ele\mu & \inlinecode{bash}{/MuonEG/Run2016F-17Jul2018-v1/MINIAOD} & $\num{277772}-\num{278808}$ & $\num{3.102}$ \\
\ele\mu & \inlinecode{bash}{/MuonEG/Run2016G-17Jul2018-v1/MINIAOD} & $\num{278820}-\num{280385}$ & $\num{7.540}$ \\
\ele\mu & \inlinecode{bash}{/MuonEG/Run2016H-17Jul2018-v1/MINIAOD} & $\num{280919}-\num{284044}$ & $\num{8.606}$ \\
\bottomrule
\end{tabular}

\caption{Jeux de données utilisés en 2016.}
\label{tab-annexe-datasets-HTT-2016_data}
\end{table}
\begin{table}[p]
\centering
\begin{tabular}{clcc}
\toprule
Canal & Jeu de données & Gamme de \emph{run} & \Lumi\ (\SI{}{\femto\barn^{-1}})\\
\midrule
\tauh\tauh & \inlinecode{bash}{/Tau/Run2017B-31Mar2018-v1/MINIAOD} & $\num{297046}-\num{299329}$ & $\num{4.823}$ \\
\tauh\tauh & \inlinecode{bash}{/Tau/Run2017C-31Mar2018-v1/MINIAOD} & $\num{299368}-\num{302029}$ & $\num{9.664}$ \\
\tauh\tauh & \inlinecode{bash}{/Tau/Run2017D-31Mar2018-v1/MINIAOD} & $\num{302030}-\num{303434}$ & $\num{4.252}$ \\
\tauh\tauh & \inlinecode{bash}{/Tau/Run2017E-31Mar2018-v1/MINIAOD} & $\num{303824}-\num{304797}$ & $\num{9.278}$ \\
\tauh\tauh & \inlinecode{bash}{/Tau/Run2017F-31Mar2018-v1/MINIAOD} & $\num{305040}-\num{306462}$ & $\num{13.54}$ \\
\midrule
\mu\tauh & \inlinecode{bash}{/SingleMuon/Run2017B-31Mar2018-v1/MINIAOD} & $\num{297046}-\num{299329}$ & $\num{4.823}$ \\
\mu\tauh & \inlinecode{bash}{/SingleMuon/Run2017C-31Mar2018-v1/MINIAOD} & $\num{299368}-\num{302029}$ & $\num{9.664}$ \\
\mu\tauh & \inlinecode{bash}{/SingleMuon/Run2017D-31Mar2018-v1/MINIAOD} & $\num{302030}-\num{303434}$ & $\num{4.252}$ \\
\mu\tauh & \inlinecode{bash}{/SingleMuon/Run2017E-31Mar2018-v1/MINIAOD} & $\num{303824}-\num{304797}$ & $\num{9.278}$ \\
\mu\tauh & \inlinecode{bash}{/SingleMuon/Run2017F-31Mar2018-v1/MINIAOD} & $\num{305040}-\num{306462}$ & $\num{13.54}$ \\
\midrule
\ele\tauh & \inlinecode{bash}{/SingleElectron/Run2017B-31Mar2018-v1/MINIAOD} & $\num{297046}-\num{299329}$ & $\num{4.823}$ \\
\ele\tauh & \inlinecode{bash}{/SingleElectron/Run2017C-31Mar2018-v1/MINIAOD} & $\num{299368}-\num{302029}$ & $\num{9.664}$ \\
\ele\tauh & \inlinecode{bash}{/SingleElectron/Run2017D-31Mar2018-v1/MINIAOD} & $\num{302030}-\num{303434}$ & $\num{4.252}$ \\
\ele\tauh & \inlinecode{bash}{/SingleElectron/Run2017E-31Mar2018-v1/MINIAOD} & $\num{303824}-\num{304797}$ & $\num{9.278}$ \\
\ele\tauh & \inlinecode{bash}{/SingleElectron/Run2017F-31Mar2018-v1/MINIAOD} & $\num{305040}-\num{306462}$ & $\num{13.54}$ \\
\midrule
\ele\mu & \inlinecode{bash}{/MuonEG/Run2017B-31Mar2018-v1/MINIAOD} & $\num{297046}-\num{299329}$ & $\num{4.823}$ \\
\ele\mu & \inlinecode{bash}{/MuonEG/Run2017C-31Mar2018-v1/MINIAOD} & $\num{299368}-\num{302029}$ & $\num{9.664}$ \\
\ele\mu & \inlinecode{bash}{/MuonEG/Run2017D-31Mar2018-v1/MINIAOD} & $\num{302030}-\num{303434}$ & $\num{4.252}$ \\
\ele\mu & \inlinecode{bash}{/MuonEG/Run2017E-31Mar2018-v1/MINIAOD} & $\num{303824}-\num{304797}$ & $\num{9.278}$ \\
\ele\mu & \inlinecode{bash}{/MuonEG/Run2017F-31Mar2018-v1/MINIAOD} & $\num{305040}-\num{306462}$ & $\num{13.54}$ \\
\bottomrule
\end{tabular}

\caption{Jeux de données utilisés en 2017.}
\label{tab-annexe-datasets-HTT-2017_data}
\end{table}
\begin{table}[p]
\centering
\begin{tabular}{clcc}
\toprule
Canal & Jeu de données & Gamme de \emph{run} & \Lumi\ (\SI{}{\femto\barn^{-1}})\\
\midrule
\tauh\tauh & \inlinecode{bash}{/Tau/Run2018A-17Sep2018-v1/MINIAOD} & $\num{315252}-\num{316995}$ & $\num{13.98}$ \\
\tauh\tauh & \inlinecode{bash}{/Tau/Run2018B-17Sep2018-v1/MINIAOD} & $\num{317080}-\num{319310}$ & $\num{7.064}$ \\
\tauh\tauh & \inlinecode{bash}{/Tau/Run2018C-17Sep2018-v1/MINIAOD} & $\num{319337}-\num{320065}$ & $\num{6.899}$ \\
\tauh\tauh & \inlinecode{bash}{/Tau/Run2018D-PromptReco-v2/MINIAOD} & $\num{320673}-\num{325175}$ & $\num{31.75}$ \\
\midrule
\mu\tauh & \inlinecode{bash}{/SingleMuon/Run2018A-17Sep2018-v1/MINIAOD} & $\num{315252}-\num{316995}$ & $\num{13.98}$ \\
\mu\tauh & \inlinecode{bash}{/SingleMuon/Run2018B-17Sep2018-v1/MINIAOD} & $\num{317080}-\num{319310}$ & $\num{7.064}$ \\
\mu\tauh & \inlinecode{bash}{/SingleMuon/Run2018C-17Sep2018-v1/MINIAOD} & $\num{319337}-\num{320065}$ & $\num{6.899}$ \\
\mu\tauh & \inlinecode{bash}{/SingleMuon/Run2018D-22Jan2019-v2/MINIAOD} & $\num{320673}-\num{325175}$ & $\num{31.75}$ \\
\midrule
\ele\tauh & \inlinecode{bash}{/EGamma/Run2018A-17Sep2018-v1/MINIAOD} & $\num{315252}-\num{316995}$ & $\num{13.98}$ \\
\ele\tauh & \inlinecode{bash}{/EGamma/Run2018B-17Sep2018-v1/MINIAOD} & $\num{317080}-\num{319310}$ & $\num{7.064}$ \\
\ele\tauh & \inlinecode{bash}{/EGamma/Run2018C-17Sep2018-v1/MINIAOD} & $\num{319337}-\num{320065}$ & $\num{6.899}$ \\
\ele\tauh & \inlinecode{bash}{/EGamma/Run2018D-22Jan2019-v2/MINIAOD} & $\num{320673}-\num{325175}$ & $\num{31.75}$ \\
\midrule
\ele\mu & \inlinecode{bash}{/MuonEG/Run2018A-17Sep2018-v1/MINIAOD} & $\num{315252}-\num{316995}$ & $\num{13.98}$ \\
\ele\mu & \inlinecode{bash}{/MuonEG/Run2018B-17Sep2018-v1/MINIAOD} & $\num{317080}-\num{319310}$ & $\num{7.064}$ \\
\ele\mu & \inlinecode{bash}{/MuonEG/Run2018C-17Sep2018-v1/MINIAOD} & $\num{319337}-\num{320065}$ & $\num{6.899}$ \\
\ele\mu & \inlinecode{bash}{/MuonEG/Run2018D-PromptReco-v2/MINIAOD} & $\num{320673}-\num{325175}$ & $\num{31.75}$ \\
\bottomrule
\end{tabular}

\caption{Jeux de données utilisés en 2018.}
\label{tab-annexe-datasets-HTT-2018_data}
\end{table}

\begin{table}[p]
\centering
\begin{tabular}{llc}
\toprule
Processus & Jeu de données simulées & $\sigma\times\BR$ (\SI{}{\pico\barn})\\
\midrule
$\gluon\gluon\higgs\to\tau\tau$ & \inlinecode{bash}{/GluGluHToTauTau_M125_13TeV}\up{$\dagger$}\up{1,2,3} & $\num{3.00}$ (N3LO) \\
VBF $\higgs\to\tau\tau$ & \inlinecode{bash}{/VBFHToTauTau_M125_13TeV}\up{$\dagger$}\up{1,2,3} & $\num{0.237}$ (NNLO) \\
$\Wbosonplus\higgs\to\tau\tau$ & \inlinecode{bash}{/WplusHToTauTau_M125_13TeV}\up{$\dagger$}\up{1} & $\num{0.0527}$ (NNLO) \\
$\Wbosonminus\higgs\to\tau\tau$ & \inlinecode{bash}{/WminusHToTauTau_M125_13TeV}\up{$\dagger$}\up{1} & $\num{0.0334}$ (NNLO) \\
$\Zboson\higgs\to\tau\tau$ & \inlinecode{bash}{/ZHToTauTau_M125_13TeV}\up{$\dagger$}\up{1} & $\num{0.0477}$ (NNLO) \\
$\gluon\gluon\Zboson\higgs\to\quark\quark\tau\tau$ & \inlinecode{bash}{/ggZH_HToTauTau_ZToQQ_M125_13TeV}\up{$\dagger$}\up{1} & $\num{0.0054}$ (NNLO) \\
$\gluon\gluon\Zboson\higgs\to\nu\nu\tau\tau$ & \inlinecode{bash}{/ggZH_HToTauTau_ZToNuNu_M125_13TeV}\up{$\dagger$}\up{1} & $\num{0.0015}$ (NNLO) \\
$\gluon\gluon\Zboson\higgs\to LL\tau\tau$ & \inlinecode{bash}{/ggZH_HToTauTau_ZToLL_M125_13TeV}\up{$\dagger$}\up{1} & $\num{0.0008}$ (NNLO) \\
$\gluon\gluon\to\higgs\to\Wboson\Wboson$ &\inlinecode{bash}{/GluGluHToWWTo2L2Nu_M125_13TeV}\up{$\ddagger$}\up{1} & $\num{1.09}$ (N3LO) \\
VBF $\higgs\to\Wboson\Wboson$ & \inlinecode{bash}{/VBFHToWWTo2L2Nu_M125_13TeV}\up{$\ddagger$}\up{1} & $\num{0.0850}$ (NNLO) \\
$\Wbosonplus\higgs\to\Wboson\Wboson$ & \inlinecode{bash}{/HWplusJ_HToWW_M125_13TeV}\up{$\dagger$}\up{1} & $\num{0.18}$ (NLO) \\
$\Wbosonminus\higgs\to\Wboson\Wboson$ & \inlinecode{bash}{/HWminusJ_HToWW_M125_13TeV}\up{$\dagger$}\up{1} & $\num{0.114}$ (NLO) \\
$\Zboson\higgs\to\Wboson\Wboson$ & \inlinecode{bash}{/HZJ_HToWW_M125_13TeV}\up{$\dagger$}\up{1} & $\num{0.163}$ (NLO) \\
$\gluon\gluon\Zboson\higgs\to\Wboson\Wboson$ & \inlinecode{bash}{/GluGluZH_HToWW_M125_13TeV}\up{$\dagger$}\up{1} & $\num{0.0262}$ (NLO) \\
$\ttbar\higgs\to\tau\tau$ & \inlinecode{bash}{/ttHJetToTT_M125_13TeV}\up{$||$}\up{4}& $\num{0.0318}$ (NLO) \\
\bottomrule
\end{tabular}
\begin{flushleft}\footnotesize
\up{1}\inlinecode{bash}{/RunIISummer16MiniAODv3-PUMoriond17_94X_mcRun2_asymptotic_v3-v*/MINIAODSIM} \\
\up{2}\inlinecode{bash}{/RunIISummer16MiniAODv3-PUMoriond17_94X_mcRun2_asymptotic_v3_ext1-v*/MINIAODSIM} \\
\up{3}\inlinecode{bash}{/RunIISummer16MiniAODv3-PUMoriond17_94X_mcRun2_asymptotic_v3_ext2-v*/MINIAODSIM} \\
\up{4}\inlinecode{bash}{/RunIISummer16MiniAODv3-PUMoriond17_94X_mcRun2_asymptotic_v3_ext4-v*/MINIAODSIM}

\begin{multicols}{2}
\up{$\dagger$}\inlinecode{bash}{_powheg_pythia8} \\
\up{$\ddagger$}\inlinecode{bash}{_powheg_JHUgenv628_pythia8} \\
\up{$||$}\inlinecode{bash}{_amcatnloFXFX_madspin_pythia8}
\end{multicols}
\end{flushleft}
\caption{Jeux de données simulées modélisant le boson de Higgs du modèle standard en 2016.}
\label{tab-annexe-datasets-HTT-2016_MC_signal_SM}
\end{table}
\begin{table}[p]
\centering
\begin{tabular}{ll}
\toprule
Processus & Jeu de données simulées \\
\midrule
$\gluon\gluon\to\Phi\to\tau\tau$ & \inlinecode{bash}{/SUSYGluGluToHToTauTau_M-*_TuneCUETP8M1_13TeV-pythia8}\up{1} \\ 
$\gluon\gluon\to\quarkb\quarkb\Phi\to\tau\tau$ & \inlinecode{bash}{/SUSYGluGluToBBHToTauTau_M-*_TuneCUETP8M1_13TeV-amcatnlo-pythia8}\up{1} \\
\bottomrule
\end{tabular}
\begin{flushleft}
\up{1}\inlinecode{bash}{/RunIISummer16MiniAODv3-PUMoriond17_94X_mcRun2_asymptotic_v3-v*/MINIAODSIM}
\end{flushleft}
\caption{Jeux de données simulées modélisant les bosons de Higgs neutres additionnels du MSSM en 2016.}
\label{tab-annexe-datasets-HTT-2016_MC_signal_MSSM}
\end{table}

\begin{table}[p]
\centering
\begin{tabular}{llc}
\toprule
Processus & Jeu de données simulées & $\sigma\times\BR$ (\SI{}{\pico\barn})\\
\midrule
$\gluon\gluon\to\higgs\to\tau\tau$ & \inlinecode{bash}{/GluGluHToTauTau_M125_13TeV}\up{$\dagger$}\up{1,2} & $\num{3.00}$ (N3LO) \\
% & \inlinecode{bash}{/GluGluToHToTauTau_M125_13TeV_amcatnloFXFX}\up{3} & $\num{3.00}$ (N3LO) \\
VBF $\higgs\to\tau\tau$ & \inlinecode{bash}{/VBFHToTauTau_M125_13TeV}\up{$\dagger$}\up{2} & $\num{0.237}$ (NNLO) \\
$\Wbosonplus\higgs\to\tau\tau$ & \inlinecode{bash}{/WplusHToTauTau_M125_13TeV}\up{$\dagger$}\up{3} & $\num{0.0527}$ (NNLO) \\
$\Wbosonminus\higgs\to\tau\tau$ & \inlinecode{bash}{/WminusHToTauTau_M125_13TeV}\up{$\dagger$}\up{3} & $\num{0.0334}$ (NNLO) \\
$\Zboson\higgs\to\tau\tau$ & \inlinecode{bash}{/ZHToTauTau_M125_13TeV}\up{$\dagger$}\up{3} & $\num{0.0477}$ (NNLO) \\
$\gluon\gluon\Zboson\higgs\to\quark\quark\tau\tau$ & \inlinecode{bash}{/ggZH_HToTauTau_ZToQQ_M125_13TeV}\up{$\dagger$}\up{3} & $\num{0.0054}$ (NNLO) \\
$\gluon\gluon\Zboson\higgs\to\nu\nu\tau\tau$ & \inlinecode{bash}{/ggZH_HToTauTau_ZToNuNu_M125_13TeV}\up{$\dagger$}\up{3} & $\num{0.0015}$ (NNLO) \\
$\gluon\gluon\Zboson\higgs\to LL\tau\tau$ & \inlinecode{bash}{/ggZH_HToTauTau_ZToLL_M125_13TeV}\up{$\dagger$}\up{3} & $\num{0.0008}$ (NNLO) \\
$\gluon\gluon\to\higgs\to\Wboson\Wboson$ &\inlinecode{bash}{/GluGluHToWWTo2L2Nu_M125_13TeV}\up{$\S$}\up{3} & $\num{1.09}$ (N3LO) \\
VBF $\higgs\to\Wboson\Wboson$ & \inlinecode{bash}{/VBFHToWWTo2L2Nu_M125_13TeV}\up{$\S$}\up{3} & $\num{0.0850}$ (NNLO) \\
$\Wbosonplus\higgs\to\Wboson\Wboson$ & \inlinecode{bash}{/HWplusJ_HToWW_M125_13TeV}\up{$\ddagger$}\up{3} & $\num{0.18}$ (NLO) \\
$\Wbosonminus\higgs\to\Wboson\Wboson$ & \inlinecode{bash}{/HWminusJ_HToWW_M125_13TeV}\up{$\ddagger$}\up{3} & $\num{0.114}$ (NLO) \\
$\Zboson\higgs\to\Wboson\Wboson$ & \inlinecode{bash}{/HZJ_HToWW_M125_13TeV}\up{$||$}\up{3} & $\num{0.163}$ (NLO) \\
$\gluon\gluon\Zboson\higgs\to\Wboson\Wboson$ & \inlinecode{bash}{/GluGluZH_HToWW_M125_13TeV}\up{$\ddagger$}\up{3} & $\num{0.0262}$ (NLO) \\
$\ttbar\higgs\to\tau\tau$ & \inlinecode{bash}{/ttHToTauTau_M125_TuneCP5_13TeV}\up{$\dagger$}\up{2}& $\num{0.0318}$ (NLO) \\
\bottomrule
\end{tabular}
\begin{flushleft}\footnotesize
\up{1}\inlinecode{bash}{/RunIIFall17MiniAODv2-PU2017_12Apr2018_94X_mc2017_realistic_v14_ext1-v*/MINIAODSIM} \\
\up{2}\inlinecode{bash}{/RunIIFall17MiniAODv2-PU2017_12Apr2018_new_pmx_94X_mc2017_realistic_v14-v*/MINIAODSIM} \\
\up{3}\inlinecode{bash}{/RunIIFall17MiniAODv2-PU2017_12Apr2018_94X_mc2017_realistic_v14-v*/MINIAODSIM}

\begin{multicols}{2}
\up{$\dagger$}\inlinecode{bash}{_powheg_pythia8} \\
\up{$\ddagger$}\inlinecode{bash}{_powheg_pythia8_TuneCP5} \\
\up{$\S$}\inlinecode{bash}{_powheg2_JHUGenV714_pythia8} \\
\up{$||$}\inlinecode{bash}{_powheg_JHUgenv714_pythia8_TuneCP5}
\end{multicols}
\end{flushleft}
\caption{Jeux de données simulées modélisant le boson de Higgs du modèle standard en 2017.}
\label{tab-annexe-datasets-HTT-2017_MC_signal_SM}
\end{table}
\begin{table}[p]
\centering
\begin{tabular}{ll}
\toprule
Processus & Jeu de données simulées\\
\midrule
%$\gluon\gluon\to\Phi\to\tau\tau$ & \inlinecode{bash}{/SUSYGluGluToHToTauTau_M-*_TuneCP5_13TeV-pythia8}\up{1} \\
%$\gluon\gluon\to\quarkb\quarkb\Phi\to\tau\tau$ & \inlinecode{bash}{/SUSYGluGluToBBHToTauTau_M-*_TuneCP5_13TeV-amcatnlo-pythia8}\up{1} \\
$\gluon\gluon\to\Phi\to\tau\tau$ & \inlinecode{bash}{/SUSYGluGluToHToTauTau_M-*_TuneCP5_13TeV-powheg-pythia8}\up{1} \\
$\gluon\gluon\to\quarkb\quarkb\Phi\to\tau\tau$ & \inlinecode{bash}{/SUSYGluGluToBBHToTauTau_M-*_TuneCP5_13TeV-powheg-pythia8}\up{1} \\
\bottomrule
\end{tabular}
\begin{flushleft}\footnotesize
\up{1}\inlinecode{bash}{/RunIIFall17MiniAODv2-PU2017_12Apr2018_94X_mc2017_realistic_v14-v1/MINIAODSIM}
\end{flushleft}
\caption{Jeux de données simulées modélisant les bosons de Higgs neutres additionnels du MSSM en 2017.}
\label{tab-annexe-datasets-HTT-2017_MC_signal_MSSM}
\end{table}

\begin{table}[p]
\centering
\begin{tabular}{llc}
\toprule
Processus & Jeu de données simulées & $\sigma\times\BR$ (\SI{}{\pico\barn})\\
\midrule
$\gluon\gluon\higgs\to\tau\tau$ & \inlinecode{bash}{/GluGluHToTauTau_M125_13TeV}\up{$\dagger$}\up{1} & $\num{3.00}$ (N3LO) \\
VBF $\higgs\to\tau\tau$ & \inlinecode{bash}{/VBFHToTauTau_M125_13TeV}\up{$\dagger$}\up{2} & $\num{0.237}$ (NNLO) \\
$\Wbosonplus\higgs\to\tau\tau$ & \inlinecode{bash}{/WplusHToTauTau_M125_13TeV}\up{$\dagger$}\up{1} & $\num{0.0527}$ (NNLO) \\
$\Wbosonminus\higgs\to\tau\tau$ & \inlinecode{bash}{/WminusHToTauTau_M125_13TeV}\up{$\dagger$}\up{1} & $\num{0.0334}$ (NNLO) \\
$\Zboson\higgs\to\tau\tau$ & \inlinecode{bash}{/ZHToTauTau_M125_13TeV}\up{$\dagger$}\up{1} & $\num{0.0477}$ (NNLO) \\
$\gluon\gluon\Zboson\higgs\to\quark\quark\tau\tau$ & \inlinecode{bash}{/ggZH_HToTauTau_ZToQQ_M125_13TeV}\up{$\dagger$}\up{1} & $\num{0.0054}$ (NNLO) \\
$\gluon\gluon\Zboson\higgs\to\nu\nu\tau\tau$ & \inlinecode{bash}{/ggZH_HToTauTau_ZToNuNu_M125_13TeV}\up{$\dagger$}\up{1} & $\num{0.0015}$ (NNLO) \\
$\gluon\gluon\Zboson\higgs\to LL\tau\tau$ & \inlinecode{bash}{/ggZH_HToTauTau_ZToLL_M125_13TeV}\up{$\dagger$}\up{1} & $\num{0.0008}$ (NNLO) \\
$\gluon\gluon\to\higgs\to\Wboson\Wboson$ &\inlinecode{bash}{/GluGluHToWWTo2L2Nu_M125_13TeV}\up{$\ddagger$}\up{1} & $\num{1.09}$ (N3LO) \\
VBF $\higgs\to\Wboson\Wboson$ & \inlinecode{bash}{/VBFHToWWTo2L2Nu_M125_13TeV}\up{$\ddagger$}\up{1} & $\num{0.0850}$ (NNLO) \\
$\Wbosonplus\higgs\to\Wboson\Wboson$ & \inlinecode{bash}{/HWplusJ_HToWW_M125_13TeV}\up{$\S$}\up{1} & $\num{0.18}$ (NLO) \\
$\Wbosonminus\higgs\to\Wboson\Wboson$ & \inlinecode{bash}{/HWminusJ_HToWW_M125_13TeV}\up{$\S$}\up{1} & $\num{0.114}$ (NLO) \\
$\Zboson\higgs\to\Wboson\Wboson$ & \inlinecode{bash}{/HZJ_HToWW_M125_13TeV}\up{$\S$}\up{1} & $\num{0.163}$ (NLO) \\
$\gluon\gluon\Zboson\higgs\to\Wboson\Wboson$ & \inlinecode{bash}{/GluGluZH_HToWW_M125_13TeV}\up{$||$}\up{1} & $\num{0.0262}$ (NLO) \\
$\ttbar\higgs\to\tau\tau$ & \inlinecode{bash}{/ttHToTauTau_M125_TuneCP5_13TeV}\up{$\dagger$}\up{1}& $\num{0.0318}$ (NLO) \\
\bottomrule
\end{tabular}
\begin{flushleft}\footnotesize
\up{1}\inlinecode{bash}{/RunIIAutumn18MiniAOD-102X_upgrade2018_realistic_v15-v*/MINIAODSIM}\\
\up{2}\inlinecode{bash}{/RunIIAutumn18MiniAOD-102X_upgrade2018_realistic_v15_ext1-v*/MINIAODSIM}


\begin{multicols}{2}
\up{$\dagger$}\inlinecode{bash}{_powheg_pythia8} \\
\up{$\ddagger$}\inlinecode{bash}{_powheg2_JHUGenV714_pythia8} \\
\up{$\S$}\inlinecode{bash}{_powheg_jhugen714_pythia8_TuneCP5} \\
\up{$||$}\inlinecode{bash}{_powheg_pythia8_TuneCP5_PSweights}
\end{multicols}
\end{flushleft}
\caption{Jeux de données simulées modélisant le boson de Higgs du modèle standard en 2018.}
\label{tab-annexe-datasets-HTT-2018_MC_signal_SM}
\end{table}
\begin{table}[p]
\centering
\begin{tabular}{ll}
\toprule
Processus & Jeu de données simulées \\
\midrule
$\gluon\gluon\to\Phi\to\tau\tau$ & \inlinecode{bash}{/SUSYGluGluToHToTauTau_M-*_TuneCP5_13TeV-pythia8}\up{1} \\
$\gluon\gluon\to\quarkb\quarkb\Phi\to\tau\tau$ & \inlinecode{bash}{/SUSYGluGluToBBHToTauTau_M-*_TuneCP5_13TeV-amcatnlo-pythia8}\up{1} \\
\bottomrule
\end{tabular}
\begin{flushleft}
\up{1}\inlinecode{bash}{/RunIIAutumn18MiniAOD-102X_upgrade2018_realistic_v15-v*/MINIAODSIM}
\end{flushleft}
\caption{Jeux de données simulées modélisant les bosons de Higgs neutres additionnels du MSSM en 2018.}
\label{tab-annexe-datasets-HTT-2018_MC_signal_MSSM}
\end{table}

\begin{table}[p]
\centering
\begin{tabular}{llc}
\toprule
Processus & Jeu de données simulées & $\sigma$ (\SI{}{\pico\barn})\\
\midrule
\ttbar & \inlinecode{bash}{/TTTo2L2Nu}\up{$\dagger$}\up{1} & $\num{88.29}$ \\
 & \inlinecode{bash}{/TTToHadronic}\up{$\dagger$}\up{1} & $\num{377.96}$ \\
 & \inlinecode{bash}{/TTToSemiLeptonic}\up{$\dagger$}\up{1} & $\num{365.35}$ \\
\emph{Single top} & \inlinecode{bash}{/ST_tW_antitop_5f_inclusiveDecays}\up{$\ddagger$}\up{2} & $\num{35.85}$ \\
 & \inlinecode{bash}{/ST_tW_top_5f_inclusiveDecays}\up{$\ddagger$}\up{2} & $\num{35.85}$ \\
 & \inlinecode{bash}{/ST_t-channel_top_4f_leptonDecays}\up{$\ddagger$}\up{1} & $\num{136.02}$ \\
 & \inlinecode{bash}{/ST_t-channel_antitop_4f_leptonDecays}\up{$\ddagger$}\up{1} & $\num{80.95}$ \\
$\Zboson \to LL$ & \inlinecode{bash}{/DYJetsToLL_M-50}\up{$\S$}\up{2,3} & $\num{6077.22}$ (NNLO) \\
$\Zboson+\text{1 jet}$ & \inlinecode{bash}{/DY1JetsToLL_M-50}\up{$\S$}\up{1} & $\num{1253.1}$\up{*} \\
$\Zboson+\text{2 jets}$ & \inlinecode{bash}{/DY2JetsToLL_M-50}\up{$\S$}\up{1} & $\num{409.4}$\up{*} \\
$\Zboson+\text{3 jets}$ & \inlinecode{bash}{/DY3JetsToLL_M-50}\up{$\S$}\up{1} & $\num{124.8}$\up{*} \\
$\Zboson+\text{4 jets}$ & \inlinecode{bash}{/DY4JetsToLL_M-50}\up{$\S$}\up{1} & $\num{67.33}$\up{*} \\
$\Zboson \to LL$ (basse masse) & \inlinecode{bash}{/DYJetsToLL_M-10to50}\up{$\S$}\up{1} & $\num{21658.0}$ (NLO)\\
$\Wboson+\text{jets}$ & \inlinecode{bash}{/WJetsToLNu}\up{$\S$}\up{1,3} & $\num{61526.7}$ (NNLO) \\
$\Wboson+\text{1 jet}$ & \inlinecode{bash}{/W1JetsToLNu}\up{$\S$}\up{1} & $\num{11805.6}$\up{*} \\
$\Wboson+\text{2 jets}$ & \inlinecode{bash}{/W2JetsToLNu}\up{$\S$}\up{2} & $\num{3891.0}$\up{*} \\
$\Wboson+\text{3 jets}$ & \inlinecode{bash}{/W3JetsToLNu}\up{$\S$}\up{2} & $\num{1153.2}$\up{*} \\
$\Wboson+\text{4 jets}$ & \inlinecode{bash}{/W4JetsToLNu}\up{$\S$}\up{2,3} & $\num{60.67}$\up{*} \\
WW & \inlinecode{bash}{/WW_TuneCP5_13TeV-pythia8}\up{1,2} & $\num{118.7}$ (NNLO) \\
WZ & \inlinecode{bash}{/WZ_TuneCP5_13TeV-pythia8}\up{1,2} & $\num{27.57}$ (LO) \\
ZZ & \inlinecode{bash}{/ZZ_TuneCP5_13TeV-pythia8}\up{1,2} & $\num{12.14}$ (LO) \\
VVTo2L2Nu & \inlinecode{bash}{/VVTo2L2Nu_13TeV}\up{$||$}\up{2} & $\num{13.84}$ \\
WZTo3LNu & \inlinecode{bash}{/WZTo3LNu_TuneCUETP8M1_13TeV}\up{$\P$}\up{1} & $\num{4.43}$ \\
WZTo2L2Q & \inlinecode{bash}{/WZTo2L2Q_13TeV}\up{$||$}\up{1} & $\num{5.52}$ \\
ZZTo2L2Q & \inlinecode{bash}{/ZZTo2L2Q_13TeV}\up{$||$}\up{1} & $\num{3.38}$ \\
ZZTo4L & \inlinecode{bash}{/ZZTo4L_13TeV}\up{$\P$}\up{2} & $\num{1.26}$ \\
EWK & \inlinecode{bash}{/EWKWMinus2Jets_WToLNu_M-50}\up{$\diamond$}\up{2,3} & $\num{23.24}$ (LO) \\
 & \inlinecode{bash}{/EWKWPlus2Jets_WToLNu_M-50}\up{$\diamond$}\up{2,3} & $\num{29.59}$ (LO) \\
 & \inlinecode{bash}{/EWKZ2Jets_ZToLL_M-50}\up{$\diamond$}\up{2,3} & $\num{4.321}$ (LO) \\
 & \inlinecode{bash}{/EWKZ2Jets_ZToNuNu}\up{$\diamond$}\up{2,3} & $\num{10.66}$ (LO) \\
$\Wboson\photon$ (canal \ele\mu) & \inlinecode{bash}{/WGToLNuG_01J_5f_TuneCUETP8M1_13TeV}\up{$\P$}\up{2,3,4} & $\num{178.4 }$ \\
$\Wboson\photon$ (canal \ele\mu) & \inlinecode{bash}{/WGstarToLNuMuMu_012Jets_13TeV-madgraph}\up{1} & $\num{2.793}$ \\
$\Wboson\photon$ (canal \ele\mu) & \inlinecode{bash}{/WGstarToLNuEE_012Jets_13TeV-madgraph}\up{1} & $\num{3.526}$ \\
\bottomrule
\end{tabular}
\begin{flushleft}\footnotesize
\up{1}\inlinecode{bash}{/RunIISummer16MiniAODv3-PUMoriond17_94X_mcRun2_asymptotic_v3-v*/MINIAODSIM} \\
\up{2}\inlinecode{bash}{/RunIISummer16MiniAODv3-PUMoriond17_94X_mcRun2_asymptotic_v3-ext1-v*/MINIAODSIM} \\
\up{3}\inlinecode{bash}{/RunIISummer16MiniAODv3-PUMoriond17_94X_mcRun2_asymptotic_v3-ext2-v*/MINIAODSIM} \\
\up{4}\inlinecode{bash}{/RunIISummer16MiniAODv3-PUMoriond17_94X_mcRun2_asymptotic_v3-ext3-v*/MINIAODSIM}

\begin{multicols}{2}
\up{$\dagger$}\inlinecode{bash}{_TuneCP5_PSweights_13TeV-powheg-pythia8} \\
\up{$\ddagger$}\inlinecode{bash}{_13TeV-powheg-pythia8_TuneCUETP8M1} \\
\up{$\S$}\inlinecode{bash}{_TuneCUETP8M1_13TeV-madgraphMLM-pythia8} \\
\up{$||$}\inlinecode{bash}{_amcatnloFXFX_madspin_pythia8} \\
\up{$\P$}\inlinecode{bash}{-amcatnloFXFX-pythia8} \\
\up{$\diamond$}\inlinecode{bash}{_TuneCP5_13TeV-madgraph-pythia8} \\
\up{*} Déterminée à partir de la section efficace du jeu inclusif.
\end{multicols}
\end{flushleft}
\caption{Jeux de données simulées modélisant le bruit de fond en 2016.}
\label{tab-annexe-datasets-HTT-2016_MC_backgrounds}
\end{table}
\begin{table}[p]
\centering
\begin{tabular}{llc}
\toprule
Processus & Jeu de données simulées & $\sigma$ (\SI{}{\pico\barn})\\
\midrule
%$\Zboson \to LL$ (basse masse) & \inlinecode{bash}{/DYJetsToLL_M-10to50}\up{$\dagger$}\up{1,2} & $\num{21658.0}$ (NLO) \\
$\Zboson \to LL$ (basse masse) & \inlinecode{bash}{/DYJetsToLL_M-10to50}\up{$\dagger$}\up{2} & $\num{21658.0}$ (NLO) \\
$\Zboson \to LL$ & \inlinecode{bash}{/DYJetsToLL_M-50}\up{$\dagger$}\up{3,4} & $\num{6077.22}$ (NNLO) \\
$\Zboson+\text{1 jet}$ & \inlinecode{bash}{/DY1JetsToLL_M-50}\up{$\dagger$}\up{5,8} & $\num{977.1}$\up{*} \\
$\Zboson+\text{2 jets}$ & \inlinecode{bash}{/DY2JetsToLL_M-50}\up{$\dagger$}\up{1,6} & $\num{347.3}$\up{*} \\
$\Zboson+\text{3 jets}$ & \inlinecode{bash}{/DY3JetsToLL_M-50}\up{$\dagger$}\up{1,2} & $\num{126.1}$\up{*} \\
$\Zboson+\text{4 jets}$ & \inlinecode{bash}{/DY4JetsToLL_M-50}\up{$\dagger$}\up{7} & $\num{71.67}$\up{*} \\
EWK $\Zboson\to LL$ & \inlinecode{bash}{/EWKZ2Jets_ZToLL_M-50}\up{$\ddagger$}\up{5} & $\num{4.321}$ (LO) \\
EWK $\Zboson\to \nu\nu$ & \inlinecode{bash}{/EWKZ2Jets_ZToNuNu}\up{$\ddagger$}\up{5} & $\num{10.66}$ (LO) \\
\ttbar & \inlinecode{bash}{/TTTo2L2Nu}\up{$\S$}\up{5} & $\num{88.29}$ \\
\ttbar & \inlinecode{bash}{/TTToHadronic}\up{$\S$}\up{5} & $\num{377.96}$ \\
\ttbar & \inlinecode{bash}{/TTToSemiLeptonic}\up{$\S$}\up{5} & $\num{365.35}$ \\
VVTo2L2Nu & \inlinecode{bash}{/VVTo2L2Nu_13TeV}\up{$||$}\up{1} & $\num{13.84}$ \\
%WW & \inlinecode{bash}{/WW_TuneCP5_13TeV-pythia8}\up{1} & $\num{118.7}$ (NNLO) \\
%WWTo1L1Nu2Q & \inlinecode{bash}{/WWTo1L1Nu2Q_13TeV}\up{$||$}\up{1} & $\num{49.997}$ \\
%WWTo2L2Nu & \inlinecode{bash}{/WWTo2L2Nu_NNPDF31}\up{$\S$}\up{1} & $\num{1.0}$ \\
%WZ & \inlinecode{bash}{/WZ_TuneCP5_13TeV-pythia8}\up{1} & $\num{27.57}$ (LO) \\
%WZTo1L1Nu2Q & \inlinecode{bash}{/WZTo1L1Nu2Q_13TeV}\up{1} & $\num{10.71}$ \\
%WZTo1L3Nu & \inlinecode{bash}{/WZTo1L3Nu_13TeV}\up{$||$}\inlinecode{bash}{_v2}\up{1} & $\num{3.05}$ \\
WZTo2L2Q & \inlinecode{bash}{/WZTo2L2Q_13TeV}\up{$||$}\up{1} & $\num{5.52}$ \\
WZTo3LNu & \inlinecode{bash}{/WZTo3LNu}\up{$\P$}\up{5} & $\num{4.43}$ \\
%ZZ & \inlinecode{bash}{/ZZ_TuneCP5_13TeV-pythia8}\up{5} & $\num{12.14}$ (LO) \\
%ZZTo2L2Nu & \inlinecode{bash}{/ZZTo2L2Nu_13TeV_powheg_pythia8}\up{1} & $\num{1.0}$ \\
ZZTo2L2Q & \inlinecode{bash}{/ZZTo2L2Q_13TeV}\up{$||$}\up{1} & $\num{3.38}$ \\
%ZZTo2Q2Nu & \inlinecode{bash}{/ZZTo2Q2Nu_TuneCP5_13TeV}\up{$||$}\up{1} & $\num{1.0}$ \\
%ZZTo4L & \inlinecode{bash}{/ZZTo4L}\up{$\P$}\up{1,2,5} & $\num{1.26}$ \\
ZZTo4L & \inlinecode{bash}{/ZZTo4L}\up{$\P$}\up{1} & $\num{1.26}$ \\
\emph{Single top} & \inlinecode{bash}{/ST_t-channel_antitop_4f_inclusiveDecays}\up{$\diamond$}\up{1} & $\num{80.95}$ \\
\emph{Single top} & \inlinecode{bash}{/ST_t-channel_top_4f_inclusiveDecays}\up{$\diamond$}\up{5} & $\num{136.02}$ \\
\emph{Single top} & \inlinecode{bash}{/ST_tW_antitop_5f_inclusiveDecays}\up{$\S$}\up{1} & $\num{35.85}$ \\
\emph{Single top} & \inlinecode{bash}{/ST_tW_top_5f_inclusiveDecays}\up{$\S$}\up{1} & $\num{35.85}$ \\
$\Wboson+\text{jets}$ & \inlinecode{bash}{/WJetsToLNu}\up{$\dagger$}\up{1,2} & $\num{61526.7}$ (NNLO) \\
$\Wboson+\text{1 jet}$ & \inlinecode{bash}{/W1JetsToLNu}\up{$\dagger$}\up{1} & $\num{9370.5}$\up{*} \\
$\Wboson+\text{2 jets}$ & \inlinecode{bash}{/W2JetsToLNu}\up{$\dagger$}\up{1} & $\num{3170.9}$\up{*} \\
$\Wboson+\text{3 jets}$ & \inlinecode{bash}{/W3JetsToLNu}\up{$\dagger$}\up{1} & $\num{1132.5}$\up{*} \\
$\Wboson+\text{4 jets}$ & \inlinecode{bash}{/W4JetsToLNu}\up{$\dagger$}\up{8} & $\num{631.5}$\up{*} \\
EWK \Wbosonminus & \inlinecode{bash}{/EWKWMinus2Jets_WToLNu_M-50}\up{$\ddagger$}\up{5} & $\num{23.24}$ (LO) \\
EWK \Wbosonplus & \inlinecode{bash}{/EWKWPlus2Jets_WToLNu_M-50}\up{$\ddagger$}\up{5} & $\num{29.59}$ (LO) \\
$\Wboson \photon$ (canal \ele\mu) & \inlinecode{bash}{/WGToLNuG}\up{$\dagger$}\up{1} & $\num{464.4}$ (LO) \\
\bottomrule
\end{tabular}
\begin{flushleft}\footnotesize
\up{1}\inlinecode{bash}{/RunIIFall17MiniAODv2-PU2017_12Apr2018_94X_mc2017_realistic_v14-v*/MINIAODSIM} \\
\up{2}\inlinecode{bash}{/RunIIFall17MiniAODv2-PU2017_12Apr2018_94X_mc2017_realistic_v14_ext1-v*/MINIAODSIM} \\
\up{3}\inlinecode{bash}{/RunIIFall17MiniAODv2-PU2017RECOSIMstep_12Apr2018_94X_mc2017_realistic_v14-v*/MINIAODSIM} \\
\up{4}\inlinecode{bash}{/RunIIFall17MiniAODv2-PU2017RECOSIMstep_12Apr2018_94X_mc2017_realistic_v14_ext1-v*/MINIAODSIM} \\
\up{5}\inlinecode{bash}{/RunIIFall17MiniAODv2-PU2017_12Apr2018_new_pmx_94X_mc2017_realistic_v14-v*/MINIAODSIM} \\
\up{6}\inlinecode{bash}{/RunIIFall17MiniAODv2-PU2017_12Apr2018_new_pmx_94X_mc2017_realistic_v14_ext1-v*/MINIAODSIM} \\
\up{7}\inlinecode{bash}{/RunIIFall17MiniAODv2-PU2017_12Apr2018_v2_94X_mc2017_realistic_v14-v*/MINIAODSIM} \\
\up{8}\inlinecode{bash}{/RunIIFall17MiniAODv2-PU2017_12Apr2018_v3_94X_mc2017_realistic_v14_ext1-v*/MINIAODSIM}

\begin{multicols}{2}
\up{$\dagger$}\inlinecode{bash}{_TuneCP5_13TeV-madgraphMLM-pythia8} \\
\up{$\ddagger$}\inlinecode{bash}{_TuneCP5_13TeV-madgraph-pythia8} \\
\up{$\S$}\inlinecode{bash}{_TuneCP5_13TeV-powheg-pythia8} \\
\up{$||$}\inlinecode{bash}{_amcatnloFXFX_madspin_pythia8} \\
\up{$\P$}\inlinecode{bash}{_TuneCP5_13TeV-amcatnloFXFX-pythia8} \\
\up{$\diamond$}\inlinecode{bash}{_TuneCP5_13TeV-powhegV2-madspin-pythia8} \\
\up{*} Déterminée à partir de la section efficace du jeu inclusif.
\end{multicols}
\end{flushleft}
\caption{Jeux de données simulées modélisant le bruit de fond en 2017.}
\label{tab-annexe-datasets-HTT-2017_MC_backgrounds}
\end{table}
\begin{table}[p]
\centering
\begin{tabular}{llc}
\toprule
Processus & Jeu de données simulées & $\sigma$ (\SI{}{\pico\barn})\\
\midrule
\ttbar & \inlinecode{bash}{/TTTo2L2Nu}\up{$\dagger$}\up{1} & $\num{88.29}$ \\
 & \inlinecode{bash}{/TTToHadronic}\up{$\dagger$}\up{1} & $\num{377.96}$ \\
 & \inlinecode{bash}{/TTToSemiLeptonic}\up{$\dagger$}\up{1} & $\num{365.35}$ \\
\emph{Single top} & \inlinecode{bash}{/ST_tW_antitop_5f_inclusiveDecays}\up{$\dagger$}\up{1} & $\num{35.85}$ \\
 & \inlinecode{bash}{/ST_tW_top_5f_inclusiveDecays}\up{$\dagger$}\up{1} & $\num{35.85}$ \\
 & \inlinecode{bash}{/ST_t-channel_top_4f_inclusiveDecays}\up{$\ddagger$}\up{1} & $\num{136.02}$ \\
 & \inlinecode{bash}{/ST_t-channel_antitop_4f_inclusiveDecays}\up{$\ddagger$}\up{1} & $\num{80.95}$ \\
$\Zboson \to LL$ & \inlinecode{bash}{/DYJetsToLL_M-50}\up{$\S$}\up{1} & $\num{6077.22}$ (NNLO) \\
$\Zboson+\text{1 jet}$ & \inlinecode{bash}{/DY1JetsToLL_M-50}\up{$\S$}\up{1} & $\num{1007.6}$\up{*} \\
$\Zboson+\text{2 jets}$ & \inlinecode{bash}{/DY2JetsToLL_M-50}\up{$\S$}\up{1} & $\num{344.3}$\up{*} \\
$\Zboson+\text{3 jets}$ & \inlinecode{bash}{/DY3JetsToLL_M-50}\up{$\S$}\up{1} & $\num{125.3}$\up{*} \\
$\Zboson+\text{4 jets}$ & \inlinecode{bash}{/DY4JetsToLL_M-50}\up{$\S$}\up{1} & $\num{71.20}$\up{*} \\
$\Zboson \to LL$ (basse masse) & \inlinecode{bash}{/DYJetsToLL_M-10to50}\up{$\S$}\up{1} & $\num{21658.0}$ (NLO) \\
$\Wboson+\text{jets}$ & \inlinecode{bash}{/WJetsToLNu}\up{$\S$}\up{1} & $\num{61526.7}$ (NNLO) \\
$\Wboson+\text{1 jet}$ & \inlinecode{bash}{/W1JetsToLNu}\up{$\S$}\up{1} & $\num{9328.1}$\up{*} \\
$\Wboson+\text{2 jets}$ & \inlinecode{bash}{/W2JetsToLNu}\up{$\S$}\up{1} & $\num{3181.5}$\up{*} \\
$\Wboson+\text{3 jets}$ & \inlinecode{bash}{/W3JetsToLNu}\up{$\S$}\up{1} & $\num{1116.2}$\up{*} \\
$\Wboson+\text{4 jets}$ & \inlinecode{bash}{/W4JetsToLNu}\up{$\S$}\up{1} & $\num{629.3}$\up{*} \\
WW & \inlinecode{bash}{/WW_TuneCP5_13TeV-pythia8}\up{1} & $\num{118.7}$ (NNLO) \\
WZ & \inlinecode{bash}{/WZ_TuneCP5_13TeV-pythia8}\up{1} & $\num{27.57}$ (LO) \\
ZZ & \inlinecode{bash}{/ZZ_TuneCP5_13TeV-pythia8}\up{1} & $\num{12.14}$ (LO) \\
VVTo2L2Nu & \inlinecode{bash}{/VVTo2L2Nu_13TeV}\up{$||$}\up{1} & $\num{13.84}$ \\
WZTo3LNu & \inlinecode{bash}{/WZTo3LNu_TuneCP5_13TeV}\up{$\P$}\up{2} & $\num{4.43}$ \\
WZTo2L2Q & \inlinecode{bash}{/WZTo2L2Q_13TeV}\up{$||$}\up{1} & $\num{5.52}$ \\
ZZTo2L2Q & \inlinecode{bash}{/ZZTo2L2Q_13TeV}\up{$||$}\up{1} & $\num{3.38}$ \\
ZZTo4L & \inlinecode{bash}{/ZZTo4L_TuneCP5_13TeV}\up{$\P$}\up{1} & $\num{1.26}$ \\
EWK \Wbosonminus & \inlinecode{bash}{/EWKWMinus2Jets_WToLNu_M-50}\up{$\diamond$}\up{1} & $\num{23.24}$ (LO) \\
EWK \Wbosonplus & \inlinecode{bash}{/EWKWPlus2Jets_WToLNu_M-50}\up{$\diamond$}\up{1} & $\num{29.59}$ (LO) \\
EWK $\Zboson\to LL$ & \inlinecode{bash}{/EWKZ2Jets_ZToLL_M-50}\up{$\diamond$}\up{1} & $\num{4.321}$ (LO) \\
EWK $\Zboson\to \nu\nu$ & \inlinecode{bash}{/EWKZ2Jets_ZToNuNu}\up{$\diamond$}\up{1} & $\num{10.66}$ (LO) \\
$\Wboson \photon$ (canal \ele\mu) & \inlinecode{bash}{/WGToLNuG}\up{$\S$}\up{1} & $\num{464.4}$ (LO) \\
\bottomrule
\end{tabular}
\begin{flushleft}\footnotesize
\up{1}\inlinecode{bash}{/RunIIAutumn18MiniAOD-102X_upgrade2018_realistic_v15-v*/MINIAODSIM} \\
\up{2}\inlinecode{bash}{/RunIIAutumn18MiniAOD-102X_upgrade2018_realistic_v15-ext1-v*/MINIAODSIM}

\begin{multicols}{2}
\up{$\dagger$}\inlinecode{bash}{_TuneCP5_13TeV-powheg-pythia8} \\
\up{$\ddagger$}\inlinecode{bash}{_TuneCP5_13TeV-powhegV2-madspin-pythia8} \\
\up{$\S$}\inlinecode{bash}{_TuneCP5_13TeV-madgraphMLM-pythia8} \\
\up{$||$}\inlinecode{bash}{_amcatnloFXFX_madspin_pythia8} \\
\up{$\P$}\inlinecode{bash}{-amcatnloFXFX-pythia8} \\
\up{$\diamond$}\inlinecode{bash}{_TuneCP5_13TeV-madgraph-pythia8} \\
\up{*} Déterminée à partir de la section efficace du jeu inclusif.
\end{multicols}
\end{flushleft}
\caption{Jeux de données simulées modélisant le bruit de fond en 2018.}
\label{tab-annexe-datasets-HTT-2018_MC_backgrounds}
\end{table}

\begin{table}[p]
\centering
\begin{tabular}{cl|cl}
\toprule
Canal & Jeu de données & Canal & Jeu de données \\
\midrule
\tauh\tauh & \inlinecode{bash}{/EmbeddingRun2016B/TauTau}\up{*} & \ele\mu & \inlinecode{bash}{/EmbeddingRun2016B/ElMu}\up{*} \\
\tauh\tauh & \inlinecode{bash}{/EmbeddingRun2016C/TauTau}\up{*} & \ele\mu & \inlinecode{bash}{/EmbeddingRun2016C/ElMu}\up{*} \\
\tauh\tauh & \inlinecode{bash}{/EmbeddingRun2016D/TauTau}\up{*} & \ele\mu & \inlinecode{bash}{/EmbeddingRun2016D/ElMu}\up{*} \\
\tauh\tauh & \inlinecode{bash}{/EmbeddingRun2016E/TauTau}\up{*} & \ele\mu & \inlinecode{bash}{/EmbeddingRun2016E/ElMu}\up{*} \\
\tauh\tauh & \inlinecode{bash}{/EmbeddingRun2016F/TauTau}\up{*} & \ele\mu & \inlinecode{bash}{/EmbeddingRun2016F/ElMu}\up{*} \\
\tauh\tauh & \inlinecode{bash}{/EmbeddingRun2016G/TauTau}\up{*} & \ele\mu & \inlinecode{bash}{/EmbeddingRun2016G/ElMu}\up{*} \\
\tauh\tauh & \inlinecode{bash}{/EmbeddingRun2016H/TauTau}\up{*} & \ele\mu & \inlinecode{bash}{/EmbeddingRun2016H/ElMu}\up{*} \\
\midrule
\mu\tauh & \inlinecode{bash}{/EmbeddingRun2016B/MuTau}\up{*} & \ele\tauh & \inlinecode{bash}{/EmbeddingRun2016B/ElTau}\up{*} \\
\mu\tauh & \inlinecode{bash}{/EmbeddingRun2016C/MuTau}\up{*} & \ele\tauh & \inlinecode{bash}{/EmbeddingRun2016C/ElTau}\up{*} \\
\mu\tauh & \inlinecode{bash}{/EmbeddingRun2016D/MuTau}\up{*} & \ele\tauh & \inlinecode{bash}{/EmbeddingRun2016D/ElTau}\up{*} \\
\mu\tauh & \inlinecode{bash}{/EmbeddingRun2016E/MuTau}\up{*} & \ele\tauh & \inlinecode{bash}{/EmbeddingRun2016E/ElTau}\up{*} \\
\mu\tauh & \inlinecode{bash}{/EmbeddingRun2016F/MuTau}\up{*} & \ele\tauh & \inlinecode{bash}{/EmbeddingRun2016F/ElTau}\up{*} \\
\mu\tauh & \inlinecode{bash}{/EmbeddingRun2016G/MuTau}\up{*} & \ele\tauh & \inlinecode{bash}{/EmbeddingRun2016G/ElTau}\up{*} \\
\mu\tauh & \inlinecode{bash}{/EmbeddingRun2016H/MuTau}\up{*} & \ele\tauh & \inlinecode{bash}{/EmbeddingRun2016H/ElTau}\up{*} \\
\bottomrule
\end{tabular}
\begin{flushleft}
\up{*}\inlinecode{bash}{FinalState-inputDoubleMu_94X_Legacy_miniAOD-v5/USER}
\end{flushleft}
\caption{Jeux de données encapsulées en 2016.}
\label{tab-annexe-datasets-HTT-2016_embedded}
\end{table}
\begin{table}[p]
\centering
\begin{tabular}{cl|cl}
\toprule
Canal & Jeu de données & Canal & Jeu de données \\
\midrule
\tauh\tauh & \inlinecode{bash}{/EmbeddingRun2017B/TauTau}\up{*} & \ele\mu & \inlinecode{bash}{/EmbeddingRun2017B/ElMu}\up{*} \\
\tauh\tauh & \inlinecode{bash}{/EmbeddingRun2017C/TauTau}\up{*} & \ele\mu & \inlinecode{bash}{/EmbeddingRun2017C/ElMu}\up{*} \\
\tauh\tauh & \inlinecode{bash}{/EmbeddingRun2017D/TauTau}\up{*} & \ele\mu & \inlinecode{bash}{/EmbeddingRun2017D/ElMu}\up{*} \\
\tauh\tauh & \inlinecode{bash}{/EmbeddingRun2017E/TauTau}\up{*} & \ele\mu & \inlinecode{bash}{/EmbeddingRun2017E/ElMu}\up{*} \\
\tauh\tauh & \inlinecode{bash}{/EmbeddingRun2017F/TauTau}\up{*} & \ele\mu & \inlinecode{bash}{/EmbeddingRun2017F/ElMu}\up{*} \\
\midrule
\mu\tauh & \inlinecode{bash}{/EmbeddingRun2017B/MuTau}\up{*} & \ele\tauh & \inlinecode{bash}{/EmbeddingRun2017B/ElTau}\up{*} \\
\mu\tauh & \inlinecode{bash}{/EmbeddingRun2017C/MuTau}\up{*} & \ele\tauh & \inlinecode{bash}{/EmbeddingRun2017C/ElTau}\up{*} \\
\mu\tauh & \inlinecode{bash}{/EmbeddingRun2017D/MuTau}\up{*} & \ele\tauh & \inlinecode{bash}{/EmbeddingRun2017D/ElTau}\up{*} \\
\mu\tauh & \inlinecode{bash}{/EmbeddingRun2017E/MuTau}\up{*} & \ele\tauh & \inlinecode{bash}{/EmbeddingRun2017E/ElTau}\up{*} \\
\mu\tauh & \inlinecode{bash}{/EmbeddingRun2017F/MuTau}\up{*} & \ele\tauh & \inlinecode{bash}{/EmbeddingRun2017F/ElTau}\up{*} \\
\bottomrule
\end{tabular}
\up{*}\inlinecode{bash}{FinalState-inputDoubleMu_94X_miniAOD-v2/USER}
\caption{Jeux de données encapsulées en 2017.}
\label{tab-annexe-datasets-HTT-2017_embedded}
\end{table}
\begin{table}[p]
\centering
\begin{tabular}{cl|cl}
\toprule
Canal & Jeu de données & Canal & Jeu de données \\
\midrule
\tauh\tauh & \inlinecode{bash}{/EmbeddingRun2018A/TauTau}\up{*} & \ele\mu & \inlinecode{bash}{/EmbeddingRun2018A/ElMu}\up{*} \\
\tauh\tauh & \inlinecode{bash}{/EmbeddingRun2018B/TauTau}\up{*} & \ele\mu & \inlinecode{bash}{/EmbeddingRun2018B/ElMu}\up{*} \\
\tauh\tauh & \inlinecode{bash}{/EmbeddingRun2018C/TauTau}\up{*} & \ele\mu & \inlinecode{bash}{/EmbeddingRun2018C/ElMu}\up{*} \\
\tauh\tauh & \inlinecode{bash}{/EmbeddingRun2018D/TauTau}\up{*} & \ele\mu & \inlinecode{bash}{/EmbeddingRun2018D/ElMu}\up{*} \\
\midrule
\mu\tauh & \inlinecode{bash}{/EmbeddingRun2018A/MuTau}\up{*} & \ele\tauh & \inlinecode{bash}{/EmbeddingRun2018A/ElTau}\up{*} \\
\mu\tauh & \inlinecode{bash}{/EmbeddingRun2018B/MuTau}\up{*} & \ele\tauh & \inlinecode{bash}{/EmbeddingRun2018B/ElTau}\up{*} \\
\mu\tauh & \inlinecode{bash}{/EmbeddingRun2018C/MuTau}\up{*} & \ele\tauh & \inlinecode{bash}{/EmbeddingRun2018C/ElTau}\up{*} \\
\mu\tauh & \inlinecode{bash}{/EmbeddingRun2018D/MuTau}\up{*} & \ele\tauh & \inlinecode{bash}{/EmbeddingRun2018D/ElTau}\up{*} \\
\bottomrule
\end{tabular}
\begin{flushleft}
\up{*}\inlinecode{bash}{FinalState-inputDoubleMu_102X_miniAOD-v1/USER}
\end{flushleft}
\caption{Jeux de données encapsulées en 2018.}
\label{tab-annexe-datasets-HTT-2018_embedded}
\end{table}
