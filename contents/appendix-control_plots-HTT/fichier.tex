\ifthenelse{\equal{\PRINTABLE}{false}}{
\chapter{Distributions de contrôle -- $\Higgs\to\tau\tau$}\label{annexe-control_plots-HTT}

Cette annexe présente des distributions de contrôle
sur les événements utilisés dans l'analyse des événements $\Higgs\to\tau\tau$
présentée dans le chapitre~\refChHTT.
La sélection est \og inclusive \fg, les événements sont ceux sélectionnés par la définition de la région de signal,
sans coupure sur $\mT^{\ell}$ (canaux \mu\tauh, \ele\tauh) ni $\Dzeta$ (canal \ele\mu).
\par
Pour chacune des trois années de prise de données (2016, 2017, 2018)
et
chacun des quatre canaux (\tauh\tauh, \mu\tauh, \ele\tauh, \ele\mu),
les distributions de plusieurs variables sont données.
\par
Dans chacun des graphiques,
les données observées (points noirs) sont comparées à la modélisation des bruits de fond (histogrammes remplis en couleur et empilés).
Les bandes grisées correspondent à l'incertitude statistique totale sur le bruit de fond.
Le rapport au bruit de fond est donné dans la partie inférieure des graphiques.

\def\EMBFFchoice{emb_ff}
\def\lolcalcurrentyear{2016}
\def\lolcalcurrentchannel{tt}
\ifthenelse{\equal{\lolcalcurrentchannel}{tt}}{\def\localchannel{\tauh\tauh}\def\localLA{\ensuremath{\tauh^{(1)}}}\def\localLB{\ensuremath{\tauh^{(2)}}}}{}
\ifthenelse{\equal{\lolcalcurrentchannel}{mt}}{\def\localchannel{\mu\tauh}\def\localLA{\mu}\def\localLB{\ensuremath{\tauh}}}{}
\ifthenelse{\equal{\lolcalcurrentchannel}{et}}{\def\localchannel{\ele\tauh}\def\localLA{\ele}\def\localLB{\ensuremath{\tauh}}}{}
\ifthenelse{\equal{\lolcalcurrentchannel}{em}}{\def\localchannel{\ele\mu}\def\localLA{\ele}\def\localLB{\mu}}{}

\begin{figure}[p]
\centering

\subcaptionbox{Impulsion transverse du jet principal.}[.475\textwidth]
{\plotHTTcontrol{\lolcalcurrentyear}{\EMBFFchoice}{\lolcalcurrentchannel}{jpt_1}\vspace{-.125\baselineskip}}
\hfill
\subcaptionbox{Impulsion transverse du jet secondaire.}[.475\textwidth]
{\plotHTTcontrol{\lolcalcurrentyear}{\EMBFFchoice}{\lolcalcurrentchannel}{jpt_2}\vspace{-.125\baselineskip}}

\subcaptionbox{Pseudo-rapidité du jet principal.}[.475\textwidth]
{\plotHTTcontrol{\lolcalcurrentyear}{\EMBFFchoice}{\lolcalcurrentchannel}{jeta_1}\vspace{-.125\baselineskip}}
\hfill
\subcaptionbox{Pseudo-rapidité du jet secondaire.}[.475\textwidth]
{\plotHTTcontrol{\lolcalcurrentyear}{\EMBFFchoice}{\lolcalcurrentchannel}{jeta_2}\vspace{-.125\baselineskip}}

\subcaptionbox{Angle azimutal du jet principal.}[.475\textwidth]
{\plotHTTcontrol{\lolcalcurrentyear}{\EMBFFchoice}{\lolcalcurrentchannel}{jphi_1}\vspace{-.125\baselineskip}}
\hfill
\subcaptionbox{Angle azimutal du jet secondaire.}[.475\textwidth]
{\plotHTTcontrol{\lolcalcurrentyear}{\EMBFFchoice}{\lolcalcurrentchannel}{jphi_2}\vspace{-.125\baselineskip}}

\caption[Distributions de contrôle, \lolcalcurrentyear\ \localchannel, cinématique des deux jets principaux.]{Canal \localchannel, \lolcalcurrentyear: cinématique des deux jets principaux.}
\end{figure}

\begin{figure}[p]
\centering

\subcaptionbox{Impulsion transverse du \quarkb-jet principal.}[.475\textwidth]
{\plotHTTcontrol{\lolcalcurrentyear}{\EMBFFchoice}{\lolcalcurrentchannel}{bpt_1}\vspace{-.125\baselineskip}}
\hfill
\subcaptionbox{Impulsion transverse du \quarkb-jet secondaire.}[.475\textwidth]
{\plotHTTcontrol{\lolcalcurrentyear}{\EMBFFchoice}{\lolcalcurrentchannel}{bpt_2}\vspace{-.125\baselineskip}}

\subcaptionbox{Impulsion transverse de l'AHA.}[.475\textwidth]
{\plotHTTcontrol{\lolcalcurrentyear}{\EMBFFchoice _jets_r}{\lolcalcurrentchannel}{jpt_r}\vspace{-.125\baselineskip}}
\hfill
\subcaptionbox{Pseudo-rapidité de l'AHA.}[.475\textwidth]
{\plotHTTcontrol{\lolcalcurrentyear}{\EMBFFchoice _jets_r}{\lolcalcurrentchannel}{jeta_r}\vspace{-.125\baselineskip}}

\subcaptionbox{Angle azimutal de l'AHA.}[.475\textwidth]
{\plotHTTcontrol{\lolcalcurrentyear}{\EMBFFchoice _jets_r}{\lolcalcurrentchannel}{jphi_r}\vspace{-.125\baselineskip}}
\hfill
\subcaptionbox{Nombre de jets dans l'AHA.}[.475\textwidth]
{\plotHTTcontrol{\lolcalcurrentyear}{\EMBFFchoice _jets_r}{\lolcalcurrentchannel}{Njet_r}\vspace{-.125\baselineskip}}

\caption[Distributions de contrôle, \lolcalcurrentyear\ \localchannel, \quarkb-jets et activité hadronique additionnelle.]{Canal \localchannel, \lolcalcurrentyear: \quarkb-jets et activité hadronique additionnelle.}
\end{figure}

\begin{figure}[p]
\centering

\subcaptionbox{Nombre de \quarkb-jets.}[.475\textwidth]
{\plotHTTcontrol{\lolcalcurrentyear}{\EMBFFchoice}{\lolcalcurrentchannel}{nbtag}\vspace{-.125\baselineskip}}
\hfill
\subcaptionbox{Nombre de jets.}[.475\textwidth]
{\plotHTTcontrol{\lolcalcurrentyear}{\EMBFFchoice}{\lolcalcurrentchannel}{njets}\vspace{-.125\baselineskip}}

\subcaptionbox{Impulsion transverse du système des deux jets.}[.475\textwidth]
{\plotHTTcontrol{\lolcalcurrentyear}{\EMBFFchoice}{\lolcalcurrentchannel}{dijetpt}\vspace{-.125\baselineskip}}
\hfill
\subcaptionbox{Distance en $\eta$ entre les deux jets.}[.475\textwidth]
{\plotHTTcontrol{\lolcalcurrentyear}{\EMBFFchoice}{\lolcalcurrentchannel}{jdeta}\vspace{-.125\baselineskip}}

\subcaptionbox{Masse invariante du système des deux jets.}[.475\textwidth]
{\plotHTTcontrol{\lolcalcurrentyear}{\EMBFFchoice}{\lolcalcurrentchannel}{mjj}\vspace{-.125\baselineskip}}
\hfill
\subcaptionbox{Nombre de vertex d'empilement.}[.475\textwidth]
{\plotHTTcontrol{\lolcalcurrentyear}{\EMBFFchoice}{\lolcalcurrentchannel}{npv}\vspace{-.125\baselineskip}}

\caption[Distributions de contrôle, \lolcalcurrentyear\ \localchannel, nombre de jets, système des deux jets principaux et empilement.]{Canal \localchannel, \lolcalcurrentyear: nombre de jets, système des deux jets principaux et empilement.}
\end{figure}


\begin{figure}[p]
\centering

\subcaptionbox{Impulsion transverse du lepton 1 (\localLA).}[.475\textwidth]
{\plotHTTcontrol{\lolcalcurrentyear}{\EMBFFchoice}{\lolcalcurrentchannel}{pt_1}\vspace{-.125\baselineskip}}
\hfill
\subcaptionbox{Impulsion transverse du lepton 2 (\localLB).}[.475\textwidth]
{\plotHTTcontrol{\lolcalcurrentyear}{\EMBFFchoice}{\lolcalcurrentchannel}{pt_2}\vspace{-.125\baselineskip}}

\subcaptionbox{Pseudo-rapidité du lepton 1 (\localLA).}[.475\textwidth]
{\plotHTTcontrol{\lolcalcurrentyear}{\EMBFFchoice}{\lolcalcurrentchannel}{eta_1}\vspace{-.125\baselineskip}}
\hfill
\subcaptionbox{Pseudo-rapidité du lepton 2 (\localLB).}[.475\textwidth]
{\plotHTTcontrol{\lolcalcurrentyear}{\EMBFFchoice}{\lolcalcurrentchannel}{eta_2}\vspace{-.125\baselineskip}}

\subcaptionbox{Angle azimutal du lepton 1 (\localLA).}[.475\textwidth]
{\plotHTTcontrol{\lolcalcurrentyear}{\EMBFFchoice}{\lolcalcurrentchannel}{phi_1}\vspace{-.125\baselineskip}}
\hfill
\subcaptionbox{Angle azimutal du lepton 2 (\localLB).}[.475\textwidth]
{\plotHTTcontrol{\lolcalcurrentyear}{\EMBFFchoice}{\lolcalcurrentchannel}{phi_2}\vspace{-.125\baselineskip}}

\caption[Distributions de contrôle, \lolcalcurrentyear\ \localchannel, cinématique des leptons (\localLA, \localLB).]{Canal \localchannel, \lolcalcurrentyear: cinématique des leptons (\localLA, \localLB).}
\end{figure}

\begin{figure}[p]
\centering

\subcaptionbox{Énergie transverse manquante.}[.475\textwidth]
{\plotHTTcontrol{\lolcalcurrentyear}{\EMBFFchoice}{\lolcalcurrentchannel}{puppimet}\vspace{-.125\baselineskip}}
\hfill
\subcaptionbox{Masse transverse du \emph{dilepton}.}[.475\textwidth]
{\plotHTTcontrol{\lolcalcurrentyear}{\EMBFFchoice}{\lolcalcurrentchannel}{mTdileptonMET_puppi}\vspace{-.125\baselineskip}}

\subcaptionbox{Impulsion transverse du \emph{dilepton}.}[.475\textwidth]
{\plotHTTcontrol{\lolcalcurrentyear}{\EMBFFchoice}{\lolcalcurrentchannel}{ptvis}\vspace{-.125\baselineskip}}
\hfill
\subcaptionbox{Masse visible du \emph{dilepton}.}[.475\textwidth]
{\plotHTTcontrol{\lolcalcurrentyear}{\EMBFFchoice}{\lolcalcurrentchannel}{m_vis}\vspace{-.125\baselineskip}}

\subcaptionbox{Impulsion transverse du système di-\tau.}[.475\textwidth]
{\plotHTTcontrol{\lolcalcurrentyear}{\EMBFFchoice}{\lolcalcurrentchannel}{pt_tt_puppi}\vspace{-.125\baselineskip}}
\hfill
\subcaptionbox{Distance $\Delta R$ entre les leptons (\localLA, \localLB).}[.475\textwidth]
{\plotHTTcontrol{\lolcalcurrentyear}{\EMBFFchoice}{\lolcalcurrentchannel}{DiTauDeltaR}\vspace{-.125\baselineskip}}

\caption[Distributions de contrôle, \lolcalcurrentyear\ \localchannel, \emph{dilepton} et énergie transverse manquante.]{Canal \localchannel, \lolcalcurrentyear: \emph{dilepton} et énergie transverse manquante.}
\end{figure}


\begin{figure}[p]
\centering

\subcaptionbox{Masse transverse du lepton 1 (\localLA).}[.475\textwidth]
{\plotHTTcontrol{\lolcalcurrentyear}{\EMBFFchoice}{\lolcalcurrentchannel}{mt_1_puppi}\vspace{-.125\baselineskip}}
\hfill
\subcaptionbox{Masse transverse du lepton 2 (\localLB).}[.475\textwidth]
{\plotHTTcontrol{\lolcalcurrentyear}{\EMBFFchoice}{\lolcalcurrentchannel}{mt_2_puppi}\vspace{-.125\baselineskip}}

\subcaptionbox{Valeur de \Dzeta.}[.475\textwidth]
{\plotHTTcontrol{\lolcalcurrentyear}{\EMBFFchoice}{\lolcalcurrentchannel}{pZetaPuppiMissVis}\vspace{-.125\baselineskip}}
\hfill
\subcaptionbox{Masse transverse totale.}[.475\textwidth]
{\plotHTTcontrol{\lolcalcurrentyear}{\EMBFFchoice}{\lolcalcurrentchannel}{mt_tot_puppi}\vspace{-.125\baselineskip}}


\subcaptionbox{Masse du système di-\tau\ d'après \SVFIT.}[.475\textwidth]
{\plotHTTcontrol{\lolcalcurrentyear}{\EMBFFchoice}{\lolcalcurrentchannel}{m_sv_puppi}\vspace{-.125\baselineskip}}
\hfill
\subcaptionbox{Masse du système di-\tau\ d'après le ML.}[.475\textwidth]
{\plotHTTcontrol{\lolcalcurrentyear}{\EMBFFchoice}{\lolcalcurrentchannel}{ml_mass}\vspace{-.125\baselineskip}}

\caption[Distributions de contrôle, \lolcalcurrentyear\ \localchannel, masses transverses, \Dzeta\ et masses.]{Canal \localchannel, \lolcalcurrentyear: masses transverses, \Dzeta\ et masses.}
\end{figure}

\def\lolcalcurrentchannel{mt}
\ifthenelse{\equal{\lolcalcurrentchannel}{tt}}{\def\localchannel{\tauh\tauh}\def\localLA{\ensuremath{\tauh^{(1)}}}\def\localLB{\ensuremath{\tauh^{(2)}}}}{}
\ifthenelse{\equal{\lolcalcurrentchannel}{mt}}{\def\localchannel{\mu\tauh}\def\localLA{\mu}\def\localLB{\ensuremath{\tauh}}}{}
\ifthenelse{\equal{\lolcalcurrentchannel}{et}}{\def\localchannel{\ele\tauh}\def\localLA{\ele}\def\localLB{\ensuremath{\tauh}}}{}
\ifthenelse{\equal{\lolcalcurrentchannel}{em}}{\def\localchannel{\ele\mu}\def\localLA{\ele}\def\localLB{\mu}}{}

\begin{figure}[p]
\centering

\subcaptionbox{Impulsion transverse du jet principal.}[.475\textwidth]
{\plotHTTcontrol{\lolcalcurrentyear}{\EMBFFchoice}{\lolcalcurrentchannel}{jpt_1}\vspace{-.125\baselineskip}}
\hfill
\subcaptionbox{Impulsion transverse du jet secondaire.}[.475\textwidth]
{\plotHTTcontrol{\lolcalcurrentyear}{\EMBFFchoice}{\lolcalcurrentchannel}{jpt_2}\vspace{-.125\baselineskip}}

\subcaptionbox{Pseudo-rapidité du jet principal.}[.475\textwidth]
{\plotHTTcontrol{\lolcalcurrentyear}{\EMBFFchoice}{\lolcalcurrentchannel}{jeta_1}\vspace{-.125\baselineskip}}
\hfill
\subcaptionbox{Pseudo-rapidité du jet secondaire.}[.475\textwidth]
{\plotHTTcontrol{\lolcalcurrentyear}{\EMBFFchoice}{\lolcalcurrentchannel}{jeta_2}\vspace{-.125\baselineskip}}

\subcaptionbox{Angle azimutal du jet principal.}[.475\textwidth]
{\plotHTTcontrol{\lolcalcurrentyear}{\EMBFFchoice}{\lolcalcurrentchannel}{jphi_1}\vspace{-.125\baselineskip}}
\hfill
\subcaptionbox{Angle azimutal du jet secondaire.}[.475\textwidth]
{\plotHTTcontrol{\lolcalcurrentyear}{\EMBFFchoice}{\lolcalcurrentchannel}{jphi_2}\vspace{-.125\baselineskip}}

\caption[Distributions de contrôle, \lolcalcurrentyear\ \localchannel, cinématique des deux jets principaux.]{Canal \localchannel, \lolcalcurrentyear: cinématique des deux jets principaux.}
\end{figure}

\begin{figure}[p]
\centering

\subcaptionbox{Impulsion transverse du \quarkb-jet principal.}[.475\textwidth]
{\plotHTTcontrol{\lolcalcurrentyear}{\EMBFFchoice}{\lolcalcurrentchannel}{bpt_1}\vspace{-.125\baselineskip}}
\hfill
\subcaptionbox{Impulsion transverse du \quarkb-jet secondaire.}[.475\textwidth]
{\plotHTTcontrol{\lolcalcurrentyear}{\EMBFFchoice}{\lolcalcurrentchannel}{bpt_2}\vspace{-.125\baselineskip}}

\subcaptionbox{Impulsion transverse de l'AHA.}[.475\textwidth]
{\plotHTTcontrol{\lolcalcurrentyear}{\EMBFFchoice _jets_r}{\lolcalcurrentchannel}{jpt_r}\vspace{-.125\baselineskip}}
\hfill
\subcaptionbox{Pseudo-rapidité de l'AHA.}[.475\textwidth]
{\plotHTTcontrol{\lolcalcurrentyear}{\EMBFFchoice _jets_r}{\lolcalcurrentchannel}{jeta_r}\vspace{-.125\baselineskip}}

\subcaptionbox{Angle azimutal de l'AHA.}[.475\textwidth]
{\plotHTTcontrol{\lolcalcurrentyear}{\EMBFFchoice _jets_r}{\lolcalcurrentchannel}{jphi_r}\vspace{-.125\baselineskip}}
\hfill
\subcaptionbox{Nombre de jets dans l'AHA.}[.475\textwidth]
{\plotHTTcontrol{\lolcalcurrentyear}{\EMBFFchoice _jets_r}{\lolcalcurrentchannel}{Njet_r}\vspace{-.125\baselineskip}}

\caption[Distributions de contrôle, \lolcalcurrentyear\ \localchannel, \quarkb-jets et activité hadronique additionnelle.]{Canal \localchannel, \lolcalcurrentyear: \quarkb-jets et activité hadronique additionnelle.}
\end{figure}

\begin{figure}[p]
\centering

\subcaptionbox{Nombre de \quarkb-jets.}[.475\textwidth]
{\plotHTTcontrol{\lolcalcurrentyear}{\EMBFFchoice}{\lolcalcurrentchannel}{nbtag}\vspace{-.125\baselineskip}}
\hfill
\subcaptionbox{Nombre de jets.}[.475\textwidth]
{\plotHTTcontrol{\lolcalcurrentyear}{\EMBFFchoice}{\lolcalcurrentchannel}{njets}\vspace{-.125\baselineskip}}

\subcaptionbox{Impulsion transverse du système des deux jets.}[.475\textwidth]
{\plotHTTcontrol{\lolcalcurrentyear}{\EMBFFchoice}{\lolcalcurrentchannel}{dijetpt}\vspace{-.125\baselineskip}}
\hfill
\subcaptionbox{Distance en $\eta$ entre les deux jets.}[.475\textwidth]
{\plotHTTcontrol{\lolcalcurrentyear}{\EMBFFchoice}{\lolcalcurrentchannel}{jdeta}\vspace{-.125\baselineskip}}

\subcaptionbox{Masse invariante du système des deux jets.}[.475\textwidth]
{\plotHTTcontrol{\lolcalcurrentyear}{\EMBFFchoice}{\lolcalcurrentchannel}{mjj}\vspace{-.125\baselineskip}}
\hfill
\subcaptionbox{Nombre de vertex d'empilement.}[.475\textwidth]
{\plotHTTcontrol{\lolcalcurrentyear}{\EMBFFchoice}{\lolcalcurrentchannel}{npv}\vspace{-.125\baselineskip}}

\caption[Distributions de contrôle, \lolcalcurrentyear\ \localchannel, nombre de jets, système des deux jets principaux et empilement.]{Canal \localchannel, \lolcalcurrentyear: nombre de jets, système des deux jets principaux et empilement.}
\end{figure}


\begin{figure}[p]
\centering

\subcaptionbox{Impulsion transverse du lepton 1 (\localLA).}[.475\textwidth]
{\plotHTTcontrol{\lolcalcurrentyear}{\EMBFFchoice}{\lolcalcurrentchannel}{pt_1}\vspace{-.125\baselineskip}}
\hfill
\subcaptionbox{Impulsion transverse du lepton 2 (\localLB).}[.475\textwidth]
{\plotHTTcontrol{\lolcalcurrentyear}{\EMBFFchoice}{\lolcalcurrentchannel}{pt_2}\vspace{-.125\baselineskip}}

\subcaptionbox{Pseudo-rapidité du lepton 1 (\localLA).}[.475\textwidth]
{\plotHTTcontrol{\lolcalcurrentyear}{\EMBFFchoice}{\lolcalcurrentchannel}{eta_1}\vspace{-.125\baselineskip}}
\hfill
\subcaptionbox{Pseudo-rapidité du lepton 2 (\localLB).}[.475\textwidth]
{\plotHTTcontrol{\lolcalcurrentyear}{\EMBFFchoice}{\lolcalcurrentchannel}{eta_2}\vspace{-.125\baselineskip}}

\subcaptionbox{Angle azimutal du lepton 1 (\localLA).}[.475\textwidth]
{\plotHTTcontrol{\lolcalcurrentyear}{\EMBFFchoice}{\lolcalcurrentchannel}{phi_1}\vspace{-.125\baselineskip}}
\hfill
\subcaptionbox{Angle azimutal du lepton 2 (\localLB).}[.475\textwidth]
{\plotHTTcontrol{\lolcalcurrentyear}{\EMBFFchoice}{\lolcalcurrentchannel}{phi_2}\vspace{-.125\baselineskip}}

\caption[Distributions de contrôle, \lolcalcurrentyear\ \localchannel, cinématique des leptons (\localLA, \localLB).]{Canal \localchannel, \lolcalcurrentyear: cinématique des leptons (\localLA, \localLB).}
\end{figure}

\begin{figure}[p]
\centering

\subcaptionbox{Énergie transverse manquante.}[.475\textwidth]
{\plotHTTcontrol{\lolcalcurrentyear}{\EMBFFchoice}{\lolcalcurrentchannel}{puppimet}\vspace{-.125\baselineskip}}
\hfill
\subcaptionbox{Masse transverse du \emph{dilepton}.}[.475\textwidth]
{\plotHTTcontrol{\lolcalcurrentyear}{\EMBFFchoice}{\lolcalcurrentchannel}{mTdileptonMET_puppi}\vspace{-.125\baselineskip}}

\subcaptionbox{Impulsion transverse du \emph{dilepton}.}[.475\textwidth]
{\plotHTTcontrol{\lolcalcurrentyear}{\EMBFFchoice}{\lolcalcurrentchannel}{ptvis}\vspace{-.125\baselineskip}}
\hfill
\subcaptionbox{Masse visible du \emph{dilepton}.}[.475\textwidth]
{\plotHTTcontrol{\lolcalcurrentyear}{\EMBFFchoice}{\lolcalcurrentchannel}{m_vis}\vspace{-.125\baselineskip}}

\subcaptionbox{Impulsion transverse du système di-\tau.}[.475\textwidth]
{\plotHTTcontrol{\lolcalcurrentyear}{\EMBFFchoice}{\lolcalcurrentchannel}{pt_tt_puppi}\vspace{-.125\baselineskip}}
\hfill
\subcaptionbox{Distance $\Delta R$ entre les leptons (\localLA, \localLB).}[.475\textwidth]
{\plotHTTcontrol{\lolcalcurrentyear}{\EMBFFchoice}{\lolcalcurrentchannel}{DiTauDeltaR}\vspace{-.125\baselineskip}}

\caption[Distributions de contrôle, \lolcalcurrentyear\ \localchannel, \emph{dilepton} et énergie transverse manquante.]{Canal \localchannel, \lolcalcurrentyear: \emph{dilepton} et énergie transverse manquante.}
\end{figure}


\begin{figure}[p]
\centering

\subcaptionbox{Masse transverse du lepton 1 (\localLA).}[.475\textwidth]
{\plotHTTcontrol{\lolcalcurrentyear}{\EMBFFchoice}{\lolcalcurrentchannel}{mt_1_puppi}\vspace{-.125\baselineskip}}
\hfill
\subcaptionbox{Masse transverse du lepton 2 (\localLB).}[.475\textwidth]
{\plotHTTcontrol{\lolcalcurrentyear}{\EMBFFchoice}{\lolcalcurrentchannel}{mt_2_puppi}\vspace{-.125\baselineskip}}

\subcaptionbox{Valeur de \Dzeta.}[.475\textwidth]
{\plotHTTcontrol{\lolcalcurrentyear}{\EMBFFchoice}{\lolcalcurrentchannel}{pZetaPuppiMissVis}\vspace{-.125\baselineskip}}
\hfill
\subcaptionbox{Masse transverse totale.}[.475\textwidth]
{\plotHTTcontrol{\lolcalcurrentyear}{\EMBFFchoice}{\lolcalcurrentchannel}{mt_tot_puppi}\vspace{-.125\baselineskip}}


\subcaptionbox{Masse du système di-\tau\ d'après \SVFIT.}[.475\textwidth]
{\plotHTTcontrol{\lolcalcurrentyear}{\EMBFFchoice}{\lolcalcurrentchannel}{m_sv_puppi}\vspace{-.125\baselineskip}}
\hfill
\subcaptionbox{Masse du système di-\tau\ d'après le ML.}[.475\textwidth]
{\plotHTTcontrol{\lolcalcurrentyear}{\EMBFFchoice}{\lolcalcurrentchannel}{ml_mass}\vspace{-.125\baselineskip}}

\caption[Distributions de contrôle, \lolcalcurrentyear\ \localchannel, masses transverses, \Dzeta\ et masses.]{Canal \localchannel, \lolcalcurrentyear: masses transverses, \Dzeta\ et masses.}
\end{figure}

\def\lolcalcurrentchannel{et}
\ifthenelse{\equal{\lolcalcurrentchannel}{tt}}{\def\localchannel{\tauh\tauh}\def\localLA{\ensuremath{\tauh^{(1)}}}\def\localLB{\ensuremath{\tauh^{(2)}}}}{}
\ifthenelse{\equal{\lolcalcurrentchannel}{mt}}{\def\localchannel{\mu\tauh}\def\localLA{\mu}\def\localLB{\ensuremath{\tauh}}}{}
\ifthenelse{\equal{\lolcalcurrentchannel}{et}}{\def\localchannel{\ele\tauh}\def\localLA{\ele}\def\localLB{\ensuremath{\tauh}}}{}
\ifthenelse{\equal{\lolcalcurrentchannel}{em}}{\def\localchannel{\ele\mu}\def\localLA{\ele}\def\localLB{\mu}}{}

\begin{figure}[p]
\centering

\subcaptionbox{Impulsion transverse du jet principal.}[.475\textwidth]
{\plotHTTcontrol{\lolcalcurrentyear}{\EMBFFchoice}{\lolcalcurrentchannel}{jpt_1}\vspace{-.125\baselineskip}}
\hfill
\subcaptionbox{Impulsion transverse du jet secondaire.}[.475\textwidth]
{\plotHTTcontrol{\lolcalcurrentyear}{\EMBFFchoice}{\lolcalcurrentchannel}{jpt_2}\vspace{-.125\baselineskip}}

\subcaptionbox{Pseudo-rapidité du jet principal.}[.475\textwidth]
{\plotHTTcontrol{\lolcalcurrentyear}{\EMBFFchoice}{\lolcalcurrentchannel}{jeta_1}\vspace{-.125\baselineskip}}
\hfill
\subcaptionbox{Pseudo-rapidité du jet secondaire.}[.475\textwidth]
{\plotHTTcontrol{\lolcalcurrentyear}{\EMBFFchoice}{\lolcalcurrentchannel}{jeta_2}\vspace{-.125\baselineskip}}

\subcaptionbox{Angle azimutal du jet principal.}[.475\textwidth]
{\plotHTTcontrol{\lolcalcurrentyear}{\EMBFFchoice}{\lolcalcurrentchannel}{jphi_1}\vspace{-.125\baselineskip}}
\hfill
\subcaptionbox{Angle azimutal du jet secondaire.}[.475\textwidth]
{\plotHTTcontrol{\lolcalcurrentyear}{\EMBFFchoice}{\lolcalcurrentchannel}{jphi_2}\vspace{-.125\baselineskip}}

\caption[Distributions de contrôle, \lolcalcurrentyear\ \localchannel, cinématique des deux jets principaux.]{Canal \localchannel, \lolcalcurrentyear: cinématique des deux jets principaux.}
\end{figure}

\begin{figure}[p]
\centering

\subcaptionbox{Impulsion transverse du \quarkb-jet principal.}[.475\textwidth]
{\plotHTTcontrol{\lolcalcurrentyear}{\EMBFFchoice}{\lolcalcurrentchannel}{bpt_1}\vspace{-.125\baselineskip}}
\hfill
\subcaptionbox{Impulsion transverse du \quarkb-jet secondaire.}[.475\textwidth]
{\plotHTTcontrol{\lolcalcurrentyear}{\EMBFFchoice}{\lolcalcurrentchannel}{bpt_2}\vspace{-.125\baselineskip}}

\subcaptionbox{Impulsion transverse de l'AHA.}[.475\textwidth]
{\plotHTTcontrol{\lolcalcurrentyear}{\EMBFFchoice _jets_r}{\lolcalcurrentchannel}{jpt_r}\vspace{-.125\baselineskip}}
\hfill
\subcaptionbox{Pseudo-rapidité de l'AHA.}[.475\textwidth]
{\plotHTTcontrol{\lolcalcurrentyear}{\EMBFFchoice _jets_r}{\lolcalcurrentchannel}{jeta_r}\vspace{-.125\baselineskip}}

\subcaptionbox{Angle azimutal de l'AHA.}[.475\textwidth]
{\plotHTTcontrol{\lolcalcurrentyear}{\EMBFFchoice _jets_r}{\lolcalcurrentchannel}{jphi_r}\vspace{-.125\baselineskip}}
\hfill
\subcaptionbox{Nombre de jets dans l'AHA.}[.475\textwidth]
{\plotHTTcontrol{\lolcalcurrentyear}{\EMBFFchoice _jets_r}{\lolcalcurrentchannel}{Njet_r}\vspace{-.125\baselineskip}}

\caption[Distributions de contrôle, \lolcalcurrentyear\ \localchannel, \quarkb-jets et activité hadronique additionnelle.]{Canal \localchannel, \lolcalcurrentyear: \quarkb-jets et activité hadronique additionnelle.}
\end{figure}

\begin{figure}[p]
\centering

\subcaptionbox{Nombre de \quarkb-jets.}[.475\textwidth]
{\plotHTTcontrol{\lolcalcurrentyear}{\EMBFFchoice}{\lolcalcurrentchannel}{nbtag}\vspace{-.125\baselineskip}}
\hfill
\subcaptionbox{Nombre de jets.}[.475\textwidth]
{\plotHTTcontrol{\lolcalcurrentyear}{\EMBFFchoice}{\lolcalcurrentchannel}{njets}\vspace{-.125\baselineskip}}

\subcaptionbox{Impulsion transverse du système des deux jets.}[.475\textwidth]
{\plotHTTcontrol{\lolcalcurrentyear}{\EMBFFchoice}{\lolcalcurrentchannel}{dijetpt}\vspace{-.125\baselineskip}}
\hfill
\subcaptionbox{Distance en $\eta$ entre les deux jets.}[.475\textwidth]
{\plotHTTcontrol{\lolcalcurrentyear}{\EMBFFchoice}{\lolcalcurrentchannel}{jdeta}\vspace{-.125\baselineskip}}

\subcaptionbox{Masse invariante du système des deux jets.}[.475\textwidth]
{\plotHTTcontrol{\lolcalcurrentyear}{\EMBFFchoice}{\lolcalcurrentchannel}{mjj}\vspace{-.125\baselineskip}}
\hfill
\subcaptionbox{Nombre de vertex d'empilement.}[.475\textwidth]
{\plotHTTcontrol{\lolcalcurrentyear}{\EMBFFchoice}{\lolcalcurrentchannel}{npv}\vspace{-.125\baselineskip}}

\caption[Distributions de contrôle, \lolcalcurrentyear\ \localchannel, nombre de jets, système des deux jets principaux et empilement.]{Canal \localchannel, \lolcalcurrentyear: nombre de jets, système des deux jets principaux et empilement.}
\end{figure}


\begin{figure}[p]
\centering

\subcaptionbox{Impulsion transverse du lepton 1 (\localLA).}[.475\textwidth]
{\plotHTTcontrol{\lolcalcurrentyear}{\EMBFFchoice}{\lolcalcurrentchannel}{pt_1}\vspace{-.125\baselineskip}}
\hfill
\subcaptionbox{Impulsion transverse du lepton 2 (\localLB).}[.475\textwidth]
{\plotHTTcontrol{\lolcalcurrentyear}{\EMBFFchoice}{\lolcalcurrentchannel}{pt_2}\vspace{-.125\baselineskip}}

\subcaptionbox{Pseudo-rapidité du lepton 1 (\localLA).}[.475\textwidth]
{\plotHTTcontrol{\lolcalcurrentyear}{\EMBFFchoice}{\lolcalcurrentchannel}{eta_1}\vspace{-.125\baselineskip}}
\hfill
\subcaptionbox{Pseudo-rapidité du lepton 2 (\localLB).}[.475\textwidth]
{\plotHTTcontrol{\lolcalcurrentyear}{\EMBFFchoice}{\lolcalcurrentchannel}{eta_2}\vspace{-.125\baselineskip}}

\subcaptionbox{Angle azimutal du lepton 1 (\localLA).}[.475\textwidth]
{\plotHTTcontrol{\lolcalcurrentyear}{\EMBFFchoice}{\lolcalcurrentchannel}{phi_1}\vspace{-.125\baselineskip}}
\hfill
\subcaptionbox{Angle azimutal du lepton 2 (\localLB).}[.475\textwidth]
{\plotHTTcontrol{\lolcalcurrentyear}{\EMBFFchoice}{\lolcalcurrentchannel}{phi_2}\vspace{-.125\baselineskip}}

\caption[Distributions de contrôle, \lolcalcurrentyear\ \localchannel, cinématique des leptons (\localLA, \localLB).]{Canal \localchannel, \lolcalcurrentyear: cinématique des leptons (\localLA, \localLB).}
\end{figure}

\begin{figure}[p]
\centering

\subcaptionbox{Énergie transverse manquante.}[.475\textwidth]
{\plotHTTcontrol{\lolcalcurrentyear}{\EMBFFchoice}{\lolcalcurrentchannel}{puppimet}\vspace{-.125\baselineskip}}
\hfill
\subcaptionbox{Masse transverse du \emph{dilepton}.}[.475\textwidth]
{\plotHTTcontrol{\lolcalcurrentyear}{\EMBFFchoice}{\lolcalcurrentchannel}{mTdileptonMET_puppi}\vspace{-.125\baselineskip}}

\subcaptionbox{Impulsion transverse du \emph{dilepton}.}[.475\textwidth]
{\plotHTTcontrol{\lolcalcurrentyear}{\EMBFFchoice}{\lolcalcurrentchannel}{ptvis}\vspace{-.125\baselineskip}}
\hfill
\subcaptionbox{Masse visible du \emph{dilepton}.}[.475\textwidth]
{\plotHTTcontrol{\lolcalcurrentyear}{\EMBFFchoice}{\lolcalcurrentchannel}{m_vis}\vspace{-.125\baselineskip}}

\subcaptionbox{Impulsion transverse du système di-\tau.}[.475\textwidth]
{\plotHTTcontrol{\lolcalcurrentyear}{\EMBFFchoice}{\lolcalcurrentchannel}{pt_tt_puppi}\vspace{-.125\baselineskip}}
\hfill
\subcaptionbox{Distance $\Delta R$ entre les leptons (\localLA, \localLB).}[.475\textwidth]
{\plotHTTcontrol{\lolcalcurrentyear}{\EMBFFchoice}{\lolcalcurrentchannel}{DiTauDeltaR}\vspace{-.125\baselineskip}}

\caption[Distributions de contrôle, \lolcalcurrentyear\ \localchannel, \emph{dilepton} et énergie transverse manquante.]{Canal \localchannel, \lolcalcurrentyear: \emph{dilepton} et énergie transverse manquante.}
\end{figure}


\begin{figure}[p]
\centering

\subcaptionbox{Masse transverse du lepton 1 (\localLA).}[.475\textwidth]
{\plotHTTcontrol{\lolcalcurrentyear}{\EMBFFchoice}{\lolcalcurrentchannel}{mt_1_puppi}\vspace{-.125\baselineskip}}
\hfill
\subcaptionbox{Masse transverse du lepton 2 (\localLB).}[.475\textwidth]
{\plotHTTcontrol{\lolcalcurrentyear}{\EMBFFchoice}{\lolcalcurrentchannel}{mt_2_puppi}\vspace{-.125\baselineskip}}

\subcaptionbox{Valeur de \Dzeta.}[.475\textwidth]
{\plotHTTcontrol{\lolcalcurrentyear}{\EMBFFchoice}{\lolcalcurrentchannel}{pZetaPuppiMissVis}\vspace{-.125\baselineskip}}
\hfill
\subcaptionbox{Masse transverse totale.}[.475\textwidth]
{\plotHTTcontrol{\lolcalcurrentyear}{\EMBFFchoice}{\lolcalcurrentchannel}{mt_tot_puppi}\vspace{-.125\baselineskip}}


\subcaptionbox{Masse du système di-\tau\ d'après \SVFIT.}[.475\textwidth]
{\plotHTTcontrol{\lolcalcurrentyear}{\EMBFFchoice}{\lolcalcurrentchannel}{m_sv_puppi}\vspace{-.125\baselineskip}}
\hfill
\subcaptionbox{Masse du système di-\tau\ d'après le ML.}[.475\textwidth]
{\plotHTTcontrol{\lolcalcurrentyear}{\EMBFFchoice}{\lolcalcurrentchannel}{ml_mass}\vspace{-.125\baselineskip}}

\caption[Distributions de contrôle, \lolcalcurrentyear\ \localchannel, masses transverses, \Dzeta\ et masses.]{Canal \localchannel, \lolcalcurrentyear: masses transverses, \Dzeta\ et masses.}
\end{figure}

\def\lolcalcurrentchannel{em}
\ifthenelse{\equal{\lolcalcurrentchannel}{tt}}{\def\localchannel{\tauh\tauh}\def\localLA{\ensuremath{\tauh^{(1)}}}\def\localLB{\ensuremath{\tauh^{(2)}}}}{}
\ifthenelse{\equal{\lolcalcurrentchannel}{mt}}{\def\localchannel{\mu\tauh}\def\localLA{\mu}\def\localLB{\ensuremath{\tauh}}}{}
\ifthenelse{\equal{\lolcalcurrentchannel}{et}}{\def\localchannel{\ele\tauh}\def\localLA{\ele}\def\localLB{\ensuremath{\tauh}}}{}
\ifthenelse{\equal{\lolcalcurrentchannel}{em}}{\def\localchannel{\ele\mu}\def\localLA{\ele}\def\localLB{\mu}}{}

\begin{figure}[p]
\centering

\subcaptionbox{Impulsion transverse du jet principal.}[.475\textwidth]
{\plotHTTcontrol{\lolcalcurrentyear}{\EMBFFchoice}{\lolcalcurrentchannel}{jpt_1}\vspace{-.125\baselineskip}}
\hfill
\subcaptionbox{Impulsion transverse du jet secondaire.}[.475\textwidth]
{\plotHTTcontrol{\lolcalcurrentyear}{\EMBFFchoice}{\lolcalcurrentchannel}{jpt_2}\vspace{-.125\baselineskip}}

\subcaptionbox{Pseudo-rapidité du jet principal.}[.475\textwidth]
{\plotHTTcontrol{\lolcalcurrentyear}{\EMBFFchoice}{\lolcalcurrentchannel}{jeta_1}\vspace{-.125\baselineskip}}
\hfill
\subcaptionbox{Pseudo-rapidité du jet secondaire.}[.475\textwidth]
{\plotHTTcontrol{\lolcalcurrentyear}{\EMBFFchoice}{\lolcalcurrentchannel}{jeta_2}\vspace{-.125\baselineskip}}

\subcaptionbox{Angle azimutal du jet principal.}[.475\textwidth]
{\plotHTTcontrol{\lolcalcurrentyear}{\EMBFFchoice}{\lolcalcurrentchannel}{jphi_1}\vspace{-.125\baselineskip}}
\hfill
\subcaptionbox{Angle azimutal du jet secondaire.}[.475\textwidth]
{\plotHTTcontrol{\lolcalcurrentyear}{\EMBFFchoice}{\lolcalcurrentchannel}{jphi_2}\vspace{-.125\baselineskip}}

\caption[Distributions de contrôle, \lolcalcurrentyear\ \localchannel, cinématique des deux jets principaux.]{Canal \localchannel, \lolcalcurrentyear: cinématique des deux jets principaux.}
\end{figure}

\begin{figure}[p]
\centering

\subcaptionbox{Impulsion transverse du \quarkb-jet principal.}[.475\textwidth]
{\plotHTTcontrol{\lolcalcurrentyear}{\EMBFFchoice}{\lolcalcurrentchannel}{bpt_1}\vspace{-.125\baselineskip}}
\hfill
\subcaptionbox{Impulsion transverse du \quarkb-jet secondaire.}[.475\textwidth]
{\plotHTTcontrol{\lolcalcurrentyear}{\EMBFFchoice}{\lolcalcurrentchannel}{bpt_2}\vspace{-.125\baselineskip}}

\subcaptionbox{Impulsion transverse de l'AHA.}[.475\textwidth]
{\plotHTTcontrol{\lolcalcurrentyear}{\EMBFFchoice _jets_r}{\lolcalcurrentchannel}{jpt_r}\vspace{-.125\baselineskip}}
\hfill
\subcaptionbox{Pseudo-rapidité de l'AHA.}[.475\textwidth]
{\plotHTTcontrol{\lolcalcurrentyear}{\EMBFFchoice _jets_r}{\lolcalcurrentchannel}{jeta_r}\vspace{-.125\baselineskip}}

\subcaptionbox{Angle azimutal de l'AHA.}[.475\textwidth]
{\plotHTTcontrol{\lolcalcurrentyear}{\EMBFFchoice _jets_r}{\lolcalcurrentchannel}{jphi_r}\vspace{-.125\baselineskip}}
\hfill
\subcaptionbox{Nombre de jets dans l'AHA.}[.475\textwidth]
{\plotHTTcontrol{\lolcalcurrentyear}{\EMBFFchoice _jets_r}{\lolcalcurrentchannel}{Njet_r}\vspace{-.125\baselineskip}}

\caption[Distributions de contrôle, \lolcalcurrentyear\ \localchannel, \quarkb-jets et activité hadronique additionnelle.]{Canal \localchannel, \lolcalcurrentyear: \quarkb-jets et activité hadronique additionnelle.}
\end{figure}

\begin{figure}[p]
\centering

\subcaptionbox{Nombre de \quarkb-jets.}[.475\textwidth]
{\plotHTTcontrol{\lolcalcurrentyear}{\EMBFFchoice}{\lolcalcurrentchannel}{nbtag}\vspace{-.125\baselineskip}}
\hfill
\subcaptionbox{Nombre de jets.}[.475\textwidth]
{\plotHTTcontrol{\lolcalcurrentyear}{\EMBFFchoice}{\lolcalcurrentchannel}{njets}\vspace{-.125\baselineskip}}

\subcaptionbox{Impulsion transverse du système des deux jets.}[.475\textwidth]
{\plotHTTcontrol{\lolcalcurrentyear}{\EMBFFchoice}{\lolcalcurrentchannel}{dijetpt}\vspace{-.125\baselineskip}}
\hfill
\subcaptionbox{Distance en $\eta$ entre les deux jets.}[.475\textwidth]
{\plotHTTcontrol{\lolcalcurrentyear}{\EMBFFchoice}{\lolcalcurrentchannel}{jdeta}\vspace{-.125\baselineskip}}

\subcaptionbox{Masse invariante du système des deux jets.}[.475\textwidth]
{\plotHTTcontrol{\lolcalcurrentyear}{\EMBFFchoice}{\lolcalcurrentchannel}{mjj}\vspace{-.125\baselineskip}}
\hfill
\subcaptionbox{Nombre de vertex d'empilement.}[.475\textwidth]
{\plotHTTcontrol{\lolcalcurrentyear}{\EMBFFchoice}{\lolcalcurrentchannel}{npv}\vspace{-.125\baselineskip}}

\caption[Distributions de contrôle, \lolcalcurrentyear\ \localchannel, nombre de jets, système des deux jets principaux et empilement.]{Canal \localchannel, \lolcalcurrentyear: nombre de jets, système des deux jets principaux et empilement.}
\end{figure}


\begin{figure}[p]
\centering

\subcaptionbox{Impulsion transverse du lepton 1 (\localLA).}[.475\textwidth]
{\plotHTTcontrol{\lolcalcurrentyear}{\EMBFFchoice}{\lolcalcurrentchannel}{pt_1}\vspace{-.125\baselineskip}}
\hfill
\subcaptionbox{Impulsion transverse du lepton 2 (\localLB).}[.475\textwidth]
{\plotHTTcontrol{\lolcalcurrentyear}{\EMBFFchoice}{\lolcalcurrentchannel}{pt_2}\vspace{-.125\baselineskip}}

\subcaptionbox{Pseudo-rapidité du lepton 1 (\localLA).}[.475\textwidth]
{\plotHTTcontrol{\lolcalcurrentyear}{\EMBFFchoice}{\lolcalcurrentchannel}{eta_1}\vspace{-.125\baselineskip}}
\hfill
\subcaptionbox{Pseudo-rapidité du lepton 2 (\localLB).}[.475\textwidth]
{\plotHTTcontrol{\lolcalcurrentyear}{\EMBFFchoice}{\lolcalcurrentchannel}{eta_2}\vspace{-.125\baselineskip}}

\subcaptionbox{Angle azimutal du lepton 1 (\localLA).}[.475\textwidth]
{\plotHTTcontrol{\lolcalcurrentyear}{\EMBFFchoice}{\lolcalcurrentchannel}{phi_1}\vspace{-.125\baselineskip}}
\hfill
\subcaptionbox{Angle azimutal du lepton 2 (\localLB).}[.475\textwidth]
{\plotHTTcontrol{\lolcalcurrentyear}{\EMBFFchoice}{\lolcalcurrentchannel}{phi_2}\vspace{-.125\baselineskip}}

\caption[Distributions de contrôle, \lolcalcurrentyear\ \localchannel, cinématique des leptons (\localLA, \localLB).]{Canal \localchannel, \lolcalcurrentyear: cinématique des leptons (\localLA, \localLB).}
\end{figure}

\begin{figure}[p]
\centering

\subcaptionbox{Énergie transverse manquante.}[.475\textwidth]
{\plotHTTcontrol{\lolcalcurrentyear}{\EMBFFchoice}{\lolcalcurrentchannel}{puppimet}\vspace{-.125\baselineskip}}
\hfill
\subcaptionbox{Masse transverse du \emph{dilepton}.}[.475\textwidth]
{\plotHTTcontrol{\lolcalcurrentyear}{\EMBFFchoice}{\lolcalcurrentchannel}{mTdileptonMET_puppi}\vspace{-.125\baselineskip}}

\subcaptionbox{Impulsion transverse du \emph{dilepton}.}[.475\textwidth]
{\plotHTTcontrol{\lolcalcurrentyear}{\EMBFFchoice}{\lolcalcurrentchannel}{ptvis}\vspace{-.125\baselineskip}}
\hfill
\subcaptionbox{Masse visible du \emph{dilepton}.}[.475\textwidth]
{\plotHTTcontrol{\lolcalcurrentyear}{\EMBFFchoice}{\lolcalcurrentchannel}{m_vis}\vspace{-.125\baselineskip}}

\subcaptionbox{Impulsion transverse du système di-\tau.}[.475\textwidth]
{\plotHTTcontrol{\lolcalcurrentyear}{\EMBFFchoice}{\lolcalcurrentchannel}{pt_tt_puppi}\vspace{-.125\baselineskip}}
\hfill
\subcaptionbox{Distance $\Delta R$ entre les leptons (\localLA, \localLB).}[.475\textwidth]
{\plotHTTcontrol{\lolcalcurrentyear}{\EMBFFchoice}{\lolcalcurrentchannel}{DiTauDeltaR}\vspace{-.125\baselineskip}}

\caption[Distributions de contrôle, \lolcalcurrentyear\ \localchannel, \emph{dilepton} et énergie transverse manquante.]{Canal \localchannel, \lolcalcurrentyear: \emph{dilepton} et énergie transverse manquante.}
\end{figure}


\begin{figure}[p]
\centering

\subcaptionbox{Masse transverse du lepton 1 (\localLA).}[.475\textwidth]
{\plotHTTcontrol{\lolcalcurrentyear}{\EMBFFchoice}{\lolcalcurrentchannel}{mt_1_puppi}\vspace{-.125\baselineskip}}
\hfill
\subcaptionbox{Masse transverse du lepton 2 (\localLB).}[.475\textwidth]
{\plotHTTcontrol{\lolcalcurrentyear}{\EMBFFchoice}{\lolcalcurrentchannel}{mt_2_puppi}\vspace{-.125\baselineskip}}

\subcaptionbox{Valeur de \Dzeta.}[.475\textwidth]
{\plotHTTcontrol{\lolcalcurrentyear}{\EMBFFchoice}{\lolcalcurrentchannel}{pZetaPuppiMissVis}\vspace{-.125\baselineskip}}
\hfill
\subcaptionbox{Masse transverse totale.}[.475\textwidth]
{\plotHTTcontrol{\lolcalcurrentyear}{\EMBFFchoice}{\lolcalcurrentchannel}{mt_tot_puppi}\vspace{-.125\baselineskip}}


\subcaptionbox{Masse du système di-\tau\ d'après \SVFIT.}[.475\textwidth]
{\plotHTTcontrol{\lolcalcurrentyear}{\EMBFFchoice}{\lolcalcurrentchannel}{m_sv_puppi}\vspace{-.125\baselineskip}}
\hfill
\subcaptionbox{Masse du système di-\tau\ d'après le ML.}[.475\textwidth]
{\plotHTTcontrol{\lolcalcurrentyear}{\EMBFFchoice}{\lolcalcurrentchannel}{ml_mass}\vspace{-.125\baselineskip}}

\caption[Distributions de contrôle, \lolcalcurrentyear\ \localchannel, masses transverses, \Dzeta\ et masses.]{Canal \localchannel, \lolcalcurrentyear: masses transverses, \Dzeta\ et masses.}
\end{figure}
\clearpage

\def\lolcalcurrentyear{2017}
\def\lolcalcurrentchannel{tt}
\ifthenelse{\equal{\lolcalcurrentchannel}{tt}}{\def\localchannel{\tauh\tauh}\def\localLA{\ensuremath{\tauh^{(1)}}}\def\localLB{\ensuremath{\tauh^{(2)}}}}{}
\ifthenelse{\equal{\lolcalcurrentchannel}{mt}}{\def\localchannel{\mu\tauh}\def\localLA{\mu}\def\localLB{\ensuremath{\tauh}}}{}
\ifthenelse{\equal{\lolcalcurrentchannel}{et}}{\def\localchannel{\ele\tauh}\def\localLA{\ele}\def\localLB{\ensuremath{\tauh}}}{}
\ifthenelse{\equal{\lolcalcurrentchannel}{em}}{\def\localchannel{\ele\mu}\def\localLA{\ele}\def\localLB{\mu}}{}

\begin{figure}[p]
\centering

\subcaptionbox{Impulsion transverse du jet principal.}[.475\textwidth]
{\plotHTTcontrol{\lolcalcurrentyear}{\EMBFFchoice}{\lolcalcurrentchannel}{jpt_1}\vspace{-.125\baselineskip}}
\hfill
\subcaptionbox{Impulsion transverse du jet secondaire.}[.475\textwidth]
{\plotHTTcontrol{\lolcalcurrentyear}{\EMBFFchoice}{\lolcalcurrentchannel}{jpt_2}\vspace{-.125\baselineskip}}

\subcaptionbox{Pseudo-rapidité du jet principal.}[.475\textwidth]
{\plotHTTcontrol{\lolcalcurrentyear}{\EMBFFchoice}{\lolcalcurrentchannel}{jeta_1}\vspace{-.125\baselineskip}}
\hfill
\subcaptionbox{Pseudo-rapidité du jet secondaire.}[.475\textwidth]
{\plotHTTcontrol{\lolcalcurrentyear}{\EMBFFchoice}{\lolcalcurrentchannel}{jeta_2}\vspace{-.125\baselineskip}}

\subcaptionbox{Angle azimutal du jet principal.}[.475\textwidth]
{\plotHTTcontrol{\lolcalcurrentyear}{\EMBFFchoice}{\lolcalcurrentchannel}{jphi_1}\vspace{-.125\baselineskip}}
\hfill
\subcaptionbox{Angle azimutal du jet secondaire.}[.475\textwidth]
{\plotHTTcontrol{\lolcalcurrentyear}{\EMBFFchoice}{\lolcalcurrentchannel}{jphi_2}\vspace{-.125\baselineskip}}

\caption[Distributions de contrôle, \lolcalcurrentyear\ \localchannel, cinématique des deux jets principaux.]{Canal \localchannel, \lolcalcurrentyear: cinématique des deux jets principaux.}
\end{figure}

\begin{figure}[p]
\centering

\subcaptionbox{Impulsion transverse du \quarkb-jet principal.}[.475\textwidth]
{\plotHTTcontrol{\lolcalcurrentyear}{\EMBFFchoice}{\lolcalcurrentchannel}{bpt_1}\vspace{-.125\baselineskip}}
\hfill
\subcaptionbox{Impulsion transverse du \quarkb-jet secondaire.}[.475\textwidth]
{\plotHTTcontrol{\lolcalcurrentyear}{\EMBFFchoice}{\lolcalcurrentchannel}{bpt_2}\vspace{-.125\baselineskip}}

\subcaptionbox{Impulsion transverse de l'AHA.}[.475\textwidth]
{\plotHTTcontrol{\lolcalcurrentyear}{\EMBFFchoice _jets_r}{\lolcalcurrentchannel}{jpt_r}\vspace{-.125\baselineskip}}
\hfill
\subcaptionbox{Pseudo-rapidité de l'AHA.}[.475\textwidth]
{\plotHTTcontrol{\lolcalcurrentyear}{\EMBFFchoice _jets_r}{\lolcalcurrentchannel}{jeta_r}\vspace{-.125\baselineskip}}

\subcaptionbox{Angle azimutal de l'AHA.}[.475\textwidth]
{\plotHTTcontrol{\lolcalcurrentyear}{\EMBFFchoice _jets_r}{\lolcalcurrentchannel}{jphi_r}\vspace{-.125\baselineskip}}
\hfill
\subcaptionbox{Nombre de jets dans l'AHA.}[.475\textwidth]
{\plotHTTcontrol{\lolcalcurrentyear}{\EMBFFchoice _jets_r}{\lolcalcurrentchannel}{Njet_r}\vspace{-.125\baselineskip}}

\caption[Distributions de contrôle, \lolcalcurrentyear\ \localchannel, \quarkb-jets et activité hadronique additionnelle.]{Canal \localchannel, \lolcalcurrentyear: \quarkb-jets et activité hadronique additionnelle.}
\end{figure}

\begin{figure}[p]
\centering

\subcaptionbox{Nombre de \quarkb-jets.}[.475\textwidth]
{\plotHTTcontrol{\lolcalcurrentyear}{\EMBFFchoice}{\lolcalcurrentchannel}{nbtag}\vspace{-.125\baselineskip}}
\hfill
\subcaptionbox{Nombre de jets.}[.475\textwidth]
{\plotHTTcontrol{\lolcalcurrentyear}{\EMBFFchoice}{\lolcalcurrentchannel}{njets}\vspace{-.125\baselineskip}}

\subcaptionbox{Impulsion transverse du système des deux jets.}[.475\textwidth]
{\plotHTTcontrol{\lolcalcurrentyear}{\EMBFFchoice}{\lolcalcurrentchannel}{dijetpt}\vspace{-.125\baselineskip}}
\hfill
\subcaptionbox{Distance en $\eta$ entre les deux jets.}[.475\textwidth]
{\plotHTTcontrol{\lolcalcurrentyear}{\EMBFFchoice}{\lolcalcurrentchannel}{jdeta}\vspace{-.125\baselineskip}}

\subcaptionbox{Masse invariante du système des deux jets.}[.475\textwidth]
{\plotHTTcontrol{\lolcalcurrentyear}{\EMBFFchoice}{\lolcalcurrentchannel}{mjj}\vspace{-.125\baselineskip}}
\hfill
\subcaptionbox{Nombre de vertex d'empilement.}[.475\textwidth]
{\plotHTTcontrol{\lolcalcurrentyear}{\EMBFFchoice}{\lolcalcurrentchannel}{npv}\vspace{-.125\baselineskip}}

\caption[Distributions de contrôle, \lolcalcurrentyear\ \localchannel, nombre de jets, système des deux jets principaux et empilement.]{Canal \localchannel, \lolcalcurrentyear: nombre de jets, système des deux jets principaux et empilement.}
\end{figure}


\begin{figure}[p]
\centering

\subcaptionbox{Impulsion transverse du lepton 1 (\localLA).}[.475\textwidth]
{\plotHTTcontrol{\lolcalcurrentyear}{\EMBFFchoice}{\lolcalcurrentchannel}{pt_1}\vspace{-.125\baselineskip}}
\hfill
\subcaptionbox{Impulsion transverse du lepton 2 (\localLB).}[.475\textwidth]
{\plotHTTcontrol{\lolcalcurrentyear}{\EMBFFchoice}{\lolcalcurrentchannel}{pt_2}\vspace{-.125\baselineskip}}

\subcaptionbox{Pseudo-rapidité du lepton 1 (\localLA).}[.475\textwidth]
{\plotHTTcontrol{\lolcalcurrentyear}{\EMBFFchoice}{\lolcalcurrentchannel}{eta_1}\vspace{-.125\baselineskip}}
\hfill
\subcaptionbox{Pseudo-rapidité du lepton 2 (\localLB).}[.475\textwidth]
{\plotHTTcontrol{\lolcalcurrentyear}{\EMBFFchoice}{\lolcalcurrentchannel}{eta_2}\vspace{-.125\baselineskip}}

\subcaptionbox{Angle azimutal du lepton 1 (\localLA).}[.475\textwidth]
{\plotHTTcontrol{\lolcalcurrentyear}{\EMBFFchoice}{\lolcalcurrentchannel}{phi_1}\vspace{-.125\baselineskip}}
\hfill
\subcaptionbox{Angle azimutal du lepton 2 (\localLB).}[.475\textwidth]
{\plotHTTcontrol{\lolcalcurrentyear}{\EMBFFchoice}{\lolcalcurrentchannel}{phi_2}\vspace{-.125\baselineskip}}

\caption[Distributions de contrôle, \lolcalcurrentyear\ \localchannel, cinématique des leptons (\localLA, \localLB).]{Canal \localchannel, \lolcalcurrentyear: cinématique des leptons (\localLA, \localLB).}
\end{figure}

\begin{figure}[p]
\centering

\subcaptionbox{Énergie transverse manquante.}[.475\textwidth]
{\plotHTTcontrol{\lolcalcurrentyear}{\EMBFFchoice}{\lolcalcurrentchannel}{puppimet}\vspace{-.125\baselineskip}}
\hfill
\subcaptionbox{Masse transverse du \emph{dilepton}.}[.475\textwidth]
{\plotHTTcontrol{\lolcalcurrentyear}{\EMBFFchoice}{\lolcalcurrentchannel}{mTdileptonMET_puppi}\vspace{-.125\baselineskip}}

\subcaptionbox{Impulsion transverse du \emph{dilepton}.}[.475\textwidth]
{\plotHTTcontrol{\lolcalcurrentyear}{\EMBFFchoice}{\lolcalcurrentchannel}{ptvis}\vspace{-.125\baselineskip}}
\hfill
\subcaptionbox{Masse visible du \emph{dilepton}.}[.475\textwidth]
{\plotHTTcontrol{\lolcalcurrentyear}{\EMBFFchoice}{\lolcalcurrentchannel}{m_vis}\vspace{-.125\baselineskip}}

\subcaptionbox{Impulsion transverse du système di-\tau.}[.475\textwidth]
{\plotHTTcontrol{\lolcalcurrentyear}{\EMBFFchoice}{\lolcalcurrentchannel}{pt_tt_puppi}\vspace{-.125\baselineskip}}
\hfill
\subcaptionbox{Distance $\Delta R$ entre les leptons (\localLA, \localLB).}[.475\textwidth]
{\plotHTTcontrol{\lolcalcurrentyear}{\EMBFFchoice}{\lolcalcurrentchannel}{DiTauDeltaR}\vspace{-.125\baselineskip}}

\caption[Distributions de contrôle, \lolcalcurrentyear\ \localchannel, \emph{dilepton} et énergie transverse manquante.]{Canal \localchannel, \lolcalcurrentyear: \emph{dilepton} et énergie transverse manquante.}
\end{figure}


\begin{figure}[p]
\centering

\subcaptionbox{Masse transverse du lepton 1 (\localLA).}[.475\textwidth]
{\plotHTTcontrol{\lolcalcurrentyear}{\EMBFFchoice}{\lolcalcurrentchannel}{mt_1_puppi}\vspace{-.125\baselineskip}}
\hfill
\subcaptionbox{Masse transverse du lepton 2 (\localLB).}[.475\textwidth]
{\plotHTTcontrol{\lolcalcurrentyear}{\EMBFFchoice}{\lolcalcurrentchannel}{mt_2_puppi}\vspace{-.125\baselineskip}}

\subcaptionbox{Valeur de \Dzeta.}[.475\textwidth]
{\plotHTTcontrol{\lolcalcurrentyear}{\EMBFFchoice}{\lolcalcurrentchannel}{pZetaPuppiMissVis}\vspace{-.125\baselineskip}}
\hfill
\subcaptionbox{Masse transverse totale.}[.475\textwidth]
{\plotHTTcontrol{\lolcalcurrentyear}{\EMBFFchoice}{\lolcalcurrentchannel}{mt_tot_puppi}\vspace{-.125\baselineskip}}


\subcaptionbox{Masse du système di-\tau\ d'après \SVFIT.}[.475\textwidth]
{\plotHTTcontrol{\lolcalcurrentyear}{\EMBFFchoice}{\lolcalcurrentchannel}{m_sv_puppi}\vspace{-.125\baselineskip}}
\hfill
\subcaptionbox{Masse du système di-\tau\ d'après le ML.}[.475\textwidth]
{\plotHTTcontrol{\lolcalcurrentyear}{\EMBFFchoice}{\lolcalcurrentchannel}{ml_mass}\vspace{-.125\baselineskip}}

\caption[Distributions de contrôle, \lolcalcurrentyear\ \localchannel, masses transverses, \Dzeta\ et masses.]{Canal \localchannel, \lolcalcurrentyear: masses transverses, \Dzeta\ et masses.}
\end{figure}

\def\lolcalcurrentchannel{mt}
\ifthenelse{\equal{\lolcalcurrentchannel}{tt}}{\def\localchannel{\tauh\tauh}\def\localLA{\ensuremath{\tauh^{(1)}}}\def\localLB{\ensuremath{\tauh^{(2)}}}}{}
\ifthenelse{\equal{\lolcalcurrentchannel}{mt}}{\def\localchannel{\mu\tauh}\def\localLA{\mu}\def\localLB{\ensuremath{\tauh}}}{}
\ifthenelse{\equal{\lolcalcurrentchannel}{et}}{\def\localchannel{\ele\tauh}\def\localLA{\ele}\def\localLB{\ensuremath{\tauh}}}{}
\ifthenelse{\equal{\lolcalcurrentchannel}{em}}{\def\localchannel{\ele\mu}\def\localLA{\ele}\def\localLB{\mu}}{}

\begin{figure}[p]
\centering

\subcaptionbox{Impulsion transverse du jet principal.}[.475\textwidth]
{\plotHTTcontrol{\lolcalcurrentyear}{\EMBFFchoice}{\lolcalcurrentchannel}{jpt_1}\vspace{-.125\baselineskip}}
\hfill
\subcaptionbox{Impulsion transverse du jet secondaire.}[.475\textwidth]
{\plotHTTcontrol{\lolcalcurrentyear}{\EMBFFchoice}{\lolcalcurrentchannel}{jpt_2}\vspace{-.125\baselineskip}}

\subcaptionbox{Pseudo-rapidité du jet principal.}[.475\textwidth]
{\plotHTTcontrol{\lolcalcurrentyear}{\EMBFFchoice}{\lolcalcurrentchannel}{jeta_1}\vspace{-.125\baselineskip}}
\hfill
\subcaptionbox{Pseudo-rapidité du jet secondaire.}[.475\textwidth]
{\plotHTTcontrol{\lolcalcurrentyear}{\EMBFFchoice}{\lolcalcurrentchannel}{jeta_2}\vspace{-.125\baselineskip}}

\subcaptionbox{Angle azimutal du jet principal.}[.475\textwidth]
{\plotHTTcontrol{\lolcalcurrentyear}{\EMBFFchoice}{\lolcalcurrentchannel}{jphi_1}\vspace{-.125\baselineskip}}
\hfill
\subcaptionbox{Angle azimutal du jet secondaire.}[.475\textwidth]
{\plotHTTcontrol{\lolcalcurrentyear}{\EMBFFchoice}{\lolcalcurrentchannel}{jphi_2}\vspace{-.125\baselineskip}}

\caption[Distributions de contrôle, \lolcalcurrentyear\ \localchannel, cinématique des deux jets principaux.]{Canal \localchannel, \lolcalcurrentyear: cinématique des deux jets principaux.}
\end{figure}

\begin{figure}[p]
\centering

\subcaptionbox{Impulsion transverse du \quarkb-jet principal.}[.475\textwidth]
{\plotHTTcontrol{\lolcalcurrentyear}{\EMBFFchoice}{\lolcalcurrentchannel}{bpt_1}\vspace{-.125\baselineskip}}
\hfill
\subcaptionbox{Impulsion transverse du \quarkb-jet secondaire.}[.475\textwidth]
{\plotHTTcontrol{\lolcalcurrentyear}{\EMBFFchoice}{\lolcalcurrentchannel}{bpt_2}\vspace{-.125\baselineskip}}

\subcaptionbox{Impulsion transverse de l'AHA.}[.475\textwidth]
{\plotHTTcontrol{\lolcalcurrentyear}{\EMBFFchoice _jets_r}{\lolcalcurrentchannel}{jpt_r}\vspace{-.125\baselineskip}}
\hfill
\subcaptionbox{Pseudo-rapidité de l'AHA.}[.475\textwidth]
{\plotHTTcontrol{\lolcalcurrentyear}{\EMBFFchoice _jets_r}{\lolcalcurrentchannel}{jeta_r}\vspace{-.125\baselineskip}}

\subcaptionbox{Angle azimutal de l'AHA.}[.475\textwidth]
{\plotHTTcontrol{\lolcalcurrentyear}{\EMBFFchoice _jets_r}{\lolcalcurrentchannel}{jphi_r}\vspace{-.125\baselineskip}}
\hfill
\subcaptionbox{Nombre de jets dans l'AHA.}[.475\textwidth]
{\plotHTTcontrol{\lolcalcurrentyear}{\EMBFFchoice _jets_r}{\lolcalcurrentchannel}{Njet_r}\vspace{-.125\baselineskip}}

\caption[Distributions de contrôle, \lolcalcurrentyear\ \localchannel, \quarkb-jets et activité hadronique additionnelle.]{Canal \localchannel, \lolcalcurrentyear: \quarkb-jets et activité hadronique additionnelle.}
\end{figure}

\begin{figure}[p]
\centering

\subcaptionbox{Nombre de \quarkb-jets.}[.475\textwidth]
{\plotHTTcontrol{\lolcalcurrentyear}{\EMBFFchoice}{\lolcalcurrentchannel}{nbtag}\vspace{-.125\baselineskip}}
\hfill
\subcaptionbox{Nombre de jets.}[.475\textwidth]
{\plotHTTcontrol{\lolcalcurrentyear}{\EMBFFchoice}{\lolcalcurrentchannel}{njets}\vspace{-.125\baselineskip}}

\subcaptionbox{Impulsion transverse du système des deux jets.}[.475\textwidth]
{\plotHTTcontrol{\lolcalcurrentyear}{\EMBFFchoice}{\lolcalcurrentchannel}{dijetpt}\vspace{-.125\baselineskip}}
\hfill
\subcaptionbox{Distance en $\eta$ entre les deux jets.}[.475\textwidth]
{\plotHTTcontrol{\lolcalcurrentyear}{\EMBFFchoice}{\lolcalcurrentchannel}{jdeta}\vspace{-.125\baselineskip}}

\subcaptionbox{Masse invariante du système des deux jets.}[.475\textwidth]
{\plotHTTcontrol{\lolcalcurrentyear}{\EMBFFchoice}{\lolcalcurrentchannel}{mjj}\vspace{-.125\baselineskip}}
\hfill
\subcaptionbox{Nombre de vertex d'empilement.}[.475\textwidth]
{\plotHTTcontrol{\lolcalcurrentyear}{\EMBFFchoice}{\lolcalcurrentchannel}{npv}\vspace{-.125\baselineskip}}

\caption[Distributions de contrôle, \lolcalcurrentyear\ \localchannel, nombre de jets, système des deux jets principaux et empilement.]{Canal \localchannel, \lolcalcurrentyear: nombre de jets, système des deux jets principaux et empilement.}
\end{figure}


\begin{figure}[p]
\centering

\subcaptionbox{Impulsion transverse du lepton 1 (\localLA).}[.475\textwidth]
{\plotHTTcontrol{\lolcalcurrentyear}{\EMBFFchoice}{\lolcalcurrentchannel}{pt_1}\vspace{-.125\baselineskip}}
\hfill
\subcaptionbox{Impulsion transverse du lepton 2 (\localLB).}[.475\textwidth]
{\plotHTTcontrol{\lolcalcurrentyear}{\EMBFFchoice}{\lolcalcurrentchannel}{pt_2}\vspace{-.125\baselineskip}}

\subcaptionbox{Pseudo-rapidité du lepton 1 (\localLA).}[.475\textwidth]
{\plotHTTcontrol{\lolcalcurrentyear}{\EMBFFchoice}{\lolcalcurrentchannel}{eta_1}\vspace{-.125\baselineskip}}
\hfill
\subcaptionbox{Pseudo-rapidité du lepton 2 (\localLB).}[.475\textwidth]
{\plotHTTcontrol{\lolcalcurrentyear}{\EMBFFchoice}{\lolcalcurrentchannel}{eta_2}\vspace{-.125\baselineskip}}

\subcaptionbox{Angle azimutal du lepton 1 (\localLA).}[.475\textwidth]
{\plotHTTcontrol{\lolcalcurrentyear}{\EMBFFchoice}{\lolcalcurrentchannel}{phi_1}\vspace{-.125\baselineskip}}
\hfill
\subcaptionbox{Angle azimutal du lepton 2 (\localLB).}[.475\textwidth]
{\plotHTTcontrol{\lolcalcurrentyear}{\EMBFFchoice}{\lolcalcurrentchannel}{phi_2}\vspace{-.125\baselineskip}}

\caption[Distributions de contrôle, \lolcalcurrentyear\ \localchannel, cinématique des leptons (\localLA, \localLB).]{Canal \localchannel, \lolcalcurrentyear: cinématique des leptons (\localLA, \localLB).}
\end{figure}

\begin{figure}[p]
\centering

\subcaptionbox{Énergie transverse manquante.}[.475\textwidth]
{\plotHTTcontrol{\lolcalcurrentyear}{\EMBFFchoice}{\lolcalcurrentchannel}{puppimet}\vspace{-.125\baselineskip}}
\hfill
\subcaptionbox{Masse transverse du \emph{dilepton}.}[.475\textwidth]
{\plotHTTcontrol{\lolcalcurrentyear}{\EMBFFchoice}{\lolcalcurrentchannel}{mTdileptonMET_puppi}\vspace{-.125\baselineskip}}

\subcaptionbox{Impulsion transverse du \emph{dilepton}.}[.475\textwidth]
{\plotHTTcontrol{\lolcalcurrentyear}{\EMBFFchoice}{\lolcalcurrentchannel}{ptvis}\vspace{-.125\baselineskip}}
\hfill
\subcaptionbox{Masse visible du \emph{dilepton}.}[.475\textwidth]
{\plotHTTcontrol{\lolcalcurrentyear}{\EMBFFchoice}{\lolcalcurrentchannel}{m_vis}\vspace{-.125\baselineskip}}

\subcaptionbox{Impulsion transverse du système di-\tau.}[.475\textwidth]
{\plotHTTcontrol{\lolcalcurrentyear}{\EMBFFchoice}{\lolcalcurrentchannel}{pt_tt_puppi}\vspace{-.125\baselineskip}}
\hfill
\subcaptionbox{Distance $\Delta R$ entre les leptons (\localLA, \localLB).}[.475\textwidth]
{\plotHTTcontrol{\lolcalcurrentyear}{\EMBFFchoice}{\lolcalcurrentchannel}{DiTauDeltaR}\vspace{-.125\baselineskip}}

\caption[Distributions de contrôle, \lolcalcurrentyear\ \localchannel, \emph{dilepton} et énergie transverse manquante.]{Canal \localchannel, \lolcalcurrentyear: \emph{dilepton} et énergie transverse manquante.}
\end{figure}


\begin{figure}[p]
\centering

\subcaptionbox{Masse transverse du lepton 1 (\localLA).}[.475\textwidth]
{\plotHTTcontrol{\lolcalcurrentyear}{\EMBFFchoice}{\lolcalcurrentchannel}{mt_1_puppi}\vspace{-.125\baselineskip}}
\hfill
\subcaptionbox{Masse transverse du lepton 2 (\localLB).}[.475\textwidth]
{\plotHTTcontrol{\lolcalcurrentyear}{\EMBFFchoice}{\lolcalcurrentchannel}{mt_2_puppi}\vspace{-.125\baselineskip}}

\subcaptionbox{Valeur de \Dzeta.}[.475\textwidth]
{\plotHTTcontrol{\lolcalcurrentyear}{\EMBFFchoice}{\lolcalcurrentchannel}{pZetaPuppiMissVis}\vspace{-.125\baselineskip}}
\hfill
\subcaptionbox{Masse transverse totale.}[.475\textwidth]
{\plotHTTcontrol{\lolcalcurrentyear}{\EMBFFchoice}{\lolcalcurrentchannel}{mt_tot_puppi}\vspace{-.125\baselineskip}}


\subcaptionbox{Masse du système di-\tau\ d'après \SVFIT.}[.475\textwidth]
{\plotHTTcontrol{\lolcalcurrentyear}{\EMBFFchoice}{\lolcalcurrentchannel}{m_sv_puppi}\vspace{-.125\baselineskip}}
\hfill
\subcaptionbox{Masse du système di-\tau\ d'après le ML.}[.475\textwidth]
{\plotHTTcontrol{\lolcalcurrentyear}{\EMBFFchoice}{\lolcalcurrentchannel}{ml_mass}\vspace{-.125\baselineskip}}

\caption[Distributions de contrôle, \lolcalcurrentyear\ \localchannel, masses transverses, \Dzeta\ et masses.]{Canal \localchannel, \lolcalcurrentyear: masses transverses, \Dzeta\ et masses.}
\end{figure}

\def\lolcalcurrentchannel{et}
\ifthenelse{\equal{\lolcalcurrentchannel}{tt}}{\def\localchannel{\tauh\tauh}\def\localLA{\ensuremath{\tauh^{(1)}}}\def\localLB{\ensuremath{\tauh^{(2)}}}}{}
\ifthenelse{\equal{\lolcalcurrentchannel}{mt}}{\def\localchannel{\mu\tauh}\def\localLA{\mu}\def\localLB{\ensuremath{\tauh}}}{}
\ifthenelse{\equal{\lolcalcurrentchannel}{et}}{\def\localchannel{\ele\tauh}\def\localLA{\ele}\def\localLB{\ensuremath{\tauh}}}{}
\ifthenelse{\equal{\lolcalcurrentchannel}{em}}{\def\localchannel{\ele\mu}\def\localLA{\ele}\def\localLB{\mu}}{}

\begin{figure}[p]
\centering

\subcaptionbox{Impulsion transverse du jet principal.}[.475\textwidth]
{\plotHTTcontrol{\lolcalcurrentyear}{\EMBFFchoice}{\lolcalcurrentchannel}{jpt_1}\vspace{-.125\baselineskip}}
\hfill
\subcaptionbox{Impulsion transverse du jet secondaire.}[.475\textwidth]
{\plotHTTcontrol{\lolcalcurrentyear}{\EMBFFchoice}{\lolcalcurrentchannel}{jpt_2}\vspace{-.125\baselineskip}}

\subcaptionbox{Pseudo-rapidité du jet principal.}[.475\textwidth]
{\plotHTTcontrol{\lolcalcurrentyear}{\EMBFFchoice}{\lolcalcurrentchannel}{jeta_1}\vspace{-.125\baselineskip}}
\hfill
\subcaptionbox{Pseudo-rapidité du jet secondaire.}[.475\textwidth]
{\plotHTTcontrol{\lolcalcurrentyear}{\EMBFFchoice}{\lolcalcurrentchannel}{jeta_2}\vspace{-.125\baselineskip}}

\subcaptionbox{Angle azimutal du jet principal.}[.475\textwidth]
{\plotHTTcontrol{\lolcalcurrentyear}{\EMBFFchoice}{\lolcalcurrentchannel}{jphi_1}\vspace{-.125\baselineskip}}
\hfill
\subcaptionbox{Angle azimutal du jet secondaire.}[.475\textwidth]
{\plotHTTcontrol{\lolcalcurrentyear}{\EMBFFchoice}{\lolcalcurrentchannel}{jphi_2}\vspace{-.125\baselineskip}}

\caption[Distributions de contrôle, \lolcalcurrentyear\ \localchannel, cinématique des deux jets principaux.]{Canal \localchannel, \lolcalcurrentyear: cinématique des deux jets principaux.}
\end{figure}

\begin{figure}[p]
\centering

\subcaptionbox{Impulsion transverse du \quarkb-jet principal.}[.475\textwidth]
{\plotHTTcontrol{\lolcalcurrentyear}{\EMBFFchoice}{\lolcalcurrentchannel}{bpt_1}\vspace{-.125\baselineskip}}
\hfill
\subcaptionbox{Impulsion transverse du \quarkb-jet secondaire.}[.475\textwidth]
{\plotHTTcontrol{\lolcalcurrentyear}{\EMBFFchoice}{\lolcalcurrentchannel}{bpt_2}\vspace{-.125\baselineskip}}

\subcaptionbox{Impulsion transverse de l'AHA.}[.475\textwidth]
{\plotHTTcontrol{\lolcalcurrentyear}{\EMBFFchoice _jets_r}{\lolcalcurrentchannel}{jpt_r}\vspace{-.125\baselineskip}}
\hfill
\subcaptionbox{Pseudo-rapidité de l'AHA.}[.475\textwidth]
{\plotHTTcontrol{\lolcalcurrentyear}{\EMBFFchoice _jets_r}{\lolcalcurrentchannel}{jeta_r}\vspace{-.125\baselineskip}}

\subcaptionbox{Angle azimutal de l'AHA.}[.475\textwidth]
{\plotHTTcontrol{\lolcalcurrentyear}{\EMBFFchoice _jets_r}{\lolcalcurrentchannel}{jphi_r}\vspace{-.125\baselineskip}}
\hfill
\subcaptionbox{Nombre de jets dans l'AHA.}[.475\textwidth]
{\plotHTTcontrol{\lolcalcurrentyear}{\EMBFFchoice _jets_r}{\lolcalcurrentchannel}{Njet_r}\vspace{-.125\baselineskip}}

\caption[Distributions de contrôle, \lolcalcurrentyear\ \localchannel, \quarkb-jets et activité hadronique additionnelle.]{Canal \localchannel, \lolcalcurrentyear: \quarkb-jets et activité hadronique additionnelle.}
\end{figure}

\begin{figure}[p]
\centering

\subcaptionbox{Nombre de \quarkb-jets.}[.475\textwidth]
{\plotHTTcontrol{\lolcalcurrentyear}{\EMBFFchoice}{\lolcalcurrentchannel}{nbtag}\vspace{-.125\baselineskip}}
\hfill
\subcaptionbox{Nombre de jets.}[.475\textwidth]
{\plotHTTcontrol{\lolcalcurrentyear}{\EMBFFchoice}{\lolcalcurrentchannel}{njets}\vspace{-.125\baselineskip}}

\subcaptionbox{Impulsion transverse du système des deux jets.}[.475\textwidth]
{\plotHTTcontrol{\lolcalcurrentyear}{\EMBFFchoice}{\lolcalcurrentchannel}{dijetpt}\vspace{-.125\baselineskip}}
\hfill
\subcaptionbox{Distance en $\eta$ entre les deux jets.}[.475\textwidth]
{\plotHTTcontrol{\lolcalcurrentyear}{\EMBFFchoice}{\lolcalcurrentchannel}{jdeta}\vspace{-.125\baselineskip}}

\subcaptionbox{Masse invariante du système des deux jets.}[.475\textwidth]
{\plotHTTcontrol{\lolcalcurrentyear}{\EMBFFchoice}{\lolcalcurrentchannel}{mjj}\vspace{-.125\baselineskip}}
\hfill
\subcaptionbox{Nombre de vertex d'empilement.}[.475\textwidth]
{\plotHTTcontrol{\lolcalcurrentyear}{\EMBFFchoice}{\lolcalcurrentchannel}{npv}\vspace{-.125\baselineskip}}

\caption[Distributions de contrôle, \lolcalcurrentyear\ \localchannel, nombre de jets, système des deux jets principaux et empilement.]{Canal \localchannel, \lolcalcurrentyear: nombre de jets, système des deux jets principaux et empilement.}
\end{figure}


\begin{figure}[p]
\centering

\subcaptionbox{Impulsion transverse du lepton 1 (\localLA).}[.475\textwidth]
{\plotHTTcontrol{\lolcalcurrentyear}{\EMBFFchoice}{\lolcalcurrentchannel}{pt_1}\vspace{-.125\baselineskip}}
\hfill
\subcaptionbox{Impulsion transverse du lepton 2 (\localLB).}[.475\textwidth]
{\plotHTTcontrol{\lolcalcurrentyear}{\EMBFFchoice}{\lolcalcurrentchannel}{pt_2}\vspace{-.125\baselineskip}}

\subcaptionbox{Pseudo-rapidité du lepton 1 (\localLA).}[.475\textwidth]
{\plotHTTcontrol{\lolcalcurrentyear}{\EMBFFchoice}{\lolcalcurrentchannel}{eta_1}\vspace{-.125\baselineskip}}
\hfill
\subcaptionbox{Pseudo-rapidité du lepton 2 (\localLB).}[.475\textwidth]
{\plotHTTcontrol{\lolcalcurrentyear}{\EMBFFchoice}{\lolcalcurrentchannel}{eta_2}\vspace{-.125\baselineskip}}

\subcaptionbox{Angle azimutal du lepton 1 (\localLA).}[.475\textwidth]
{\plotHTTcontrol{\lolcalcurrentyear}{\EMBFFchoice}{\lolcalcurrentchannel}{phi_1}\vspace{-.125\baselineskip}}
\hfill
\subcaptionbox{Angle azimutal du lepton 2 (\localLB).}[.475\textwidth]
{\plotHTTcontrol{\lolcalcurrentyear}{\EMBFFchoice}{\lolcalcurrentchannel}{phi_2}\vspace{-.125\baselineskip}}

\caption[Distributions de contrôle, \lolcalcurrentyear\ \localchannel, cinématique des leptons (\localLA, \localLB).]{Canal \localchannel, \lolcalcurrentyear: cinématique des leptons (\localLA, \localLB).}
\end{figure}

\begin{figure}[p]
\centering

\subcaptionbox{Énergie transverse manquante.}[.475\textwidth]
{\plotHTTcontrol{\lolcalcurrentyear}{\EMBFFchoice}{\lolcalcurrentchannel}{puppimet}\vspace{-.125\baselineskip}}
\hfill
\subcaptionbox{Masse transverse du \emph{dilepton}.}[.475\textwidth]
{\plotHTTcontrol{\lolcalcurrentyear}{\EMBFFchoice}{\lolcalcurrentchannel}{mTdileptonMET_puppi}\vspace{-.125\baselineskip}}

\subcaptionbox{Impulsion transverse du \emph{dilepton}.}[.475\textwidth]
{\plotHTTcontrol{\lolcalcurrentyear}{\EMBFFchoice}{\lolcalcurrentchannel}{ptvis}\vspace{-.125\baselineskip}}
\hfill
\subcaptionbox{Masse visible du \emph{dilepton}.}[.475\textwidth]
{\plotHTTcontrol{\lolcalcurrentyear}{\EMBFFchoice}{\lolcalcurrentchannel}{m_vis}\vspace{-.125\baselineskip}}

\subcaptionbox{Impulsion transverse du système di-\tau.}[.475\textwidth]
{\plotHTTcontrol{\lolcalcurrentyear}{\EMBFFchoice}{\lolcalcurrentchannel}{pt_tt_puppi}\vspace{-.125\baselineskip}}
\hfill
\subcaptionbox{Distance $\Delta R$ entre les leptons (\localLA, \localLB).}[.475\textwidth]
{\plotHTTcontrol{\lolcalcurrentyear}{\EMBFFchoice}{\lolcalcurrentchannel}{DiTauDeltaR}\vspace{-.125\baselineskip}}

\caption[Distributions de contrôle, \lolcalcurrentyear\ \localchannel, \emph{dilepton} et énergie transverse manquante.]{Canal \localchannel, \lolcalcurrentyear: \emph{dilepton} et énergie transverse manquante.}
\end{figure}


\begin{figure}[p]
\centering

\subcaptionbox{Masse transverse du lepton 1 (\localLA).}[.475\textwidth]
{\plotHTTcontrol{\lolcalcurrentyear}{\EMBFFchoice}{\lolcalcurrentchannel}{mt_1_puppi}\vspace{-.125\baselineskip}}
\hfill
\subcaptionbox{Masse transverse du lepton 2 (\localLB).}[.475\textwidth]
{\plotHTTcontrol{\lolcalcurrentyear}{\EMBFFchoice}{\lolcalcurrentchannel}{mt_2_puppi}\vspace{-.125\baselineskip}}

\subcaptionbox{Valeur de \Dzeta.}[.475\textwidth]
{\plotHTTcontrol{\lolcalcurrentyear}{\EMBFFchoice}{\lolcalcurrentchannel}{pZetaPuppiMissVis}\vspace{-.125\baselineskip}}
\hfill
\subcaptionbox{Masse transverse totale.}[.475\textwidth]
{\plotHTTcontrol{\lolcalcurrentyear}{\EMBFFchoice}{\lolcalcurrentchannel}{mt_tot_puppi}\vspace{-.125\baselineskip}}


\subcaptionbox{Masse du système di-\tau\ d'après \SVFIT.}[.475\textwidth]
{\plotHTTcontrol{\lolcalcurrentyear}{\EMBFFchoice}{\lolcalcurrentchannel}{m_sv_puppi}\vspace{-.125\baselineskip}}
\hfill
\subcaptionbox{Masse du système di-\tau\ d'après le ML.}[.475\textwidth]
{\plotHTTcontrol{\lolcalcurrentyear}{\EMBFFchoice}{\lolcalcurrentchannel}{ml_mass}\vspace{-.125\baselineskip}}

\caption[Distributions de contrôle, \lolcalcurrentyear\ \localchannel, masses transverses, \Dzeta\ et masses.]{Canal \localchannel, \lolcalcurrentyear: masses transverses, \Dzeta\ et masses.}
\end{figure}

\def\lolcalcurrentchannel{em}
\ifthenelse{\equal{\lolcalcurrentchannel}{tt}}{\def\localchannel{\tauh\tauh}\def\localLA{\ensuremath{\tauh^{(1)}}}\def\localLB{\ensuremath{\tauh^{(2)}}}}{}
\ifthenelse{\equal{\lolcalcurrentchannel}{mt}}{\def\localchannel{\mu\tauh}\def\localLA{\mu}\def\localLB{\ensuremath{\tauh}}}{}
\ifthenelse{\equal{\lolcalcurrentchannel}{et}}{\def\localchannel{\ele\tauh}\def\localLA{\ele}\def\localLB{\ensuremath{\tauh}}}{}
\ifthenelse{\equal{\lolcalcurrentchannel}{em}}{\def\localchannel{\ele\mu}\def\localLA{\ele}\def\localLB{\mu}}{}

\begin{figure}[p]
\centering

\subcaptionbox{Impulsion transverse du jet principal.}[.475\textwidth]
{\plotHTTcontrol{\lolcalcurrentyear}{\EMBFFchoice}{\lolcalcurrentchannel}{jpt_1}\vspace{-.125\baselineskip}}
\hfill
\subcaptionbox{Impulsion transverse du jet secondaire.}[.475\textwidth]
{\plotHTTcontrol{\lolcalcurrentyear}{\EMBFFchoice}{\lolcalcurrentchannel}{jpt_2}\vspace{-.125\baselineskip}}

\subcaptionbox{Pseudo-rapidité du jet principal.}[.475\textwidth]
{\plotHTTcontrol{\lolcalcurrentyear}{\EMBFFchoice}{\lolcalcurrentchannel}{jeta_1}\vspace{-.125\baselineskip}}
\hfill
\subcaptionbox{Pseudo-rapidité du jet secondaire.}[.475\textwidth]
{\plotHTTcontrol{\lolcalcurrentyear}{\EMBFFchoice}{\lolcalcurrentchannel}{jeta_2}\vspace{-.125\baselineskip}}

\subcaptionbox{Angle azimutal du jet principal.}[.475\textwidth]
{\plotHTTcontrol{\lolcalcurrentyear}{\EMBFFchoice}{\lolcalcurrentchannel}{jphi_1}\vspace{-.125\baselineskip}}
\hfill
\subcaptionbox{Angle azimutal du jet secondaire.}[.475\textwidth]
{\plotHTTcontrol{\lolcalcurrentyear}{\EMBFFchoice}{\lolcalcurrentchannel}{jphi_2}\vspace{-.125\baselineskip}}

\caption[Distributions de contrôle, \lolcalcurrentyear\ \localchannel, cinématique des deux jets principaux.]{Canal \localchannel, \lolcalcurrentyear: cinématique des deux jets principaux.}
\end{figure}

\begin{figure}[p]
\centering

\subcaptionbox{Impulsion transverse du \quarkb-jet principal.}[.475\textwidth]
{\plotHTTcontrol{\lolcalcurrentyear}{\EMBFFchoice}{\lolcalcurrentchannel}{bpt_1}\vspace{-.125\baselineskip}}
\hfill
\subcaptionbox{Impulsion transverse du \quarkb-jet secondaire.}[.475\textwidth]
{\plotHTTcontrol{\lolcalcurrentyear}{\EMBFFchoice}{\lolcalcurrentchannel}{bpt_2}\vspace{-.125\baselineskip}}

\subcaptionbox{Impulsion transverse de l'AHA.}[.475\textwidth]
{\plotHTTcontrol{\lolcalcurrentyear}{\EMBFFchoice _jets_r}{\lolcalcurrentchannel}{jpt_r}\vspace{-.125\baselineskip}}
\hfill
\subcaptionbox{Pseudo-rapidité de l'AHA.}[.475\textwidth]
{\plotHTTcontrol{\lolcalcurrentyear}{\EMBFFchoice _jets_r}{\lolcalcurrentchannel}{jeta_r}\vspace{-.125\baselineskip}}

\subcaptionbox{Angle azimutal de l'AHA.}[.475\textwidth]
{\plotHTTcontrol{\lolcalcurrentyear}{\EMBFFchoice _jets_r}{\lolcalcurrentchannel}{jphi_r}\vspace{-.125\baselineskip}}
\hfill
\subcaptionbox{Nombre de jets dans l'AHA.}[.475\textwidth]
{\plotHTTcontrol{\lolcalcurrentyear}{\EMBFFchoice _jets_r}{\lolcalcurrentchannel}{Njet_r}\vspace{-.125\baselineskip}}

\caption[Distributions de contrôle, \lolcalcurrentyear\ \localchannel, \quarkb-jets et activité hadronique additionnelle.]{Canal \localchannel, \lolcalcurrentyear: \quarkb-jets et activité hadronique additionnelle.}
\end{figure}

\begin{figure}[p]
\centering

\subcaptionbox{Nombre de \quarkb-jets.}[.475\textwidth]
{\plotHTTcontrol{\lolcalcurrentyear}{\EMBFFchoice}{\lolcalcurrentchannel}{nbtag}\vspace{-.125\baselineskip}}
\hfill
\subcaptionbox{Nombre de jets.}[.475\textwidth]
{\plotHTTcontrol{\lolcalcurrentyear}{\EMBFFchoice}{\lolcalcurrentchannel}{njets}\vspace{-.125\baselineskip}}

\subcaptionbox{Impulsion transverse du système des deux jets.}[.475\textwidth]
{\plotHTTcontrol{\lolcalcurrentyear}{\EMBFFchoice}{\lolcalcurrentchannel}{dijetpt}\vspace{-.125\baselineskip}}
\hfill
\subcaptionbox{Distance en $\eta$ entre les deux jets.}[.475\textwidth]
{\plotHTTcontrol{\lolcalcurrentyear}{\EMBFFchoice}{\lolcalcurrentchannel}{jdeta}\vspace{-.125\baselineskip}}

\subcaptionbox{Masse invariante du système des deux jets.}[.475\textwidth]
{\plotHTTcontrol{\lolcalcurrentyear}{\EMBFFchoice}{\lolcalcurrentchannel}{mjj}\vspace{-.125\baselineskip}}
\hfill
\subcaptionbox{Nombre de vertex d'empilement.}[.475\textwidth]
{\plotHTTcontrol{\lolcalcurrentyear}{\EMBFFchoice}{\lolcalcurrentchannel}{npv}\vspace{-.125\baselineskip}}

\caption[Distributions de contrôle, \lolcalcurrentyear\ \localchannel, nombre de jets, système des deux jets principaux et empilement.]{Canal \localchannel, \lolcalcurrentyear: nombre de jets, système des deux jets principaux et empilement.}
\end{figure}


\begin{figure}[p]
\centering

\subcaptionbox{Impulsion transverse du lepton 1 (\localLA).}[.475\textwidth]
{\plotHTTcontrol{\lolcalcurrentyear}{\EMBFFchoice}{\lolcalcurrentchannel}{pt_1}\vspace{-.125\baselineskip}}
\hfill
\subcaptionbox{Impulsion transverse du lepton 2 (\localLB).}[.475\textwidth]
{\plotHTTcontrol{\lolcalcurrentyear}{\EMBFFchoice}{\lolcalcurrentchannel}{pt_2}\vspace{-.125\baselineskip}}

\subcaptionbox{Pseudo-rapidité du lepton 1 (\localLA).}[.475\textwidth]
{\plotHTTcontrol{\lolcalcurrentyear}{\EMBFFchoice}{\lolcalcurrentchannel}{eta_1}\vspace{-.125\baselineskip}}
\hfill
\subcaptionbox{Pseudo-rapidité du lepton 2 (\localLB).}[.475\textwidth]
{\plotHTTcontrol{\lolcalcurrentyear}{\EMBFFchoice}{\lolcalcurrentchannel}{eta_2}\vspace{-.125\baselineskip}}

\subcaptionbox{Angle azimutal du lepton 1 (\localLA).}[.475\textwidth]
{\plotHTTcontrol{\lolcalcurrentyear}{\EMBFFchoice}{\lolcalcurrentchannel}{phi_1}\vspace{-.125\baselineskip}}
\hfill
\subcaptionbox{Angle azimutal du lepton 2 (\localLB).}[.475\textwidth]
{\plotHTTcontrol{\lolcalcurrentyear}{\EMBFFchoice}{\lolcalcurrentchannel}{phi_2}\vspace{-.125\baselineskip}}

\caption[Distributions de contrôle, \lolcalcurrentyear\ \localchannel, cinématique des leptons (\localLA, \localLB).]{Canal \localchannel, \lolcalcurrentyear: cinématique des leptons (\localLA, \localLB).}
\end{figure}

\begin{figure}[p]
\centering

\subcaptionbox{Énergie transverse manquante.}[.475\textwidth]
{\plotHTTcontrol{\lolcalcurrentyear}{\EMBFFchoice}{\lolcalcurrentchannel}{puppimet}\vspace{-.125\baselineskip}}
\hfill
\subcaptionbox{Masse transverse du \emph{dilepton}.}[.475\textwidth]
{\plotHTTcontrol{\lolcalcurrentyear}{\EMBFFchoice}{\lolcalcurrentchannel}{mTdileptonMET_puppi}\vspace{-.125\baselineskip}}

\subcaptionbox{Impulsion transverse du \emph{dilepton}.}[.475\textwidth]
{\plotHTTcontrol{\lolcalcurrentyear}{\EMBFFchoice}{\lolcalcurrentchannel}{ptvis}\vspace{-.125\baselineskip}}
\hfill
\subcaptionbox{Masse visible du \emph{dilepton}.}[.475\textwidth]
{\plotHTTcontrol{\lolcalcurrentyear}{\EMBFFchoice}{\lolcalcurrentchannel}{m_vis}\vspace{-.125\baselineskip}}

\subcaptionbox{Impulsion transverse du système di-\tau.}[.475\textwidth]
{\plotHTTcontrol{\lolcalcurrentyear}{\EMBFFchoice}{\lolcalcurrentchannel}{pt_tt_puppi}\vspace{-.125\baselineskip}}
\hfill
\subcaptionbox{Distance $\Delta R$ entre les leptons (\localLA, \localLB).}[.475\textwidth]
{\plotHTTcontrol{\lolcalcurrentyear}{\EMBFFchoice}{\lolcalcurrentchannel}{DiTauDeltaR}\vspace{-.125\baselineskip}}

\caption[Distributions de contrôle, \lolcalcurrentyear\ \localchannel, \emph{dilepton} et énergie transverse manquante.]{Canal \localchannel, \lolcalcurrentyear: \emph{dilepton} et énergie transverse manquante.}
\end{figure}


\begin{figure}[p]
\centering

\subcaptionbox{Masse transverse du lepton 1 (\localLA).}[.475\textwidth]
{\plotHTTcontrol{\lolcalcurrentyear}{\EMBFFchoice}{\lolcalcurrentchannel}{mt_1_puppi}\vspace{-.125\baselineskip}}
\hfill
\subcaptionbox{Masse transverse du lepton 2 (\localLB).}[.475\textwidth]
{\plotHTTcontrol{\lolcalcurrentyear}{\EMBFFchoice}{\lolcalcurrentchannel}{mt_2_puppi}\vspace{-.125\baselineskip}}

\subcaptionbox{Valeur de \Dzeta.}[.475\textwidth]
{\plotHTTcontrol{\lolcalcurrentyear}{\EMBFFchoice}{\lolcalcurrentchannel}{pZetaPuppiMissVis}\vspace{-.125\baselineskip}}
\hfill
\subcaptionbox{Masse transverse totale.}[.475\textwidth]
{\plotHTTcontrol{\lolcalcurrentyear}{\EMBFFchoice}{\lolcalcurrentchannel}{mt_tot_puppi}\vspace{-.125\baselineskip}}


\subcaptionbox{Masse du système di-\tau\ d'après \SVFIT.}[.475\textwidth]
{\plotHTTcontrol{\lolcalcurrentyear}{\EMBFFchoice}{\lolcalcurrentchannel}{m_sv_puppi}\vspace{-.125\baselineskip}}
\hfill
\subcaptionbox{Masse du système di-\tau\ d'après le ML.}[.475\textwidth]
{\plotHTTcontrol{\lolcalcurrentyear}{\EMBFFchoice}{\lolcalcurrentchannel}{ml_mass}\vspace{-.125\baselineskip}}

\caption[Distributions de contrôle, \lolcalcurrentyear\ \localchannel, masses transverses, \Dzeta\ et masses.]{Canal \localchannel, \lolcalcurrentyear: masses transverses, \Dzeta\ et masses.}
\end{figure}
\clearpage

\def\lolcalcurrentyear{2018}
\def\lolcalcurrentchannel{tt}
\ifthenelse{\equal{\lolcalcurrentchannel}{tt}}{\def\localchannel{\tauh\tauh}\def\localLA{\ensuremath{\tauh^{(1)}}}\def\localLB{\ensuremath{\tauh^{(2)}}}}{}
\ifthenelse{\equal{\lolcalcurrentchannel}{mt}}{\def\localchannel{\mu\tauh}\def\localLA{\mu}\def\localLB{\ensuremath{\tauh}}}{}
\ifthenelse{\equal{\lolcalcurrentchannel}{et}}{\def\localchannel{\ele\tauh}\def\localLA{\ele}\def\localLB{\ensuremath{\tauh}}}{}
\ifthenelse{\equal{\lolcalcurrentchannel}{em}}{\def\localchannel{\ele\mu}\def\localLA{\ele}\def\localLB{\mu}}{}

\begin{figure}[p]
\centering

\subcaptionbox{Impulsion transverse du jet principal.}[.475\textwidth]
{\plotHTTcontrol{\lolcalcurrentyear}{\EMBFFchoice}{\lolcalcurrentchannel}{jpt_1}\vspace{-.125\baselineskip}}
\hfill
\subcaptionbox{Impulsion transverse du jet secondaire.}[.475\textwidth]
{\plotHTTcontrol{\lolcalcurrentyear}{\EMBFFchoice}{\lolcalcurrentchannel}{jpt_2}\vspace{-.125\baselineskip}}

\subcaptionbox{Pseudo-rapidité du jet principal.}[.475\textwidth]
{\plotHTTcontrol{\lolcalcurrentyear}{\EMBFFchoice}{\lolcalcurrentchannel}{jeta_1}\vspace{-.125\baselineskip}}
\hfill
\subcaptionbox{Pseudo-rapidité du jet secondaire.}[.475\textwidth]
{\plotHTTcontrol{\lolcalcurrentyear}{\EMBFFchoice}{\lolcalcurrentchannel}{jeta_2}\vspace{-.125\baselineskip}}

\subcaptionbox{Angle azimutal du jet principal.}[.475\textwidth]
{\plotHTTcontrol{\lolcalcurrentyear}{\EMBFFchoice}{\lolcalcurrentchannel}{jphi_1}\vspace{-.125\baselineskip}}
\hfill
\subcaptionbox{Angle azimutal du jet secondaire.}[.475\textwidth]
{\plotHTTcontrol{\lolcalcurrentyear}{\EMBFFchoice}{\lolcalcurrentchannel}{jphi_2}\vspace{-.125\baselineskip}}

\caption[Distributions de contrôle, \lolcalcurrentyear\ \localchannel, cinématique des deux jets principaux.]{Canal \localchannel, \lolcalcurrentyear: cinématique des deux jets principaux.}
\end{figure}

\begin{figure}[p]
\centering

\subcaptionbox{Impulsion transverse du \quarkb-jet principal.}[.475\textwidth]
{\plotHTTcontrol{\lolcalcurrentyear}{\EMBFFchoice}{\lolcalcurrentchannel}{bpt_1}\vspace{-.125\baselineskip}}
\hfill
\subcaptionbox{Impulsion transverse du \quarkb-jet secondaire.}[.475\textwidth]
{\plotHTTcontrol{\lolcalcurrentyear}{\EMBFFchoice}{\lolcalcurrentchannel}{bpt_2}\vspace{-.125\baselineskip}}

\subcaptionbox{Impulsion transverse de l'AHA.}[.475\textwidth]
{\plotHTTcontrol{\lolcalcurrentyear}{\EMBFFchoice _jets_r}{\lolcalcurrentchannel}{jpt_r}\vspace{-.125\baselineskip}}
\hfill
\subcaptionbox{Pseudo-rapidité de l'AHA.}[.475\textwidth]
{\plotHTTcontrol{\lolcalcurrentyear}{\EMBFFchoice _jets_r}{\lolcalcurrentchannel}{jeta_r}\vspace{-.125\baselineskip}}

\subcaptionbox{Angle azimutal de l'AHA.}[.475\textwidth]
{\plotHTTcontrol{\lolcalcurrentyear}{\EMBFFchoice _jets_r}{\lolcalcurrentchannel}{jphi_r}\vspace{-.125\baselineskip}}
\hfill
\subcaptionbox{Nombre de jets dans l'AHA.}[.475\textwidth]
{\plotHTTcontrol{\lolcalcurrentyear}{\EMBFFchoice _jets_r}{\lolcalcurrentchannel}{Njet_r}\vspace{-.125\baselineskip}}

\caption[Distributions de contrôle, \lolcalcurrentyear\ \localchannel, \quarkb-jets et activité hadronique additionnelle.]{Canal \localchannel, \lolcalcurrentyear: \quarkb-jets et activité hadronique additionnelle.}
\end{figure}

\begin{figure}[p]
\centering

\subcaptionbox{Nombre de \quarkb-jets.}[.475\textwidth]
{\plotHTTcontrol{\lolcalcurrentyear}{\EMBFFchoice}{\lolcalcurrentchannel}{nbtag}\vspace{-.125\baselineskip}}
\hfill
\subcaptionbox{Nombre de jets.}[.475\textwidth]
{\plotHTTcontrol{\lolcalcurrentyear}{\EMBFFchoice}{\lolcalcurrentchannel}{njets}\vspace{-.125\baselineskip}}

\subcaptionbox{Impulsion transverse du système des deux jets.}[.475\textwidth]
{\plotHTTcontrol{\lolcalcurrentyear}{\EMBFFchoice}{\lolcalcurrentchannel}{dijetpt}\vspace{-.125\baselineskip}}
\hfill
\subcaptionbox{Distance en $\eta$ entre les deux jets.}[.475\textwidth]
{\plotHTTcontrol{\lolcalcurrentyear}{\EMBFFchoice}{\lolcalcurrentchannel}{jdeta}\vspace{-.125\baselineskip}}

\subcaptionbox{Masse invariante du système des deux jets.}[.475\textwidth]
{\plotHTTcontrol{\lolcalcurrentyear}{\EMBFFchoice}{\lolcalcurrentchannel}{mjj}\vspace{-.125\baselineskip}}
\hfill
\subcaptionbox{Nombre de vertex d'empilement.}[.475\textwidth]
{\plotHTTcontrol{\lolcalcurrentyear}{\EMBFFchoice}{\lolcalcurrentchannel}{npv}\vspace{-.125\baselineskip}}

\caption[Distributions de contrôle, \lolcalcurrentyear\ \localchannel, nombre de jets, système des deux jets principaux et empilement.]{Canal \localchannel, \lolcalcurrentyear: nombre de jets, système des deux jets principaux et empilement.}
\end{figure}


\begin{figure}[p]
\centering

\subcaptionbox{Impulsion transverse du lepton 1 (\localLA).}[.475\textwidth]
{\plotHTTcontrol{\lolcalcurrentyear}{\EMBFFchoice}{\lolcalcurrentchannel}{pt_1}\vspace{-.125\baselineskip}}
\hfill
\subcaptionbox{Impulsion transverse du lepton 2 (\localLB).}[.475\textwidth]
{\plotHTTcontrol{\lolcalcurrentyear}{\EMBFFchoice}{\lolcalcurrentchannel}{pt_2}\vspace{-.125\baselineskip}}

\subcaptionbox{Pseudo-rapidité du lepton 1 (\localLA).}[.475\textwidth]
{\plotHTTcontrol{\lolcalcurrentyear}{\EMBFFchoice}{\lolcalcurrentchannel}{eta_1}\vspace{-.125\baselineskip}}
\hfill
\subcaptionbox{Pseudo-rapidité du lepton 2 (\localLB).}[.475\textwidth]
{\plotHTTcontrol{\lolcalcurrentyear}{\EMBFFchoice}{\lolcalcurrentchannel}{eta_2}\vspace{-.125\baselineskip}}

\subcaptionbox{Angle azimutal du lepton 1 (\localLA).}[.475\textwidth]
{\plotHTTcontrol{\lolcalcurrentyear}{\EMBFFchoice}{\lolcalcurrentchannel}{phi_1}\vspace{-.125\baselineskip}}
\hfill
\subcaptionbox{Angle azimutal du lepton 2 (\localLB).}[.475\textwidth]
{\plotHTTcontrol{\lolcalcurrentyear}{\EMBFFchoice}{\lolcalcurrentchannel}{phi_2}\vspace{-.125\baselineskip}}

\caption[Distributions de contrôle, \lolcalcurrentyear\ \localchannel, cinématique des leptons (\localLA, \localLB).]{Canal \localchannel, \lolcalcurrentyear: cinématique des leptons (\localLA, \localLB).}
\end{figure}

\begin{figure}[p]
\centering

\subcaptionbox{Énergie transverse manquante.}[.475\textwidth]
{\plotHTTcontrol{\lolcalcurrentyear}{\EMBFFchoice}{\lolcalcurrentchannel}{puppimet}\vspace{-.125\baselineskip}}
\hfill
\subcaptionbox{Masse transverse du \emph{dilepton}.}[.475\textwidth]
{\plotHTTcontrol{\lolcalcurrentyear}{\EMBFFchoice}{\lolcalcurrentchannel}{mTdileptonMET_puppi}\vspace{-.125\baselineskip}}

\subcaptionbox{Impulsion transverse du \emph{dilepton}.}[.475\textwidth]
{\plotHTTcontrol{\lolcalcurrentyear}{\EMBFFchoice}{\lolcalcurrentchannel}{ptvis}\vspace{-.125\baselineskip}}
\hfill
\subcaptionbox{Masse visible du \emph{dilepton}.}[.475\textwidth]
{\plotHTTcontrol{\lolcalcurrentyear}{\EMBFFchoice}{\lolcalcurrentchannel}{m_vis}\vspace{-.125\baselineskip}}

\subcaptionbox{Impulsion transverse du système di-\tau.}[.475\textwidth]
{\plotHTTcontrol{\lolcalcurrentyear}{\EMBFFchoice}{\lolcalcurrentchannel}{pt_tt_puppi}\vspace{-.125\baselineskip}}
\hfill
\subcaptionbox{Distance $\Delta R$ entre les leptons (\localLA, \localLB).}[.475\textwidth]
{\plotHTTcontrol{\lolcalcurrentyear}{\EMBFFchoice}{\lolcalcurrentchannel}{DiTauDeltaR}\vspace{-.125\baselineskip}}

\caption[Distributions de contrôle, \lolcalcurrentyear\ \localchannel, \emph{dilepton} et énergie transverse manquante.]{Canal \localchannel, \lolcalcurrentyear: \emph{dilepton} et énergie transverse manquante.}
\end{figure}


\begin{figure}[p]
\centering

\subcaptionbox{Masse transverse du lepton 1 (\localLA).}[.475\textwidth]
{\plotHTTcontrol{\lolcalcurrentyear}{\EMBFFchoice}{\lolcalcurrentchannel}{mt_1_puppi}\vspace{-.125\baselineskip}}
\hfill
\subcaptionbox{Masse transverse du lepton 2 (\localLB).}[.475\textwidth]
{\plotHTTcontrol{\lolcalcurrentyear}{\EMBFFchoice}{\lolcalcurrentchannel}{mt_2_puppi}\vspace{-.125\baselineskip}}

\subcaptionbox{Valeur de \Dzeta.}[.475\textwidth]
{\plotHTTcontrol{\lolcalcurrentyear}{\EMBFFchoice}{\lolcalcurrentchannel}{pZetaPuppiMissVis}\vspace{-.125\baselineskip}}
\hfill
\subcaptionbox{Masse transverse totale.}[.475\textwidth]
{\plotHTTcontrol{\lolcalcurrentyear}{\EMBFFchoice}{\lolcalcurrentchannel}{mt_tot_puppi}\vspace{-.125\baselineskip}}


\subcaptionbox{Masse du système di-\tau\ d'après \SVFIT.}[.475\textwidth]
{\plotHTTcontrol{\lolcalcurrentyear}{\EMBFFchoice}{\lolcalcurrentchannel}{m_sv_puppi}\vspace{-.125\baselineskip}}
\hfill
\subcaptionbox{Masse du système di-\tau\ d'après le ML.}[.475\textwidth]
{\plotHTTcontrol{\lolcalcurrentyear}{\EMBFFchoice}{\lolcalcurrentchannel}{ml_mass}\vspace{-.125\baselineskip}}

\caption[Distributions de contrôle, \lolcalcurrentyear\ \localchannel, masses transverses, \Dzeta\ et masses.]{Canal \localchannel, \lolcalcurrentyear: masses transverses, \Dzeta\ et masses.}
\end{figure}

\def\lolcalcurrentchannel{mt}
\ifthenelse{\equal{\lolcalcurrentchannel}{tt}}{\def\localchannel{\tauh\tauh}\def\localLA{\ensuremath{\tauh^{(1)}}}\def\localLB{\ensuremath{\tauh^{(2)}}}}{}
\ifthenelse{\equal{\lolcalcurrentchannel}{mt}}{\def\localchannel{\mu\tauh}\def\localLA{\mu}\def\localLB{\ensuremath{\tauh}}}{}
\ifthenelse{\equal{\lolcalcurrentchannel}{et}}{\def\localchannel{\ele\tauh}\def\localLA{\ele}\def\localLB{\ensuremath{\tauh}}}{}
\ifthenelse{\equal{\lolcalcurrentchannel}{em}}{\def\localchannel{\ele\mu}\def\localLA{\ele}\def\localLB{\mu}}{}

\begin{figure}[p]
\centering

\subcaptionbox{Impulsion transverse du jet principal.}[.475\textwidth]
{\plotHTTcontrol{\lolcalcurrentyear}{\EMBFFchoice}{\lolcalcurrentchannel}{jpt_1}\vspace{-.125\baselineskip}}
\hfill
\subcaptionbox{Impulsion transverse du jet secondaire.}[.475\textwidth]
{\plotHTTcontrol{\lolcalcurrentyear}{\EMBFFchoice}{\lolcalcurrentchannel}{jpt_2}\vspace{-.125\baselineskip}}

\subcaptionbox{Pseudo-rapidité du jet principal.}[.475\textwidth]
{\plotHTTcontrol{\lolcalcurrentyear}{\EMBFFchoice}{\lolcalcurrentchannel}{jeta_1}\vspace{-.125\baselineskip}}
\hfill
\subcaptionbox{Pseudo-rapidité du jet secondaire.}[.475\textwidth]
{\plotHTTcontrol{\lolcalcurrentyear}{\EMBFFchoice}{\lolcalcurrentchannel}{jeta_2}\vspace{-.125\baselineskip}}

\subcaptionbox{Angle azimutal du jet principal.}[.475\textwidth]
{\plotHTTcontrol{\lolcalcurrentyear}{\EMBFFchoice}{\lolcalcurrentchannel}{jphi_1}\vspace{-.125\baselineskip}}
\hfill
\subcaptionbox{Angle azimutal du jet secondaire.}[.475\textwidth]
{\plotHTTcontrol{\lolcalcurrentyear}{\EMBFFchoice}{\lolcalcurrentchannel}{jphi_2}\vspace{-.125\baselineskip}}

\caption[Distributions de contrôle, \lolcalcurrentyear\ \localchannel, cinématique des deux jets principaux.]{Canal \localchannel, \lolcalcurrentyear: cinématique des deux jets principaux.}
\end{figure}

\begin{figure}[p]
\centering

\subcaptionbox{Impulsion transverse du \quarkb-jet principal.}[.475\textwidth]
{\plotHTTcontrol{\lolcalcurrentyear}{\EMBFFchoice}{\lolcalcurrentchannel}{bpt_1}\vspace{-.125\baselineskip}}
\hfill
\subcaptionbox{Impulsion transverse du \quarkb-jet secondaire.}[.475\textwidth]
{\plotHTTcontrol{\lolcalcurrentyear}{\EMBFFchoice}{\lolcalcurrentchannel}{bpt_2}\vspace{-.125\baselineskip}}

\subcaptionbox{Impulsion transverse de l'AHA.}[.475\textwidth]
{\plotHTTcontrol{\lolcalcurrentyear}{\EMBFFchoice _jets_r}{\lolcalcurrentchannel}{jpt_r}\vspace{-.125\baselineskip}}
\hfill
\subcaptionbox{Pseudo-rapidité de l'AHA.}[.475\textwidth]
{\plotHTTcontrol{\lolcalcurrentyear}{\EMBFFchoice _jets_r}{\lolcalcurrentchannel}{jeta_r}\vspace{-.125\baselineskip}}

\subcaptionbox{Angle azimutal de l'AHA.}[.475\textwidth]
{\plotHTTcontrol{\lolcalcurrentyear}{\EMBFFchoice _jets_r}{\lolcalcurrentchannel}{jphi_r}\vspace{-.125\baselineskip}}
\hfill
\subcaptionbox{Nombre de jets dans l'AHA.}[.475\textwidth]
{\plotHTTcontrol{\lolcalcurrentyear}{\EMBFFchoice _jets_r}{\lolcalcurrentchannel}{Njet_r}\vspace{-.125\baselineskip}}

\caption[Distributions de contrôle, \lolcalcurrentyear\ \localchannel, \quarkb-jets et activité hadronique additionnelle.]{Canal \localchannel, \lolcalcurrentyear: \quarkb-jets et activité hadronique additionnelle.}
\end{figure}

\begin{figure}[p]
\centering

\subcaptionbox{Nombre de \quarkb-jets.}[.475\textwidth]
{\plotHTTcontrol{\lolcalcurrentyear}{\EMBFFchoice}{\lolcalcurrentchannel}{nbtag}\vspace{-.125\baselineskip}}
\hfill
\subcaptionbox{Nombre de jets.}[.475\textwidth]
{\plotHTTcontrol{\lolcalcurrentyear}{\EMBFFchoice}{\lolcalcurrentchannel}{njets}\vspace{-.125\baselineskip}}

\subcaptionbox{Impulsion transverse du système des deux jets.}[.475\textwidth]
{\plotHTTcontrol{\lolcalcurrentyear}{\EMBFFchoice}{\lolcalcurrentchannel}{dijetpt}\vspace{-.125\baselineskip}}
\hfill
\subcaptionbox{Distance en $\eta$ entre les deux jets.}[.475\textwidth]
{\plotHTTcontrol{\lolcalcurrentyear}{\EMBFFchoice}{\lolcalcurrentchannel}{jdeta}\vspace{-.125\baselineskip}}

\subcaptionbox{Masse invariante du système des deux jets.}[.475\textwidth]
{\plotHTTcontrol{\lolcalcurrentyear}{\EMBFFchoice}{\lolcalcurrentchannel}{mjj}\vspace{-.125\baselineskip}}
\hfill
\subcaptionbox{Nombre de vertex d'empilement.}[.475\textwidth]
{\plotHTTcontrol{\lolcalcurrentyear}{\EMBFFchoice}{\lolcalcurrentchannel}{npv}\vspace{-.125\baselineskip}}

\caption[Distributions de contrôle, \lolcalcurrentyear\ \localchannel, nombre de jets, système des deux jets principaux et empilement.]{Canal \localchannel, \lolcalcurrentyear: nombre de jets, système des deux jets principaux et empilement.}
\end{figure}


\begin{figure}[p]
\centering

\subcaptionbox{Impulsion transverse du lepton 1 (\localLA).}[.475\textwidth]
{\plotHTTcontrol{\lolcalcurrentyear}{\EMBFFchoice}{\lolcalcurrentchannel}{pt_1}\vspace{-.125\baselineskip}}
\hfill
\subcaptionbox{Impulsion transverse du lepton 2 (\localLB).}[.475\textwidth]
{\plotHTTcontrol{\lolcalcurrentyear}{\EMBFFchoice}{\lolcalcurrentchannel}{pt_2}\vspace{-.125\baselineskip}}

\subcaptionbox{Pseudo-rapidité du lepton 1 (\localLA).}[.475\textwidth]
{\plotHTTcontrol{\lolcalcurrentyear}{\EMBFFchoice}{\lolcalcurrentchannel}{eta_1}\vspace{-.125\baselineskip}}
\hfill
\subcaptionbox{Pseudo-rapidité du lepton 2 (\localLB).}[.475\textwidth]
{\plotHTTcontrol{\lolcalcurrentyear}{\EMBFFchoice}{\lolcalcurrentchannel}{eta_2}\vspace{-.125\baselineskip}}

\subcaptionbox{Angle azimutal du lepton 1 (\localLA).}[.475\textwidth]
{\plotHTTcontrol{\lolcalcurrentyear}{\EMBFFchoice}{\lolcalcurrentchannel}{phi_1}\vspace{-.125\baselineskip}}
\hfill
\subcaptionbox{Angle azimutal du lepton 2 (\localLB).}[.475\textwidth]
{\plotHTTcontrol{\lolcalcurrentyear}{\EMBFFchoice}{\lolcalcurrentchannel}{phi_2}\vspace{-.125\baselineskip}}

\caption[Distributions de contrôle, \lolcalcurrentyear\ \localchannel, cinématique des leptons (\localLA, \localLB).]{Canal \localchannel, \lolcalcurrentyear: cinématique des leptons (\localLA, \localLB).}
\end{figure}

\begin{figure}[p]
\centering

\subcaptionbox{Énergie transverse manquante.}[.475\textwidth]
{\plotHTTcontrol{\lolcalcurrentyear}{\EMBFFchoice}{\lolcalcurrentchannel}{puppimet}\vspace{-.125\baselineskip}}
\hfill
\subcaptionbox{Masse transverse du \emph{dilepton}.}[.475\textwidth]
{\plotHTTcontrol{\lolcalcurrentyear}{\EMBFFchoice}{\lolcalcurrentchannel}{mTdileptonMET_puppi}\vspace{-.125\baselineskip}}

\subcaptionbox{Impulsion transverse du \emph{dilepton}.}[.475\textwidth]
{\plotHTTcontrol{\lolcalcurrentyear}{\EMBFFchoice}{\lolcalcurrentchannel}{ptvis}\vspace{-.125\baselineskip}}
\hfill
\subcaptionbox{Masse visible du \emph{dilepton}.}[.475\textwidth]
{\plotHTTcontrol{\lolcalcurrentyear}{\EMBFFchoice}{\lolcalcurrentchannel}{m_vis}\vspace{-.125\baselineskip}}

\subcaptionbox{Impulsion transverse du système di-\tau.}[.475\textwidth]
{\plotHTTcontrol{\lolcalcurrentyear}{\EMBFFchoice}{\lolcalcurrentchannel}{pt_tt_puppi}\vspace{-.125\baselineskip}}
\hfill
\subcaptionbox{Distance $\Delta R$ entre les leptons (\localLA, \localLB).}[.475\textwidth]
{\plotHTTcontrol{\lolcalcurrentyear}{\EMBFFchoice}{\lolcalcurrentchannel}{DiTauDeltaR}\vspace{-.125\baselineskip}}

\caption[Distributions de contrôle, \lolcalcurrentyear\ \localchannel, \emph{dilepton} et énergie transverse manquante.]{Canal \localchannel, \lolcalcurrentyear: \emph{dilepton} et énergie transverse manquante.}
\end{figure}


\begin{figure}[p]
\centering

\subcaptionbox{Masse transverse du lepton 1 (\localLA).}[.475\textwidth]
{\plotHTTcontrol{\lolcalcurrentyear}{\EMBFFchoice}{\lolcalcurrentchannel}{mt_1_puppi}\vspace{-.125\baselineskip}}
\hfill
\subcaptionbox{Masse transverse du lepton 2 (\localLB).}[.475\textwidth]
{\plotHTTcontrol{\lolcalcurrentyear}{\EMBFFchoice}{\lolcalcurrentchannel}{mt_2_puppi}\vspace{-.125\baselineskip}}

\subcaptionbox{Valeur de \Dzeta.}[.475\textwidth]
{\plotHTTcontrol{\lolcalcurrentyear}{\EMBFFchoice}{\lolcalcurrentchannel}{pZetaPuppiMissVis}\vspace{-.125\baselineskip}}
\hfill
\subcaptionbox{Masse transverse totale.}[.475\textwidth]
{\plotHTTcontrol{\lolcalcurrentyear}{\EMBFFchoice}{\lolcalcurrentchannel}{mt_tot_puppi}\vspace{-.125\baselineskip}}


\subcaptionbox{Masse du système di-\tau\ d'après \SVFIT.}[.475\textwidth]
{\plotHTTcontrol{\lolcalcurrentyear}{\EMBFFchoice}{\lolcalcurrentchannel}{m_sv_puppi}\vspace{-.125\baselineskip}}
\hfill
\subcaptionbox{Masse du système di-\tau\ d'après le ML.}[.475\textwidth]
{\plotHTTcontrol{\lolcalcurrentyear}{\EMBFFchoice}{\lolcalcurrentchannel}{ml_mass}\vspace{-.125\baselineskip}}

\caption[Distributions de contrôle, \lolcalcurrentyear\ \localchannel, masses transverses, \Dzeta\ et masses.]{Canal \localchannel, \lolcalcurrentyear: masses transverses, \Dzeta\ et masses.}
\end{figure}

\def\lolcalcurrentchannel{et}
\ifthenelse{\equal{\lolcalcurrentchannel}{tt}}{\def\localchannel{\tauh\tauh}\def\localLA{\ensuremath{\tauh^{(1)}}}\def\localLB{\ensuremath{\tauh^{(2)}}}}{}
\ifthenelse{\equal{\lolcalcurrentchannel}{mt}}{\def\localchannel{\mu\tauh}\def\localLA{\mu}\def\localLB{\ensuremath{\tauh}}}{}
\ifthenelse{\equal{\lolcalcurrentchannel}{et}}{\def\localchannel{\ele\tauh}\def\localLA{\ele}\def\localLB{\ensuremath{\tauh}}}{}
\ifthenelse{\equal{\lolcalcurrentchannel}{em}}{\def\localchannel{\ele\mu}\def\localLA{\ele}\def\localLB{\mu}}{}

\begin{figure}[p]
\centering

\subcaptionbox{Impulsion transverse du jet principal.}[.475\textwidth]
{\plotHTTcontrol{\lolcalcurrentyear}{\EMBFFchoice}{\lolcalcurrentchannel}{jpt_1}\vspace{-.125\baselineskip}}
\hfill
\subcaptionbox{Impulsion transverse du jet secondaire.}[.475\textwidth]
{\plotHTTcontrol{\lolcalcurrentyear}{\EMBFFchoice}{\lolcalcurrentchannel}{jpt_2}\vspace{-.125\baselineskip}}

\subcaptionbox{Pseudo-rapidité du jet principal.}[.475\textwidth]
{\plotHTTcontrol{\lolcalcurrentyear}{\EMBFFchoice}{\lolcalcurrentchannel}{jeta_1}\vspace{-.125\baselineskip}}
\hfill
\subcaptionbox{Pseudo-rapidité du jet secondaire.}[.475\textwidth]
{\plotHTTcontrol{\lolcalcurrentyear}{\EMBFFchoice}{\lolcalcurrentchannel}{jeta_2}\vspace{-.125\baselineskip}}

\subcaptionbox{Angle azimutal du jet principal.}[.475\textwidth]
{\plotHTTcontrol{\lolcalcurrentyear}{\EMBFFchoice}{\lolcalcurrentchannel}{jphi_1}\vspace{-.125\baselineskip}}
\hfill
\subcaptionbox{Angle azimutal du jet secondaire.}[.475\textwidth]
{\plotHTTcontrol{\lolcalcurrentyear}{\EMBFFchoice}{\lolcalcurrentchannel}{jphi_2}\vspace{-.125\baselineskip}}

\caption[Distributions de contrôle, \lolcalcurrentyear\ \localchannel, cinématique des deux jets principaux.]{Canal \localchannel, \lolcalcurrentyear: cinématique des deux jets principaux.}
\end{figure}

\begin{figure}[p]
\centering

\subcaptionbox{Impulsion transverse du \quarkb-jet principal.}[.475\textwidth]
{\plotHTTcontrol{\lolcalcurrentyear}{\EMBFFchoice}{\lolcalcurrentchannel}{bpt_1}\vspace{-.125\baselineskip}}
\hfill
\subcaptionbox{Impulsion transverse du \quarkb-jet secondaire.}[.475\textwidth]
{\plotHTTcontrol{\lolcalcurrentyear}{\EMBFFchoice}{\lolcalcurrentchannel}{bpt_2}\vspace{-.125\baselineskip}}

\subcaptionbox{Impulsion transverse de l'AHA.}[.475\textwidth]
{\plotHTTcontrol{\lolcalcurrentyear}{\EMBFFchoice _jets_r}{\lolcalcurrentchannel}{jpt_r}\vspace{-.125\baselineskip}}
\hfill
\subcaptionbox{Pseudo-rapidité de l'AHA.}[.475\textwidth]
{\plotHTTcontrol{\lolcalcurrentyear}{\EMBFFchoice _jets_r}{\lolcalcurrentchannel}{jeta_r}\vspace{-.125\baselineskip}}

\subcaptionbox{Angle azimutal de l'AHA.}[.475\textwidth]
{\plotHTTcontrol{\lolcalcurrentyear}{\EMBFFchoice _jets_r}{\lolcalcurrentchannel}{jphi_r}\vspace{-.125\baselineskip}}
\hfill
\subcaptionbox{Nombre de jets dans l'AHA.}[.475\textwidth]
{\plotHTTcontrol{\lolcalcurrentyear}{\EMBFFchoice _jets_r}{\lolcalcurrentchannel}{Njet_r}\vspace{-.125\baselineskip}}

\caption[Distributions de contrôle, \lolcalcurrentyear\ \localchannel, \quarkb-jets et activité hadronique additionnelle.]{Canal \localchannel, \lolcalcurrentyear: \quarkb-jets et activité hadronique additionnelle.}
\end{figure}

\begin{figure}[p]
\centering

\subcaptionbox{Nombre de \quarkb-jets.}[.475\textwidth]
{\plotHTTcontrol{\lolcalcurrentyear}{\EMBFFchoice}{\lolcalcurrentchannel}{nbtag}\vspace{-.125\baselineskip}}
\hfill
\subcaptionbox{Nombre de jets.}[.475\textwidth]
{\plotHTTcontrol{\lolcalcurrentyear}{\EMBFFchoice}{\lolcalcurrentchannel}{njets}\vspace{-.125\baselineskip}}

\subcaptionbox{Impulsion transverse du système des deux jets.}[.475\textwidth]
{\plotHTTcontrol{\lolcalcurrentyear}{\EMBFFchoice}{\lolcalcurrentchannel}{dijetpt}\vspace{-.125\baselineskip}}
\hfill
\subcaptionbox{Distance en $\eta$ entre les deux jets.}[.475\textwidth]
{\plotHTTcontrol{\lolcalcurrentyear}{\EMBFFchoice}{\lolcalcurrentchannel}{jdeta}\vspace{-.125\baselineskip}}

\subcaptionbox{Masse invariante du système des deux jets.}[.475\textwidth]
{\plotHTTcontrol{\lolcalcurrentyear}{\EMBFFchoice}{\lolcalcurrentchannel}{mjj}\vspace{-.125\baselineskip}}
\hfill
\subcaptionbox{Nombre de vertex d'empilement.}[.475\textwidth]
{\plotHTTcontrol{\lolcalcurrentyear}{\EMBFFchoice}{\lolcalcurrentchannel}{npv}\vspace{-.125\baselineskip}}

\caption[Distributions de contrôle, \lolcalcurrentyear\ \localchannel, nombre de jets, système des deux jets principaux et empilement.]{Canal \localchannel, \lolcalcurrentyear: nombre de jets, système des deux jets principaux et empilement.}
\end{figure}


\begin{figure}[p]
\centering

\subcaptionbox{Impulsion transverse du lepton 1 (\localLA).}[.475\textwidth]
{\plotHTTcontrol{\lolcalcurrentyear}{\EMBFFchoice}{\lolcalcurrentchannel}{pt_1}\vspace{-.125\baselineskip}}
\hfill
\subcaptionbox{Impulsion transverse du lepton 2 (\localLB).}[.475\textwidth]
{\plotHTTcontrol{\lolcalcurrentyear}{\EMBFFchoice}{\lolcalcurrentchannel}{pt_2}\vspace{-.125\baselineskip}}

\subcaptionbox{Pseudo-rapidité du lepton 1 (\localLA).}[.475\textwidth]
{\plotHTTcontrol{\lolcalcurrentyear}{\EMBFFchoice}{\lolcalcurrentchannel}{eta_1}\vspace{-.125\baselineskip}}
\hfill
\subcaptionbox{Pseudo-rapidité du lepton 2 (\localLB).}[.475\textwidth]
{\plotHTTcontrol{\lolcalcurrentyear}{\EMBFFchoice}{\lolcalcurrentchannel}{eta_2}\vspace{-.125\baselineskip}}

\subcaptionbox{Angle azimutal du lepton 1 (\localLA).}[.475\textwidth]
{\plotHTTcontrol{\lolcalcurrentyear}{\EMBFFchoice}{\lolcalcurrentchannel}{phi_1}\vspace{-.125\baselineskip}}
\hfill
\subcaptionbox{Angle azimutal du lepton 2 (\localLB).}[.475\textwidth]
{\plotHTTcontrol{\lolcalcurrentyear}{\EMBFFchoice}{\lolcalcurrentchannel}{phi_2}\vspace{-.125\baselineskip}}

\caption[Distributions de contrôle, \lolcalcurrentyear\ \localchannel, cinématique des leptons (\localLA, \localLB).]{Canal \localchannel, \lolcalcurrentyear: cinématique des leptons (\localLA, \localLB).}
\end{figure}

\begin{figure}[p]
\centering

\subcaptionbox{Énergie transverse manquante.}[.475\textwidth]
{\plotHTTcontrol{\lolcalcurrentyear}{\EMBFFchoice}{\lolcalcurrentchannel}{puppimet}\vspace{-.125\baselineskip}}
\hfill
\subcaptionbox{Masse transverse du \emph{dilepton}.}[.475\textwidth]
{\plotHTTcontrol{\lolcalcurrentyear}{\EMBFFchoice}{\lolcalcurrentchannel}{mTdileptonMET_puppi}\vspace{-.125\baselineskip}}

\subcaptionbox{Impulsion transverse du \emph{dilepton}.}[.475\textwidth]
{\plotHTTcontrol{\lolcalcurrentyear}{\EMBFFchoice}{\lolcalcurrentchannel}{ptvis}\vspace{-.125\baselineskip}}
\hfill
\subcaptionbox{Masse visible du \emph{dilepton}.}[.475\textwidth]
{\plotHTTcontrol{\lolcalcurrentyear}{\EMBFFchoice}{\lolcalcurrentchannel}{m_vis}\vspace{-.125\baselineskip}}

\subcaptionbox{Impulsion transverse du système di-\tau.}[.475\textwidth]
{\plotHTTcontrol{\lolcalcurrentyear}{\EMBFFchoice}{\lolcalcurrentchannel}{pt_tt_puppi}\vspace{-.125\baselineskip}}
\hfill
\subcaptionbox{Distance $\Delta R$ entre les leptons (\localLA, \localLB).}[.475\textwidth]
{\plotHTTcontrol{\lolcalcurrentyear}{\EMBFFchoice}{\lolcalcurrentchannel}{DiTauDeltaR}\vspace{-.125\baselineskip}}

\caption[Distributions de contrôle, \lolcalcurrentyear\ \localchannel, \emph{dilepton} et énergie transverse manquante.]{Canal \localchannel, \lolcalcurrentyear: \emph{dilepton} et énergie transverse manquante.}
\end{figure}


\begin{figure}[p]
\centering

\subcaptionbox{Masse transverse du lepton 1 (\localLA).}[.475\textwidth]
{\plotHTTcontrol{\lolcalcurrentyear}{\EMBFFchoice}{\lolcalcurrentchannel}{mt_1_puppi}\vspace{-.125\baselineskip}}
\hfill
\subcaptionbox{Masse transverse du lepton 2 (\localLB).}[.475\textwidth]
{\plotHTTcontrol{\lolcalcurrentyear}{\EMBFFchoice}{\lolcalcurrentchannel}{mt_2_puppi}\vspace{-.125\baselineskip}}

\subcaptionbox{Valeur de \Dzeta.}[.475\textwidth]
{\plotHTTcontrol{\lolcalcurrentyear}{\EMBFFchoice}{\lolcalcurrentchannel}{pZetaPuppiMissVis}\vspace{-.125\baselineskip}}
\hfill
\subcaptionbox{Masse transverse totale.}[.475\textwidth]
{\plotHTTcontrol{\lolcalcurrentyear}{\EMBFFchoice}{\lolcalcurrentchannel}{mt_tot_puppi}\vspace{-.125\baselineskip}}


\subcaptionbox{Masse du système di-\tau\ d'après \SVFIT.}[.475\textwidth]
{\plotHTTcontrol{\lolcalcurrentyear}{\EMBFFchoice}{\lolcalcurrentchannel}{m_sv_puppi}\vspace{-.125\baselineskip}}
\hfill
\subcaptionbox{Masse du système di-\tau\ d'après le ML.}[.475\textwidth]
{\plotHTTcontrol{\lolcalcurrentyear}{\EMBFFchoice}{\lolcalcurrentchannel}{ml_mass}\vspace{-.125\baselineskip}}

\caption[Distributions de contrôle, \lolcalcurrentyear\ \localchannel, masses transverses, \Dzeta\ et masses.]{Canal \localchannel, \lolcalcurrentyear: masses transverses, \Dzeta\ et masses.}
\end{figure}

\def\lolcalcurrentchannel{em}
\ifthenelse{\equal{\lolcalcurrentchannel}{tt}}{\def\localchannel{\tauh\tauh}\def\localLA{\ensuremath{\tauh^{(1)}}}\def\localLB{\ensuremath{\tauh^{(2)}}}}{}
\ifthenelse{\equal{\lolcalcurrentchannel}{mt}}{\def\localchannel{\mu\tauh}\def\localLA{\mu}\def\localLB{\ensuremath{\tauh}}}{}
\ifthenelse{\equal{\lolcalcurrentchannel}{et}}{\def\localchannel{\ele\tauh}\def\localLA{\ele}\def\localLB{\ensuremath{\tauh}}}{}
\ifthenelse{\equal{\lolcalcurrentchannel}{em}}{\def\localchannel{\ele\mu}\def\localLA{\ele}\def\localLB{\mu}}{}

\begin{figure}[p]
\centering

\subcaptionbox{Impulsion transverse du jet principal.}[.475\textwidth]
{\plotHTTcontrol{\lolcalcurrentyear}{\EMBFFchoice}{\lolcalcurrentchannel}{jpt_1}\vspace{-.125\baselineskip}}
\hfill
\subcaptionbox{Impulsion transverse du jet secondaire.}[.475\textwidth]
{\plotHTTcontrol{\lolcalcurrentyear}{\EMBFFchoice}{\lolcalcurrentchannel}{jpt_2}\vspace{-.125\baselineskip}}

\subcaptionbox{Pseudo-rapidité du jet principal.}[.475\textwidth]
{\plotHTTcontrol{\lolcalcurrentyear}{\EMBFFchoice}{\lolcalcurrentchannel}{jeta_1}\vspace{-.125\baselineskip}}
\hfill
\subcaptionbox{Pseudo-rapidité du jet secondaire.}[.475\textwidth]
{\plotHTTcontrol{\lolcalcurrentyear}{\EMBFFchoice}{\lolcalcurrentchannel}{jeta_2}\vspace{-.125\baselineskip}}

\subcaptionbox{Angle azimutal du jet principal.}[.475\textwidth]
{\plotHTTcontrol{\lolcalcurrentyear}{\EMBFFchoice}{\lolcalcurrentchannel}{jphi_1}\vspace{-.125\baselineskip}}
\hfill
\subcaptionbox{Angle azimutal du jet secondaire.}[.475\textwidth]
{\plotHTTcontrol{\lolcalcurrentyear}{\EMBFFchoice}{\lolcalcurrentchannel}{jphi_2}\vspace{-.125\baselineskip}}

\caption[Distributions de contrôle, \lolcalcurrentyear\ \localchannel, cinématique des deux jets principaux.]{Canal \localchannel, \lolcalcurrentyear: cinématique des deux jets principaux.}
\end{figure}

\begin{figure}[p]
\centering

\subcaptionbox{Impulsion transverse du \quarkb-jet principal.}[.475\textwidth]
{\plotHTTcontrol{\lolcalcurrentyear}{\EMBFFchoice}{\lolcalcurrentchannel}{bpt_1}\vspace{-.125\baselineskip}}
\hfill
\subcaptionbox{Impulsion transverse du \quarkb-jet secondaire.}[.475\textwidth]
{\plotHTTcontrol{\lolcalcurrentyear}{\EMBFFchoice}{\lolcalcurrentchannel}{bpt_2}\vspace{-.125\baselineskip}}

\subcaptionbox{Impulsion transverse de l'AHA.}[.475\textwidth]
{\plotHTTcontrol{\lolcalcurrentyear}{\EMBFFchoice _jets_r}{\lolcalcurrentchannel}{jpt_r}\vspace{-.125\baselineskip}}
\hfill
\subcaptionbox{Pseudo-rapidité de l'AHA.}[.475\textwidth]
{\plotHTTcontrol{\lolcalcurrentyear}{\EMBFFchoice _jets_r}{\lolcalcurrentchannel}{jeta_r}\vspace{-.125\baselineskip}}

\subcaptionbox{Angle azimutal de l'AHA.}[.475\textwidth]
{\plotHTTcontrol{\lolcalcurrentyear}{\EMBFFchoice _jets_r}{\lolcalcurrentchannel}{jphi_r}\vspace{-.125\baselineskip}}
\hfill
\subcaptionbox{Nombre de jets dans l'AHA.}[.475\textwidth]
{\plotHTTcontrol{\lolcalcurrentyear}{\EMBFFchoice _jets_r}{\lolcalcurrentchannel}{Njet_r}\vspace{-.125\baselineskip}}

\caption[Distributions de contrôle, \lolcalcurrentyear\ \localchannel, \quarkb-jets et activité hadronique additionnelle.]{Canal \localchannel, \lolcalcurrentyear: \quarkb-jets et activité hadronique additionnelle.}
\end{figure}

\begin{figure}[p]
\centering

\subcaptionbox{Nombre de \quarkb-jets.}[.475\textwidth]
{\plotHTTcontrol{\lolcalcurrentyear}{\EMBFFchoice}{\lolcalcurrentchannel}{nbtag}\vspace{-.125\baselineskip}}
\hfill
\subcaptionbox{Nombre de jets.}[.475\textwidth]
{\plotHTTcontrol{\lolcalcurrentyear}{\EMBFFchoice}{\lolcalcurrentchannel}{njets}\vspace{-.125\baselineskip}}

\subcaptionbox{Impulsion transverse du système des deux jets.}[.475\textwidth]
{\plotHTTcontrol{\lolcalcurrentyear}{\EMBFFchoice}{\lolcalcurrentchannel}{dijetpt}\vspace{-.125\baselineskip}}
\hfill
\subcaptionbox{Distance en $\eta$ entre les deux jets.}[.475\textwidth]
{\plotHTTcontrol{\lolcalcurrentyear}{\EMBFFchoice}{\lolcalcurrentchannel}{jdeta}\vspace{-.125\baselineskip}}

\subcaptionbox{Masse invariante du système des deux jets.}[.475\textwidth]
{\plotHTTcontrol{\lolcalcurrentyear}{\EMBFFchoice}{\lolcalcurrentchannel}{mjj}\vspace{-.125\baselineskip}}
\hfill
\subcaptionbox{Nombre de vertex d'empilement.}[.475\textwidth]
{\plotHTTcontrol{\lolcalcurrentyear}{\EMBFFchoice}{\lolcalcurrentchannel}{npv}\vspace{-.125\baselineskip}}

\caption[Distributions de contrôle, \lolcalcurrentyear\ \localchannel, nombre de jets, système des deux jets principaux et empilement.]{Canal \localchannel, \lolcalcurrentyear: nombre de jets, système des deux jets principaux et empilement.}
\end{figure}


\begin{figure}[p]
\centering

\subcaptionbox{Impulsion transverse du lepton 1 (\localLA).}[.475\textwidth]
{\plotHTTcontrol{\lolcalcurrentyear}{\EMBFFchoice}{\lolcalcurrentchannel}{pt_1}\vspace{-.125\baselineskip}}
\hfill
\subcaptionbox{Impulsion transverse du lepton 2 (\localLB).}[.475\textwidth]
{\plotHTTcontrol{\lolcalcurrentyear}{\EMBFFchoice}{\lolcalcurrentchannel}{pt_2}\vspace{-.125\baselineskip}}

\subcaptionbox{Pseudo-rapidité du lepton 1 (\localLA).}[.475\textwidth]
{\plotHTTcontrol{\lolcalcurrentyear}{\EMBFFchoice}{\lolcalcurrentchannel}{eta_1}\vspace{-.125\baselineskip}}
\hfill
\subcaptionbox{Pseudo-rapidité du lepton 2 (\localLB).}[.475\textwidth]
{\plotHTTcontrol{\lolcalcurrentyear}{\EMBFFchoice}{\lolcalcurrentchannel}{eta_2}\vspace{-.125\baselineskip}}

\subcaptionbox{Angle azimutal du lepton 1 (\localLA).}[.475\textwidth]
{\plotHTTcontrol{\lolcalcurrentyear}{\EMBFFchoice}{\lolcalcurrentchannel}{phi_1}\vspace{-.125\baselineskip}}
\hfill
\subcaptionbox{Angle azimutal du lepton 2 (\localLB).}[.475\textwidth]
{\plotHTTcontrol{\lolcalcurrentyear}{\EMBFFchoice}{\lolcalcurrentchannel}{phi_2}\vspace{-.125\baselineskip}}

\caption[Distributions de contrôle, \lolcalcurrentyear\ \localchannel, cinématique des leptons (\localLA, \localLB).]{Canal \localchannel, \lolcalcurrentyear: cinématique des leptons (\localLA, \localLB).}
\end{figure}

\begin{figure}[p]
\centering

\subcaptionbox{Énergie transverse manquante.}[.475\textwidth]
{\plotHTTcontrol{\lolcalcurrentyear}{\EMBFFchoice}{\lolcalcurrentchannel}{puppimet}\vspace{-.125\baselineskip}}
\hfill
\subcaptionbox{Masse transverse du \emph{dilepton}.}[.475\textwidth]
{\plotHTTcontrol{\lolcalcurrentyear}{\EMBFFchoice}{\lolcalcurrentchannel}{mTdileptonMET_puppi}\vspace{-.125\baselineskip}}

\subcaptionbox{Impulsion transverse du \emph{dilepton}.}[.475\textwidth]
{\plotHTTcontrol{\lolcalcurrentyear}{\EMBFFchoice}{\lolcalcurrentchannel}{ptvis}\vspace{-.125\baselineskip}}
\hfill
\subcaptionbox{Masse visible du \emph{dilepton}.}[.475\textwidth]
{\plotHTTcontrol{\lolcalcurrentyear}{\EMBFFchoice}{\lolcalcurrentchannel}{m_vis}\vspace{-.125\baselineskip}}

\subcaptionbox{Impulsion transverse du système di-\tau.}[.475\textwidth]
{\plotHTTcontrol{\lolcalcurrentyear}{\EMBFFchoice}{\lolcalcurrentchannel}{pt_tt_puppi}\vspace{-.125\baselineskip}}
\hfill
\subcaptionbox{Distance $\Delta R$ entre les leptons (\localLA, \localLB).}[.475\textwidth]
{\plotHTTcontrol{\lolcalcurrentyear}{\EMBFFchoice}{\lolcalcurrentchannel}{DiTauDeltaR}\vspace{-.125\baselineskip}}

\caption[Distributions de contrôle, \lolcalcurrentyear\ \localchannel, \emph{dilepton} et énergie transverse manquante.]{Canal \localchannel, \lolcalcurrentyear: \emph{dilepton} et énergie transverse manquante.}
\end{figure}


\begin{figure}[p]
\centering

\subcaptionbox{Masse transverse du lepton 1 (\localLA).}[.475\textwidth]
{\plotHTTcontrol{\lolcalcurrentyear}{\EMBFFchoice}{\lolcalcurrentchannel}{mt_1_puppi}\vspace{-.125\baselineskip}}
\hfill
\subcaptionbox{Masse transverse du lepton 2 (\localLB).}[.475\textwidth]
{\plotHTTcontrol{\lolcalcurrentyear}{\EMBFFchoice}{\lolcalcurrentchannel}{mt_2_puppi}\vspace{-.125\baselineskip}}

\subcaptionbox{Valeur de \Dzeta.}[.475\textwidth]
{\plotHTTcontrol{\lolcalcurrentyear}{\EMBFFchoice}{\lolcalcurrentchannel}{pZetaPuppiMissVis}\vspace{-.125\baselineskip}}
\hfill
\subcaptionbox{Masse transverse totale.}[.475\textwidth]
{\plotHTTcontrol{\lolcalcurrentyear}{\EMBFFchoice}{\lolcalcurrentchannel}{mt_tot_puppi}\vspace{-.125\baselineskip}}


\subcaptionbox{Masse du système di-\tau\ d'après \SVFIT.}[.475\textwidth]
{\plotHTTcontrol{\lolcalcurrentyear}{\EMBFFchoice}{\lolcalcurrentchannel}{m_sv_puppi}\vspace{-.125\baselineskip}}
\hfill
\subcaptionbox{Masse du système di-\tau\ d'après le ML.}[.475\textwidth]
{\plotHTTcontrol{\lolcalcurrentyear}{\EMBFFchoice}{\lolcalcurrentchannel}{ml_mass}\vspace{-.125\baselineskip}}

\caption[Distributions de contrôle, \lolcalcurrentyear\ \localchannel, masses transverses, \Dzeta\ et masses.]{Canal \localchannel, \lolcalcurrentyear: masses transverses, \Dzeta\ et masses.}
\end{figure}
\clearpage

}{
\addtocontents{toc}{\protect\contentsline {chapter}{\numberline {\refApHTTctrlplotsLETTER}Distributions de contrôle -- $\Higgs\to\tau\tau$}{voir version en ligne}{0}}
}