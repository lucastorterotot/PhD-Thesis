\ifthenelse{\equal{\PRINTABLE}{false}}{
\chapter{Distributions de contrôle -- $\Higgs\to\tau\tau$}\label{annexe-control_plots-HTT}

Cette annexe présente des distributions de contrôle
avant ajustement des paramètres de nuisance
sur les événements utilisés dans l'analyse des événements $\Higgs\to\tau\tau$
présentée dans le chapitre~\refChHTT.
La sélection est \og inclusive \fg, les événements sont ceux sélectionnés par la définition de la région de signal,
sans coupure sur $\mT^{\ell}$ (canaux \mu\tauh, \ele\tauh) ni $\Dzeta$ (canal \ele\mu).
\par
Pour chacune des trois années de prise de données (2016, 2017, 2018)
et
chacun des quatre canaux (\tauh\tauh, \mu\tauh, \ele\tauh, \ele\mu),
les distributions de plusieurs variables sont données.
\par
Dans chacun des graphiques,
les données observées (points noirs) sont comparées à la modélisation des bruits de fond (histogrammes remplis en couleur et empilés).
Les bandes grisées correspondent à l'incertitude statistique totale sur le bruit de fond.
Le rapport au bruit de fond est donné dans la partie inférieure des graphiques.

\bigskip
\par
Ces distributions montrent un bon accord entre données observées et estimations des bruits de fond,
à l'exception:
\begin{itemize}
\item des pseudo-rapidités des \tauh, $\eta(\tauh)$;
%\item de la quantité de jets issus de quarks~\quarkb;
\item du nombre de vertex d'empilement \Npu.
\end{itemize}
Les membres du groupe de l'analyse MSSM \HAtoTauTau\ 
ont récemment été mis au courant des écarts observés
et de plus amples investigations sont prévues.
Ci-après, les pistes envisagées sont présentées.
\paragraph{Pseudo-rapidités des \tauh}
L'écart observé pourrait être réduit
en utilisant des \fakefactors\ dépendants de $\eta(\tauh)$,
ce qui n'est pas le cas actuellement.
D'autres variables dépendant directement de la pseudo-rapidité d'un \tauh\ telles que
la distance $\Delta R$ entre les deux éléments du dilepton
montrent également des écarts entre
données observées et estimation des bruits de fond,
potentiellement dus à ceux sur $\eta(\tauh)$.
L'effet attendu sur
la variable discriminante \mTtot\ 
est faible,
cette dernière étant fonction d'impulsions transverses et d'angles azimutaux.
Une étude plus approfondie sur la corrélation de \mTtot\ avec $\eta(\tauh)$
ainsi que l'effet de l'utilisation de \fakefactors\ dépendants de $\eta(\tauh)$ sur les distributions de \mTtot\
permettrait de quantifier cet effet.
\paragraph{Nombre de vertex d'empilement}
%Les distributions de \Npu\
%montrent un mauvais accord entre données observées et estimations des bruits de fond.
L'effet de la mauvaise modélisation de \Npu\ constatée dans les distributions de contrôle
devrait être marginal sur \mTtot,
car cette observable ne dépend pas de \Npu.
Les autres sources d'incertitudes sur \mTtot\ permettent alors de couvrir cet effet.
Les valeurs de \mml, \ie\ des prédictions du réseau de neurones présenté au chapitre~\refChML,
dépendent de \Npu.
Les bons accords
entre observations et estimations des bruits de fond
obtenus sur les distributions de \mml\
montrent que cette observable semble peu affectée par la mauvaise modélisation de \Npu.
Ainsi,
l'effet limité de \Npu\ sur les variables discriminantes utilisées
permet de conserver des résultats finaux pertinents
bien que l'écart sur \Npu\
entre observations et estimations des bruits de fond
puisse être de l'ordre de \SI{40}{\%}.
Plusieurs effets peuvent causer un tel écart sur les distributions de \Npu:
\subparagraph{Modélisation des \HLTpaths\ dans les données encapsulées}
Les données encapsulées,
introduites dans le chapitre~\refChHTT,
sont des hybrides entre données réelles et simulées.
Les \HLTpaths\ activés proviennent de la partie simulée uniquement,
\ie\ d'un événement vide à l'exception des leptons~\tau\ qui remplacent les muons.
Or, l'empilement provient de la partie réelle des données encapsulées
et l'acceptation des \HLTpaths\ en dépend.
Actuellement,
des facteurs correctifs sont appliqués en fonction des propriétés cinématiques des \tau\
mais ils ne dépendent pas de \Npu.
Ainsi, les distributions de \Npu\ peuvent être biaisées dans les données encapsulées.
\subparagraph{Dépendance en \Npu\ de l'identification des \tauh}
L'identification d'un \tauh\ dépend de son isolation,
elle-même sensible à l'empilement.
Différents points de fonctionnement
d'identification des \tauh\ sont utilisés dans le cadre des \fakefactors,
introduits au chapitre~\refChHTT.
Ces derniers ne sont pas déterminés en fonction de \Npu,
ce qui peut introduire un biais.
\subparagraph{Modélisation de \Npu\ dans les données simulées}
Un désaccord sur \Npu\ est bien attendu entre données réelles et simulées
et ces dernières sont pondérées afin de le corriger, comme exposé au chapitre~\refChLHCCMS.
Toutefois, cette pondération ne permet pas d'obtenir un accord parfait.

\def\EMBFFchoice{emb_ff}
\def\lolcalcurrentyear{2016}
\def\lolcalcurrentchannel{tt}
\def\localLA{\ensuremath{\tauh^{(1)}}}
\def\localLB{\ensuremath{\tauh^{(2)}}}
\def\localchannel{\tauh\tauh}

\begin{figure}[p]
\centering

\subcaptionbox{Catégorie \CATxxh.}[.475\textwidth]
{\plotHTTshapes{mt_tot}{mssm_vs_sm_h125}{\lolcalcurrentyear}{\lolcalcurrentchannel}{1}{prefit}}
\hfill
\subcaptionbox{Catégorie \CATemb.}[.475\textwidth]
{\plotHTTshapes{mt_tot}{mssm_vs_sm_h125}{\lolcalcurrentyear}{\lolcalcurrentchannel}{20}{prefit_linear}}

\subcaptionbox{Catégorie \CATfake.}[.475\textwidth]
{\plotHTTshapes{mt_tot}{mssm_vs_sm_h125}{\lolcalcurrentyear}{\lolcalcurrentchannel}{21}{prefit_linear}}
\hfill
\subcaptionbox{Catégorie \CATmisc.}[.475\textwidth]
{\plotHTTshapes{mt_tot}{mssm_vs_sm_h125}{\lolcalcurrentyear}{\lolcalcurrentchannel}{16}{prefit_linear}}

\caption{Distributions de \NNscore\ en \lolcalcurrentyear\ dans le canal \localchannel.}
\end{figure}

\begin{figure}[p]
\centering

\subcaptionbox{Catégorie \CATbtag, \mTtot.}[.475\textwidth]
{\plotHTTshapes{mt_tot}{mssm_classic}{\lolcalcurrentyear}{\lolcalcurrentchannel}{35}{prefit_linear_nosignal}}
\hfill
\subcaptionbox{Catégorie \CATbtag, \mml.}[.475\textwidth]
{\plotHTTshapes{m_ml}{mssm_classic}{\lolcalcurrentyear}{\lolcalcurrentchannel}{35}{prefit_linear_nosignal}}

\subcaptionbox{Catégorie \CATnobtag, \mTtot.}[.475\textwidth]
{\plotHTTshapes{mt_tot}{mssm_classic}{\lolcalcurrentyear}{\lolcalcurrentchannel}{32}{prefit_linear_nosignal}}
\hfill
\subcaptionbox{Catégorie \CATnobtag, \mml.}[.475\textwidth]
{\plotHTTshapes{m_ml}{mssm_classic}{\lolcalcurrentyear}{\lolcalcurrentchannel}{32}{prefit_linear_nosignal}}

\subcaptionbox{Catégorie \CATbsm~\CATnobtag, \mTtot.}[.475\textwidth]
{\plotHTTshapes{mt_tot}{mssm_vs_sm_h125}{\lolcalcurrentyear}{\lolcalcurrentchannel}{32}{prefit_linear_nosignal}}
\hfill
\subcaptionbox{Catégorie \CATbsm~\CATnobtag, \mml.}[.475\textwidth]
{\plotHTTshapes{m_ml}{mssm_vs_sm_h125}{\lolcalcurrentyear}{\lolcalcurrentchannel}{32}{prefit_linear_nosignal}}

\caption{Distributions de \mTtot\ et \mml\ en \lolcalcurrentyear\ dans le canal \localchannel.}
\end{figure}
\def\lolcalcurrentchannel{mt}
\def\localLA{\mu}
\def\localLB{\tauh}
\def\localchannel{\localLA\localLB}

\input{\PhDthesisdir/contents/appendix-discriminating_variables-HTT/fichier_base_year_lt.tex}
\def\lolcalcurrentchannel{et}
\def\localLA{\ele}
\def\localLB{\tauh}
\def\localchannel{\localLA\localLB}

\input{\PhDthesisdir/contents/appendix-discriminating_variables-HTT/fichier_base_year_lt.tex}
\def\lolcalcurrentchannel{em}
\def\localLA{\ele}
\def\localLB{\mu}
\def\localchannel{\localLA\localLB}

\begin{figure}[p]
\centering

\subcaptionbox{Catégorie \CATxxh.}[.475\textwidth]
{\plotHTTshapes{mt_tot}{mssm_vs_sm_h125}{\lolcalcurrentyear}{\lolcalcurrentchannel}{1}{prefit}}
\hfill
\subcaptionbox{Catégorie \CATemb.}[.475\textwidth]
{\plotHTTshapes{mt_tot}{mssm_vs_sm_h125}{\lolcalcurrentyear}{\lolcalcurrentchannel}{20}{prefit_linear}}

\subcaptionbox{Catégorie \CATttbar.}[.475\textwidth]
{\plotHTTshapes{mt_tot}{mssm_vs_sm_h125}{\lolcalcurrentyear}{\lolcalcurrentchannel}{13}{prefit_linear}}
\hfill
\subcaptionbox{Catégorie \CATdib.}[.475\textwidth]
{\plotHTTshapes{mt_tot}{mssm_vs_sm_h125}{\lolcalcurrentyear}{\lolcalcurrentchannel}{19}{prefit_linear}}

\subcaptionbox{Catégorie \CATqcd.}[.475\textwidth]
{\plotHTTshapes{mt_tot}{mssm_vs_sm_h125}{\lolcalcurrentyear}{\lolcalcurrentchannel}{14}{prefit_linear}}
\hfill
\subcaptionbox{Catégorie \CATmisc.}[.475\textwidth]
{\plotHTTshapes{mt_tot}{mssm_vs_sm_h125}{\lolcalcurrentyear}{\lolcalcurrentchannel}{16}{prefit_linear}}

\caption{Distributions de \NNscore\ en \lolcalcurrentyear\ dans le canal \localchannel.}
\end{figure}

\begin{figure}[p]
\centering

\subcaptionbox{Catégorie \CATbtag~\CAThighdz, \mTtot.}[.475\textwidth]
{\plotHTTshapes{mt_tot}{mssm_classic}{\lolcalcurrentyear}{\lolcalcurrentchannel}{35}{prefit_linear_nosignal}}
\hfill
\subcaptionbox{Catégorie \CATbtag~\CAThighdz, \mml.}[.475\textwidth]
{\plotHTTshapes{m_ml}{mssm_classic}{\lolcalcurrentyear}{\lolcalcurrentchannel}{35}{prefit_linear_nosignal}}

\subcaptionbox{Catégorie \CATbtag~\CATmediumdz, \mTtot.}[.475\textwidth]
{\plotHTTshapes{mt_tot}{mssm_classic}{\lolcalcurrentyear}{\lolcalcurrentchannel}{36}{prefit_linear_nosignal}}
\hfill
\subcaptionbox{Catégorie \CATbtag~\CATmediumdz, \mml.}[.475\textwidth]
{\plotHTTshapes{m_ml}{mssm_classic}{\lolcalcurrentyear}{\lolcalcurrentchannel}{36}{prefit_linear_nosignal}}

\subcaptionbox{Catégorie \CATbtag~\CATlowdz, \mTtot.}[.475\textwidth]
{\plotHTTshapes{mt_tot}{mssm_classic}{\lolcalcurrentyear}{\lolcalcurrentchannel}{37}{prefit_linear_nosignal}}
\hfill
\subcaptionbox{Catégorie \CATbtag~\CATlowdz, \mml.}[.475\textwidth]
{\plotHTTshapes{m_ml}{mssm_classic}{\lolcalcurrentyear}{\lolcalcurrentchannel}{37}{prefit_linear_nosignal}}

\caption{Distributions de \mTtot\ et \mml\ en \lolcalcurrentyear\ dans le canal \localchannel, catégories \CATbtag.}
\end{figure}

\begin{figure}[p]
\centering

\subcaptionbox{Catégorie \CATnobtag~\CAThighdz, \mTtot.}[.475\textwidth]
{\plotHTTshapes{mt_tot}{mssm_classic}{\lolcalcurrentyear}{\lolcalcurrentchannel}{32}{prefit_linear_nosignal}}
\hfill
\subcaptionbox{Catégorie \CATnobtag~\CAThighdz, \mml.}[.475\textwidth]
{\plotHTTshapes{m_ml}{mssm_classic}{\lolcalcurrentyear}{\lolcalcurrentchannel}{32}{prefit_linear_nosignal}}

\subcaptionbox{Catégorie \CATnobtag~\CATmediumdz, \mTtot.}[.475\textwidth]
{\plotHTTshapes{mt_tot}{mssm_classic}{\lolcalcurrentyear}{\lolcalcurrentchannel}{33}{prefit_linear_nosignal}}
\hfill
\subcaptionbox{Catégorie \CATnobtag~\CATmediumdz, \mml.}[.475\textwidth]
{\plotHTTshapes{m_ml}{mssm_classic}{\lolcalcurrentyear}{\lolcalcurrentchannel}{33}{prefit_linear_nosignal}}

\subcaptionbox{Catégorie \CATnobtag~\CATlowdz, \mTtot.}[.475\textwidth]
{\plotHTTshapes{mt_tot}{mssm_classic}{\lolcalcurrentyear}{\lolcalcurrentchannel}{34}{prefit_linear_nosignal}}
\hfill
\subcaptionbox{Catégorie \CATnobtag~\CATlowdz, \mml.}[.475\textwidth]
{\plotHTTshapes{m_ml}{mssm_classic}{\lolcalcurrentyear}{\lolcalcurrentchannel}{34}{prefit_linear_nosignal}}

\caption{Distributions de \mTtot\ et \mml\ en \lolcalcurrentyear\ dans le canal \localchannel, catégories \CATnobtag.}
\end{figure}

\begin{figure}[p]
\centering

\subcaptionbox{Catégorie \CATnobtag~\CAThighdz, \mTtot.}[.475\textwidth]
{\plotHTTshapes{mt_tot}{mssm_vs_sm_h125}{\lolcalcurrentyear}{\lolcalcurrentchannel}{32}{prefit_linear_nosignal}}
\hfill
\subcaptionbox{Catégorie \CATnobtag~\CAThighdz, \mml.}[.475\textwidth]
{\plotHTTshapes{m_ml}{mssm_vs_sm_h125}{\lolcalcurrentyear}{\lolcalcurrentchannel}{32}{prefit_linear_nosignal}}

\subcaptionbox{Catégorie \CATnobtag~\CATmediumdz, \mTtot.}[.475\textwidth]
{\plotHTTshapes{mt_tot}{mssm_vs_sm_h125}{\lolcalcurrentyear}{\lolcalcurrentchannel}{33}{prefit_linear_nosignal}}
\hfill
\subcaptionbox{Catégorie \CATnobtag~\CATmediumdz, \mml.}[.475\textwidth]
{\plotHTTshapes{m_ml}{mssm_vs_sm_h125}{\lolcalcurrentyear}{\lolcalcurrentchannel}{33}{prefit_linear_nosignal}}

\subcaptionbox{Catégorie \CATnobtag~\CATlowdz, \mTtot.}[.475\textwidth]
{\plotHTTshapes{mt_tot}{mssm_vs_sm_h125}{\lolcalcurrentyear}{\lolcalcurrentchannel}{34}{prefit_linear_nosignal}}
\hfill
\subcaptionbox{Catégorie \CATnobtag~\CATlowdz, \mml.}[.475\textwidth]
{\plotHTTshapes{m_ml}{mssm_vs_sm_h125}{\lolcalcurrentyear}{\lolcalcurrentchannel}{34}{prefit_linear_nosignal}}

\caption{Distributions de \mTtot\ et \mml\ en \lolcalcurrentyear\ dans le canal \localchannel, catégories \CATnobtag\ avec $\mCutForCategories \geq \SI{250}{\GeV}$.}
\end{figure}
\clearpage

\def\lolcalcurrentyear{2017}
\def\lolcalcurrentchannel{tt}
\def\localLA{\ensuremath{\tauh^{(1)}}}
\def\localLB{\ensuremath{\tauh^{(2)}}}
\def\localchannel{\tauh\tauh}

\begin{figure}[p]
\centering

\subcaptionbox{Catégorie \CATxxh.}[.475\textwidth]
{\plotHTTshapes{mt_tot}{mssm_vs_sm_h125}{\lolcalcurrentyear}{\lolcalcurrentchannel}{1}{prefit}}
\hfill
\subcaptionbox{Catégorie \CATemb.}[.475\textwidth]
{\plotHTTshapes{mt_tot}{mssm_vs_sm_h125}{\lolcalcurrentyear}{\lolcalcurrentchannel}{20}{prefit_linear}}

\subcaptionbox{Catégorie \CATfake.}[.475\textwidth]
{\plotHTTshapes{mt_tot}{mssm_vs_sm_h125}{\lolcalcurrentyear}{\lolcalcurrentchannel}{21}{prefit_linear}}
\hfill
\subcaptionbox{Catégorie \CATmisc.}[.475\textwidth]
{\plotHTTshapes{mt_tot}{mssm_vs_sm_h125}{\lolcalcurrentyear}{\lolcalcurrentchannel}{16}{prefit_linear}}

\caption{Distributions de \NNscore\ en \lolcalcurrentyear\ dans le canal \localchannel.}
\end{figure}

\begin{figure}[p]
\centering

\subcaptionbox{Catégorie \CATbtag, \mTtot.}[.475\textwidth]
{\plotHTTshapes{mt_tot}{mssm_classic}{\lolcalcurrentyear}{\lolcalcurrentchannel}{35}{prefit_linear_nosignal}}
\hfill
\subcaptionbox{Catégorie \CATbtag, \mml.}[.475\textwidth]
{\plotHTTshapes{m_ml}{mssm_classic}{\lolcalcurrentyear}{\lolcalcurrentchannel}{35}{prefit_linear_nosignal}}

\subcaptionbox{Catégorie \CATnobtag, \mTtot.}[.475\textwidth]
{\plotHTTshapes{mt_tot}{mssm_classic}{\lolcalcurrentyear}{\lolcalcurrentchannel}{32}{prefit_linear_nosignal}}
\hfill
\subcaptionbox{Catégorie \CATnobtag, \mml.}[.475\textwidth]
{\plotHTTshapes{m_ml}{mssm_classic}{\lolcalcurrentyear}{\lolcalcurrentchannel}{32}{prefit_linear_nosignal}}

\subcaptionbox{Catégorie \CATbsm~\CATnobtag, \mTtot.}[.475\textwidth]
{\plotHTTshapes{mt_tot}{mssm_vs_sm_h125}{\lolcalcurrentyear}{\lolcalcurrentchannel}{32}{prefit_linear_nosignal}}
\hfill
\subcaptionbox{Catégorie \CATbsm~\CATnobtag, \mml.}[.475\textwidth]
{\plotHTTshapes{m_ml}{mssm_vs_sm_h125}{\lolcalcurrentyear}{\lolcalcurrentchannel}{32}{prefit_linear_nosignal}}

\caption{Distributions de \mTtot\ et \mml\ en \lolcalcurrentyear\ dans le canal \localchannel.}
\end{figure}
\def\lolcalcurrentchannel{mt}
\def\localLA{\mu}
\def\localLB{\tauh}
\def\localchannel{\localLA\localLB}

\input{\PhDthesisdir/contents/appendix-discriminating_variables-HTT/fichier_base_year_lt.tex}
\def\lolcalcurrentchannel{et}
\def\localLA{\ele}
\def\localLB{\tauh}
\def\localchannel{\localLA\localLB}

\input{\PhDthesisdir/contents/appendix-discriminating_variables-HTT/fichier_base_year_lt.tex}
\def\lolcalcurrentchannel{em}
\def\localLA{\ele}
\def\localLB{\mu}
\def\localchannel{\localLA\localLB}

\begin{figure}[p]
\centering

\subcaptionbox{Catégorie \CATxxh.}[.475\textwidth]
{\plotHTTshapes{mt_tot}{mssm_vs_sm_h125}{\lolcalcurrentyear}{\lolcalcurrentchannel}{1}{prefit}}
\hfill
\subcaptionbox{Catégorie \CATemb.}[.475\textwidth]
{\plotHTTshapes{mt_tot}{mssm_vs_sm_h125}{\lolcalcurrentyear}{\lolcalcurrentchannel}{20}{prefit_linear}}

\subcaptionbox{Catégorie \CATttbar.}[.475\textwidth]
{\plotHTTshapes{mt_tot}{mssm_vs_sm_h125}{\lolcalcurrentyear}{\lolcalcurrentchannel}{13}{prefit_linear}}
\hfill
\subcaptionbox{Catégorie \CATdib.}[.475\textwidth]
{\plotHTTshapes{mt_tot}{mssm_vs_sm_h125}{\lolcalcurrentyear}{\lolcalcurrentchannel}{19}{prefit_linear}}

\subcaptionbox{Catégorie \CATqcd.}[.475\textwidth]
{\plotHTTshapes{mt_tot}{mssm_vs_sm_h125}{\lolcalcurrentyear}{\lolcalcurrentchannel}{14}{prefit_linear}}
\hfill
\subcaptionbox{Catégorie \CATmisc.}[.475\textwidth]
{\plotHTTshapes{mt_tot}{mssm_vs_sm_h125}{\lolcalcurrentyear}{\lolcalcurrentchannel}{16}{prefit_linear}}

\caption{Distributions de \NNscore\ en \lolcalcurrentyear\ dans le canal \localchannel.}
\end{figure}

\begin{figure}[p]
\centering

\subcaptionbox{Catégorie \CATbtag~\CAThighdz, \mTtot.}[.475\textwidth]
{\plotHTTshapes{mt_tot}{mssm_classic}{\lolcalcurrentyear}{\lolcalcurrentchannel}{35}{prefit_linear_nosignal}}
\hfill
\subcaptionbox{Catégorie \CATbtag~\CAThighdz, \mml.}[.475\textwidth]
{\plotHTTshapes{m_ml}{mssm_classic}{\lolcalcurrentyear}{\lolcalcurrentchannel}{35}{prefit_linear_nosignal}}

\subcaptionbox{Catégorie \CATbtag~\CATmediumdz, \mTtot.}[.475\textwidth]
{\plotHTTshapes{mt_tot}{mssm_classic}{\lolcalcurrentyear}{\lolcalcurrentchannel}{36}{prefit_linear_nosignal}}
\hfill
\subcaptionbox{Catégorie \CATbtag~\CATmediumdz, \mml.}[.475\textwidth]
{\plotHTTshapes{m_ml}{mssm_classic}{\lolcalcurrentyear}{\lolcalcurrentchannel}{36}{prefit_linear_nosignal}}

\subcaptionbox{Catégorie \CATbtag~\CATlowdz, \mTtot.}[.475\textwidth]
{\plotHTTshapes{mt_tot}{mssm_classic}{\lolcalcurrentyear}{\lolcalcurrentchannel}{37}{prefit_linear_nosignal}}
\hfill
\subcaptionbox{Catégorie \CATbtag~\CATlowdz, \mml.}[.475\textwidth]
{\plotHTTshapes{m_ml}{mssm_classic}{\lolcalcurrentyear}{\lolcalcurrentchannel}{37}{prefit_linear_nosignal}}

\caption{Distributions de \mTtot\ et \mml\ en \lolcalcurrentyear\ dans le canal \localchannel, catégories \CATbtag.}
\end{figure}

\begin{figure}[p]
\centering

\subcaptionbox{Catégorie \CATnobtag~\CAThighdz, \mTtot.}[.475\textwidth]
{\plotHTTshapes{mt_tot}{mssm_classic}{\lolcalcurrentyear}{\lolcalcurrentchannel}{32}{prefit_linear_nosignal}}
\hfill
\subcaptionbox{Catégorie \CATnobtag~\CAThighdz, \mml.}[.475\textwidth]
{\plotHTTshapes{m_ml}{mssm_classic}{\lolcalcurrentyear}{\lolcalcurrentchannel}{32}{prefit_linear_nosignal}}

\subcaptionbox{Catégorie \CATnobtag~\CATmediumdz, \mTtot.}[.475\textwidth]
{\plotHTTshapes{mt_tot}{mssm_classic}{\lolcalcurrentyear}{\lolcalcurrentchannel}{33}{prefit_linear_nosignal}}
\hfill
\subcaptionbox{Catégorie \CATnobtag~\CATmediumdz, \mml.}[.475\textwidth]
{\plotHTTshapes{m_ml}{mssm_classic}{\lolcalcurrentyear}{\lolcalcurrentchannel}{33}{prefit_linear_nosignal}}

\subcaptionbox{Catégorie \CATnobtag~\CATlowdz, \mTtot.}[.475\textwidth]
{\plotHTTshapes{mt_tot}{mssm_classic}{\lolcalcurrentyear}{\lolcalcurrentchannel}{34}{prefit_linear_nosignal}}
\hfill
\subcaptionbox{Catégorie \CATnobtag~\CATlowdz, \mml.}[.475\textwidth]
{\plotHTTshapes{m_ml}{mssm_classic}{\lolcalcurrentyear}{\lolcalcurrentchannel}{34}{prefit_linear_nosignal}}

\caption{Distributions de \mTtot\ et \mml\ en \lolcalcurrentyear\ dans le canal \localchannel, catégories \CATnobtag.}
\end{figure}

\begin{figure}[p]
\centering

\subcaptionbox{Catégorie \CATnobtag~\CAThighdz, \mTtot.}[.475\textwidth]
{\plotHTTshapes{mt_tot}{mssm_vs_sm_h125}{\lolcalcurrentyear}{\lolcalcurrentchannel}{32}{prefit_linear_nosignal}}
\hfill
\subcaptionbox{Catégorie \CATnobtag~\CAThighdz, \mml.}[.475\textwidth]
{\plotHTTshapes{m_ml}{mssm_vs_sm_h125}{\lolcalcurrentyear}{\lolcalcurrentchannel}{32}{prefit_linear_nosignal}}

\subcaptionbox{Catégorie \CATnobtag~\CATmediumdz, \mTtot.}[.475\textwidth]
{\plotHTTshapes{mt_tot}{mssm_vs_sm_h125}{\lolcalcurrentyear}{\lolcalcurrentchannel}{33}{prefit_linear_nosignal}}
\hfill
\subcaptionbox{Catégorie \CATnobtag~\CATmediumdz, \mml.}[.475\textwidth]
{\plotHTTshapes{m_ml}{mssm_vs_sm_h125}{\lolcalcurrentyear}{\lolcalcurrentchannel}{33}{prefit_linear_nosignal}}

\subcaptionbox{Catégorie \CATnobtag~\CATlowdz, \mTtot.}[.475\textwidth]
{\plotHTTshapes{mt_tot}{mssm_vs_sm_h125}{\lolcalcurrentyear}{\lolcalcurrentchannel}{34}{prefit_linear_nosignal}}
\hfill
\subcaptionbox{Catégorie \CATnobtag~\CATlowdz, \mml.}[.475\textwidth]
{\plotHTTshapes{m_ml}{mssm_vs_sm_h125}{\lolcalcurrentyear}{\lolcalcurrentchannel}{34}{prefit_linear_nosignal}}

\caption{Distributions de \mTtot\ et \mml\ en \lolcalcurrentyear\ dans le canal \localchannel, catégories \CATnobtag\ avec $\mCutForCategories \geq \SI{250}{\GeV}$.}
\end{figure}
\clearpage

\def\lolcalcurrentyear{2018}
\def\lolcalcurrentchannel{tt}
\def\localLA{\ensuremath{\tauh^{(1)}}}
\def\localLB{\ensuremath{\tauh^{(2)}}}
\def\localchannel{\tauh\tauh}

\begin{figure}[p]
\centering

\subcaptionbox{Catégorie \CATxxh.}[.475\textwidth]
{\plotHTTshapes{mt_tot}{mssm_vs_sm_h125}{\lolcalcurrentyear}{\lolcalcurrentchannel}{1}{prefit}}
\hfill
\subcaptionbox{Catégorie \CATemb.}[.475\textwidth]
{\plotHTTshapes{mt_tot}{mssm_vs_sm_h125}{\lolcalcurrentyear}{\lolcalcurrentchannel}{20}{prefit_linear}}

\subcaptionbox{Catégorie \CATfake.}[.475\textwidth]
{\plotHTTshapes{mt_tot}{mssm_vs_sm_h125}{\lolcalcurrentyear}{\lolcalcurrentchannel}{21}{prefit_linear}}
\hfill
\subcaptionbox{Catégorie \CATmisc.}[.475\textwidth]
{\plotHTTshapes{mt_tot}{mssm_vs_sm_h125}{\lolcalcurrentyear}{\lolcalcurrentchannel}{16}{prefit_linear}}

\caption{Distributions de \NNscore\ en \lolcalcurrentyear\ dans le canal \localchannel.}
\end{figure}

\begin{figure}[p]
\centering

\subcaptionbox{Catégorie \CATbtag, \mTtot.}[.475\textwidth]
{\plotHTTshapes{mt_tot}{mssm_classic}{\lolcalcurrentyear}{\lolcalcurrentchannel}{35}{prefit_linear_nosignal}}
\hfill
\subcaptionbox{Catégorie \CATbtag, \mml.}[.475\textwidth]
{\plotHTTshapes{m_ml}{mssm_classic}{\lolcalcurrentyear}{\lolcalcurrentchannel}{35}{prefit_linear_nosignal}}

\subcaptionbox{Catégorie \CATnobtag, \mTtot.}[.475\textwidth]
{\plotHTTshapes{mt_tot}{mssm_classic}{\lolcalcurrentyear}{\lolcalcurrentchannel}{32}{prefit_linear_nosignal}}
\hfill
\subcaptionbox{Catégorie \CATnobtag, \mml.}[.475\textwidth]
{\plotHTTshapes{m_ml}{mssm_classic}{\lolcalcurrentyear}{\lolcalcurrentchannel}{32}{prefit_linear_nosignal}}

\subcaptionbox{Catégorie \CATbsm~\CATnobtag, \mTtot.}[.475\textwidth]
{\plotHTTshapes{mt_tot}{mssm_vs_sm_h125}{\lolcalcurrentyear}{\lolcalcurrentchannel}{32}{prefit_linear_nosignal}}
\hfill
\subcaptionbox{Catégorie \CATbsm~\CATnobtag, \mml.}[.475\textwidth]
{\plotHTTshapes{m_ml}{mssm_vs_sm_h125}{\lolcalcurrentyear}{\lolcalcurrentchannel}{32}{prefit_linear_nosignal}}

\caption{Distributions de \mTtot\ et \mml\ en \lolcalcurrentyear\ dans le canal \localchannel.}
\end{figure}
\def\lolcalcurrentchannel{mt}
\def\localLA{\mu}
\def\localLB{\tauh}
\def\localchannel{\localLA\localLB}

\input{\PhDthesisdir/contents/appendix-discriminating_variables-HTT/fichier_base_year_lt.tex}
\def\lolcalcurrentchannel{et}
\def\localLA{\ele}
\def\localLB{\tauh}
\def\localchannel{\localLA\localLB}

\input{\PhDthesisdir/contents/appendix-discriminating_variables-HTT/fichier_base_year_lt.tex}
\def\lolcalcurrentchannel{em}
\def\localLA{\ele}
\def\localLB{\mu}
\def\localchannel{\localLA\localLB}

\begin{figure}[p]
\centering

\subcaptionbox{Catégorie \CATxxh.}[.475\textwidth]
{\plotHTTshapes{mt_tot}{mssm_vs_sm_h125}{\lolcalcurrentyear}{\lolcalcurrentchannel}{1}{prefit}}
\hfill
\subcaptionbox{Catégorie \CATemb.}[.475\textwidth]
{\plotHTTshapes{mt_tot}{mssm_vs_sm_h125}{\lolcalcurrentyear}{\lolcalcurrentchannel}{20}{prefit_linear}}

\subcaptionbox{Catégorie \CATttbar.}[.475\textwidth]
{\plotHTTshapes{mt_tot}{mssm_vs_sm_h125}{\lolcalcurrentyear}{\lolcalcurrentchannel}{13}{prefit_linear}}
\hfill
\subcaptionbox{Catégorie \CATdib.}[.475\textwidth]
{\plotHTTshapes{mt_tot}{mssm_vs_sm_h125}{\lolcalcurrentyear}{\lolcalcurrentchannel}{19}{prefit_linear}}

\subcaptionbox{Catégorie \CATqcd.}[.475\textwidth]
{\plotHTTshapes{mt_tot}{mssm_vs_sm_h125}{\lolcalcurrentyear}{\lolcalcurrentchannel}{14}{prefit_linear}}
\hfill
\subcaptionbox{Catégorie \CATmisc.}[.475\textwidth]
{\plotHTTshapes{mt_tot}{mssm_vs_sm_h125}{\lolcalcurrentyear}{\lolcalcurrentchannel}{16}{prefit_linear}}

\caption{Distributions de \NNscore\ en \lolcalcurrentyear\ dans le canal \localchannel.}
\end{figure}

\begin{figure}[p]
\centering

\subcaptionbox{Catégorie \CATbtag~\CAThighdz, \mTtot.}[.475\textwidth]
{\plotHTTshapes{mt_tot}{mssm_classic}{\lolcalcurrentyear}{\lolcalcurrentchannel}{35}{prefit_linear_nosignal}}
\hfill
\subcaptionbox{Catégorie \CATbtag~\CAThighdz, \mml.}[.475\textwidth]
{\plotHTTshapes{m_ml}{mssm_classic}{\lolcalcurrentyear}{\lolcalcurrentchannel}{35}{prefit_linear_nosignal}}

\subcaptionbox{Catégorie \CATbtag~\CATmediumdz, \mTtot.}[.475\textwidth]
{\plotHTTshapes{mt_tot}{mssm_classic}{\lolcalcurrentyear}{\lolcalcurrentchannel}{36}{prefit_linear_nosignal}}
\hfill
\subcaptionbox{Catégorie \CATbtag~\CATmediumdz, \mml.}[.475\textwidth]
{\plotHTTshapes{m_ml}{mssm_classic}{\lolcalcurrentyear}{\lolcalcurrentchannel}{36}{prefit_linear_nosignal}}

\subcaptionbox{Catégorie \CATbtag~\CATlowdz, \mTtot.}[.475\textwidth]
{\plotHTTshapes{mt_tot}{mssm_classic}{\lolcalcurrentyear}{\lolcalcurrentchannel}{37}{prefit_linear_nosignal}}
\hfill
\subcaptionbox{Catégorie \CATbtag~\CATlowdz, \mml.}[.475\textwidth]
{\plotHTTshapes{m_ml}{mssm_classic}{\lolcalcurrentyear}{\lolcalcurrentchannel}{37}{prefit_linear_nosignal}}

\caption{Distributions de \mTtot\ et \mml\ en \lolcalcurrentyear\ dans le canal \localchannel, catégories \CATbtag.}
\end{figure}

\begin{figure}[p]
\centering

\subcaptionbox{Catégorie \CATnobtag~\CAThighdz, \mTtot.}[.475\textwidth]
{\plotHTTshapes{mt_tot}{mssm_classic}{\lolcalcurrentyear}{\lolcalcurrentchannel}{32}{prefit_linear_nosignal}}
\hfill
\subcaptionbox{Catégorie \CATnobtag~\CAThighdz, \mml.}[.475\textwidth]
{\plotHTTshapes{m_ml}{mssm_classic}{\lolcalcurrentyear}{\lolcalcurrentchannel}{32}{prefit_linear_nosignal}}

\subcaptionbox{Catégorie \CATnobtag~\CATmediumdz, \mTtot.}[.475\textwidth]
{\plotHTTshapes{mt_tot}{mssm_classic}{\lolcalcurrentyear}{\lolcalcurrentchannel}{33}{prefit_linear_nosignal}}
\hfill
\subcaptionbox{Catégorie \CATnobtag~\CATmediumdz, \mml.}[.475\textwidth]
{\plotHTTshapes{m_ml}{mssm_classic}{\lolcalcurrentyear}{\lolcalcurrentchannel}{33}{prefit_linear_nosignal}}

\subcaptionbox{Catégorie \CATnobtag~\CATlowdz, \mTtot.}[.475\textwidth]
{\plotHTTshapes{mt_tot}{mssm_classic}{\lolcalcurrentyear}{\lolcalcurrentchannel}{34}{prefit_linear_nosignal}}
\hfill
\subcaptionbox{Catégorie \CATnobtag~\CATlowdz, \mml.}[.475\textwidth]
{\plotHTTshapes{m_ml}{mssm_classic}{\lolcalcurrentyear}{\lolcalcurrentchannel}{34}{prefit_linear_nosignal}}

\caption{Distributions de \mTtot\ et \mml\ en \lolcalcurrentyear\ dans le canal \localchannel, catégories \CATnobtag.}
\end{figure}

\begin{figure}[p]
\centering

\subcaptionbox{Catégorie \CATnobtag~\CAThighdz, \mTtot.}[.475\textwidth]
{\plotHTTshapes{mt_tot}{mssm_vs_sm_h125}{\lolcalcurrentyear}{\lolcalcurrentchannel}{32}{prefit_linear_nosignal}}
\hfill
\subcaptionbox{Catégorie \CATnobtag~\CAThighdz, \mml.}[.475\textwidth]
{\plotHTTshapes{m_ml}{mssm_vs_sm_h125}{\lolcalcurrentyear}{\lolcalcurrentchannel}{32}{prefit_linear_nosignal}}

\subcaptionbox{Catégorie \CATnobtag~\CATmediumdz, \mTtot.}[.475\textwidth]
{\plotHTTshapes{mt_tot}{mssm_vs_sm_h125}{\lolcalcurrentyear}{\lolcalcurrentchannel}{33}{prefit_linear_nosignal}}
\hfill
\subcaptionbox{Catégorie \CATnobtag~\CATmediumdz, \mml.}[.475\textwidth]
{\plotHTTshapes{m_ml}{mssm_vs_sm_h125}{\lolcalcurrentyear}{\lolcalcurrentchannel}{33}{prefit_linear_nosignal}}

\subcaptionbox{Catégorie \CATnobtag~\CATlowdz, \mTtot.}[.475\textwidth]
{\plotHTTshapes{mt_tot}{mssm_vs_sm_h125}{\lolcalcurrentyear}{\lolcalcurrentchannel}{34}{prefit_linear_nosignal}}
\hfill
\subcaptionbox{Catégorie \CATnobtag~\CATlowdz, \mml.}[.475\textwidth]
{\plotHTTshapes{m_ml}{mssm_vs_sm_h125}{\lolcalcurrentyear}{\lolcalcurrentchannel}{34}{prefit_linear_nosignal}}

\caption{Distributions de \mTtot\ et \mml\ en \lolcalcurrentyear\ dans le canal \localchannel, catégories \CATnobtag\ avec $\mCutForCategories \geq \SI{250}{\GeV}$.}
\end{figure}
\clearpage

}{
\addtocontents{toc}{\protect\contentsline {chapter}{\numberline {\refApHTTctrlplotsLETTER}Distributions de contrôle -- $\Higgs\to\tau\tau$}{voir version en ligne}{0}}
}