\ifthenelse{\equal{\PRINTABLE}{false}}{
\chapter{Distributions de contrôle -- $\Higgs\to\tau\tau$}\label{annexe-control_plots-HTT}

Cette annexe présente des distributions de contrôle
avant ajustement des paramètres de nuisance
sur les événements utilisés dans l'analyse des événements $\Higgs\to\tau\tau$
présentée dans le chapitre~\refChHTT.
La sélection est \og inclusive \fg, les événements sont ceux sélectionnés par la définition de la région de signal,
sans coupure sur $\mT^{\ell}$ (canaux \mu\tauh, \ele\tauh) ni $\Dzeta$ (canal \ele\mu).
\par
Pour chacune des trois années de prise de données (2016, 2017, 2018)
et
chacun des quatre canaux (\tauh\tauh, \mu\tauh, \ele\tauh, \ele\mu),
les distributions de plusieurs variables sont données.
\par
Dans chacun des graphiques,
les données observées (points noirs) sont comparées à la modélisation des bruits de fond (histogrammes remplis en couleur et empilés).
Les bandes grisées correspondent à l'incertitude statistique totale sur le bruit de fond.
Le rapport au bruit de fond est donné dans la partie inférieure des graphiques.

\encadre[ltcolorred]{Certaines de ces distributions de contrôle, en particulier:
\begin{itemize}
\item le nombre de vertex d'empilement;
\item la pseudo-rapidité du \tauh\ de plus haut \pT;
\item les jets issus de quarks~\quarkb;
\end{itemize}
présentent un mauvais accord entre données observées et estimations.
De plus amples investigations seront menées en vue de la soutenance.}

\def\EMBFFchoice{emb_ff}
\def\lolcalcurrentyear{2016}
\def\lolcalcurrentchannel{tt}
\def\localLA{\ensuremath{\tauh^{(1)}}}
\def\localLB{\ensuremath{\tauh^{(2)}}}
\def\localchannel{\tauh\tauh}

\begin{figure}[p]
\centering

\subcaptionbox{Catégorie \CATxxh.}[.475\textwidth]
{\plotHTTshapes{mt_tot}{mssm_vs_sm_h125}{\lolcalcurrentyear}{\lolcalcurrentchannel}{1}{prefit}}
\hfill
\subcaptionbox{Catégorie \CATemb.}[.475\textwidth]
{\plotHTTshapes{mt_tot}{mssm_vs_sm_h125}{\lolcalcurrentyear}{\lolcalcurrentchannel}{20}{prefit_linear}}

\subcaptionbox{Catégorie \CATfake.}[.475\textwidth]
{\plotHTTshapes{mt_tot}{mssm_vs_sm_h125}{\lolcalcurrentyear}{\lolcalcurrentchannel}{21}{prefit_linear}}
\hfill
\subcaptionbox{Catégorie \CATmisc.}[.475\textwidth]
{\plotHTTshapes{mt_tot}{mssm_vs_sm_h125}{\lolcalcurrentyear}{\lolcalcurrentchannel}{16}{prefit_linear}}

\caption{Distributions de \NNscore\ en \lolcalcurrentyear\ dans le canal \localchannel.}
\end{figure}

\begin{figure}[p]
\centering

\subcaptionbox{Catégorie \CATbtag, \mTtot.}[.475\textwidth]
{\plotHTTshapes{mt_tot}{mssm_classic}{\lolcalcurrentyear}{\lolcalcurrentchannel}{35}{prefit_linear_nosignal}}
\hfill
\subcaptionbox{Catégorie \CATbtag, \mml.}[.475\textwidth]
{\plotHTTshapes{m_ml}{mssm_classic}{\lolcalcurrentyear}{\lolcalcurrentchannel}{35}{prefit_linear_nosignal}}

\subcaptionbox{Catégorie \CATnobtag, \mTtot.}[.475\textwidth]
{\plotHTTshapes{mt_tot}{mssm_classic}{\lolcalcurrentyear}{\lolcalcurrentchannel}{32}{prefit_linear_nosignal}}
\hfill
\subcaptionbox{Catégorie \CATnobtag, \mml.}[.475\textwidth]
{\plotHTTshapes{m_ml}{mssm_classic}{\lolcalcurrentyear}{\lolcalcurrentchannel}{32}{prefit_linear_nosignal}}

\subcaptionbox{Catégorie \CATbsm~\CATnobtag, \mTtot.}[.475\textwidth]
{\plotHTTshapes{mt_tot}{mssm_vs_sm_h125}{\lolcalcurrentyear}{\lolcalcurrentchannel}{32}{prefit_linear_nosignal}}
\hfill
\subcaptionbox{Catégorie \CATbsm~\CATnobtag, \mml.}[.475\textwidth]
{\plotHTTshapes{m_ml}{mssm_vs_sm_h125}{\lolcalcurrentyear}{\lolcalcurrentchannel}{32}{prefit_linear_nosignal}}

\caption{Distributions de \mTtot\ et \mml\ en \lolcalcurrentyear\ dans le canal \localchannel.}
\end{figure}
\def\lolcalcurrentchannel{mt}
\def\localLA{\mu}
\def\localLB{\tauh}
\def\localchannel{\localLA\localLB}

\input{\PhDthesisdir/contents/appendix-discriminating_variables-HTT/fichier_base_year_lt.tex}
\def\lolcalcurrentchannel{et}
\def\localLA{\ele}
\def\localLB{\tauh}
\def\localchannel{\localLA\localLB}

\input{\PhDthesisdir/contents/appendix-discriminating_variables-HTT/fichier_base_year_lt.tex}
\def\lolcalcurrentchannel{em}
\def\localLA{\ele}
\def\localLB{\mu}
\def\localchannel{\localLA\localLB}

\begin{figure}[p]
\centering

\subcaptionbox{Catégorie \CATxxh.}[.475\textwidth]
{\plotHTTshapes{mt_tot}{mssm_vs_sm_h125}{\lolcalcurrentyear}{\lolcalcurrentchannel}{1}{prefit}}
\hfill
\subcaptionbox{Catégorie \CATemb.}[.475\textwidth]
{\plotHTTshapes{mt_tot}{mssm_vs_sm_h125}{\lolcalcurrentyear}{\lolcalcurrentchannel}{20}{prefit_linear}}

\subcaptionbox{Catégorie \CATttbar.}[.475\textwidth]
{\plotHTTshapes{mt_tot}{mssm_vs_sm_h125}{\lolcalcurrentyear}{\lolcalcurrentchannel}{13}{prefit_linear}}
\hfill
\subcaptionbox{Catégorie \CATdib.}[.475\textwidth]
{\plotHTTshapes{mt_tot}{mssm_vs_sm_h125}{\lolcalcurrentyear}{\lolcalcurrentchannel}{19}{prefit_linear}}

\subcaptionbox{Catégorie \CATqcd.}[.475\textwidth]
{\plotHTTshapes{mt_tot}{mssm_vs_sm_h125}{\lolcalcurrentyear}{\lolcalcurrentchannel}{14}{prefit_linear}}
\hfill
\subcaptionbox{Catégorie \CATmisc.}[.475\textwidth]
{\plotHTTshapes{mt_tot}{mssm_vs_sm_h125}{\lolcalcurrentyear}{\lolcalcurrentchannel}{16}{prefit_linear}}

\caption{Distributions de \NNscore\ en \lolcalcurrentyear\ dans le canal \localchannel.}
\end{figure}

\begin{figure}[p]
\centering

\subcaptionbox{Catégorie \CATbtag~\CAThighdz, \mTtot.}[.475\textwidth]
{\plotHTTshapes{mt_tot}{mssm_classic}{\lolcalcurrentyear}{\lolcalcurrentchannel}{35}{prefit_linear_nosignal}}
\hfill
\subcaptionbox{Catégorie \CATbtag~\CAThighdz, \mml.}[.475\textwidth]
{\plotHTTshapes{m_ml}{mssm_classic}{\lolcalcurrentyear}{\lolcalcurrentchannel}{35}{prefit_linear_nosignal}}

\subcaptionbox{Catégorie \CATbtag~\CATmediumdz, \mTtot.}[.475\textwidth]
{\plotHTTshapes{mt_tot}{mssm_classic}{\lolcalcurrentyear}{\lolcalcurrentchannel}{36}{prefit_linear_nosignal}}
\hfill
\subcaptionbox{Catégorie \CATbtag~\CATmediumdz, \mml.}[.475\textwidth]
{\plotHTTshapes{m_ml}{mssm_classic}{\lolcalcurrentyear}{\lolcalcurrentchannel}{36}{prefit_linear_nosignal}}

\subcaptionbox{Catégorie \CATbtag~\CATlowdz, \mTtot.}[.475\textwidth]
{\plotHTTshapes{mt_tot}{mssm_classic}{\lolcalcurrentyear}{\lolcalcurrentchannel}{37}{prefit_linear_nosignal}}
\hfill
\subcaptionbox{Catégorie \CATbtag~\CATlowdz, \mml.}[.475\textwidth]
{\plotHTTshapes{m_ml}{mssm_classic}{\lolcalcurrentyear}{\lolcalcurrentchannel}{37}{prefit_linear_nosignal}}

\caption{Distributions de \mTtot\ et \mml\ en \lolcalcurrentyear\ dans le canal \localchannel, catégories \CATbtag.}
\end{figure}

\begin{figure}[p]
\centering

\subcaptionbox{Catégorie \CATnobtag~\CAThighdz, \mTtot.}[.475\textwidth]
{\plotHTTshapes{mt_tot}{mssm_classic}{\lolcalcurrentyear}{\lolcalcurrentchannel}{32}{prefit_linear_nosignal}}
\hfill
\subcaptionbox{Catégorie \CATnobtag~\CAThighdz, \mml.}[.475\textwidth]
{\plotHTTshapes{m_ml}{mssm_classic}{\lolcalcurrentyear}{\lolcalcurrentchannel}{32}{prefit_linear_nosignal}}

\subcaptionbox{Catégorie \CATnobtag~\CATmediumdz, \mTtot.}[.475\textwidth]
{\plotHTTshapes{mt_tot}{mssm_classic}{\lolcalcurrentyear}{\lolcalcurrentchannel}{33}{prefit_linear_nosignal}}
\hfill
\subcaptionbox{Catégorie \CATnobtag~\CATmediumdz, \mml.}[.475\textwidth]
{\plotHTTshapes{m_ml}{mssm_classic}{\lolcalcurrentyear}{\lolcalcurrentchannel}{33}{prefit_linear_nosignal}}

\subcaptionbox{Catégorie \CATnobtag~\CATlowdz, \mTtot.}[.475\textwidth]
{\plotHTTshapes{mt_tot}{mssm_classic}{\lolcalcurrentyear}{\lolcalcurrentchannel}{34}{prefit_linear_nosignal}}
\hfill
\subcaptionbox{Catégorie \CATnobtag~\CATlowdz, \mml.}[.475\textwidth]
{\plotHTTshapes{m_ml}{mssm_classic}{\lolcalcurrentyear}{\lolcalcurrentchannel}{34}{prefit_linear_nosignal}}

\caption{Distributions de \mTtot\ et \mml\ en \lolcalcurrentyear\ dans le canal \localchannel, catégories \CATnobtag.}
\end{figure}

\begin{figure}[p]
\centering

\subcaptionbox{Catégorie \CATnobtag~\CAThighdz, \mTtot.}[.475\textwidth]
{\plotHTTshapes{mt_tot}{mssm_vs_sm_h125}{\lolcalcurrentyear}{\lolcalcurrentchannel}{32}{prefit_linear_nosignal}}
\hfill
\subcaptionbox{Catégorie \CATnobtag~\CAThighdz, \mml.}[.475\textwidth]
{\plotHTTshapes{m_ml}{mssm_vs_sm_h125}{\lolcalcurrentyear}{\lolcalcurrentchannel}{32}{prefit_linear_nosignal}}

\subcaptionbox{Catégorie \CATnobtag~\CATmediumdz, \mTtot.}[.475\textwidth]
{\plotHTTshapes{mt_tot}{mssm_vs_sm_h125}{\lolcalcurrentyear}{\lolcalcurrentchannel}{33}{prefit_linear_nosignal}}
\hfill
\subcaptionbox{Catégorie \CATnobtag~\CATmediumdz, \mml.}[.475\textwidth]
{\plotHTTshapes{m_ml}{mssm_vs_sm_h125}{\lolcalcurrentyear}{\lolcalcurrentchannel}{33}{prefit_linear_nosignal}}

\subcaptionbox{Catégorie \CATnobtag~\CATlowdz, \mTtot.}[.475\textwidth]
{\plotHTTshapes{mt_tot}{mssm_vs_sm_h125}{\lolcalcurrentyear}{\lolcalcurrentchannel}{34}{prefit_linear_nosignal}}
\hfill
\subcaptionbox{Catégorie \CATnobtag~\CATlowdz, \mml.}[.475\textwidth]
{\plotHTTshapes{m_ml}{mssm_vs_sm_h125}{\lolcalcurrentyear}{\lolcalcurrentchannel}{34}{prefit_linear_nosignal}}

\caption{Distributions de \mTtot\ et \mml\ en \lolcalcurrentyear\ dans le canal \localchannel, catégories \CATnobtag\ avec $\mCutForCategories \geq \SI{250}{\GeV}$.}
\end{figure}
\clearpage

\def\lolcalcurrentyear{2017}
\def\lolcalcurrentchannel{tt}
\def\localLA{\ensuremath{\tauh^{(1)}}}
\def\localLB{\ensuremath{\tauh^{(2)}}}
\def\localchannel{\tauh\tauh}

\begin{figure}[p]
\centering

\subcaptionbox{Catégorie \CATxxh.}[.475\textwidth]
{\plotHTTshapes{mt_tot}{mssm_vs_sm_h125}{\lolcalcurrentyear}{\lolcalcurrentchannel}{1}{prefit}}
\hfill
\subcaptionbox{Catégorie \CATemb.}[.475\textwidth]
{\plotHTTshapes{mt_tot}{mssm_vs_sm_h125}{\lolcalcurrentyear}{\lolcalcurrentchannel}{20}{prefit_linear}}

\subcaptionbox{Catégorie \CATfake.}[.475\textwidth]
{\plotHTTshapes{mt_tot}{mssm_vs_sm_h125}{\lolcalcurrentyear}{\lolcalcurrentchannel}{21}{prefit_linear}}
\hfill
\subcaptionbox{Catégorie \CATmisc.}[.475\textwidth]
{\plotHTTshapes{mt_tot}{mssm_vs_sm_h125}{\lolcalcurrentyear}{\lolcalcurrentchannel}{16}{prefit_linear}}

\caption{Distributions de \NNscore\ en \lolcalcurrentyear\ dans le canal \localchannel.}
\end{figure}

\begin{figure}[p]
\centering

\subcaptionbox{Catégorie \CATbtag, \mTtot.}[.475\textwidth]
{\plotHTTshapes{mt_tot}{mssm_classic}{\lolcalcurrentyear}{\lolcalcurrentchannel}{35}{prefit_linear_nosignal}}
\hfill
\subcaptionbox{Catégorie \CATbtag, \mml.}[.475\textwidth]
{\plotHTTshapes{m_ml}{mssm_classic}{\lolcalcurrentyear}{\lolcalcurrentchannel}{35}{prefit_linear_nosignal}}

\subcaptionbox{Catégorie \CATnobtag, \mTtot.}[.475\textwidth]
{\plotHTTshapes{mt_tot}{mssm_classic}{\lolcalcurrentyear}{\lolcalcurrentchannel}{32}{prefit_linear_nosignal}}
\hfill
\subcaptionbox{Catégorie \CATnobtag, \mml.}[.475\textwidth]
{\plotHTTshapes{m_ml}{mssm_classic}{\lolcalcurrentyear}{\lolcalcurrentchannel}{32}{prefit_linear_nosignal}}

\subcaptionbox{Catégorie \CATbsm~\CATnobtag, \mTtot.}[.475\textwidth]
{\plotHTTshapes{mt_tot}{mssm_vs_sm_h125}{\lolcalcurrentyear}{\lolcalcurrentchannel}{32}{prefit_linear_nosignal}}
\hfill
\subcaptionbox{Catégorie \CATbsm~\CATnobtag, \mml.}[.475\textwidth]
{\plotHTTshapes{m_ml}{mssm_vs_sm_h125}{\lolcalcurrentyear}{\lolcalcurrentchannel}{32}{prefit_linear_nosignal}}

\caption{Distributions de \mTtot\ et \mml\ en \lolcalcurrentyear\ dans le canal \localchannel.}
\end{figure}
\def\lolcalcurrentchannel{mt}
\def\localLA{\mu}
\def\localLB{\tauh}
\def\localchannel{\localLA\localLB}

\input{\PhDthesisdir/contents/appendix-discriminating_variables-HTT/fichier_base_year_lt.tex}
\def\lolcalcurrentchannel{et}
\def\localLA{\ele}
\def\localLB{\tauh}
\def\localchannel{\localLA\localLB}

\input{\PhDthesisdir/contents/appendix-discriminating_variables-HTT/fichier_base_year_lt.tex}
\def\lolcalcurrentchannel{em}
\def\localLA{\ele}
\def\localLB{\mu}
\def\localchannel{\localLA\localLB}

\begin{figure}[p]
\centering

\subcaptionbox{Catégorie \CATxxh.}[.475\textwidth]
{\plotHTTshapes{mt_tot}{mssm_vs_sm_h125}{\lolcalcurrentyear}{\lolcalcurrentchannel}{1}{prefit}}
\hfill
\subcaptionbox{Catégorie \CATemb.}[.475\textwidth]
{\plotHTTshapes{mt_tot}{mssm_vs_sm_h125}{\lolcalcurrentyear}{\lolcalcurrentchannel}{20}{prefit_linear}}

\subcaptionbox{Catégorie \CATttbar.}[.475\textwidth]
{\plotHTTshapes{mt_tot}{mssm_vs_sm_h125}{\lolcalcurrentyear}{\lolcalcurrentchannel}{13}{prefit_linear}}
\hfill
\subcaptionbox{Catégorie \CATdib.}[.475\textwidth]
{\plotHTTshapes{mt_tot}{mssm_vs_sm_h125}{\lolcalcurrentyear}{\lolcalcurrentchannel}{19}{prefit_linear}}

\subcaptionbox{Catégorie \CATqcd.}[.475\textwidth]
{\plotHTTshapes{mt_tot}{mssm_vs_sm_h125}{\lolcalcurrentyear}{\lolcalcurrentchannel}{14}{prefit_linear}}
\hfill
\subcaptionbox{Catégorie \CATmisc.}[.475\textwidth]
{\plotHTTshapes{mt_tot}{mssm_vs_sm_h125}{\lolcalcurrentyear}{\lolcalcurrentchannel}{16}{prefit_linear}}

\caption{Distributions de \NNscore\ en \lolcalcurrentyear\ dans le canal \localchannel.}
\end{figure}

\begin{figure}[p]
\centering

\subcaptionbox{Catégorie \CATbtag~\CAThighdz, \mTtot.}[.475\textwidth]
{\plotHTTshapes{mt_tot}{mssm_classic}{\lolcalcurrentyear}{\lolcalcurrentchannel}{35}{prefit_linear_nosignal}}
\hfill
\subcaptionbox{Catégorie \CATbtag~\CAThighdz, \mml.}[.475\textwidth]
{\plotHTTshapes{m_ml}{mssm_classic}{\lolcalcurrentyear}{\lolcalcurrentchannel}{35}{prefit_linear_nosignal}}

\subcaptionbox{Catégorie \CATbtag~\CATmediumdz, \mTtot.}[.475\textwidth]
{\plotHTTshapes{mt_tot}{mssm_classic}{\lolcalcurrentyear}{\lolcalcurrentchannel}{36}{prefit_linear_nosignal}}
\hfill
\subcaptionbox{Catégorie \CATbtag~\CATmediumdz, \mml.}[.475\textwidth]
{\plotHTTshapes{m_ml}{mssm_classic}{\lolcalcurrentyear}{\lolcalcurrentchannel}{36}{prefit_linear_nosignal}}

\subcaptionbox{Catégorie \CATbtag~\CATlowdz, \mTtot.}[.475\textwidth]
{\plotHTTshapes{mt_tot}{mssm_classic}{\lolcalcurrentyear}{\lolcalcurrentchannel}{37}{prefit_linear_nosignal}}
\hfill
\subcaptionbox{Catégorie \CATbtag~\CATlowdz, \mml.}[.475\textwidth]
{\plotHTTshapes{m_ml}{mssm_classic}{\lolcalcurrentyear}{\lolcalcurrentchannel}{37}{prefit_linear_nosignal}}

\caption{Distributions de \mTtot\ et \mml\ en \lolcalcurrentyear\ dans le canal \localchannel, catégories \CATbtag.}
\end{figure}

\begin{figure}[p]
\centering

\subcaptionbox{Catégorie \CATnobtag~\CAThighdz, \mTtot.}[.475\textwidth]
{\plotHTTshapes{mt_tot}{mssm_classic}{\lolcalcurrentyear}{\lolcalcurrentchannel}{32}{prefit_linear_nosignal}}
\hfill
\subcaptionbox{Catégorie \CATnobtag~\CAThighdz, \mml.}[.475\textwidth]
{\plotHTTshapes{m_ml}{mssm_classic}{\lolcalcurrentyear}{\lolcalcurrentchannel}{32}{prefit_linear_nosignal}}

\subcaptionbox{Catégorie \CATnobtag~\CATmediumdz, \mTtot.}[.475\textwidth]
{\plotHTTshapes{mt_tot}{mssm_classic}{\lolcalcurrentyear}{\lolcalcurrentchannel}{33}{prefit_linear_nosignal}}
\hfill
\subcaptionbox{Catégorie \CATnobtag~\CATmediumdz, \mml.}[.475\textwidth]
{\plotHTTshapes{m_ml}{mssm_classic}{\lolcalcurrentyear}{\lolcalcurrentchannel}{33}{prefit_linear_nosignal}}

\subcaptionbox{Catégorie \CATnobtag~\CATlowdz, \mTtot.}[.475\textwidth]
{\plotHTTshapes{mt_tot}{mssm_classic}{\lolcalcurrentyear}{\lolcalcurrentchannel}{34}{prefit_linear_nosignal}}
\hfill
\subcaptionbox{Catégorie \CATnobtag~\CATlowdz, \mml.}[.475\textwidth]
{\plotHTTshapes{m_ml}{mssm_classic}{\lolcalcurrentyear}{\lolcalcurrentchannel}{34}{prefit_linear_nosignal}}

\caption{Distributions de \mTtot\ et \mml\ en \lolcalcurrentyear\ dans le canal \localchannel, catégories \CATnobtag.}
\end{figure}

\begin{figure}[p]
\centering

\subcaptionbox{Catégorie \CATnobtag~\CAThighdz, \mTtot.}[.475\textwidth]
{\plotHTTshapes{mt_tot}{mssm_vs_sm_h125}{\lolcalcurrentyear}{\lolcalcurrentchannel}{32}{prefit_linear_nosignal}}
\hfill
\subcaptionbox{Catégorie \CATnobtag~\CAThighdz, \mml.}[.475\textwidth]
{\plotHTTshapes{m_ml}{mssm_vs_sm_h125}{\lolcalcurrentyear}{\lolcalcurrentchannel}{32}{prefit_linear_nosignal}}

\subcaptionbox{Catégorie \CATnobtag~\CATmediumdz, \mTtot.}[.475\textwidth]
{\plotHTTshapes{mt_tot}{mssm_vs_sm_h125}{\lolcalcurrentyear}{\lolcalcurrentchannel}{33}{prefit_linear_nosignal}}
\hfill
\subcaptionbox{Catégorie \CATnobtag~\CATmediumdz, \mml.}[.475\textwidth]
{\plotHTTshapes{m_ml}{mssm_vs_sm_h125}{\lolcalcurrentyear}{\lolcalcurrentchannel}{33}{prefit_linear_nosignal}}

\subcaptionbox{Catégorie \CATnobtag~\CATlowdz, \mTtot.}[.475\textwidth]
{\plotHTTshapes{mt_tot}{mssm_vs_sm_h125}{\lolcalcurrentyear}{\lolcalcurrentchannel}{34}{prefit_linear_nosignal}}
\hfill
\subcaptionbox{Catégorie \CATnobtag~\CATlowdz, \mml.}[.475\textwidth]
{\plotHTTshapes{m_ml}{mssm_vs_sm_h125}{\lolcalcurrentyear}{\lolcalcurrentchannel}{34}{prefit_linear_nosignal}}

\caption{Distributions de \mTtot\ et \mml\ en \lolcalcurrentyear\ dans le canal \localchannel, catégories \CATnobtag\ avec $\mCutForCategories \geq \SI{250}{\GeV}$.}
\end{figure}
\clearpage

\def\lolcalcurrentyear{2018}
\def\lolcalcurrentchannel{tt}
\def\localLA{\ensuremath{\tauh^{(1)}}}
\def\localLB{\ensuremath{\tauh^{(2)}}}
\def\localchannel{\tauh\tauh}

\begin{figure}[p]
\centering

\subcaptionbox{Catégorie \CATxxh.}[.475\textwidth]
{\plotHTTshapes{mt_tot}{mssm_vs_sm_h125}{\lolcalcurrentyear}{\lolcalcurrentchannel}{1}{prefit}}
\hfill
\subcaptionbox{Catégorie \CATemb.}[.475\textwidth]
{\plotHTTshapes{mt_tot}{mssm_vs_sm_h125}{\lolcalcurrentyear}{\lolcalcurrentchannel}{20}{prefit_linear}}

\subcaptionbox{Catégorie \CATfake.}[.475\textwidth]
{\plotHTTshapes{mt_tot}{mssm_vs_sm_h125}{\lolcalcurrentyear}{\lolcalcurrentchannel}{21}{prefit_linear}}
\hfill
\subcaptionbox{Catégorie \CATmisc.}[.475\textwidth]
{\plotHTTshapes{mt_tot}{mssm_vs_sm_h125}{\lolcalcurrentyear}{\lolcalcurrentchannel}{16}{prefit_linear}}

\caption{Distributions de \NNscore\ en \lolcalcurrentyear\ dans le canal \localchannel.}
\end{figure}

\begin{figure}[p]
\centering

\subcaptionbox{Catégorie \CATbtag, \mTtot.}[.475\textwidth]
{\plotHTTshapes{mt_tot}{mssm_classic}{\lolcalcurrentyear}{\lolcalcurrentchannel}{35}{prefit_linear_nosignal}}
\hfill
\subcaptionbox{Catégorie \CATbtag, \mml.}[.475\textwidth]
{\plotHTTshapes{m_ml}{mssm_classic}{\lolcalcurrentyear}{\lolcalcurrentchannel}{35}{prefit_linear_nosignal}}

\subcaptionbox{Catégorie \CATnobtag, \mTtot.}[.475\textwidth]
{\plotHTTshapes{mt_tot}{mssm_classic}{\lolcalcurrentyear}{\lolcalcurrentchannel}{32}{prefit_linear_nosignal}}
\hfill
\subcaptionbox{Catégorie \CATnobtag, \mml.}[.475\textwidth]
{\plotHTTshapes{m_ml}{mssm_classic}{\lolcalcurrentyear}{\lolcalcurrentchannel}{32}{prefit_linear_nosignal}}

\subcaptionbox{Catégorie \CATbsm~\CATnobtag, \mTtot.}[.475\textwidth]
{\plotHTTshapes{mt_tot}{mssm_vs_sm_h125}{\lolcalcurrentyear}{\lolcalcurrentchannel}{32}{prefit_linear_nosignal}}
\hfill
\subcaptionbox{Catégorie \CATbsm~\CATnobtag, \mml.}[.475\textwidth]
{\plotHTTshapes{m_ml}{mssm_vs_sm_h125}{\lolcalcurrentyear}{\lolcalcurrentchannel}{32}{prefit_linear_nosignal}}

\caption{Distributions de \mTtot\ et \mml\ en \lolcalcurrentyear\ dans le canal \localchannel.}
\end{figure}
\def\lolcalcurrentchannel{mt}
\def\localLA{\mu}
\def\localLB{\tauh}
\def\localchannel{\localLA\localLB}

\input{\PhDthesisdir/contents/appendix-discriminating_variables-HTT/fichier_base_year_lt.tex}
\def\lolcalcurrentchannel{et}
\def\localLA{\ele}
\def\localLB{\tauh}
\def\localchannel{\localLA\localLB}

\input{\PhDthesisdir/contents/appendix-discriminating_variables-HTT/fichier_base_year_lt.tex}
\def\lolcalcurrentchannel{em}
\def\localLA{\ele}
\def\localLB{\mu}
\def\localchannel{\localLA\localLB}

\begin{figure}[p]
\centering

\subcaptionbox{Catégorie \CATxxh.}[.475\textwidth]
{\plotHTTshapes{mt_tot}{mssm_vs_sm_h125}{\lolcalcurrentyear}{\lolcalcurrentchannel}{1}{prefit}}
\hfill
\subcaptionbox{Catégorie \CATemb.}[.475\textwidth]
{\plotHTTshapes{mt_tot}{mssm_vs_sm_h125}{\lolcalcurrentyear}{\lolcalcurrentchannel}{20}{prefit_linear}}

\subcaptionbox{Catégorie \CATttbar.}[.475\textwidth]
{\plotHTTshapes{mt_tot}{mssm_vs_sm_h125}{\lolcalcurrentyear}{\lolcalcurrentchannel}{13}{prefit_linear}}
\hfill
\subcaptionbox{Catégorie \CATdib.}[.475\textwidth]
{\plotHTTshapes{mt_tot}{mssm_vs_sm_h125}{\lolcalcurrentyear}{\lolcalcurrentchannel}{19}{prefit_linear}}

\subcaptionbox{Catégorie \CATqcd.}[.475\textwidth]
{\plotHTTshapes{mt_tot}{mssm_vs_sm_h125}{\lolcalcurrentyear}{\lolcalcurrentchannel}{14}{prefit_linear}}
\hfill
\subcaptionbox{Catégorie \CATmisc.}[.475\textwidth]
{\plotHTTshapes{mt_tot}{mssm_vs_sm_h125}{\lolcalcurrentyear}{\lolcalcurrentchannel}{16}{prefit_linear}}

\caption{Distributions de \NNscore\ en \lolcalcurrentyear\ dans le canal \localchannel.}
\end{figure}

\begin{figure}[p]
\centering

\subcaptionbox{Catégorie \CATbtag~\CAThighdz, \mTtot.}[.475\textwidth]
{\plotHTTshapes{mt_tot}{mssm_classic}{\lolcalcurrentyear}{\lolcalcurrentchannel}{35}{prefit_linear_nosignal}}
\hfill
\subcaptionbox{Catégorie \CATbtag~\CAThighdz, \mml.}[.475\textwidth]
{\plotHTTshapes{m_ml}{mssm_classic}{\lolcalcurrentyear}{\lolcalcurrentchannel}{35}{prefit_linear_nosignal}}

\subcaptionbox{Catégorie \CATbtag~\CATmediumdz, \mTtot.}[.475\textwidth]
{\plotHTTshapes{mt_tot}{mssm_classic}{\lolcalcurrentyear}{\lolcalcurrentchannel}{36}{prefit_linear_nosignal}}
\hfill
\subcaptionbox{Catégorie \CATbtag~\CATmediumdz, \mml.}[.475\textwidth]
{\plotHTTshapes{m_ml}{mssm_classic}{\lolcalcurrentyear}{\lolcalcurrentchannel}{36}{prefit_linear_nosignal}}

\subcaptionbox{Catégorie \CATbtag~\CATlowdz, \mTtot.}[.475\textwidth]
{\plotHTTshapes{mt_tot}{mssm_classic}{\lolcalcurrentyear}{\lolcalcurrentchannel}{37}{prefit_linear_nosignal}}
\hfill
\subcaptionbox{Catégorie \CATbtag~\CATlowdz, \mml.}[.475\textwidth]
{\plotHTTshapes{m_ml}{mssm_classic}{\lolcalcurrentyear}{\lolcalcurrentchannel}{37}{prefit_linear_nosignal}}

\caption{Distributions de \mTtot\ et \mml\ en \lolcalcurrentyear\ dans le canal \localchannel, catégories \CATbtag.}
\end{figure}

\begin{figure}[p]
\centering

\subcaptionbox{Catégorie \CATnobtag~\CAThighdz, \mTtot.}[.475\textwidth]
{\plotHTTshapes{mt_tot}{mssm_classic}{\lolcalcurrentyear}{\lolcalcurrentchannel}{32}{prefit_linear_nosignal}}
\hfill
\subcaptionbox{Catégorie \CATnobtag~\CAThighdz, \mml.}[.475\textwidth]
{\plotHTTshapes{m_ml}{mssm_classic}{\lolcalcurrentyear}{\lolcalcurrentchannel}{32}{prefit_linear_nosignal}}

\subcaptionbox{Catégorie \CATnobtag~\CATmediumdz, \mTtot.}[.475\textwidth]
{\plotHTTshapes{mt_tot}{mssm_classic}{\lolcalcurrentyear}{\lolcalcurrentchannel}{33}{prefit_linear_nosignal}}
\hfill
\subcaptionbox{Catégorie \CATnobtag~\CATmediumdz, \mml.}[.475\textwidth]
{\plotHTTshapes{m_ml}{mssm_classic}{\lolcalcurrentyear}{\lolcalcurrentchannel}{33}{prefit_linear_nosignal}}

\subcaptionbox{Catégorie \CATnobtag~\CATlowdz, \mTtot.}[.475\textwidth]
{\plotHTTshapes{mt_tot}{mssm_classic}{\lolcalcurrentyear}{\lolcalcurrentchannel}{34}{prefit_linear_nosignal}}
\hfill
\subcaptionbox{Catégorie \CATnobtag~\CATlowdz, \mml.}[.475\textwidth]
{\plotHTTshapes{m_ml}{mssm_classic}{\lolcalcurrentyear}{\lolcalcurrentchannel}{34}{prefit_linear_nosignal}}

\caption{Distributions de \mTtot\ et \mml\ en \lolcalcurrentyear\ dans le canal \localchannel, catégories \CATnobtag.}
\end{figure}

\begin{figure}[p]
\centering

\subcaptionbox{Catégorie \CATnobtag~\CAThighdz, \mTtot.}[.475\textwidth]
{\plotHTTshapes{mt_tot}{mssm_vs_sm_h125}{\lolcalcurrentyear}{\lolcalcurrentchannel}{32}{prefit_linear_nosignal}}
\hfill
\subcaptionbox{Catégorie \CATnobtag~\CAThighdz, \mml.}[.475\textwidth]
{\plotHTTshapes{m_ml}{mssm_vs_sm_h125}{\lolcalcurrentyear}{\lolcalcurrentchannel}{32}{prefit_linear_nosignal}}

\subcaptionbox{Catégorie \CATnobtag~\CATmediumdz, \mTtot.}[.475\textwidth]
{\plotHTTshapes{mt_tot}{mssm_vs_sm_h125}{\lolcalcurrentyear}{\lolcalcurrentchannel}{33}{prefit_linear_nosignal}}
\hfill
\subcaptionbox{Catégorie \CATnobtag~\CATmediumdz, \mml.}[.475\textwidth]
{\plotHTTshapes{m_ml}{mssm_vs_sm_h125}{\lolcalcurrentyear}{\lolcalcurrentchannel}{33}{prefit_linear_nosignal}}

\subcaptionbox{Catégorie \CATnobtag~\CATlowdz, \mTtot.}[.475\textwidth]
{\plotHTTshapes{mt_tot}{mssm_vs_sm_h125}{\lolcalcurrentyear}{\lolcalcurrentchannel}{34}{prefit_linear_nosignal}}
\hfill
\subcaptionbox{Catégorie \CATnobtag~\CATlowdz, \mml.}[.475\textwidth]
{\plotHTTshapes{m_ml}{mssm_vs_sm_h125}{\lolcalcurrentyear}{\lolcalcurrentchannel}{34}{prefit_linear_nosignal}}

\caption{Distributions de \mTtot\ et \mml\ en \lolcalcurrentyear\ dans le canal \localchannel, catégories \CATnobtag\ avec $\mCutForCategories \geq \SI{250}{\GeV}$.}
\end{figure}
\clearpage

}{
\addtocontents{toc}{\protect\contentsline {chapter}{\numberline {\refApHTTctrlplotsLETTER}Distributions de contrôle -- $\Higgs\to\tau\tau$}{voir version en ligne}{0}}
}