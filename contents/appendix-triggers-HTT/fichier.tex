\chapter{Chemins de déclenchement -- $\Higgs\to\tau\tau$}\label{annexe-triggers-HTT}

Pour l'analyse détaillée dans le chapitre~\ifref{chapter-HTT_analysis}{\ref{chapter-HTT_analysis}}{5}, l'enregistrement des données à CMS est activé selon les chemins de déclenchements listés dans les tableaux~\ref{tab-annexe-triggers-HTT-2016_tt} à~\ref{tab-annexe-triggers-HTT-2018_em}, pour les années 2016, 2017 et 2018 et pour les canaux \tauh\tauh, \mu\tauh, \ele\tauh\ et \ele\mu\ selon la répartition donnée dans le tableau~\ref{tab-annexe-triggers-HTT-refs}.
\begin{table}[h]
\centering
\begin{tabular}{ccccc}
\toprule
Année & \tauh\tauh & \mu\tauh & \ele\tauh\ & \ele\mu \\
\midrule
2016 & \ref{tab-annexe-triggers-HTT-2016_tt} & \ref{tab-annexe-triggers-HTT-2016_mt} & \ref{tab-annexe-triggers-HTT-2016_et} & \ref{tab-annexe-triggers-HTT-2016_em} \\
2017 & \ref{tab-annexe-triggers-HTT-2017_tt} & \ref{tab-annexe-triggers-HTT-2017_mt} & \ref{tab-annexe-triggers-HTT-2017_et} & \ref{tab-annexe-triggers-HTT-2017_em} \\
2018 & \ref{tab-annexe-triggers-HTT-2018_tt} & \ref{tab-annexe-triggers-HTT-2018_mt} & \ref{tab-annexe-triggers-HTT-2018_et} & \ref{tab-annexe-triggers-HTT-2018_em} \\
\bottomrule
\end{tabular}
\caption{Tableaux contenant les informations des chemins de déclenchement pour chaque année et canal de l'analyse.}
\label{tab-annexe-triggers-HTT-refs}
\end{table}
\newcommand{\AllSatisfyingFollowing}[2]{Tout #1 respectant les critères listés ci-après est retenu pour jouer le rôle de #2 dans le \emph{dilepton}}

\newcommand{\TauHdz}{la distance $d_z$ entre la trace principale du tau hadronique et le vertex primaire d'interaction vérifie $d_z < \SI{0.2}{\centi\meter}$}
\newcommand{\Leptondzdxy}{paramètres d'impact $d_z < \SI{0.2}{\centi\meter}$ et $d_{xy} < \SI{0.045}{\centi\meter}$}

\newcommand{\RelIsoBelow}[2]{$I_{#1} < \num{#2} \, \pT^{#1}$}

\newcommand{\MuonIDWP}[2]{point de fonctionnement #1 (\emph{#2}) du \muonID}
\newcommand{\MediumMuonID}{\MuonIDWP{moyen}{medium}}
\newcommand{\LooseMuonID}{\MuonIDWP{relâché}{loose}}

\newcommand{\NinetyNineEleMVA}{point de fonctionnement à \SI{90}{\%} d'efficacité de l'\EleIDMVA}
\newcommand{\NinetyNineEleMVAnoIso}{\NinetyNineEleMVA\ sans utilisation des variables d'isolation}

\newcommand{\TwoProngsFootnote}{\footnote{Un \emph{prong} est un hadron chargé issu directement de la désintégration initiale. Par conservation de la charge, il en faut nécessairement un nombre impair dans un tau hadronique. Lorsqu'un des hadrons chargés n'est pas reconstruit, il est possible d'obtenir les DM 5, 6 ou 7. Ces cas de figure sont largement contaminés par le bruit de fond \og QCD multijet \fg, ils sont donc rejetés dans l'analyse.}}

\newcommand{\NewDecayModeFinding}[1][]{passer le discriminateur \texttt{NewDecayModeFinding} avec les modes de désintégration 5, 6, et 7 (deux \emph{prongs}\ifthenelse{\equal{#1}{footnote}}{\TwoProngsFootnote}{}) interdits}
\newcommand{\PassDeepTau}[3]{#1 (\emph{#2}) du discriminateur \texttt{deepTau #3}}

\newcommand{\HLTpath}{chemin de déclenchement}
\newcommand{\HLTpaths}{chemins de déclenchement}

\newcommand{\HLTDoubleTau}{\og double \tauh \fg}
\newcommand{\HLTSingleTau}{\og \tauh\ seul \fg}
\newcommand{\HLTSingleMu}{\og muon seul \fg}
\newcommand{\HLTMuTauCross}{\og muon et \tauh \fg}
\newcommand{\HLTSingleEle}{\og électron seul \fg}
\newcommand{\HLTEleTauCross}{\og électron et \tauh \fg}

\newcommand{\AtLeastOneOSPair}[1]{L'événement est retenu à condition qu'au moins une paire $L_1L_2=#1$ puisse être construite avec $L_1$ et $L_2$ de charges électriques opposées.}
\newcommand{\DeltaRPair}[1]{Il est de plus requis que $L_1$ et $L_2$ soient séparés dans le plan $(\eta,\phi)$ tel que $\Delta R > \num{#1}$.}
\newcommand{\IfMoreOnePair}{Si plus d'une paire possible existe dans l'événement, une seule est retenue selon la logique exposée dans la section~\ref{chapter-HTT_analysis-section-offline-dilepton}.}

\newcommand{\FromPairMatchToHLTObjects}{de la paire sélectionnée doivent correspondre, le cas échéant, aux objets ayant activé un des \HLTpaths\ utilisé pour enregistrer l'événement.
Les objets considérés pour chacun des \HLTpaths\ sont donnés dans l'annexe~\ifref{annexe-triggers-HTT}{\ref{annexe-triggers-HTT}}{E}.
La correspondance est établie lorsque la particule reconstruite et l'objet du \HLTpath\ sont séparés de moins de \num{0.5} dans le plan $(\eta,\phi)$, \ie\ $\Delta R < \num{0.5}$.
Ce critère est appliqué de manière cohérente dans les données réelles, simulées et encapsulées.}
\newcommand{\HLTregionsDefined}{catégories sont définies pour les événements enregistrés en 2016 (respectivement 2017 et 2018)}

\newcommand{\TransverseMassWjetsFootnote}{\footnote{\TransverseMassWjetsFootnoteTXT}}
\newcommand{\TransverseMassWjetsFootnoteTXT}{Cette coupure permet de s'assurer que la région de signal soit orthogonale à la région de détermination des facteurs de faux des événements $\Wboson+\text{jets}$. Les facteurs de faux sont abordés dans la section~\ref{chapter-HTT_analysis-section-bg_estimation-FF_method}.}

\newcommand{\TransverseMassWjets}[4][]{La masse transverse #2, définie par
\begin{equation}
\mT^{(#3)} = \sqrt{2 \, \pT^{(#3)} \, \MET \, (1-\cos\Delta\phi)} \label{eq-mT_def-#4}
\end{equation}
avec $\Delta\phi = \phi^{(#3)} - \phi^{(\MET)}$
doit vérifier $\mT < \SI{70}{\GeV}$\ifthenelse{\equal{#1}{footnote}}{\TransverseMassWjetsFootnote}.}
\renewcommand{\TransverseMassWjets}[4][]{La masse transverse #2, définie par
\begin{equation}
\mT^{(#3)} = \sqrt{2 \, \pT^{(#3)} \, \MET \, (1-\cos\Delta\phi)}
\end{equation}
avec $\Delta\phi = \phi^{(#3)} - \phi^{\MET}$
doit vérifier $\mT < \SI{70}{\GeV}$. \TransverseMassWjetsFootnoteTXT}

\newcommand{\LeptonVetoes}{Les vetos de leptons supplémentaires doivent être respectés, \ie\ que l'événement ne contient pas:}

\newcommand{\LeptonVetoesExtra}[6]{#1 tel que $\pT^{#2} > \SI{#3}{\GeV}$, $\abs{\eta^{#2}} < \num{#4}$, passant le #5 et d'isolation \RelIsoBelow{#2}{#6}}
\newcommand{\LeptonVetoesExtraMuon}{\LeptonVetoesExtra{de muon}{\mu}{10}{2.4}{\MediumMuonID}{0.3}}
\newcommand{\LeptonVetoesSecondMuon}{\LeptonVetoesExtra{de second muon}{\mu}{10}{2.4}{\MediumMuonID}{0.3}}
\newcommand{\LeptonVetoesExtraEle}{\LeptonVetoesExtra{d'électron}{\ele}{10}{2.5}{\NinetyNineEleMVA}{0.3}}
\newcommand{\LeptonVetoesSecondEle}{\LeptonVetoesExtra{de second électron}{\ele}{10}{2.5}{\NinetyNineEleMVA}{0.3}}

\newcommand{\LeptonVetoesPair}[7]{de paire #1 de charges opposées avec $\Delta R > \num{#2}$, tous deux vérifiant $\pT^{#3} > \SI{#4}{\GeV}$, $\abs{\eta^{#3}} < \num{#5}$, passant le #6, de paramètres d'impact $d_z < \SI{0.2}{\centi\meter}$ et $d_{xy} < \SI{0.045}{\centi\meter}$ et d'isolation \RelIsoBelow{#3}{#7}}
\newcommand{\LeptonVetoesMuonPair}{\LeptonVetoesPair{de muons}{0.15}{\mu}{15}{2.4}{\LooseMuonID}{0.3}}
\newcommand{\LeptonVetoesElectronPair}{\LeptonVetoesPair{d'électrons}{0.15}{\ele}{15}{2.5}{\CutBasedEleIDVeto}{0.3}}

\newcommand{\LessTwoMissingHitsVertex}{présenter moins de deux points de passage manquants dans le trajectographe}
\newcommand{\PassConversionVeto}{passer le veto d'électron de conversion}

%%%%%%

\renewcommand{\MuonIDWP}[2]{point de fonctionnement \emph{#2} du \muonID}
\renewcommand{\PassDeepTau}[3]{\emph{#2} du discriminateur \texttt{deepTau #3}}
\par
La fréquence des collisions ainsi que la variété des objets à reconstruire rendent impossible la reconstruction chaque trace du trajectographe en temps réel~\cite{CMS-TRG-12-001}.
Afin d'estimer les objets physique en présence, une \og graine \fg{} (\LoneSeed) est générée à partir de quelques informations directement issues du détecteur.
Elle définit une estimation initiale de la trajectoire d'une particule d'un type donné ainsi que l'incertitude sur celle-ci.
Pour chaque chemin de déclenchement (\HLTpath), une liste de \LoneSeed\ utilisées est indiquée.
\par
De plus, dans l'analyse détaillée dans le chapitre~\ifref{chapter-HTT_analysis}{\ref{chapter-HTT_analysis}}{5}, il est requis que les objets d'intérêts correspondent aux objets ayant déclenché le \HLTpath.
Des filtres permettent de déterminer l'objet ayant déclenché le \HLTpath\ à comparer avec l'objet d'intérêt et sont également renseignés.

\begin{table}[p]
\centering
{\footnotesize
\begin{tabularx}{\textwidth}{llX}
\toprule
\HLTpath & \LoneSeed & \TauFilterToMatch \\
\midrule
HLT\_VLooseIsoPFTau
&
L1\_SingleTau100er
&
hltPFTau120TrackPt50LooseAbsOrRelVLooseIso
\\
120\_Trk50\_eta2p1\_v
\\\hline
HLT\_VLooseIsoPFTau
&
L1\_SingleTau100er
&
hltPFTau140TrackPt50LooseAbsOrRelVLooseIso
\\
140\_Trk50\_eta2p1\_v
\\\hline
HLT\_DoubleMedium
&
L1\_DoubleIsoTau*er,
&
hltDoublePFTau35TrackPt1MediumIsolationDz02Reg
\\
Iso\_PFTau35\_Trk1
&
* in 26, 27, 28, 30, 32, 39
\\
\_eta2p1\_Reg\_v
\\\hline
HLT\_DoubleMedium
&
L1\_DoubleIsoTau*er,
&
hltDoublePFTau35TrackPt1MediumCombinedIsolationDz02Reg
\\
CombinedIsoPFTau
&
* in 26, 27, 28, 30, 32, 39
\\
35Trk1\_eta2p1\_Reg\_v
\\
\bottomrule
\end{tabularx}
}
\caption{Chemins de déclenchement utilisés en 2016 pour le canal \tauh\tauh.}
\label{tab-annexe-triggers-HTT-2016_tt}
\end{table}
\begin{table}[p]
\centering
{\footnotesize
\begin{tabular}{|p{2.5cm} p{2.5cm} p{4.5cm} p{4.5cm}|}
\hline
HLT Path & L1 Seed & Muon filter to match & Tau filter to match \\
\hline
HLT\_IsoMu22\_v
&
L1\_SingleMu20
&
hltL3crIsoL1sMu20L1f0L2f10QL3f22QL3trkIsoFiltered0p09
&
-
\\
HLT\_IsoMu22
\_eta2p1\_v
&
L1\_SingleMu20er
&
hltL3crIsoL1sSingleMu20erL1f0L2f10QL3f22QL3trkIsoFiltered0p09
&
-
\\
HLT\_VLooseIsoPFTau120\_Trk50\_eta2p1\_v
&
L1\_SingleTau100er
&
-
&
hltPFTau120TrackPt50LooseAbsOrRelVLooseIso
\\
HLT\_VLooseIsoPFTau140\_Trk50\_eta2p1\_v
&
L1\_SingleTau100er
&
-
&
hltPFTau140TrackPt50LooseAbsOrRelVLooseIso
\\
HLT\_IsoMu19\_eta2p1\_LooseIsoPFTau20\_v
&
L1\_Mu18er\_Tau20er
&
hltL3crIsoL1sMu18erTauJet20erL1f0L2f10QL3f19QL3trkIsoFiltered0p09hltOverlapFilterIsoMu19LooseIsoPFTau20
&
hltPFTau20TrackLooseIsoAgainstMuonhltOverlapFilterIsoMu19LooseIsoPFTau20
\\
HLT\_IsoMu19\_eta2p1\_LooseIsoPFTau20\_SingleL1\_v
&
L1\_SingleMu18er \textbf{or} L1\_SingleMu20er
&
hltL3crIsoL1sSingleMu18erIorSingleMu20erL1f0L2f10QL3f19QL3trkIsoFiltered0p09hltOverlapFilterSingleIsoMu19LooseIsoPFTau20
&
hltPFTau20TrackLooseIsoAgainstMuonhltOverlapFilterSingleIsoMu19LooseIsoPFTau20
\\
\hline
\end{tabular}
}
\caption{Chemins de déclenchement utilisés en 2016 pour le canal \mu\tauh.}
\label{tab-annexe-triggers-HTT-2016_mt}
\end{table}
\begin{table}[p]
\centering
{\footnotesize
\begin{tabular}{|p{2.5cm} p{2.5cm} p{4.5cm} p{4.5cm}|}
\hline
HLT Path & L1 Seed & Electron filter to match & Tau filter to match \\
\hline
HLT\_Ele25\_eta2p1\_WPTight\_Gsf\_v
&
L1\_SingleEG40 \textbf{or} SingleIsoEG24er \textbf{or} L1\_SingleIsoEG22er
&
hltEle25erWPTightGsfTrackIsoFilter
&
-
\\
HLT\_VLooseIsoPFTau120\_Trk50\_eta2p1\_v
&
L1\_SingleTau100er
&
-
&
hltPFTau120TrackPt50LooseAbsOrRelVLooseIso
\\
HLT\_VLooseIsoPFTau140\_Trk50\_eta2p1\_v
&
L1\_SingleTau100er
&
-
&
hltPFTau140TrackPt50LooseAbsOrRelVLooseIso
\\
\hline
&
&
Run $<$ 276215 and MC
&
\\
HLT\_Ele24\_eta2p1\_WPLoose\_Gsf\_LooseIsoPFTau20\_SingleL1\_v
&
L1\_SingleEG40 \textbf{or} SingleIsoEG24er \textbf{or} L1\_SingleIsoEG22er
&
hltEle24WPLooseL1SingleIsoEG22erGsfTrackIsoFilterhltOverlapFilterSingleIsoEle24WPLooseGsfLooseIsoPFTau20
&
hltPFTau20TrackLooseIsohltOverlapFilterSingleIsoEle24WPLooseGsfLooseIsoPFTau20
\\
\hline
&
&
276215 $\leq$ Run $<$ 278270
&
\\
HLT\_Ele24\_eta2p1\_WPLoose\_Gsf\_LooseIsoPFTau20\_v
&
L1\_IsoEG22er\_Tau20er\_dEta\_Min0p2
&
hltEle24WPLooseL1IsoEG22erTau20erGsfTrackIsoFilterhltOverlapFilterIsoEle24WPLooseGsfLooseIsoPFTau20
&
hltPFTau20TrackLooseIsohltOverlapFilterIsoEle24WPLooseGsfLooseIsoPFTau20
\\
\hline
&
&
278270 $\leq$ Run
&
\\
HLT\_Ele24\_Eta2p1\_WPLoose\_Gsf\_LooseIsoPFTau30\_v
&
L1\_IsoEG22er\_IsoTau26er\_dEta\_Min0p2
&
hltEle24WPLooseL1IsoEG22erIsoTau26erGsfTrackIsoFilterhltOverlapFilterIsoEle24WPLooseGsfLooseIsoPFTau30
&
hltPFTau30TrackLooseIsohltOverlapFilterIsoEle24WPLooseGsfLooseIsoPFTau30
\\
\hline
\end{tabular}
}
\caption{Chemins de déclenchement utilisés en 2016 pour le canal \ele\tauh.}
\label{tab-annexe-triggers-HTT-2016_et}
\end{table}
\begin{table}[p]
\centering
{\footnotesize
\begin{tabularx}{\textwidth}{llXX}
\toprule
HLT Path & L1 Seed & Muon filter to match & Electron filter to match \\
\midrule
\multicolumn{4}{c}{Runs B-F and MC}
\\\hline
HLT\_Mu23\_Trk
&
L1\_Mu12\_EG10
&
hltMu23TrkIsoVVLEle12CaloIdL
&
hltMu23TrkIsoVVLEle12CaloIdL
\\
IsoVVL\_Ele12\_
&
&
TrackIdLIsoVLMuonlegL3
&
TrackIdLIsoVLElectronlegTrack
\\
CaloIdL\_Track
&
&
IsoFiltered23
&
IsoFilter
\\
IdL\_IsoVL\_v
\\\hline
HLT\_Mu8\_Trk
&
L1\_Mu5\_EG15
&
hltMu8TrkIsoVVLEle23CaloIdL
&
hltMu8TrkIsoVVLEle23CaloIdL
\\
IsoVVL\_Ele23\_
&
&
TrackIdLIsoVLMuonlegL3
&
TrackIdLIsoVLElectronlegTrack
\\
CaloIdL\_Track
&
&
IsoFiltered8
&
IsoFilter
\\
IdL\_IsoVL\_v
\\
\midrule
\multicolumn{4}{c}{Runs G-H}
\\\hline
HLT\_Mu23\_Trk
&
L1\_Mu12\_EG10
&
hltMu23TrkIsoVVLEle12CaloIdLTrack
&
hltMu23TrkIsoVVLEle12CaloIdLTrack
\\
IsoVVL\_Ele12\_
&
&
IdLIsoVLMuonlegL3IsoFiltered23 \textbf{et}
&
IdLIsoVLElectronlegTrackIsoFilter \textbf{et}
\\
CaloIdL\_Track
&
&
hltMu23TrkIsoVVLEle12CaloIdL
&
hltMu23TrkIsoVVLEle12CaloIdL
\\
IdL\_IsoVL\_DZ\_v
&
&
TrackIdLIsoVLDZFilter
&
TrackIdLIsoVLDZFilter
\\\hline
HLT\_Mu8\_Trk
&
L1\_Mu5\_EG15
&
hltMu8TrkIsoVVLEle23CaloIdLTrack
&
hltMu8TrkIsoVVLEle23CaloIdLTrack
\\
IsoVVL\_Ele23\_
&
&
IdLIsoVLMuonlegL3IsoFiltered8 \textbf{et}
&
IdLIsoVLElectronlegTrackIsoFilter \textbf{et}
\\
CaloIdL\_Track
&
&
hltMu8TrkIsoVVLEle23CaloIdL
&
hltMu8TrkIsoVVLEle23CaloIdL
\\
IdL\_IsoVL\_DZ\_v
&
&
TrackIdLIsoVLDZFilter
&
TrackIdLIsoVLDZFilter
\\
\bottomrule
\end{tabularx}
}
\caption{Chemins de déclenchement utilisés en 2016 pour le canal \ele\mu.}
\label{tab-annexe-triggers-HTT-2016_em}
\end{table}
\begin{table}[p]
\centering
{\footnotesize
\begin{tabularx}{\textwidth}{llX}
\toprule
\HLTPATH & \LoneSeed & \TauFilterToMatch \\
\midrule
HLT\_MediumCharged
&
L1\_SingleTau80to140er
&
hltPFTau180TrackPt50LooseAbsOrRelMediumHighPtRelaxedIsoIso\!\!\!
\\
IsoPFTau180HighPt
&
&
\textbf{et} hltSelectedPFTau180MediumChargedIsolationL1HLTMatched
\\
\multicolumn{2}{l}{RelaxedIso\_Trk50\_eta2p1\_v}
\\\hline
HLT\_DoubleMedium
&
L1\_DoubleIsoTau*er2p1
&
hltDoublePFTau40TrackPt1MediumChargedIsolationAnd
\\
ChargedIsoPFTau40
&
&
TightOOSCPhotonsDz02Reg
\\
\multicolumn{2}{l}{\_Trk1\_TightID\_eta2p1\_Reg\_v}
\\\hline
HLT\_DoubleTight
&
L1\_DoubleIsoTau*er2p1
&
hltDoublePFTau40TrackPt1TightChargedIsolationDz02Reg
\\
ChargedIsoPFTau40
\\
\_Trk1\_eta2p1\_Reg\_v
\\\hline
HLT\_DoubleTight
&
L1\_DoubleIsoTau*er2p1
&
hltDoublePFTau35TrackPt1TightChargedIsolationAnd
\\
ChargedIsoPFTau35
&
&
TightOOSCPhotonsDz02Reg
\\
\multicolumn{2}{l}{\_Trk1\_TightID\_eta2p1\_Reg\_v}
\\
\bottomrule
\end{tabularx}

\begin{flushleft}
\up{*} in 28, 30, 32, 33, 34, 35, 36, 38, 70
\end{flushleft}
}
\caption{Chemins de déclenchement utilisés en 2017 pour le canal \tauh\tauh.}
\label{tab-annexe-triggers-HTT-2017_tt}
\end{table}
\begin{table}[p]
\centering
{\footnotesize
\begin{tabularx}{\textwidth}{llXX}
\toprule
\HLTpath & \LoneSeed & \MuonFilterToMatch & \TauFilterToMatch \\
\midrule
HLT\_IsoMu24\_v
&
L1\_SingleMu22
&
hltL3crIsoL1sSingleMu22L1f0L2
&
-
\\
&
&
f10QL3f24QL3trkIsoFiltered0p07
\\\hline
HLT\_IsoMu27\_v
&
L1\_SingleMu22
&
hltL3crIsoL1sMu22Or25L1f0L2
&
-
\\
&
\textbf{ou} L1\_SingleMu25
&
f10QL3f27QL3trkIsoFiltered0p07
\\\hline
HLT\_Medium
&
L1\_SingleTau
&
-
&
hltPFTau180TrackPt50LooseAbsOr
\\
ChargedIsoPFTau
&
80to140er
&
&
RelMediumHighPtRelaxedIsoIso \textbf{et} hltSelectedPFTau180Medium
\\
180HighPtRelaxed
&
&
&
ChargedIsolationL1HLTMatched
\\
\multicolumn{2}{l}{Iso\_Trk50\_eta2p1\_v}
\\\hline
HLT\_IsoMu20
&
L1\_Mu18er2p1
&
hltL3crIsoL1sMu18erTau24erIorMu
&
hltSelectedPFTau27LooseCharged
\\
\_eta2p1\_Loose
&
\_Tau24er2p1
&
20erTau24erL1f0L2f10QL3f20QL3trk\!\!
&
IsolationAgainstMuonL1HLT
\\
ChargedIsoPFTau27
&
&
IsoFiltered0p07hltOverlapFilterIso
&
MatchedhltOverlapFilterIsoMu20
\\
\_eta2p1\_CrossL1\_v
&
&
Mu20LooseChargedIsoPFTau27L1
&
LooseChargedIsoPFTau27L1Seeded
\\
&
&
Seeded
\\
\bottomrule
\end{tabularx}
}
\caption{Chemins de déclenchement utilisés en 2017 pour le canal \mu\tauh.}
\label{tab-annexe-triggers-HTT-2017_mt}
\end{table}
\begin{table}[p]
\centering
{\footnotesize
\begin{tabularx}{\textwidth}{llXX}
\toprule
HLT Path & L1 Seed & Electron filter to match & Tau filter to match \\
\midrule
HLT\_Ele27\_
&
L1\_SingleEGXX \textbf{ou}
&
hltEle27WPTightGsf
&
-
\\
WPTight\_Gsf\_v
&
L1\_SingleEGXXer2p1 \textbf{ou}
&
TrackIsoFilter
\\
&
L1\_SingleIsoEGXX \textbf{ou}
\\
&
L1\_SingleIsoEGXXer2p1
\\\hline
HLT\_Ele32\_
&
L1\_SingleEGXX \textbf{ou}
&
hltEle32WPTightGsf
&
-
\\
WPTight\_Gsf\_v
&
L1\_SingleEGXXer2p1 \textbf{ou}
&
TrackIsoFilter
\\
&
L1\_SingleIsoEGXX \textbf{ou}
\\
&
L1\_SingleIsoEGXXer2p1
\\\hline
HLT\_Ele35\_
&
L1\_SingleEGXX \textbf{ou}
&
hltEle35noerWPTightGsf
&
-
\\
WPTight\_Gsf\_v
&
L1\_SingleEGXXer2p1 \textbf{ou}
&
TrackIsoFilter
\\
&
L1\_SingleIsoEGXX \textbf{ou}
\\
&
L1\_SingleIsoEGXXer2p1
\\\hline
HLT\_Medium
&
L1\_SingleTau80to140er
&
-
&
hltPFTau180TrackPt50LooseAbs
\\
ChargedIsoPFTau
&
&
&
OrRelMediumHighPtRelaxed
\\
180HighPtRelaxed
&
&
&
IsoIso \textbf{et} hltSelectedPFTau180
\\
Iso\_Trk50\_eta2p1\_v
&
&
&
MediumChargedIsolationL1
\\
&
&
&
HLTMatched
\\\hline
HLT\_Ele24\_
&
L1\_LooseIsoEG22er2p1
&
hltEle24erWPTightGsfTrackIso
&
hltSelectedPFTau30Loose
\\
eta2p1\_WPTight
&
\_IsoTau26er2p1
&
FilterForTauhltOverlapFilterIso
&
ChargedIsolationL1HLTMatched\!
\\
\_Gsf\_LooseCharged
&
\_dR\_Min0p3
&
Ele24WPTightGsfLooseIso
&
hltOverlapFilterIsoEle24WPTight\!
\\
IsoPFTau30
&
&
PFTau30
&
GsfLooseIsoPFTau30
\\
\_eta2p1\_CrossL1\_v
\\
\bottomrule
\end{tabularx}
}
\caption{Chemins de déclenchement utilisés en 2017 pour le canal \ele\tauh.}
\label{tab-annexe-triggers-HTT-2017_et}
\end{table}
\begin{table}[p]
\centering
{\footnotesize
\begin{tabularx}{\textwidth}{llXX}
\toprule
\HLTPATH & \LoneSeed & \MuonFilterToMatch & \ElectronFilterToMatch \\
\midrule
HLT\_Mu23\_Trk
&
\todo{???}
&
hltMu23TrkIsoVVLEle12CaloIdL
&
hltMu23TrkIsoVVLEle12CaloIdL
\\
IsoVVL\_Ele12\_
&
&
TrackIdLIsoVLMuonlegL3
&
TrackIdLIsoVLElectronleg
\\
CaloIdL\_Track
&
&
IsoFiltered23
&
TrackIsoFilter
\\
IdL\_IsoVL\_DZ\_v
\\\hline
HLT\_Mu8\_Trk
&
\todo{???}
&
hltL3fL1sMu7EG23f0Filtered8 \textbf{ou}
&
hltMu8TrkIsoVVLEle23CaloIdL
\\
IsoVVL\_Ele23\_
&
&
hltMu8TrkIsoVVLEle23CaloIdLTrack
&
TrackIdLIsoVLElectronlegTrack
\\
CaloIdL\_Track
&
&
IdLIsoVLMuonlegL3IsoFiltered8
&
IsoFilter
\\
IdL\_IsoVL\_DZ\_v
\\
\bottomrule
\end{tabularx}
}

\caption{Chemins de déclenchement utilisés en 2017 pour le canal \ele\mu.}
\label{tab-annexe-triggers-HTT-2017_em}
\end{table}
\begin{table}[p]
\centering
{\footnotesize
\begin{tabularx}{\textwidth}{llX}
\toprule
\HLTPATH & \LoneSeed & \TauFilterToMatch \\
\midrule
HLT\_MediumCharged
&
L1\_SingleTau80to140er
&
hltPFTau180TrackPt50LooseAbsOrRelMediumHighPtRelaxedIsoIso\!\!\!
\\
IsoPFTau180HighPt
&
&
\textbf{et} hltSelectedPFTau180MediumChargedIsolationL1HLTMatched
\\
\multicolumn{2}{l}{RelaxedIso\_Trk50\_eta2p1\_v}
\\\hline
HLT\_DoubleMedium
&
L1\_DoubleIsoTau*er2p1
&
hltDoublePFTau40TrackPt1MediumChargedIsolationAnd
\\
ChargedIsoPFTau40
&
&
TightOOSCPhotonsDz02Reg
\\
\multicolumn{2}{l}{\_Trk1\_TightID\_eta2p1\_Reg\_v}
\\\hline
HLT\_DoubleTight
&
L1\_DoubleIsoTau*er2p1
&
hltDoublePFTau40TrackPt1TightChargedIsolationDz02Reg
\\
ChargedIsoPFTau40
\\
\_Trk1\_eta2p1\_Reg\_v
\\\hline
HLT\_DoubleTight
&
L1\_DoubleIsoTau*er2p1
&
hltDoublePFTau35TrackPt1TightChargedIsolationAnd
\\
ChargedIsoPFTau35
&
&
TightOOSCPhotonsDz02Reg
\\
\multicolumn{2}{l}{\_Trk1\_TightID\_eta2p1\_Reg\_v}
\\\hline
HLT\_DoubleMedium
&
L1\_DoubleIsoTau*er2p1
&
hltHpsDoublePFTau35TrackPt1MediumChargedIsolationDz02Reg
\\
\multicolumn{2}{l}{ChargedIsoPFTauHPS35\_Trk1\_eta2p1\_Reg\_v}
\\
\bottomrule
\end{tabularx}

\begin{flushleft}
\up{*} in 28, 30, 32, 33, 34, 35, 36, 38, 70
\end{flushleft}
}

\caption{Chemins de déclenchement utilisés en 2018 pour le canal \tauh\tauh.}
\label{tab-annexe-triggers-HTT-2018_tt}
\end{table}
\begin{table}[p]
\centering
{\footnotesize
\begin{tabularx}{\textwidth}{llXX}
\toprule
\HLTpath & \LoneSeed & \MuonFilterToMatch & \TauFilterToMatch \\
\midrule
HLT\_IsoMu24\_v
&
L1\_SingleMu22
&
hltL3crIsoL1sSingleMu22L1f0L2
&
-
\\
&
&
f10QL3f24QL3trkIsoFiltered0p07
\\\hline
HLT\_IsoMu27\_v
&
L1\_SingleMu22
&
hltL3crIsoL1sMu22Or25L1f0L2
&
-
\\
&
\textbf{ou} L1\_SingleMu25
&
f10QL3f27QL3trkIsoFiltered0p07
\\\hline
HLT\_Medium
&
L1\_SingleTau
&
-
&
hltPFTau180TrackPt50LooseAbsOr
\\
ChargedIsoPFTau
&
80to140er
&
&
RelMediumHighPtRelaxedIsoIso \textbf{et} hltSelectedPFTau180Medium
\\
180HighPtRelaxed
&
&
&
ChargedIsolationL1HLTMatched
\\
\multicolumn{2}{l}{Iso\_Trk50\_eta2p1\_v}
\\\hline
HLT\_IsoMu20
&
L1\_Mu18er2p1
&
hltL3crIsoL1sMu18erTau24erIorMu
&
hltSelectedPFTau27LooseCharged
\\
\_eta2p1\_Loose
&
\_Tau24er2p1
&
20erTau24erL1f0L2f10QL3f20QL3trk\!\!
&
IsolationAgainstMuonL1HLT
\\
ChargedIsoPFTau27
&
&
IsoFiltered0p07hltOverlapFilterIso
&
MatchedhltOverlapFilterIsoMu20
\\
\_eta2p1\_CrossL1\_v
&
&
Mu20LooseChargedIsoPFTau27L1
&
LooseChargedIsoPFTau27L1Seeded
\\
&
&
Seeded
\\
\bottomrule
\end{tabularx}
}
\caption{Chemins de déclenchement utilisés en 2018 pour le canal \mu\tauh.}
\label{tab-annexe-triggers-HTT-2018_mt}
\end{table}
\begin{table}[p]
\centering
{\footnotesize
\begin{tabularx}{\textwidth}{llXX}
\toprule
HLT Path & L1 Seed & Electron filter to match & Tau filter to match \\
\midrule
HLT\_Ele32\_
&
L1\_SingleEGXX \textbf{ou}
&
hltEle32WPTightGsf
&
-
\\
WPTight\_Gsf\_v
&
L1\_SingleEGXXer2p1 \textbf{ou}
&
TrackIsoFilter
\\
&
L1\_SingleIsoEGXX \textbf{ou}
\\
&
L1\_SingleIsoEGXXer2p1
\\\hline
HLT\_Ele35\_
&
L1\_SingleEGXX \textbf{ou}
&
hltEle35noerWPTightGsf
&
-
\\
WPTight\_Gsf\_v
&
L1\_SingleEGXXer2p1 \textbf{ou}
&
TrackIsoFilter
\\
&
L1\_SingleIsoEGXX \textbf{ou}
\\
&
L1\_SingleIsoEGXXer2p1
\\\hline
HLT\_Medium
&
L1\_SingleTau80to140er
&
-
&
hltPFTau180TrackPt50LooseAbs
\\
ChargedIsoPFTau
&
&
&
OrRelMediumHighPtRelaxed
\\
180HighPtRelaxed
&
&
&
IsoIso \textbf{et} hltSelectedPFTau180
\\
Iso\_Trk50\_eta2p1\_v
&
&
&
MediumChargedIsolationL1
\\
&
&
&
HLTMatched
\\\hline
HLT\_Ele24\_
&
L1\_LooseIsoEG22er2p1
&
hltEle24erWPTightGsfTrackIso
&
hltSelectedPFTau30Loose
\\
eta2p1\_WPTight
&
\_IsoTau26er2p1
&
FilterForTauhltOverlapFilterIso
&
ChargedIsolationL1HLTMatched\!
\\
\_Gsf\_LooseCharged
&
\_dR\_Min0p3
&
Ele24WPTightGsfLooseIso
&
hltOverlapFilterIsoEle24WPTight\!
\\
IsoPFTau30
&
&
PFTau30
&
GsfLooseIsoPFTau30
\\
\_eta2p1\_CrossL1\_v
\\\hline
HLT\_Ele24\_
&
L1\_LooseIsoEG22er2p1
&
hltEle24erWPTightGsfTrackIso
&
hltSelectedPFTau30Loose
\\
eta2p1\_WPTight
&
\_IsoTau26er2p1
&
FilterForTauhltOverlapFilterIso
&
ChargedIsolationL1HLTMatched\!
\\
\_Gsf\_LooseCharged
&
\_dR\_Min0p3
&
Ele24WPTightGsfLooseIso
&
hltOverlapFilterIsoEle24WPTight\!
\\
IsoPFTauHPS30
&
&
PFTau30
&
GsfLooseIsoPFTau30
\\
\_eta2p1\_CrossL1\_v
\\
\bottomrule
\end{tabularx}
}
\caption{Chemins de déclenchement utilisés en 2018 pour le canal \ele\tauh.}
\label{tab-annexe-triggers-HTT-2018_et}
\end{table}
\begin{table}[p]
\centering
{\footnotesize
\begin{tabularx}{\textwidth}{llXX}
\toprule
\HLTPATH & \LoneSeed & \MuonFilterToMatch & \ElectronFilterToMatch \\
\midrule
HLT\_Mu23\_Trk
&
\todo{???}
&
hltMu23TrkIsoVVLEle12CaloIdL
&
hltMu23TrkIsoVVLEle12CaloIdL
\\
IsoVVL\_Ele12\_
&
&
TrackIdLIsoVLMuonlegL3
&
TrackIdLIsoVLElectronleg
\\
CaloIdL\_Track
&
&
IsoFiltered23
&
TrackIsoFilter
\\
IdL\_IsoVL\_DZ\_v
\\\hline
HLT\_Mu8\_Trk
&
\todo{???}
&
hltL3fL1sMu7EG23f0Filtered8 \textbf{ou}
&
hltMu8TrkIsoVVLEle23CaloIdL
\\
IsoVVL\_Ele23\_
&
&
hltMu8TrkIsoVVLEle23CaloIdLTrack
&
TrackIdLIsoVLElectronlegTrack
\\
CaloIdL\_Track
&
&
IdLIsoVLMuonlegL3IsoFiltered8
&
IsoFilter
\\
IdL\_IsoVL\_DZ\_v
\\
\bottomrule
\end{tabularx}
}

\caption{Chemins de déclenchement utilisés en 2018 pour le canal \ele\mu.}
\label{tab-annexe-triggers-HTT-2018_em}
\end{table}