\section{Incertitudes systématiques}\label{chapter-HTT_analysis-section-systematics}
La statistique n'est pas la seule source d'incertitudes sur les distributions de l'estimation des bruits de fond et du signal.
Des incertitudes expérimentales, liées à la reconstruction des objets physiques et leur identification par exemple, sont à prendre en compte.
Des incertitudes théoriques existent également.
Chaque incertitude peut affecter les distributions sous la forme d'un facteur de normalisation, d'une modification des formes de celles-ci voire les deux.
Les incertitudes de normalisation sont présentées dans la section~\ref{chapter-HTT_analysis-section-systematics-normalization},
celles pouvant modifier les formes des distributions dans la section~\ref{chapter-HTT_analysis-section-systematics-shapes}.
Il en résulte des paramètres de nuisance, exploités dans la section~\ref{chapter-HTT_analysis-section-signal_extraction}.
\subsection{Incertitudes de normalisation}\label{chapter-HTT_analysis-section-systematics-normalization}

\paragraph{Luminosité}
lumi unc.~\cite{LumiTwiki}

\paragraph{Pondération du \emph{prefiring}}

\paragraph{Taux de mauvaise identification $\mu\to\tauh$ (\emph{$\mu\to\tauh$ fake rate})}

\paragraph{Taux de mauvaise identification $\ele\to\tauh$ (\emph{$\ele\to\tauh$ fake rate})}

\paragraph{Efficacité d'identification et isolation des muons et des électrons (\emph{muon/electron ID/iso efficiency})}

\paragraph{Efficacité du \quarkb-\emph{tagging} (\emph{Btag efficiency})}

\paragraph{Bruits de fond simulés}

\paragraph{Bruits de fond estimés par les \fakefactors}

\paragraph{Incertitudes théoriques}

\subsection{Incertitudes de forme}\label{chapter-HTT_analysis-section-systematics-shapes}


