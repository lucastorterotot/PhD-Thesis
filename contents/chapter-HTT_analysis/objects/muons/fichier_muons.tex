\subsection{Muons}\label{chapter-HTT_analysis-section-objects-muons}
Les muons utilisés sont les objets physiques identifiés comme étant des muons par l'algorithme de \PF~\cite{particle-flow}.
La sélection des muons se fait sur la base de critères supplémentaires discutés ci-après.

\subsubsection{Identification}\label{chapter-HTT_analysis-section-objects-muons-ID}
Il est possible de définir un critère de qualité sur l'objet reconstruit devant correspondre à un muon à partir des propriétés des éléments du \PF\ ayant servi à sa reconstruction, exposée dans le chapitre~\refChLHCCMS.
Il s'agit du \muonID~\cite{CMS-MUO-16-001,cmsMediumMuon}.
Le $\chi^2$ de l'ajustement de la trajectoire ainsi que la fraction des signaux du trajectographe valides associés au muon sont des métriques utilisées pour le \muonID.
Des plus, un algorithme (\emph{kink finder}) sépare la trajectoire du muon et détermine un $\chi^2$ afin de vérifier si cette trajectoire reconstruite peut en réalité provenir de deux traces réelles distinctes.
Ce dernier cas de figure peut survenir suite à une déviation du muon par le matériel constituant le détecteur, par exemple.
\par
Pour des critères d'identification stricts, il est possible d'utiliser le nombre de points de passage (\emph{hits}) dans les chambres à muons utilisés pour l'ajustement global de la trajectoire du muon, \Nmdhits.
Le nombre de stations de chambres à muons associées à la trajectoire, \Nms, est aussi exploité.
Les informations issues du trajectographe sont également utilisées.
Il s'agit du nombre de \emph{hits} dans la partie à pixels, \Npixelhits, et du nombre total de \emph{hits} dans le trajectographe, \Ntrkhits.
\par
Trois niveaux d'exigence ou points de fonctionnement (\emph{Working Point}) sont ainsi définis, de plus en plus exigeants.
En particulier, le \emph{Medium} \muonID\ est utilisé dans les canaux \mu\tauh, \ele\tauh\ et \ele\mu\ de l'analyse, comme le recommande le \emph{POG}\footnote{POG: \emph{Physics Object Group}, groupe responsable d'un objet physique, ici les muons.} Muons~\cite{cmsMediumMuon}.
\paragraph{\emph{Loose} \muonID} (exigence lâche)
\begin{itemize}
\item le muon est issu du \PF;
\item le muon est reconstruit comme muon global ou du trajectographe.
\end{itemize}
\paragraph{\emph{Medium} \muonID} (exigence moyenne)
\begin{itemize}
\item le muon passe le \emph{loose} \muonID;
\item au moins \SI{80}{\%} des signaux du trajectographe associés au muon sont valides.
\end{itemize}
De plus, un des deux ensembles de critères suivants doit être respecté:
\begin{itemize}
\item le muon est un muon global;
\item l'ajustement de la trajectoire vérifie $\chi^2/\Ndof<3$, avec \Ndof\ le nombre de degrés de liberté de l'ajustement;
\item l'accord entre le muon seul et le muon du trajectographe issus des mêmes éléments de reconstruction que le muon global vérifie $\chi^2<12$;
\item la compatibilité avec une déviation du muon due au matériel du détecteur (\emph{kink finder}) vérifie $\chi^2<20$;
\item la compatibilité du segment est supérieure à \num{0.303};
\end{itemize}
ou
\begin{itemize}
\item le muon est un muon du trajectographe;
\item la compatibilité du segment est supérieure à \num{0.451}.
\end{itemize}
\paragraph{\emph{Tight} \muonID} (exigence stricte)
\begin{itemize}
\item le muon est issu du \PF;
\item le muon est reconstruit comme muon global;
\item l'ajustement de la trajectoire vérifie $\chi^2/\Ndof<10$;
\item les chambres à muon vérifient $\Nmdhits>0$ et $\Nms>1$;
\item le trajectographe vérifie $\Npixelhits>0$ et $\Ntrkhits>5$;
\item les paramètres d'impact du muon vis-à-vis du vertex primaire principal vérifient $d_{xy} < \SI{2}{\milli\meter}$ et $d_z<\SI{5}{\milli\meter}$.
\end{itemize}

\subsubsection{Isolation}\label{chapter-HTT_analysis-section-objects-muons-iso}
Des muons peuvent être produits lors de la désintégration de quarks de saveur lourde.
Ces désintégrations sont accompagnées de jets, comme exposé dans le chapitre~\refChJERC.
Afin de réduire la contamination de l'analyse par ces muons, un critère d'isolation est appliqué.
\par
L'isolation d'un muon est quantifiée à partir des particules situées dans un cône de rayon
\begin{equation}
\Delta R = \sqrt{\Delta\eta^2+\Delta\phi^2} < R_\mu=\num{0.4}
\end{equation}
autour de la direction du muon au niveau du vertex primaire principal,
avec $\Delta\eta$ et $\Delta\phi$ les distances angulaires des particules au muon dans les directions $\eta$ et $\phi$,
selon
\begin{equation}
I^{(\mu)}
=
\left.
\sum_{\text{ch},\text{PV}} \pT^{\text{ch}}
+
\max\left(
0
,
\sum_{\hadron^0}\ET^{\hadron^0}
+
\sum_{\gamma}\ET^{\gamma}
- \Delta\beta
\sum_{\text{ch},\text{PU}} \pT^{\text{ch}}
\right)
\right|_{\Delta R < R_\mu}
\label{eq-muons-iso}
\end{equation}
où
$\sum_{\text{ch},\text{PV}} \pT^{\text{ch}}$ est la somme scalaire des impulsions transverses des particules chargées provenant du vertex primaire principal à l'exception de ce muon,
$\sum_{\hadron^0}\ET^{\hadron^0}$ est la somme des énergies dans le plan transverse de tous les hadrons neutres,
$\sum_{\gamma}\ET^{\gamma}$ est la somme des énergies dans le plan transverse de tous les photons,
$\sum_{\text{ch},\text{PU}} \pT^{\text{ch}}$ est la somme scalaire des impulsions transverses des particules chargées provenant de l'empilement et
$\Delta\beta$ est une estimation du rapport entre particules neutres et particules chargées créées lors des collisions de protons.
Le second terme de l'équation~\eqref{eq-muons-iso} permet ainsi d'estimer la contribution des particules neutres à l'isolation.
La variable d'isolation ainsi construite est basse pour des particules isolées, haute pour des particules non isolées.
Il est possible de définir l'isolation relative comme étant le rapport entre l'isolation et l'impulsion transverse de la particule,
\begin{equation}
I_\text{rel}^{i}
=
\frac{1}{\pT^{i}}
I^{i}
\mend
\label{eq-muons-reliso}
\end{equation}