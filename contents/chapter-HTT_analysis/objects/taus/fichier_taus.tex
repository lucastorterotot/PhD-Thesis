\subsection{Taus hadroniques}\label{chapter-HTT_analysis-section-objects-taus}

\begin{wrapfigure}{R}{.4\textwidth}
\centering
\begin{fmffile}{tau_to_tauh-3prongs}%\fmfstraight
\begin{fmfchar*}(30,20)
  \fmfleft{taui}
  \fmfright{l1,l2,l3,f1,f2,f3,nuout}
  \fmf{fermion, label=$\leptau$, l.side=left, tension=2}{taui,v1}
  \fmf{fermion}{v1,nuout}
  \fmf{phantom}{v1,l1}
  \fmffreeze
  \fmflabel{$\nutau$}{nuout}
  \fmf{boson, label=$\Wbosonminus$, l.side=right, tension=2}{v1,v2}
  \fmf{phantom}{v2,td1,l1}
  \fmf{phantom}{v2,td2,l2}
  \fmf{phantom}{v2,td3,l3}
  \fmffreeze
  \fmf{plain}{td3,v2,td1}
  \fmfblob{.15w}{td2}
  \fmf{plain}{td2,l1}
  \fmf{plain}{td2,l2}
  \fmf{plain}{td2,l3}
  \fmflabel{$\hadron^-$}{l1}
  \fmflabel{$\hadron^-$}{l2}
  \fmflabel{$\hadron^+$}{l3}
  \fmfdot{v1,v2}
\end{fmfchar*}
\end{fmffile}

\vspace{\baselineskip}
\caption{Diagramme de Feynman de désintégration hadronique d'un \leptau.}
\label{fig-fgraph-tau_to_tauh}
\end{wrapfigure}
Lors d'une désintégration hadronique d'un lepton tau, une paire de quarks est émise.
Il s'en suit donc un processus d'hadronisation, phénomène à l'origine de la formation des jets, ce qui est exposé dans le chapitre~\refChJERC.
Du lepton tau résulte alors un ensemble de hadrons, comme illustré sur la figure~\ref{fig-fgraph-tau_to_tauh}.
Ces hadrons, en général trois ou moins, sont éventuellement accompagnés de particules neutres, principalement des \pionnull.
Ces derniers se désintégrant majoritairement en deux photons.
L'ensemble de ces particule forme un \og tau hadronique \fg, noté \tauh, et est initialement identifié comme un jet.
\subsubsection{Obtention de candidats tau hadronique}
L'identification des taus hadroniques est réalisée par l'algorithme \emph{Hadrons Plus Strips} (HPS)~\cite{Khachatryan:2015dfa,Sirunyan:2018pgf} à partir des jets reconstruits par l'algorithme de \PF\ vérifiant $\pT>\SI{14}{\GeV}$ et $\abs{\eta}<\num{2.5}$.
Les hadrons chargés contenus dans le jet initial tels que $\pT>\SI{0.5}{\GeV}$ et de paramètre d'impact transverse $d_{xy}<\SI{0.1}{\centi\meter}$ vis-à-vis du vertex primaire principal sont utilisés pour former des candidats \tauh.
\par
Afin d'identifier les dépôts d'énergie dans le ECAL dus aux \pionnull, les photons et les électrons contenus dans le jet initial sont regroupés en bandes (\emph{strips}).
La construction d'une bande est un procédé itératif:
\begin{enumerate}
\item Une bande est créée à partir de l'électron ou du photon ($\ele/\photon$) de plus haut \pT\ contenu dans le jet initial et n'ayant pas déjà été associé à une bande. La position de cette particule dans le plan $(\eta,\phi)$, ainsi que son \pT, sont associés à la bande.
\item L'électron ou photon de plus haut \pT\ restant est ajouté à la bande s'il est situé à une distance par rapport à la bande dans le plan $(\eta,\phi)$ telle que
\begin{align}
\Delta\eta < f\left(\pT^{(\ele/\photon)}\right) + f\left(\pT^\text{bande}\right) \msep& f(\pT) = \num{0.20} (\pT [\SI{}{\GeV}])^{-\num{0.66}}\\
\Delta\phi < g\left(\pT^{(\ele/\photon)}\right) + g\left(\pT^\text{bande}\right) \msep& g(\pT) = \num{0.35} (\pT [\SI{}{\GeV}])^{-\num{0.71}}
\end{align}
avec $\pT^{(\ele/\photon)}$ l'impulsion transverse de l'électron ou du photon à ajouter à la bande et $\pT^\text{bande}$ l'impulsion transverse associée à la bande avant ajout de l'électron ou du photon.\\
Si l'ajout se fait, la bande est mise à jour selon
\begin{align}
\pT^\text{bande} &= \sum_{\ele/\photon\in\text{bande}}\pT^{(\ele/\photon)}\mend[,]\\
\eta^\text{bande} &= \frac{1}{\pT^\text{bande}} \sum_{\ele/\photon\in\text{bande}}\pT^{(\ele/\photon)} \eta_{(\ele/\photon)}\mend[,]\\
\phi^\text{bande} &= \frac{1}{\pT^\text{bande}} \sum_{\ele/\photon\in\text{bande}}\pT^{(\ele/\photon)} \phi_{(\ele/\photon)}\mend[,]
\end{align}
ce qui rend la bande dynamique lors de sa construction.
Les dimensions de la bande sont limitées à $\num{0.05}<\Delta\eta<\num{0.15}$ et $\num{0.05}<\Delta\phi<\num{0.3}$.
\item L'étape précédente est répétée jusqu'à ce qu'une limite de taille de la bande soit atteinte ou qu'il ne reste plus d'électron ni de photon tels que $\pT>\SI{0.5}{\GeV}$ dans la zone de la bande.
\item Les éléments associés à la bande sont retirés de la liste des électrons et photons en attente d'association à une bande.
\item Le procédé reprend à l'étape 1.
\end{enumerate}
Toute bande vérifiant $\pT>\SI{2.5}{\GeV}$ est considérée comme un candidat \pionnull.
\par
Des candidats \tauh\ compatibles avec un des modes de désintégration hadronique du tau sont ainsi formés à partir de toutes les combinaisons possibles de hadrons chargés et de candidats \pionnull.
\subsubsection{Modes de désintégration hadronique du tau}
Les modes de désintégration (\emph{Decay Modes}, DM) considérés dans l'analyse sont ainsi listé dans le tableau~\ref{tab-tauh-DMs}.
À chaque DM correspond une valeur afin de le désigner, définie comme
\begin{equation}
\text{DM} = 5\times(N_{\hadronpm}-1) - N_{\pionnull}
\end{equation}
où $N_{\hadronpm}$ est le nombre de hadrons chargés et $N_{\pionnull}$ le nombre de \pionnull\ contenus dans le \tauh.
Lorsqu'un des hadrons chargés n'est pas reconstruit, il est possible d'obtenir les DM 5, 6 ou 7.
Ces cas de figure sont largement contaminés par le bruit de fond \og QCD multijet \fg, ils sont donc rejetés dans l'analyse.
\begin{table}[h]
\centering
\begin{tabular}{clc}
\toprule
Code & Mode de désintégration & \BR{} (\SI{}{\%})\\
\midrule
0 & $\leptau\to \hadronminus \antinutau$ & \num{11.51} \\
1 & $\leptau\to \hadronminus \pionnull \antinutau$ & \num{25.93} \\
2 & $\leptau\to \hadronminus \pionnull\pionnull \antinutau$ & \num{9.48} \\
10 & $\leptau\to \hadronminus \hadronminus\hadronplus \antinutau$ & \num{9.80} \\
11 & $\leptau\to \hadronminus \hadronminus\hadronplus\pionnull\antinutau$ & \num{4.76} \\
\bottomrule
\end{tabular}
\caption[Modes de désintégration du \tau\ considérés.]{Modes de désintégration du \tau\ considérés. La désintégration d'un \leptau\ correspondant au DM, ainsi que le rapport de branchement $\leptau\to\tauh^-$ correspondant~\cite{PDG_booklet_2020} sont également donnés.}
\label{tab-tauh-DMs}
\end{table}
\par
Certains DM présentent des contraintes supplémentaires sur la masse du \tauh:
\begin{align*}
\text{DM 1:} && \SI{0.3}{\GeV} < m_{\tauh} < \num{1.3}\sqrt{\frac{\pT[\SI{}{\GeV}]}{100}} \SI{}{\GeV}\mend[,]\\
\text{DM 2:} && \SI{0.4}{\GeV} < m_{\tauh} < \num{1.2}\sqrt{\frac{\pT[\SI{}{\GeV}]}{100}} \SI{}{\GeV}\mend[,]\\
\text{DM 10 et 11:} && \SI{0.8}{\GeV} < m_{\tauh} < \SI{1.5}{\GeV}\mend[,]\\
\end{align*}
et, dans le cas du DM 10, les traces des hadrons chargés doivent provenir du même vertex dans la limite de $\Delta z < \SI{0.4}{\centi\meter}$.
\subsubsection{Sélection d'un candidat \tauh}
Il est possible d'obtenir plusieurs candidats \tauh\ au sein d'un même jet.
Des critères de qualité sur les candidats leur sont alors imposés.
\par
Par conservation, la somme des charges électriques des hadrons contenus dans le candidat \tauh\ doit valoir $\pm1$.
Ces hadrons chargés doivent de plus être contenus dans le cône dit \og de signal \fg{} défini et contraint selon
\begin{equation}
\Delta R_\text{sig} = \frac{\SI{3}{\GeV}}{\pT^{(\tauh)}}
\msep
\num{0.05} < \Delta R_\text{sig} < \num{0.1}
\mend
\end{equation}
Les centres des bandes du candidat \tauh\ doivent également se situer dans ce cône.
S'il reste plusieurs candidats à ce stade, celui de plus haut \pT\ est retenu.
Il existe donc au plus un \tauh\ par jet.
\subsubsection{Mauvaises reconstruction de \tauh}
Un \tauh\ peut être reconstruit à partir de jets n'étant pas des \tauh, d'électrons ou de muons.
Afin de réduire la quantité de mauvais \tauh\ (\ftauh), un réseau de neurones profond convolutionnel (DNN)~\cite{DNN} a été développé à CMS.
Il s'agit de l'algorithme \DEEPTAU~\cite{CMS-DP-2019-033} qui fournit les discriminateurs
\texttt{deepTau vs jet},
\texttt{deepTau anti-electron} et
\texttt{deepTau anti-muon}
utilisés dans cette analyse.
\par
Les points de fonctionnement utilisés pour chacun de ces discriminateurs dépendent de l'état final, comme présenté dans la section~\ref{chapter-HTT_analysis-section-selection}.
Les efficacités d'identification de chacun des points de fonctionnement existants sont donnés dans le tableau~\ref{tab-HTT_analysis-section-objects-taus-DEEPTAU_eff}.
Les taux d'identification de jet, électron ou muon comme étant un \tauh, \ie\ les faux positifs, dépendent de la nature des événements sur lesquels ces discriminateurs sont appliqués et se situent entre \num{e-4} et \num{e-2}.
\begin{table}[h]
\centering
\begin{tabular}{lcccccccc}
\toprule
Discriminateur & \emph{VVTight} & \emph{VTight} & \emph{Tight} & \emph{Medium} & \emph{Loose} & \emph{VLoose} & \emph{VVLoose} & \emph{VVVLoose}\\
\midrule
\texttt{vs jet} & \num{40} & \num{50} & \num{60} & \num{70} & \num{80} & \num{90} & \num{95} & \num{98} \\
\texttt{anti-electron} & \num{60} & \num{70} & \num{80} & \num{90} & \num{95} & \num{98} & \num{99} & \num{99.5} \\
\texttt{anti-muon} & - & - & \num{99.5} & \num{99.8} & \num{99.9} & \num{99.95} & - & - \\
\bottomrule
\end{tabular}
\caption[Efficacités d'identification de l'algorithme \DEEPTAU.]{Efficacités d'identification en \SI{}{\%} de l'algorithme \DEEPTAU\ pour chacun des points de fonctionnement disponibles~\cite{CMS-DP-2019-033,Androsov_deeptau}.}
\label{tab-HTT_analysis-section-objects-taus-DEEPTAU_eff}
\end{table}