\subsection{Électrons}\label{chapter-HTT_analysis-section-objects-electrons}
Les électrons utilisés sont les objets physiques identifiés comme étant des électrons par l'algorithme de \PF~\cite{particle-flow}.
La sélection des électrons se fait sur la base de critères supplémentaires discutés ci-après.

\subsubsection{Identification}\label{chapter-HTT_analysis-section-objects-muons-ID}
Les électrons retenus pour l'analyse doivent passer un discriminateur d'identification basé sur un arbre de décision (\emph{Bossted Decision Tree}, BDT).
Le BDT utilise plusieurs variables afin de fournir l'identification d'électron \og \EleIDMVA \fg~\cite{cmsElectronMVA}:
\begin{itemize}
\item l'étalement en $\eta$ et en $\phi$ du dépôt d'énergie dans le ECAL, $\sigma_{i\eta i\eta}$ et $\sigma_{i\phi i\phi}$ où $i\eta$ et $i\phi$ correspondent au nombre entier désignant le cristal du calorimètre;
\item la circularité du dépôt d'énergie, $1- E_{1\times5}/E_{5\times5}$, où $ E_{1\times5}$ et $E_{5\times5}$ correspondent respectivement aux énergies dans une région de $1\times5$ et $5\times5$ cristaux centré sur le cristal contenant le plus d'énergie dans le \emph{supercluster}, défini dans la section~\ifref{chapter-LHC-section-evt_reco-subsec-ptc_ID}{\ref{chapter-LHC-section-evt_reco-subsec-ptc_ID}}{4.2} du chapitre~\ifref{chapter-LHC}{\ref{chapter-LHC}}{3};
\item $R_9 = \frac{E_{3\times3}}{E_{SC}}$, où $E_{SC}$ est l'énergie contenue dans le \emph{supercluster};
\item le $\chi^2$ de l'ajustement de la trajectoire;
\item le nombre de \emph{hits} valides utilisés pour l'ajustement de la trajectoire;
\item le $\chi^2$ de l'ajustement de la trajectoire GSF (\emph{GSFtrack}). Le \emph{Gaussian Sum Filter} est une méthode de traitement du signal~\cite{GSF};
\item le nombre de \emph{hits} utilisés pour l'ajustement de la trajectoire GSF;
\item le nombre attendu de \emph{hits} manquants;
\item l'ajustement à un vertex de conversion (cas de la conversion $\photon\to\positron\electron$);
\item les écarts $\Delta\eta$ et $\Delta\phi$ entre le \emph{supercluster} et la direction de la trace associée à l'électron au niveau du vertex primaire;
\item l'écart $\Delta\eta$ entre le \emph{supercluster} et la direction de la trace associée à l'électron au niveau de la surface du ECAL;
\item $H/E$ le rapport de l'énergie hadronique sur l'énergie électromagnétique associées à l'électron;
\item $E/P$ le rapport de l'énergie du \emph{supercluster} et de l'impulsion de l'électron;
\item $1/E_{\ele}-1/P_{\ele}$ où $E_{\ele}$ est l'énergie de l'électron et $P_{\ele}$ son impulsion.
\end{itemize}
Le BDT est ainsi entraîné sur des événements $\Zboson/\photon^*$ simulés à l'aide de \MADGRAPH~\cite{madgraph5}.

\subsubsection{Isolation}\label{chapter-HTT_analysis-section-objects-electrons-iso}
Tout comme les muons, des électrons peuvent être produits lors de la désintégration de quarks de saveur lourde.
Un critère d'isolation, similaire à celui des muons présenté dans la section~\ref{chapter-HTT_analysis-section-objects-muons-iso}, est ainsi défini.
\begin{wraptable}{R}{.33\textwidth}
\centering
\begin{tabular}{cc}
\toprule
Région & $\mathcal{E_A}$ \\
\midrule
$\abs{\eta} \leq \num{1.0}$ & \num{0.1440} \\
$\num{1.0} < \abs{\eta} \leq \num{1.479}$ & \num{0.1562} \\
$\num{1.479} < \abs{\eta} \leq \num{2.0}$ & \num{0.1032} \\
$\num{2.0} < \abs{\eta} \leq \num{2.2}$ & \num{0.0859}  \\
$\num{2.2} < \abs{\eta} \leq \num{2.3}$ & \num{0.1116} \\
$\num{2.3} < \abs{\eta} \leq \num{2.4}$ & \num{0.1321} \\
$\abs{\eta} > \num{2.4}$ & \num{0.1654} \\
\bottomrule
\end{tabular}
\caption[Aires effectives de correction de l'isolation de l'électron.]{Valeurs de l'aire effective $\mathcal{E_A}$ utilisée pour corriger la contribution de l'empilement aux isolations des électrons vis-à-vis des autres particules.}
\label{tab-electron-effective_areas}
\end{wraptable}
\par
L'isolation d'un électron est quantifiée à partir des photons et des hadrons, neutres et chargés, situés dans un cône de rayon $R_\ele=\num{0.3}$ selon
\begin{equation}
I^{(\ele)}
=
\left.
\sum_{\hadron^\pm,\text{PV}} \pT^{\hadron^\pm}
+
\max\left(
0
,
\sum_{\hadron^0}\ET^{\hadron^0}
+
\sum_{\gamma}\ET^{\gamma}
- \rho \times \mathcal{E_A}
\right)
\right|_{\Delta R < R_\ele}
\label{eq-electrons-iso}
\end{equation}
où
$\sum_{\hadron^\pm,\text{PV}} \pT^{\hadron^\pm}$ est la somme scalaire des impulsions transverses des hadrons chargés provenant du vertex primaire principal,
$\sum_{\hadron^0}\ET^{\hadron^0}$ est la somme des énergies dans le plan transverse de tous les hadrons neutres,
$\sum_{\gamma}\ET^{\gamma}$ est la somme des énergies dans le plan transverse de tous les photons,
$\rho$ est la densité d'énergie issue de l'empilement dans l'événement et
$\mathcal{E_A}$ est l'aire effective, \ie\ la fraction de l'espace $(\eta,\phi)$ correspondant à la zone d'isolation à corriger pour l'empilement.
Les valeurs des aires effectives utilisées sont présentées dans le tableau~\ref{tab-electron-effective_areas}.