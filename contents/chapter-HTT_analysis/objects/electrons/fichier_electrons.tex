\subsection{Électrons}\label{chapter-HTT_analysis-section-objects-electrons}
Les électrons utilisés sont les objets physiques identifiés comme étant des électrons par l'algorithme de \PF~\cite{particle-flow}.
La sélection des électrons se fait sur la base de critères supplémentaires discutés ci-après.

\subsubsection{Identification}\label{chapter-HTT_analysis-section-objects-electron-ID}
\paragraph{\EleIDMVA}
Les électrons retenus pour l'analyse doivent passer un discriminateur d'identification issu d'une analyse multivariée (\emph{MultiVariate Analysis}, MVA).
Cette analysée est basée sur un arbre de décision (\emph{Bossted Decision Tree}, BDT) afin de fournir l'identification d'électron \og \EleIDMVA \fg~\cite{cmsElectronMVA}.
Les variables prises en compte sont:
\begin{itemize}
\item l'impulsion transverse de l'électron $\pT^{(\ele)}$;
\item la pseudo-rapidité du \emph{supercluster}, défini dans la section~\ifref{chapter-LHC-section-evt_reco-subsec-ptc_ID}{\ref{chapter-LHC-section-evt_reco-subsec-ptc_ID}}{4.2} du chapitre~\ifref{chapter-LHC}{\ref{chapter-LHC}}{3};
\item la densité d'énergie issue de l'empilement dans l'événement $\rho$;

\item l'étalement en $\eta$ et en $\phi$ du dépôt d'énergie dans le ECAL, $\sigma_{i\eta i\eta}$ et $\sigma_{i\phi i\phi}$ où $i\eta$ et $i\phi$ correspondent au nombre entier désignant le cristal du calorimètre;
\item la circularité du dépôt d'énergie, $1- E_{1\times5}/E_{5\times5}$, où $ E_{1\times5}$ et $E_{5\times5}$ correspondent respectivement aux énergies dans une région de $1\times5$ et $5\times5$ cristaux centré sur le cristal contenant le plus d'énergie dans le \emph{supercluster};
\item $R_9 = \frac{E_{3\times3}}{E_{SC}}$, où $E_{SC}$ est l'énergie contenue dans le \emph{supercluster};
\item $H/E_{SC}$ où $H$ est l'énergie hadronique située dans un cône de $\Delta R < \num{0.15}$ autour de l'électron;
\item $E_{SC}^\text{PS}/E_{SC}^\text{raw}$ le rapport de l'énergie du \emph{supercluster} située dans le \emph{PreShower} sur son énergie totale non corrigée.
Le \emph{PreShower} est défini dans le chapitre~\ifref{chapter-LHC}{\ref{chapter-LHC}}{3};
\item la largeur du \emph{supercluster}, $\Delta \eta_{SC}$ et $\Delta \phi_{SC}$;

\item le $\chi^2$ de l'ajustement de la trajectoire;
\item le nombre de \emph{hits} valides utilisés pour l'ajustement de la trajectoire;
\item le $\chi^2$ de l'ajustement de la trajectoire GSF (\emph{GSFtrack}). Le \emph{Gaussian Sum Filter} est une méthode de traitement du signal~\cite{GSF};
\item le nombre de \emph{hits} utilisés pour l'ajustement de la trajectoire GSF, $N_\text{lost}^\text{GSF}$;
\item le nombre attendu de \emph{hits} manquants;
\item la fraction d'énergie perdue par \emph{bremsstrahlung}, $f_\text{brem} = 1-p_\text{out}/p_\text{in}$ où
$p_\text{in}$ est l'impulsion de l'électron obtenue d'après la courbe de sa trajectoire près du vertex primaire et
$p_\text{out}$ l'impulsion de l'électron obtenue d'après la courbe de sa trajectoire près de la surface de ECAL;

\item $E_{SC}/p_\text{in}$;
\item $E_{\PF}/p_\text{in}$ avec $E_{\PF}$ est l'énergie du \emph{supercluster} le plus proche du point d'entrée de l'électron dans le ECAL;
\item les écarts $\Delta\eta$ et $\Delta\phi$ entre le \emph{supercluster} et la direction de la trace associée à l'électron au niveau du vertex primaire;
\item l'écart $\Delta\eta$ entre le \emph{supercluster} et la direction de la trace associée à l'électron au niveau de la surface du ECAL;
\item $1/E_{\ele}-1/P_{\ele}$ où $E_{\ele}$ est l'énergie de l'électron et $P_{\ele}$ son impulsion.

\item la probabilité que l'électron soit issu d'une conversion $\photon\to\positron\electron$;
\end{itemize}
\par
Le BDT est ainsi entraîné sur des événements $\text{Drell-Yann ($\Zboson/\photon^*$)} + \text{jets}$ simulés à l'aide de \MADGRAPH~\cite{madgraph5}.
L'entraînement se fait à l'aide de \XGBOOST~\cite{xgboost}.
Le point de fonctionnement à \SI{90}{\%} d'efficacité est défini à partir d'une valeur minimale de sortie du BDT.
Cette valeur dépend de $\pT^{(\ele)}$ et $\eta^{(\ele)}$ ainsi que de l'année de prise de données.

\paragraph{\CutBasedEleID}
Un autre critère d'identification est utilisé pour les électrons et consiste en une liste de coupures (\emph{cut}) sur certaines variables.
Il s'agit du \CutBasedEleID.
Les valeurs des coupures dépendent du point de fonctionnement.
Dans l'analyse, seul le point de fonctionnement \og veto \fg{} est utilisé, les coupures associées sont listées dans le tableau~\ref{tab-CutBasedEleIDVeto}.
Les variables utilisées sont définies précédemment, à l'exception de
\begin{itemize}
\item $\abs{\Delta\eta_\text{in}^\text{seed}}$ l'écart en $\eta$ entre le point d'entrée de l'électron dans le ECAL et la position du \emph{supercluster} identifié par l'algorithme de \PF;
\item $I_\text{rel}^{\Delta\beta}$ l'isolation relative de l'électron obtenue avec la même formule que pour les muons~\eqref{eq-muons-reliso}, à l'exception de la taille du cône valant ici $R_{\ele}=\num{0.3}$.
\end{itemize}
\begin{table}[h]
\centering
\begin{tabular}{ccc}
\toprule
Variable & $\abs{\eta^{(\ele)}} < \num{1.479}$ & $\abs{\eta^{(\ele)}} \geq \num{1.479}$ \\
\midrule
$\sigma_{i\eta i\eta}$ & $<\num{0.0126}$ & $<\num{0.0457}$ \\
$\abs{\Delta\eta_\text{in}^\text{seed}}$ & $<\num{0.00463}$ & $<\num{0.00814}$ \\
$\abs{\Delta\phi_\text{in}}$ & $<\num{0.148}$ & $<\num{0.19}$ \\
$H/E_{SC}$ & $<\num{0.05}+\frac{\num{1.16}}{E_{SC} [\SI{}{\GeV}]} + \num{0.0324}\frac{\rho}{E_{SC}}$ & $<\num{0.05}+\frac{\num{2.54}}{E_{SC} [\SI{}{\GeV}]} + \num{0.183}\frac{\rho}{E_{SC}}$ \\
$I_\text{rel}^{\Delta\beta}$ & $<\num{0.198} + \frac{\num{0.506}}{\pT^{(\ele)} [\SI{}{\GeV}]}$ & $<\num{0.203} + \frac{\num{0.96}}{\pT^{(\ele)} [\SI{}{\GeV}]}$ \\
$\abs{1/E_{SC}-1/p_\text{in}}$ & $<\SI{0.209}{\GeV^{-1}}$ & $<\SI{0.132}{\GeV^{-1}}$ \\
$N_\text{lost}^\text{GSF}$ & $\leq\num{2}$ & $\leq\num{3}$ \\
veto de conversion & passé & passé \\
\bottomrule
\end{tabular}
\caption[Coupures du \CutBasedEleIDVeto.]{Coupures du \CutBasedEleIDVeto\ pour les deux régions en $\eta$ du \emph{supercluster} possibles. Les variables sont détaillées dans le texte.}
\label{tab-CutBasedEleIDVeto}
\end{table}

\subsubsection{Isolation}\label{chapter-HTT_analysis-section-objects-electrons-iso}
Tout comme les muons, des électrons peuvent être produits lors de la désintégration de quarks de saveur lourde.
Un critère d'isolation, similaire à celui des muons présenté dans la section~\ref{chapter-HTT_analysis-section-objects-muons-iso}, est ainsi défini.
\begin{wraptable}{R}{.33\textwidth}
\centering
\begin{tabular}{cc}
\toprule
Région & $\mathcal{E_A}$ \\
\midrule
$\abs{\eta} \leq \num{1.0}$ & \num{0.1440} \\
$\num{1.0} < \abs{\eta} \leq \num{1.479}$ & \num{0.1562} \\
$\num{1.479} < \abs{\eta} \leq \num{2.0}$ & \num{0.1032} \\
$\num{2.0} < \abs{\eta} \leq \num{2.2}$ & \num{0.0859}  \\
$\num{2.2} < \abs{\eta} \leq \num{2.3}$ & \num{0.1116} \\
$\num{2.3} < \abs{\eta} \leq \num{2.4}$ & \num{0.1321} \\
$\abs{\eta} > \num{2.4}$ & \num{0.1654} \\
\bottomrule
\end{tabular}
\caption[Aires effectives de correction de l'isolation de l'électron.]{Valeurs de l'aire effective $\mathcal{E_A}$ utilisée pour corriger la contribution de l'empilement aux isolations des électrons vis-à-vis des autres particules.}
\label{tab-electron-effective_areas}
\end{wraptable}
\par
L'isolation d'un électron est quantifiée à partir des particules situées dans un cône de rayon
\begin{equation}
\Delta R = \sqrt{\Delta\eta^2+\Delta\phi^2} < R_\ele=\num{0.3}
\end{equation}
autour de la direction de l'électron au niveau du vertex primaire principal,
avec $\Delta\eta$ et $\Delta\phi$ les distances angulaires des particules à l'électron dans les directions $\eta$ et $\phi$,
selon
\begin{equation}
I^{(\ele)}
=
\left.
\sum_{\text{ch},\text{PV}} \pT^{\text{ch}}
+
\max\left(
0
,
\sum_{\hadron^0}\ET^{\hadron^0}
+
\sum_{\gamma}\ET^{\gamma}
- \rho \times \mathcal{E_A}
\right)
\right|_{\Delta R < R_\ele}
\label{eq-electrons-iso}
\end{equation}
où
$\sum_{\text{ch},\text{PV}} \pT^{\text{ch}}$ est la somme scalaire des impulsions transverses des particules chargées provenant du vertex primaire principal à l'exception de cet électron,
$\sum_{\hadron^0}\ET^{\hadron^0}$ est la somme des énergies dans le plan transverse de tous les hadrons neutres,
$\sum_{\gamma}\ET^{\gamma}$ est la somme des énergies dans le plan transverse de tous les photons,
$\rho$ est la densité d'énergie issue de l'empilement dans l'événement et
$\mathcal{E_A}$ est l'aire effective, \ie\ la fraction de l'espace $(\eta,\phi)$ correspondant à la zone d'isolation à corriger pour l'empilement.
Les valeurs des aires effectives utilisées sont présentées dans le tableau~\ref{tab-electron-effective_areas}.