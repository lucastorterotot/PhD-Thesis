\subsection{Estimation du bruit de fond QCD dans le canal \ele\mu}\label{chapter-HTT_analysis-section-bg_estimation-QCD-SS}
Dans le cas du canal \ele\mu, le bruit de fond QCD contribue à la sélection des événements lorsqu'au moins un jet est identifié à tort comme un électron ou un muon.
Une estimation de cette contribution est réalisée à partir des données réelles en suivant le principe de la méthode \og ABCD \fg.
\begin{wrapfigure}{R}{7.25cm}
\centering
\begin{tikzpicture}[scale=.8]
\draw (-2,0)--(2,0);
\draw (0,-2)--(0,2);
\draw (-1,1) node {A = SR};
\draw (1,1) node {C};
\draw (-1,-1) node {B = AR};
\draw (1,-1) node {D};
\draw (-2,1) node [left] {OS};
\draw (-2,-1) node [left] {SS};
\draw (-2,2) node [above left] {\mu:};
\draw (-1,2) node [above] {isolé\vphantom{\mu:}};
\draw (1,2) node [above] {anti-isolé\vphantom{\mu:}};
\end{tikzpicture}
\caption{Définition schématique des régions A, B, C et D pour l'estimation du bruit de fond QCD.}
\label{fig-ABCD_regions_schem}
\end{wrapfigure}
\par
Quatre régions pouvant se résumer schématiquement comme illustré sur la figure~\ref{fig-ABCD_regions_schem} sont définies:
\begin{description}
\item[A] région de signal (SR), définie dans la section~\ref{chapter-HTT_analysis-section-selection};
\item[B] définie comme la SR mais avec les charges électriques de l'électron et du muon de même signe (SS, \emph{Same Signs}) et non de signes opposés (OS, \emph{Opposite Signs}) comme dans la SR;
\item[C] définie comme la SR mais avec un muon \og anti-isolé \fg{}, \ie\ que la coupure sur son isolation est inversée, $\num{0.2}\leq I_\text{rel}^{(\mu)} < \num{0.5}$ au lieu de $I_\text{rel}^{(\mu)} < \num{0.2}$;
\item[D] définie comme la SR mais avec muon anti-isolé et SS.
\end{description}
Les hypothèses d'application de cette méthode sont:
\begin{itemize}
\item la forme de la distribution de la variable $v$ issue du bruit de fond QCD est identique dans la région A à déterminer et dans la région B connue;
\item le rapport du nombre d'événements entre A et B est le même qu'entre C et D.
\end{itemize}
La région B est ainsi également nommée région d'application (AR, \emph{Application Region}) du facteur $C/D$.
Les contributions des bruits de fond autres que QCD aux régions B, C et D sont soustraits à partir de données simulées.
\par
La méthode ABCD permet alors d'obtenir le bruit de fond QCD dans la région de signal A selon ce qui s'assimile à un produit en croix,
\begin{equation}
A = B \times \frac{C}{D} \Leftrightarrow h_v^\text{A} = h_v^\text{B} \times \frac{\int h_v^\text{C}}{\int h_v^\text{D}}
\end{equation}
où $h_v^\text{X}$ correspond à la distribution de la variable $v$ dans la région X et $\int h_v^\text{X}$ à son intégrale, \ie\ la quantité d'événements (indépendante de $v$).
\par
Afin d'augmenter la quantité d'événements exploités, et donc de réduire l'incertitude statistique, la coupure sur \Dzeta\ n'est pas appliquée dans les régions C et D.
Un facteur $C/D$ global donne une estimation trop peu précise~\cite{CMS-PAS-HIG-18-032} car l'hypothèse d'indépendance de la forme de la distribution n'est pas vérifiée.
Afin de corriger cet effet, le facteur $C/D$ est déterminé en fonction de:
\begin{itemize}
\item la distance entre l'électron et le muon dans le plan $(\eta,\phi)$, $\Delta R$;
\item le nombre de jets \Njets;
\item l'impulsion transverse de l'électron, $\pT^{(\ele)}$;
\item l'impulsion transverse du muon, $\pT^{(\mu)}$;
\end{itemize}
La dépendance en $\Delta R$ est majoritairement due à la contribution \quarkb\antiquarkb\ au bruit de fond QCD.
Elle est modélisée par un polynôme de degré 2.
\par
Pour corriger le biais introduit par le changement de critère d'isolation du muon, le facteur $C/D$ est également déterminé dans le cas d'un électron anti-isolé ($\num{0.15}\leq I_\text{rel}^{(\ele)} < \num{0.5}$ au lieu de $I_\text{rel}^{(\ele)} < \num{0.15}$) et d'un muon isolé et pour électron et muon anti-isolés.
Le rapport de ces facteurs donne la correction relative au passage des muons isolés à anti-isolés.