\subsection{Estimation du bruit de fond QCD dans le canal \ele\mu}\label{chapter-HTT_analysis-section-bg_estimation-QCD-SS}

%In the eμ channel, the QCD background is estimated by inverting the charge requirement on
%the selected electron-muon pair. This way, a control region enriched in QCD events is cre-
%ated. Correction factors for the extrapolation of the QCD background from the region where
%the electron-muon pair has same-sign charges to the signal region where the electron and the
%muon candidate are of opposite charge are determined in the control region where the muon
%candidate fails the isolation requirement of the baseline selection but still passes I rel < 0.5.
%Non-QCD background processes are subtracted from the data in the same-sign region using
%their estimates from simulation. The OS/SS extrapolation factors, depend on the p T of the se-
%lected electron candidate, the p T of the selected muon candidate, the number of jets in the event
%and the distance between the electron and the muon candidate measured in the η-φ plane.
%The following detailed explanation of the method closely follows the description in
%
%A. Loeliger, C. Caillol, A. Mallampalli, J. Madhusudanan, S. Dasu, T. Bose, D. Kim, T.
%Mitchell, Y. Maravin, A. Mohammadi, K. Kaadze, S. Duric, I. Ojalvo, S. Higginbotham, D.
%Marlow, “Measurement of the higgs boson production and decay to a pair of tau leptons
%on the full run 2 data set using a cut-based approach”, CMS Note 2019/109 (2019).