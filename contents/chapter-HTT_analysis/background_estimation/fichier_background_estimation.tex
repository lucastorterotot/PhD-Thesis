\section{Modélisation du bruit de fond}\label{chapter-HTT_analysis-section-bg_estimation}
Le bruit de fond est constitué de tous les événements sélectionnés selon la procédure décrite section~\ref{chapter-HTT_analysis-section-selection} mais ne correspondant pas au signal recherché, \ie\ hors $\higgs,\Higgs,\HiggsA\to\tau\tau$.
Plusieurs processus contribuent ainsi au bruit de fond de cette analyse.
En effet, ils peuvent donner des états finaux similaires à ceux attendus avec le signal recherché, comme illustré sur la figure~\ref{fig-chapter-HTT_analysis-section-bg_estimation-procs}.
\begin{figure}[b]
\centering

\vspace{\baselineskip}

\subcaptionbox{Signal $\gluon\gluon\to\Higgs\to\tau\tau\to\mu\tauh$.\label{subfig-chapter-HTT_analysis-section-bg_estimation-procs-ggH}}[.45\textwidth]
{\input{\PhDthesisdir/plots_and_images/Feynman_diagrams/BG/fgraph-H-tautau_mutau_small.tex}\vspace{\baselineskip}}
\hfill
\subcaptionbox{Signal $\gluon\gluon\to\quarkb\antiquarkb\Higgs\to\quarkb\antiquarkb\tau\tau\to\mu\tauh+2\text{\quarkb-jets}$.\label{subfig-chapter-HTT_analysis-section-bg_estimation-procs-bbH}}[.45\textwidth]
{\input{\PhDthesisdir/plots_and_images/Feynman_diagrams/BG/fgraph-H-BBtautau_mutau_small.tex}\vspace{\baselineskip}}

\vspace{2\baselineskip}

\subcaptionbox{Drell-Yann $\quark\antiquark\to\Zboson\to\tau\tau\to\mu\tauh$.\label{subfig-chapter-HTT_analysis-section-bg_estimation-procs-DY}}[.45\textwidth]
{\begin{fmffile}{DY_small}\fmfstraight
\begin{fmfchar*}(50,25)
  \fmfleft{a,g1,b,c,g2,d}
  \fmfright{nu1,upq,doq,muout,antinumu,nu2}
  
  \fmf{phantom}{g1,g1loop}
  \fmf{phantom}{g2,g2loop}
  \fmf{phantom, tension=.2}{g1loop,upq}
  \fmf{phantom, tension=.2}{g2loop,antinumu}
  \fmffreeze
  \fmf{phantom}{g1loop,hloop,g2loop,g1loop}
  \fmf{phantom}{g2loop,g1loop}
  \fmf{boson, label=$\Zboson$, l.side=left, tension=1}{hloop,v}
  \fmf{phantom, tension=.2}{nu1,v,nu2}
  \fmffreeze
  
  \fmf{fermion}{g1,hloop,g2}
  
  \fmf{fermion, label=$\antitau$, l.side=left, tension=4}{t1d,v}
  \fmf{fermion, label=$\leptau$, l.side=left, tension=4}{v,t2d}
  \fmf{fermion, tension=3}{nu1,t1d}
  \fmf{boson, l.side=left, tension=2}{t1d,W1d}
  \fmf{phantom}{doq,W1d,upq}
  \fmf{fermion, tension=3}{t2d,nu2}
  \fmf{boson, tension=2}{t2d,W2d}
  \fmf{fermion}{antinumu,W2d,muout}
  \fmffreeze
  \fmf{plain}{W1d,upq}
  
  \fmflabel{\quark}{g1}
  \fmflabel{\antiquark}{g2}
  
  \fmflabel{{\color{ltcolorgray2}$\antinutau$}}{nu1}
  \fmflabel{{\color{ltcolorgray2}$\nutau$}}{nu2}
  \fmflabel{{\color{\muoncolor}$\muon$}}{muout}
  \fmflabel{{\color{ltcolorgray2}$\antinumu$}}{antinumu}
  \fmflabel{{\color{\tauhcolor}$\tauh$}}{upq}
  \fmfdot{hloop,v,t1d,t2d,W2d}
  \fmfblob{.07w}{W1d}
\end{fmfchar*}
\end{fmffile}
\vspace{\baselineskip}}
\hfill
\subcaptionbox{\ttbar\ $\quark\antiquark\to\gluon\to\quarkt\antiquarkt\to\mu\tauh+2\text{\quarkb-jets}$.\label{subfig-chapter-HTT_analysis-section-bg_estimation-procs-ttbar}}[.45\textwidth]
{\input{\PhDthesisdir/plots_and_images/Feynman_diagrams/BG/fgraph-ttbar_small.tex}\vspace{\baselineskip}}

\vspace{2\baselineskip}

\subcaptionbox{\Wjets\ avec un muon dans l'état final.\label{subfig-chapter-HTT_analysis-section-bg_estimation-procs-WJ}}[.45\textwidth]
{\begin{fmffile}{WJ_small}\fmfstraight
\begin{fmfchar*}(50,25)
  \fmfleft{a,g1,b,c,g2,d}
  \fmfright{nu1,upq,doq,muout,antinumu,nu2}
  
  
  \fmf{fermion}{g1,v1,v2,g2}
  \fmf{phantom, tension=.5}{v1,upq}
  \fmf{phantom, tension=.5}{v2,antinumu}
  \fmffreeze
  
  \fmf{gluon}{v1,W1d}
  \fmf{plain}{W1d,upq}
  
  \fmf{boson, label=$\Wboson$, l.side=left}{v2,W2d}
  \fmf{fermion, tension=.5}{antinumu,W2d,muout}
  
  \fmflabel{\quark}{g1}
  \fmflabel{\antiquark}{g2}
  
  \fmflabel{{\color{\muoncolor}$\muon$}}{muout}
  \fmflabel{{\color{ltcolorgray2}$\antinumu$}}{antinumu}
  \fmflabel{{\color{\jetcolor}jet}}{upq}
  \fmfdot{v1,v2,W2d}
  \fmfblob{.07w}{W1d}
\end{fmfchar*}
\end{fmffile}
\vspace{\baselineskip}}
\hfill
\subcaptionbox{QCD.\label{subfig-chapter-HTT_analysis-section-bg_estimation-procs-QCD}}[.45\textwidth]
{\begin{fmffile}{QCD_small}\fmfstraight
\begin{fmfchar*}(50,25)
  \fmfleft{a,g1,b,c,g2,d}
  \fmfright{nu1,upq,doq,muout,antinumu,nu2}
  
  \fmf{gluon}{g1,v3}
  \fmf{gluon}{g2,v4}
%  \fmf{gluon}{v1,v2}
%  \fmf{gluon}{v1,v3}
%  \fmf{gluon}{v2,v4}
  \fmf{fermion}{v3,v4,v6,v5,v3}
  \fmf{phantom}{v5,upq}
  \fmf{phantom}{v6,antinumu}

  \fmffreeze
  
  \fmf{gluon}{v6,v7}
  \fmf{plain}{v7,antinumu}
  
  \fmf{gluon}{v5,v8}
  \fmf{plain}{v8,upq}
  
  \fmflabel{\gluon}{g1}
  \fmflabel{\gluon}{g2}

  \fmflabel{{\color{\jetcolor}jet}}{upq}
  \fmflabel{{\color{\jetcolor}jet}}{antinumu}
  \fmfdot{v3,v4,v5,v6}
  \fmfblob{.07w}{v7,v8}
\end{fmfchar*}
\end{fmffile}
\vspace{\baselineskip}}

\caption[Diagrammes de Feynman des signaux et principaux bruits de fond de l'analyse.]{Diagrammes de Feynman complets des signaux $\gluon\gluon\Higgs$ (\ref{subfig-chapter-HTT_analysis-section-bg_estimation-procs-ggH}) et $\quarkb\antiquarkb\Higgs$ (\ref{subfig-chapter-HTT_analysis-section-bg_estimation-procs-bbH}) et bruits de fond Drell-Yann (\ref{subfig-chapter-HTT_analysis-section-bg_estimation-procs-DY}), \ttbar\ (\ref{subfig-chapter-HTT_analysis-section-bg_estimation-procs-ttbar}), \Wjets\ (\ref{subfig-chapter-HTT_analysis-section-bg_estimation-procs-WJ}) et QCD (\ref{subfig-chapter-HTT_analysis-section-bg_estimation-procs-QCD}) de l'analyse illustrés dans le cas du canal \mu\tauh.}
\label{fig-chapter-HTT_analysis-section-bg_estimation-procs}
\end{figure}
\begin{wraptable}{R}{.45\textwidth}
\centering
\begin{tabular}{lcccc}
\toprule
 & \multicolumn{4}{c}{Canal}\\
Bruit de fond & \tauh\tauh & \mu\tauh & \ele\tauh & \ele\mu \\
\midrule
$\Zboson\to\tau\tau$ & $\num{33}$ & $\num{46}$ & $\num{27}$ & $\num{20}$ \\
$\Zboson\to\ell\ell$, $\ell\in\set{\ele, \mu}$ & $\sim\num{1}$ & $\num{2}$ & $\num{9}$ & $\num{1}$ \\
\ttbar & $<\num{1}$ & $\num{13}$ & $\num{18}$ & $\num{54}$ \\
\Wjets & $<\num{1}$ & \multirow{2}{*}{$\num{36}$} & \multirow{2}{*}{$\num{42}$} & $\num{3}$ \\
QCD & $\num{66}$ & & & $\num{11}$ \\
Diboson & $<\num{1}$ & $\num{3}$ & $\num{4}$ & $\num{11}$ \\
\bottomrule
\end{tabular}
\caption{Contributions en pourcent des bruits de fond aux canaux étudiés.}
\label{tab-chapter-HTT_analysis-section-bg_estimation-procs_contribs}
\end{wraptable}
Ils peuvent également produire des objets physiques pouvant être interprétés comme des produits de désintégration de leptons~\tau.
Ces processus, résumés dans le tableau~\ref{tab-chapter-HTT_analysis-section-bg_estimation-procs_contribs} avec les pourcentages de leurs contributions au bruit de fond total, sont:
\begin{description}
\item[$\bm{\Zboson\to\tau\tau}$, $\bm{\Zboson\to\ell\ell}$] La désintégration du boson \Zboson\ en paire de leptons~\tau\ ($\Zboson\to\tau\tau$),
ainsi qu'en paire de muons ou d'électrons ($\Zboson\to\ell\ell$) lorsque l'un de ces leptons est mal identifié (les canaux \mu\mu\ et \ele\ele\ n'étant pas exploités).
La production du \Zboson\ peut se faire par annihilation d'une paire de quarks, comme illustré sur la figure~\ref{subfig-chapter-HTT_analysis-section-bg_estimation-procs-DY}.
Il s'agit des processus \og Drell-Yan \fg.
Le \Zboson\ peut également être produit par fusion de bosons électrofaibles (EWK, \emph{ElectroWeaK}).
Dans ce cas, deux jets supplémentaires sont présents dans l'état final.
\item[$\bm{\Wjets}$] La production d'un boson \Wboson, en particulier dans les canaux semi-leptoniques, où le muon ou l'électron issu de la désintégration du \Wboson\ est associé à un jet identifié à tort comme un \tauh.
Ce processus est illustré figure~\ref{subfig-chapter-HTT_analysis-section-bg_estimation-procs-WJ}.
Le \Wboson\ peut être produit par annihilation d'une paire de quarks, comme sur la figure~\ref{subfig-chapter-HTT_analysis-section-bg_estimation-procs-WJ}, ou par fusion de bosons électrofaibles (EWK).
\item[$\bm{\ttbar}$] La production d'une paire de quarks~\quarkt, en particulier pour les événements contenant des jets issus de quarks~\quarkb.
Ce cas est illustré figure~\ref{subfig-chapter-HTT_analysis-section-bg_estimation-procs-ttbar}.
Les désintégrations par interaction faible des quarks~\quarkt\ forment des bosons \Wboson, comme lors des désintégrations des leptons~\tau, d'où la contribution au bruit de fond de ces processus \ttbar.
\item[Diboson] Les productions de paires de bosons vecteurs ainsi que de quark~\quarkt\ seul (\emph{Single top}) contribuent également au bruit de fond, en particulier dans le canal \ele\mu.
\item[QCD] Enfin, les événements contenant des jets produits par interaction forte (QCD), lorsque ces jets sont identifiés à tort comme des éléments de désintégration d'une paire de leptons~\tau, forment la dernière source de bruit de fond considérée.
Cette source de bruit de fond est particulièrement importante dans le canal \tauh\tauh.
\end{description}
Les contenus exacts en processus physiques de ces six sortes de bruit de fond sont détaillés dans l'annexe~\refApHTTdatasets.
Plusieurs techniques sont utilisées afin de modéliser leurs contributions.
\par
De plus, la désintégration du boson de Higgs du \SM\ en paire de bosons \Wboson, $\higgs\to\Wbosonplus\Wbosonminus$, constitue également un bruit de fond vis-à-vis de l'analyse $\higgs\to\tau\tau$.
En effet, les leptons~\tau\ se désintègrent par interaction faible en produisant un neutrino, invisible dans le détecteur, et un \Wboson\ virtuel.
La désintégration d'un~\tau\ forme ainsi un état final très similaire à celle d'un \Wboson.
\par
Pour tous les processus à part QCD, des jeux de données simulées par générateur Monte-Carlo sont disponibles.
Toutefois, une large partie des bruits de fond est estimée à partir des données réelles, ce qui permet d'améliorer l'accord entre données réelles et estimation du bruit de fond tout en réduisant les incertitudes systématiques.
Tous les événements simulés contenant deux authentiques (\emph{genuine}) leptons~\tau\ sont ainsi remplacés par les données encapsulées (\emph{embedded}) présentées dans la section~\ref{chapter-HTT_analysis-section-bg_estimation-embedding}.
Les événements $\Zboson\to\tau\tau$ sont ainsi couverts par cette méthode mais également une partie des bruits de fond \ttbar\ et Diboson.
De plus, la contribution du bruit de fond QCD dans le canal \ele\mu\ est estimée à partir d'une région de contrôle où les charges électriques de l'électron et du muon sont de même signe.
Cette méthode est dénommée \og QCD SS \fg{} (\emph{Same Sign}) et est exposée dans la section~\ref{chapter-HTT_analysis-section-bg_estimation-QCD-SS}.
Enfin, les événements contenant au moins un jet identifié à tort comme provenant d'un lepton~\tau\ sont estimés par la méthode des facteurs de faux (\emph{Fake Factors}) décrite section~\ref{chapter-HTT_analysis-section-bg_estimation-FF_method}.
Tous les autres bruits de fond sont modélisés par des données simulées.
Les jeux de données ainsi utilisés dans l'analyse sont donnés dans l'annexe~\refApHTTdatasets.
\par
Afin de séparer les contributions estimées à partir des différentes techniques et de procéder à ces remplacements de manière cohérente, les événements simulés sont répartis selon la provenance des produits de désintégration visibles des leptons~\tau\ au niveau générateur.
Pour cela, un \emph{generator matching} est appliqué.
Les particules reconstruites sélectionnées (électrons, muons et taus hadroniques) sont associées à l'objet physique généré le plus proche dans le plan $(\eta, \phi)$ et à moins de $\Delta R = \num{0.2}$.
Si aucun objet généré ne respecte cette condition, l'objet reconstruit est considéré comme provenant d'un jet.
Il est ainsi possible de déterminer la provenance de l'objet reconstruit en connaissant la provenance de l'objet généré correspondant.
Il existe six cas de figure différents:
\begin{itemize}
\item électron natif (\emph{prompt electron}), \ie\ un électron ne provenant pas de la désintégration d'un lepton~\tau;
\item muon natif (\emph{prompt muon}), \ie\ un muon ne provenant pas de la désintégration d'un lepton~\tau;
\item électron provenant de la désintégration d'un lepton~\tau;
\item muon provenant de la désintégration d'un lepton~\tau;
\item tau hadronique;
\item jet ou particule issue de l'empilement.
\end{itemize}
Les définitions exactes de chacun de ces cas de figure sont données dans le tableau~\ref{tab-chapter-HTT_analysis-gen_match_values}.
Un \tauh\ généré est reconstruit à partir des produits de désintégration générés visibles hors électrons et muons.
Seuls les produits de désintégration du lepton~\tau\ généré tels que \inlinecode{python}{IsPrompt == True} sont considérés.
Il est de plus requis que l'impulsion transverse de ce \tauh\ généré reconstruit soit supérieure à \SI{15}{\GeV} afin d'éviter la limite de reconstruction des \tauh\ et d'éliminer des faux électrons et muons issus des \tauh.
Dans le cas des électrons et muons natifs, la coupure $\pT > \SI{8}{\GeV}$ permet de supprimer les leptons issus du FSR $\photon\to\ell^+\ell^-$.
Le FSR est introduit au chapitre~\refChJERC.
Les remplacements des événements simulés se font ainsi sur la base des valeurs de \inlinecode{python}{gen_match}, donnés dans le tableau~\ref{tab-chapter-HTT_analysis-gen_match_values}, pour $L_1$ et $L_2$ selon les coupures données dans le tableau~\ref{tab-chapter-HTT_analysis-gen_match_cuts}.
\begin{table}[h]
\centering
\begin{tabular}{ccl}
\toprule
\inlinecode{python}{gen_match} & Type de particule & Propriétés de l'objet au niveau générateur\\
\midrule
1 & électron natif & $\abs{\text{pdgID}} = 11$, $\pT > \SI{8}{\GeV}$, \inlinecode{python}{IsPrompt == True} \\
2 & muon natif & $\abs{\text{pdgID}} = 13$, $\pT > \SI{8}{\GeV}$, \inlinecode{python}{IsPrompt == True} \\
3 & $\tau\to\ele$  & $\abs{\text{pdgID}} = 11$, $\pT > \SI{8}{\GeV}$, \\
  & &  \inlinecode{python}{IsDirectPromptTauDecayProduct == True} \\
4 & $\tau\to\mu$  & $\abs{\text{pdgID}} = 13$, $\pT > \SI{8}{\GeV}$, \\
  & & \inlinecode{python}{IsDirectPromptTauDecayProduct == True} \\
5 & $\tau\to\tauh$ & Tau hadronique généré\\
6 & Faux \tauh, \tauh\ de l'empilement & Tout objet ne rentrant pas dans les catégories 1 à 5\\
\bottomrule
\end{tabular}
\caption[Valeurs prises par {\rm\texttt{gen\_match}}.]{Valeurs prises par \inlinecode{python}{gen_match}.}
\label{tab-chapter-HTT_analysis-gen_match_values}
\end{table}
\begin{table}[h]
\centering
\begin{tabular}{cccl}
\toprule
Canal & \inlinecode{python}{gen_match} $L_1$ & \inlinecode{python}{gen_match} $L_2$ & Simulations remplacées par la méthode \\
\midrule
\tauh\tauh & 5 & 5 & Données encapsulées \\
\tauh\tauh & ? & 6 & Facteurs de faux \\
\tauh\tauh & 6 & ? & Facteurs de faux \\
\mu\tauh & 4 & 5 & Données encapsulées \\
\mu\tauh & ? & 6 & Facteurs de faux \\
\ele\tauh & 3 & 5 & Données encapsulées \\
\ele\tauh & ? & 6 & Facteurs de faux \\
\ele\mu & 3 & 4 & Données encapsulées \\
\bottomrule
\end{tabular}
\caption[Remplacement des événements simulés par une estimation basée sur les données.]{Remplacement des événements simulés par une estimation basée sur les données. Un \og ? \fg{} signifie \og toute valeur possible \fg.}
\label{tab-chapter-HTT_analysis-gen_match_cuts}
\end{table}
\subsection{Méthode des données encapsulées ou \emph{embedding}}\label{chapter-HTT_analysis-section-bg_estimation-embedding}
La méthode des données encapsulées (\emph{embedding}) permet d'estimer le bruit de fond issu du modèle standard donnant une paire de leptons tau dans l'état final en minimisant l'utilisation de simulations.
La technique, présentée en détails dans la référence~\cite{embedding}, se déroule en quatre étapes, résumées sur la figure~\ref{fig-embedding_recap} et listées ci-après:
\begin{enumerate}
\item Sélection d'une paire de muons:\\
Dans les données réelles, des paires de muons sont formées.
La paire de masse invariante la plus proche de celle du boson \Zboson\ est choisie pour la suite.
Il existe ainsi des contributions issues des processus $\Zboson\to\mu\mu$, \ttbar\ et Diboson.
\item Suppression de la paire de muons:\\
Les signaux dans le détecteur correspondant aux muons sont retirés.
Les autres signaux sont conservés pour la reconstruction de l'événement.
\item Génération d'une paire de taus:\\
Deux leptons tau sont générés.
Les propriétés cinématiques des muons initiaux sont utilisées afin d'obtenir celles des leptons tau.
Leurs valeurs exactes sont modifiées afin de rendre compte de la différence de masse entre les muons et les taus.
Plus de détails sont disponibles dans la section 5.3 de la référence~\cite{embedding}.
Les désintégrations respectives des taus en électron, muon ou tau hadronique et leurs propagations dans le détecteur sont simulées.
\item Assemblage des données sans la paire de muons et des taus générés:\\
Les traces et dépôts d'énergie des objets simulés à l'étape précédente sont ajoutés à ceux de l'événement réel, auquel les signaux associés à la paire de muons initiaux ont été retirés.
La reconstruction des particules des événements présentée au chapitre~\refChLHCCMS\ ainsi que des objets de haut niveau introduite chapitre~\refChJERC\ peut alors être réalisée.
\end{enumerate}
\begin{figure}[h]
\centering
\def\EmbedPictsWidth{3.75}
\def\EmbedPictsMarginX{1.25}
\def\EmbedPictsMarginY{.5}
\def\EmbedPictsTxtSize{\small}
\begin{tikzpicture}
\draw (\EmbedPictsWidth/2, \EmbedPictsWidth) node [above] {\EmbedPictsTxtSize Données réelles $\Zboson\to\mu\mu$};
\node[anchor=south west,inner sep=0] at (0,0) {\frame{\includegraphics[width=\EmbedPictsWidth cm]{\PhDthesisdir/plots_and_images/from_embedding/Z_to_mumu_data.png}}};

\draw [-latex, very thick] (\EmbedPictsWidth+\EmbedPictsMarginX/5, \EmbedPictsWidth/2) -- + (3*\EmbedPictsMarginX/5,0);
\draw (\EmbedPictsWidth/2+\EmbedPictsWidth+\EmbedPictsMarginX, \EmbedPictsWidth) node [above] {\EmbedPictsTxtSize Supression de la paire $\mu\mu$};
\node[anchor=south west,inner sep=0] at (\EmbedPictsWidth+\EmbedPictsMarginX,0) {\frame{\includegraphics[width=\EmbedPictsWidth cm]{\PhDthesisdir/plots_and_images/from_embedding/remove_mumu.png}}};

\draw (\EmbedPictsWidth+\EmbedPictsMarginX,-\EmbedPictsWidth/2-\EmbedPictsMarginY) node [above left] {\EmbedPictsTxtSize Simulation des \tau};
\draw (\EmbedPictsWidth+\EmbedPictsMarginX,-\EmbedPictsWidth/2-\EmbedPictsMarginY) node [below left] {\EmbedPictsTxtSize ($\vec{p}_{\tau} \Leftrightarrow \vec{p}_{\mu}$)};
\node[anchor=south west,inner sep=0] at (\EmbedPictsWidth+\EmbedPictsMarginX,-\EmbedPictsWidth-\EmbedPictsMarginY) {\frame{\includegraphics[width=\EmbedPictsWidth cm]{\PhDthesisdir/plots_and_images/from_embedding/Z_to_tautau_simulation.png}}};

\draw [-latex, very thick] (2*\EmbedPictsWidth+6*\EmbedPictsMarginX/5, \EmbedPictsWidth/4) -- (2*\EmbedPictsWidth+2*\EmbedPictsMarginX-\EmbedPictsMarginX/5,0);
\draw [-latex, very thick] (2*\EmbedPictsWidth+6*\EmbedPictsMarginX/5,-\EmbedPictsWidth/4-\EmbedPictsMarginY) -- (2*\EmbedPictsWidth+2*\EmbedPictsMarginX-\EmbedPictsMarginX/5,-\EmbedPictsMarginY);
\draw (2*\EmbedPictsWidth+2*\EmbedPictsMarginX+\EmbedPictsWidth/2,\EmbedPictsWidth/2-\EmbedPictsMarginY/2) node [above] {\EmbedPictsTxtSize Données encapsulées};
\node[anchor=south west,inner sep=0] at (2*\EmbedPictsWidth+2*\EmbedPictsMarginX,-\EmbedPictsWidth/2-\EmbedPictsMarginY/2) {\frame{\includegraphics[width=\EmbedPictsWidth cm]{\PhDthesisdir/plots_and_images/from_embedding/embedded_event.png}}};

\clip (-1,-\EmbedPictsWidth-\EmbedPictsMarginY-.25) rectangle (3*\EmbedPictsWidth+5*\EmbedPictsMarginX/2+.25, \EmbedPictsWidth+.5);
\end{tikzpicture}
\caption[Schéma récapitulatif de la méthode des données encapsulées.]{Schéma récapitulatif de la méthode des données encapsulées~\cite{embedding}, illustrée dans le cas de l'état final \mu\tauh.}
\label{fig-embedding_recap}
\end{figure}
\par
Les données encapsulées nécessitent ainsi l'utilisation de simulation uniquement pour la paire de leptons taus et leurs désintégrations.
Tous les autres objets présents sont issus de données réelles.
L'empilement, l'événement sous-jacent et les jets de l'événement principal sont donc décrits de manière parfaitement identique à la réalité, dans la mesure où ils ne sont pas simulés.
De plus, l'incertitude sur la luminosité est supprimée pour les données encapsulées, car leur quantité est directement reliée à celle des données réelles, ce qui n'est pas le cas pour les données entièrement simulées.
Enfin, les effets dus au détecteur tels que le bruit inhérent à la mesure, les pièces défectueuses et son vieillissement sont naturellement inclus dans les données encapsulées.
\par
L'amélioration de la description des données ainsi obtenue grâce à l'encapsulement est visible sur la figure~\ref{fig-embedding_2018mt_puppimet_illustration}, où les distributions de l'énergie transverse manquante dans les données et dans l'estimation du bruit de fond sans et avec cette méthode sont tracées à titre d'illustration.
L'accord est sensiblement amélioré pour $\MET < \SI{30}{\GeV}$.
\begin{figure}[h]
\centering

\subcaptionbox{Sans l'encapsulement.}[.475\textwidth]
{\plotHTTcontrol{2018}{fully_classic}{mt}{puppimet}}
\hfill
\subcaptionbox{Avec l'encapsulement.}[.475\textwidth]
{\plotHTTcontrol{2018}{emb_classic}{mt}{puppimet}}

\caption{Distributions de \MET\ pour le canal \mu\tauh\ en 2018.}
\label{fig-embedding_2018mt_puppimet_illustration}
\end{figure}

\subsection{Estimation du bruit de fond QCD dans le canal \ele\mu}\label{chapter-HTT_analysis-section-bg_estimation-QCD-SS}


\subsection{Méthode du facteur de faux ou \emph{fake factor}}\label{chapter-HTT_analysis-section-bg_estimation-FF_method}


\begin{figure}
\centering

\subcaptionbox{bla}[.45\textwidth]
{\plotHTTcontrol{2018}{fully_classic}{et}{mt_1_puppi}}
\hfill
\subcaptionbox{bla}[.45\textwidth]
{\plotHTTcontrol{2018}{classic_ff}{et}{mt_1_puppi}}

\caption{bla}
\end{figure}