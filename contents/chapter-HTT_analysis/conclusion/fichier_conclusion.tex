\section{Conclusion}\label{chapter-HTT_analysis-section-conclusion}
La recherche de bosons de Higgs supplémentaires de haute masse se désintégrant en paire de taus
%dans l'expérience CMS au LHC
a été présentée.
Ces particules sont les deux bosons neutres \Higgs\ et \HiggsA\ prédits par le MSSM.
Les données analysées sont celles récoltées par CMS
lors des collisions de protons avec une énergie dans le centre de masse de $\sqrt{s}=\SI{13}{\TeV}$ du Run~II du LHC,
correspondant à une luminosité intégrée de \SI{137}{\femto\barn^{-1}}.
Parmi les six canaux ou états finaux possibles pour la paire de leptons~\tau,
quatre ont été exploités (\tauh\tauh, \mu\tauh, \ele\tauh, \ele\mu).
\par
Afin d'interpréter les données observées,
une modélisation des processus physiques attendus tels que les bruits de fond
est nécessaire.
La méthode des données encapsulées permet de limiter l'utilisation de simulations afin de modéliser
certains processus contenant une véritable paire de leptons~\tau,
en particulier $\Zboson\to\tau\tau$.
L'estimation des \ftauhs, jets identifiés à tort comme étant des \tauh,
est également basée presque exclusivement sur les données réelles.
Ces méthodes permettent de réduire les incertitudes inhérentes à la simulation.
\par
Le signal correspondant aux bosons supplémentaires est modélisé par les processus
de fusion de gluons ($\gluon\gluon\to\Phi\to\tau\tau$)
et
en association avec des quarks~\quarkb\ ($\gluon\gluon\to\quarkb\antiquarkb\Phi\to\tau\tau$).
Afin de maximiser la sensibilité de l'analyse à ces deux modes de production,
une catégorisation des événements basée sur le nombre de jets issus de quarks~\quarkb\
est utilisée.
Une catégorisation supplémentaire, basée sur la masse transverse du lepton $\ell$ dans les canaux $\ell\tauh$ (\mu\tauh, \ele\tauh) et sur \Dzeta\ dans le canal \ele\mu, permet d'augmenter encore cette sensibilité.
Des limites d'exclusion sur le produit de la section efficace de production des bosons supplémentaires avec leur rapport de branchement à la désintégration en paire de leptons~\tau\ ont été données pour les deux modes de production étudiés pour des masses comprises entre \SI{110}{\GeV} et \SI{3.2}{\TeV}.
\par
Le scénario $M_{\higgs}^{125}$
fixe les valeurs de certains paramètres du MSSM.
Des limites d'exclusion de ce modèle
en faveur du \SM\
ont été données
dans le plan
$(m_{\HiggsA},\tan\beta)$.
Les valeurs de $m_{\HiggsA}$ inférieures à \SI{600}{\GeV} sont exclues
et cette valeur augmente à \SI{2}{\TeV} lorsque $\tan\beta\gtrsim\num{50}$.
Un second scénario, $M_{\Higgs_1}^{125}(\text{CPV})$,
a également été étudié.
Dans celui-ci, le choix des valeurs des paramètres du MSSM
autorise la violation de $CP$ par les bosons de Higgs
car les états propres de $CP$ ne sont plus les mêmes que ceux de masse.
Ces derniers, pour les bosons neutres, sont
$\Higgs_1$, $\Higgs_2$ et $\Higgs_3$.
En particulier, \HiggsA\ n'est plus un état propre de masse et c'est donc
dans le plan $(m_{\Higgspm},\tan\beta)$ que sont données les limites d'exclusion en faveur du \SM.
Pour $\tan\beta\simeq\num{9}$ et $m_{\Higgspm}\simeq\SI{700}{\GeV}$,
les interférences entre $\Higgs_2$ et $\Higgs_3$ réduisent fortement la sensibilité de l'analyse,
limitant alors l'étendue de la région d'exclusion du MSSM.
L'analyse menée permet toutefois d'exclure $m_{\Higgspm} < \SI{400}{\GeV}$
et cette valeur augmente à \SI{1.4}{\TeV} lorsque $\tan\beta\simeq\num{20}$.
\par
Cependant,
les résultats présentés dans ce manuscrit sont donnés à titre d'illustration de l'état actuel de l'analyse.
L'ajustement des paramètres de nuisance ainsi que les résultats obtenus nécessitent de plus amples investigations.
\par
À ce jour,
aucun des modèles BSM proposés ne permet de mieux décrire les résultats expérimentaux que le \SM.
Augmenter l'échelle d'énergie des collisions ou la luminosité intégrée exploitable permettra d'aller plus loin dans l'analyse des processus physiques.
Les modèles proposés par les théoriciens sont alors mieux contraints et sont soit rejetés, soit affinés.
L'énergie dans le centre de masse des collisions de protons au LHC doit passer à \SI{14}{\TeV} lors du Run~III, dont le début est prévu en 2022, avec une luminosité intégrée de l'ordre de \SI{300}{\femto\barn^{-1}}.
Puis, de 2027 à 2035, le LHC à haute luminosité (HL-LHC) générera près de \SI{3000}{\femto\barn^{-1}} de données.
Cette augmentation considérable de la statistique
permettra d'améliorer les mesures de précision des paramètres du \SM\
ainsi que les analyses telles que celle présentée dans ce chapitre.
Cependant, le tracé du LHC ne permet pas, avec la technologie actuelle, d'espérer augmenter encore l'énergie de collision.
Le projet FCC (Future Collisionneur Circulaire) \cite{FCC} pourrait proposer une énergie de collision de \SI{100}{\TeV} à l'horizon 2040 avec des protons.
\par
La présence de neutrinos dans l'état final des événements exploités dans cette analyse
rend difficile l'estimation de la masse invariante d'une résonance se désintégrant en deux~\tau.
Cet effet, inhérent à cette analyse, limite sa sensibilité.
Le \emph{Machine Learning} peut permettre une estimation plus précise de la masse d'une résonance se désintégrant en paire de~\tau.
Ce sujet est présenté dans le chapitre~\refChML.