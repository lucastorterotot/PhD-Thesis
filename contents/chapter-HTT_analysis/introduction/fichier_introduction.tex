\section{Introduction}\label{chapter-HTT_analysis-section-introduction}
Dans le chapitre~\refChMSSM,
il a été montré que le modèle standard (SM, \emph{Standard Model}) souffre de lacunes quant à l'explication à apporter à certaines observations.
Des modèles allant au-delà (BSM, \emph{Beyond Standard Model}),
comme l'extension supersymétrique minimale du modèle standard ou \og MSSM \fg,
peuvent combler certaines d'entre elles.
Le MSSM prédit l'existence de cinq bosons de Higgs, dont trois neutres, \higgs, \Higgs\ et \HiggsA.
L'un d'entre-eux doit correspondre au boson découvert en 2012 et interprété comme étant le boson de Higgs du \SM~\cite{ATLAS_Higgs_discovery,CMS_Higgs_discovery,CMS_Higgs_discovery_2013,ATLAS-CMS-Higgs_combined_1,ATLAS-CMS-Higgs_combined_2}.
L'existence des deux bosons de Higgs neutres supplémentaires peut être testée expérimentalement avec des accélérateurs de particules,
comme cela a été fait au LEP~\cite{Schael:2006cr}.
Ces bosons se désintègrent préférentiellement en paire de quarks~\quarkb\ ou de leptons~\tau.
Bien que le rapport de branchement (\BR) de ces bosons aux \quarkb\ soit 5 à 10 fois supérieur que celui aux~\tau,
ces derniers proposent une meilleure accessibilité expérimentale dans les collisionneurs hadroniques comme le Tevatron,
où ces désintégrations en~\tau\ ont été étudiées~\cite{Aaltonen:2009vf,Abazov:2011jh}.
\par
L'expérience CMS installée au LHC et présentée dans le chapitre~\refChLHCCMS\ permet elle aussi de tester expérimentalement le MSSM, dans des conditions de collision inédites.
La recherche de bosons de Higgs supplémentaires se désintégrant en paire de leptons~\tau\ a été menée dans les collisions de protons avec une énergie dans le centre de masse de $\sqrt{s}=\num{7}$ et $\SI{8}{\TeV}$ (Run~I) \cite{Chatrchyan:2012vp,CMS-MSSM-HTT_2014,CMS-PAS-HIG-13-021,CMS-PAS-HIG-14-029} ainsi qu'avec les données récoltées en 2016 avec une énergie de $\sqrt{s}=\SI{13}{\TeV}$ \cite{CMS-PAS-HIG-17-020}.
Plusieurs thèses portent sur l'analyse des événements où un boson de Higgs se désintègre en paire de~\tau~\cite{Gael_thesis,Artur_thesis}.
La désintégration en paire de \quarkb\ est également exploitée~\cite{Chatrchyan:2013qga,Khachatryan:2015tra},
ainsi que celle en paire de muons~\cite{CMS:2015ooa}.
L'expérience ATLAS mène des recherches similaires~\cite{Aad:2012cfr,ATLAS-MSSM-HTT_2018,ATLAS-MSSM-HTT_2020}.
\par
Ce chapitre présente la
recherche de bosons de Higgs supplémentaires de haute masse se désintégrant en paire de~\tau\
avec les données récoltées par l'expérience CMS
lors du Run~II du LHC (années 2016, 2017 et 2018),
correspondant à une luminosité intégrée de \SI{137}{\femto\barn^{-1}} ($\num{35.9}+\num{41.5}+\SI{59.7}{\femto\barn^{-1}}$)
à une énergie dans le centre de masse de $\sqrt{s}=\SI{13}{\TeV}$.
Sur les six canaux de désintégration de la paire de leptons~\tau\ introduits dans le chapitre~\refChMSSM,
les quatre présentant les plus grands \BR\ sont considérés dans l'analyse.
Il s'agit des canaux
hadronique (\tauh\tauh),
semi-leptoniques (\mu\tauh, \ele\tauh)
et
leptonique asymétrique (\ele\mu).
Les canaux leptoniques symétriques (\mu\mu, \ele\ele) ne sont pas exploités.
\par
Dans les données réelles, les particules doivent forcément être reconstruites à partir des signaux qu'elles produisent dans le détecteur.
Dans le cas des données simulées, la réponse du détecteur aux particules est modélisée.
À partir des signaux réels comme simulés, les particules individuelles et les objets physiques de haut niveau sont reconstruits comme exposé dans le chapitre~\refChLHCCMS.
Les simulations n'étant pas exemptées de défauts, des corrections déterminées à l'aide d'analyses annexes leurs sont appliquées.
Les corrections génériques sont présentées au chapitre~\refChLHCCMS,
à l'exception de la calibration en énergie des jets détaillée dans le chapitre~\refChJERC.
Les corrections spécifiques à la présente analyse sont introduites dans la section~\ref{chapter-HTT_analysis-section-corrections}.
Les objets reconstruits et corrigés permettent de sélectionner les événements d'intérêt pour l'analyse selon la procédure explicitée en section~\ref{chapter-HTT_analysis-section-selection}.
Des processus physiques différents de ceux du signal recherché passent cette sélection et constituent le bruit de fond.
Afin d'interpréter les observations, il est nécessaire de modéliser ce bruit de fond.
Cette modélisation est présentée section~\ref{chapter-HTT_analysis-section-bg_estimation}.
En plus de l'utilisation de données simulées, des techniques basées sur les données réelles sont exploitées.
Des données dites \og encapsulées \fg{} (\emph{embedded}) sont ainsi produites selon la procédure exposée section~\ref{chapter-HTT_analysis-section-bg_estimation-embedding} et décrivent les événements contenant une vraie paire de leptons~\tau.
Une estimation du bruit de fond dû aux jets identifiés à tort comme des taus hadroniques (\ftauhs) est quant à elle obtenue grâce à la méthode des facteurs de faux (\emph{fake factors}) introduite section~\ref{chapter-HTT_analysis-section-bg_estimation-FF_method}.
Les événements sont par la suite catégorisés afin d'augmenter la sensibilité de l'analyse.
Les catégories utilisées sont présentées en section~\ref{chapter-HTT_analysis-section-categorisation}.
Les sources d'incertitudes systématiques sont données section~\ref{chapter-HTT_analysis-section-systematics}.
Leur prise en compte dans l'extraction du signal ainsi que la modélisation de celui-ci sont exposées dans la section~\ref{chapter-HTT_analysis-section-signal_extraction}.
Enfin, les résultats obtenus sont disponibles section~\ref{chapter-HTT_analysis-section-results}.
Certains sont indépendants de tout modèle, d'autres sont obtenus dans le cadre de scénarios spécifiques du MSSM~\cite{Bagnaschi_2019}.
\par
Une note d'analyse \cite{CMS-NOTE-2020-218} est déjà disponible pour les membres de la collaboration et un article est en préparation~\cite{HIG-21-001}.
Ces travaux sont réalisés au sein d'une équipe regroupant:
\begin{itemize}
\item l'Institut de Physique des 2 Infinis (IP2I) de l'Université Claude Bernard de Lyon, mon laboratoire de rattachement;
\item l'\emph{Institut für Experimentelle Teilchenphysik} (ETP) du \emph{Karlsruher Institut für Technologie} (KIT) de Karlsruhe;
\item le \emph{Deutsches Elektronen-Synchrotron} (DESY) de Hambourg;
\item l'\emph{Imperial College} de Londres;
\item l'\emph{Institut für Hochenergiephysik} (HEPHY)
% de l'\emph{Österreichische Akademie der Wissenschaften}
 de Vienne;
\item le \emph{Tata Institute of Fundamental Research} de Bombay.
\end{itemize}
\par
En début de thèse, j'ai travaillé sur les données de l'année 2017 en équipe avec Gaël \textsc{Touquet} qui a exploité le canal \tauh\tauh\ dans sa thèse~\cite{Gael_thesis}.
Je me suis concentré sur les  canaux semi-leptoniques et plus particulièrement le canal \mu\tauh\ \cite{jrjc2019_torterotot}.
La présence de \tauh\ dans nos canaux respectifs nous a mené à de travailler en étroite collaboration.
Les événements étaient analysés à l'aide d'un code basé sur \HEPPY~\cite{heppy},
indépendant de celui utilisé par les autres instituts listés précédemment,
ce qui a permis à l'ensemble des acteurs de cette analyse de valider la bonne implémentation des différentes corrections et sélections détaillées dans ce chapitre.
À cette occasion, j'ai découvert une erreur dans le code de \COMBINE, l'outil de combination statistique utilisé par la collaboration CMS et basé sur \ROOSTATS~\cite{RooStats}.
Cette erreur a été comprise et corrigée comme présenté dans la section~\ref{chapter-HTT_analysis-section-signal_extraction-likelihood-stat_uncs}.
Le correctif~\cite{BBB_PR} a été transmis à la collaboration CMS qui l'a pris en compte.
\par
J'ai par la suite travaillé directement avec le groupe de Karlsruhe dans le cadre de l'analyse du Run~II.
J'ai implémenté le traitement du scénario avec violation de la symétrie $CP$
et
participé au traitement des jeux de données utilisés, listés dans l'annexe~\refApHTTdatasets.
Il s'agissait de s'assurer du bon déroulement de plusieurs milliers de tâches informatiques et du regroupement de leurs résultats.
Enfin, j'ai activement participé à la rédaction
de la note d'analyse CMS~\cite{CMS-NOTE-2020-218}
et
de la publication~\cite{HIG-21-001}
correspondantes.
J'ai ainsi apporté une contribution significative à ces travaux.
%Homework concerning the bbH and ggH samples
%1) Monitor the production ---> ask in case there are invalid samples
%2) Process the new samples
%3) Rederive the ggH weights based on input POWHEG samples
%4) Update the signal modelling for new samples in CombineHarvester
%And on the experimental side I recall:
%newes FF's
%MET tail correction and uncertainty
%Are these the only items left before wrapping up or am I missing anything in addition?