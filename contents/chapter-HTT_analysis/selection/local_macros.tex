\newcommand{\AllSatisfyingFollowing}[2]{Tout #1{} respectant les critères listés ci-après est retenu pour jouer le rôle de #2 dans le \emph{dilepton}}

\newcommand{\TauHdz}{$d_z < \SI{0.2}{\centi\meter}$ avec $d_z$ la distance entre la trace principale du \tauh\ et le vertex primaire d'interaction}
\newcommand{\Leptondzdxy}{paramètres d'impact $d_z < \SI{0.2}{\centi\meter}$ et $d_{xy} < \SI{0.045}{\centi\meter}$}

\newcommand{\RelIsoBelow}[2]{$I_{#1} < \num{#2} \, \pT^{#1}$}

\newcommand{\MuonIDWP}[2]{point de fonctionnement #1 (\emph{#2}) du \muonID}
\newcommand{\MediumMuonID}{\MuonIDWP{moyen}{medium}}
\newcommand{\LooseMuonID}{\MuonIDWP{relâché}{loose}}

\newcommand{\NinetyNineEleMVA}{point de fonctionnement à \SI{90}{\%} d'efficacité de l'\EleIDMVA}
\newcommand{\NinetyNineEleMVAnoIso}{\NinetyNineEleMVA\ sans utilisation des variables d'isolation}

\newcommand{\NewDecayModeFinding}[1][]{passer le discriminateur \texttt{NewDecayModeFinding} (modes de désintégration 5, 6, et 7 interdits)}
\newcommand{\PassDeepTau}[3]{#1 (\emph{#2}) du discriminateur \texttt{deepTau #3}}

\newcommand{\AtLeastOneOSPair}[1]{L'événement est retenu à condition qu'au moins une paire $L_1L_2=#1$ puisse être construite avec $L_1$ et $L_2$ de charges électriques opposées.}
\newcommand{\DeltaRPair}[1]{Il est de plus requis que $L_1$ et $L_2$ soient séparés dans le plan $(\eta,\phi)$ tel que $\Delta R > \num{#1}$.}
\newcommand{\IfMoreOnePair}{Si plus d'une paire possible existe dans l'événement, une seule est retenue selon la logique exposée dans la section~\ref{chapter-HTT_analysis-section-offline-dilepton}.}

\newcommand{\FromPairMatchToHLTObjects}{de la paire sélectionnée doivent correspondre, le cas échéant, aux objets ayant activé un des \HLTpaths\ utilisé pour enregistrer l'événement.
Les objets considérés pour chacun des \HLTpaths\ sont donnés dans l'annexe~\refApHTTtrg.
La correspondance est établie lorsque la particule reconstruite et l'objet du \HLTpath\ sont séparés de moins de \num{0.5} dans le plan $(\eta,\phi)$, \ie\ $\Delta R < \num{0.5}$.
Ce critère est appliqué de manière cohérente dans les données réelles, simulées et encapsulées.}
\newcommand{\HLTregionsDefined}{catégories sont définies pour les événements enregistrés en 2016 (respectivement 2017 et 2018)}

\newcommand{\TransverseMassWjetsInfosTXT}{Cette coupure permet de s'assurer que la région de signal soit orthogonale à la région de détermination des facteurs de faux des événements $\Wboson+\text{jets}$. Les facteurs de faux sont abordés dans la section~\ref{chapter-HTT_analysis-section-bg_estimation-FF_method}.}

\newcommand{\TransverseMassWjets}[4][]{La masse transverse #2, définie par
\begin{equation}
\mT^{(#3)} = \sqrt{2 \, \pT^{(#3)} \, \MET \, (1-\cos\Delta\phi)}
\end{equation}
avec $\Delta\phi = \phi^{(#3)} - \phi^{\MET}$
doit vérifier $\mT < \SI{70}{\GeV}$. \TransverseMassWjetsInfosTXT}

\newcommand{\DzetaEleMU}{

\begin{wrapfigure}[17]{R}{.45\textwidth}
\centering
\vspace{-2\baselineskip}
\begin{tikzpicture}
%% base
\draw [->] (0,0)--(2,0) node [right] {$\bvec_x$};
\draw [->] (0,0)--(0,2) node [above] {$\bvec_y$};

\def\xMET{-1.5}
\def\yMET{-1.25}
\def\xELE{2}
\def\yELE{3}
\def\xMU{.5}
\def\yMU{-2}

\draw [thick, -latex, ltcolorred] (0,0) -- (\xMET,\yMET) coordinate (vMET);
\draw [ltcolorred] (vMET) node [below] {\vMET};
\draw [thick, -latex, ltcolorblue] (0,0) -- (\xELE,\yELE) coordinate (vE);
\draw [ltcolorblue] (vE) node [right] {$\vpT (\ele)$};
\draw [thick, -latex, ltcolorblue] (0,0) -- (\xMU,\yMU) coordinate (vM);
\draw [ltcolorblue] (vM) node [right] {$\vpT (\mu)$};

\draw [dashed, -latex] ({-\xELE+-\xMU}, {-\yELE+-\yMU}) -- ({1.25*(\xELE+\xMU)}, {1.25*(\yELE+\yMU)}) node [above] {$\vec{\zeta}$};

\draw [thick, ltcolorgreen, -latex] (0,0) -- ({\xELE+\xMU}, {\yELE+\yMU}) coordinate (vVIS);
\draw [ltcolorgreen] (vVIS) node [below right] {$\vpTvis$};

\draw [thick, -latex, ltcolorred4] (0,0) --+ ({180+acos((\xELE+\xMU)/(((\xELE+\xMU)*(\xELE+\xMU)+(\yELE+\yMU)*(\yELE+\yMU))^(0.5)))}:{(\xMET*\xMET+\yMET*\yMET)^(0.5)*cos(180-acos((\xELE+\xMU)/(((\xELE+\xMU)*(\xELE+\xMU)+(\yELE+\yMU)*(\yELE+\yMU))^(0.5)))-acos((\xMET)/((\xMET*\xMET+\yMET*\yMET)^(0.5))))})  coordinate (vMETzeta) ;
\draw [ltcolorred4] (vMETzeta) node [above] {$p_\zeta^\text{miss}\hat{\zeta}$};

\draw [dotted] (vMET) -- (vMETzeta);
\draw [dotted] (vE) -- (vVIS);
\draw [dotted] (vM) -- (vVIS);

\end{tikzpicture}
\caption{Illustration de la définition de $\hat{\zeta}$~\cite{Jang_thesis}. Le plan de ce schéma est le plan transverse.}
\label{fig-zeta_illustration}
\end{wrapfigure}

La variable \Dzeta\ est définie selon
\begin{equation}
\Dzeta = p_\zeta^\text{miss} - \num{0.85} p_\zeta^{(\tau\tau)}
\label{eq-Dzeta_def}
\end{equation}
avec
\begin{equation}
p_\zeta^\text{miss} = \vMET \cdot \hat{\zeta}
\msep
p_\zeta^{(\tau\tau)} = \vpTvis \cdot \hat{\zeta}
\end{equation}
où $\hat{\zeta}$ est la direction bisectionnelle entre l'électron et le muon dans le plan transverse~\cite{Jang_thesis}
et
\begin{equation}
\vpTvis = \left( \vpT^{\ele} + \vpT^{\mu} \right)
\label{eq-pTvis_def}
\end{equation}
comme illustré sur la figure~\ref{fig-zeta_illustration}.
Il est requis que $\Dzeta \geq \num{-35}$ afin de s'assurer que la région de signal soit orthogonale à la région de contrôle (CR) du bruit de fond \ttbar.}

\newcommand{\LeptonVetoes}{Les vetos de leptons supplémentaires doivent être respectés, \ie\ que l'événement ne contient pas:}

\newcommand{\LeptonVetoesExtra}[6]{#1 tel que $\pT^{#2} > \SI{#3}{\GeV}$, $\abs{\eta^{#2}} < \num{#4}$, passant le #5 et d'isolation \RelIsoBelow{#2}{#6}}
\newcommand{\LeptonVetoesExtraMuon}{\LeptonVetoesExtra{de muon}{\mu}{10}{2.4}{\MediumMuonID}{0.3}}
\newcommand{\LeptonVetoesSecondMuon}{\LeptonVetoesExtra{de second muon}{\mu}{10}{2.4}{\MediumMuonID}{0.3}}
\newcommand{\LeptonVetoesExtraEle}{\LeptonVetoesExtra{d'électron}{\ele}{10}{2.5}{\NinetyNineEleMVA}{0.3}}
\newcommand{\LeptonVetoesSecondEle}{\LeptonVetoesExtra{de second électron}{\ele}{10}{2.5}{\NinetyNineEleMVA}{0.3}}

\newcommand{\LeptonVetoesPair}[7]{de paire #1 de charges opposées avec $\Delta R > \num{#2}$, tous deux vérifiant $\pT^{#3} > \SI{#4}{\GeV}$, $\abs{\eta^{#3}} < \num{#5}$, passant le #6, de paramètres d'impact $d_z < \SI{0.2}{\centi\meter}$ et $d_{xy} < \SI{0.045}{\centi\meter}$ et d'isolation \RelIsoBelow{#3}{#7}}
\newcommand{\LeptonVetoesMuonPair}{\LeptonVetoesPair{de muons}{0.15}{\mu}{15}{2.4}{\LooseMuonID}{0.3}}
\newcommand{\LeptonVetoesElectronPair}{\LeptonVetoesPair{d'électrons}{0.15}{\ele}{15}{2.5}{\CutBasedEleIDVeto}{0.3}}

\newcommand{\LessTwoMissingHitsVertex}{présenter moins de deux points de passage manquants dans le trajectographe}
\newcommand{\PassConversionVeto}{passer le veto d'électron de conversion}

%%%%%%

\renewcommand{\MuonIDWP}[2]{point de fonctionnement \emph{#2} du \muonID}
\renewcommand{\PassDeepTau}[3]{\emph{#2} du discriminateur \texttt{deepTau #3}}