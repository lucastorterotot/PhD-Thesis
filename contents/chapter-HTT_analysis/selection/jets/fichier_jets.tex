\subsection{Sélection des jets}\label{chapter-HTT_analysis-section-jets}
Les événements sont répartis en différentes catégories d'après les différents mécanismes de production des bosons de Higgs.
Cette catégorisation est détaillée dans la section~\ref{chapter-HTT_analysis-section-categorisation}.
À cette fin, les jets présents dans l'événement sont exploités.
\par
Pour réduire la contamination par les jets issus de l'empilement, les hadrons chargés sont soumis à la procédure de \emph{pile-up Charged Hadron Subtraction} (CHS)~\cite{CMS-PAS-JME-14-001} décrite dans le chapitre~\refChJERC.
Les jets utilisés sont ceux obtenus à partir des particules restantes à l'aide de l'algorithme anti-\kT~\cite{Cacciari_antikT} avec un paramètre $R=\num{0.4}$.
\par
Ces jets doivent également passer les critères d'identification présentés dans le chapitre~\refChLHCCMS.
L'identification des jets issus de quarks~\quarkb\ (\quarkb-\emph{tagging}) est réalisée par l'algorithme \DeepCSV~\cite{Sirunyan_heavy_flavor_jets_2018,DeepJet}.
Les jets tels que $\pT > \SI{20}{\GeV}$ et $\abs{\eta}<\num{2.4} (\num{2.5})$ en 2016 (2017, 2018) sont considérés comme issus d'un \quarkb\ si leur score est supérieur à \num{0.3093} en 2016, \num{0.3033} en 2017 et \num{0.2770} en 2018.
De plus, tout jet tel que $\pT > \SI{30}{\GeV}$ et $\abs{\eta}<\num{4.7}$ est retenu.
%\par
Afin d'exclure les électrons, muons et taus hadroniques de la liste des jets, il est requis que les jets soient distants du dilepton de $\Delta R > \num{0.5}$.
\par
Lors de la prise de données en 2017, le \CMSendcap\ du ECAL présentait un bruit important, perturbant la reconstruction des jets.
Conformément aux recommandations du \POG\ (\emph{Physics Object Group}, groupe responsable d'un objet physique) \emph{JetMET}, les jets reconstruits tels que $\num{2.65} < \abs{\eta} < \num{3.139}$ ayant une impulsion transverse avant correction inférieure à \SI{50}{\GeV} sont rejetés.
L'énergie transverse manquante est corrigée en conséquence.