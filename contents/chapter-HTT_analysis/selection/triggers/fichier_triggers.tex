\subsection{Chemins de déclenchement}\label{chapter-HTT_analysis-section-triggers}
Les chemins de déclenchement permettent une présélection en temps réel des événements observés à CMS.
Pour chacune des trois années (2016, 2017 et 2018) et chacun des états finaux considérés (\tauh\tauh, \mu\tauh, \ele\tauh\ et \ele\mu), une liste donnée de chemins de déclenchement est utilisée.
Ces listes sont données dans l'annexe~\ifref{annexe-triggers-HTT}{\ref{annexe-triggers-HTT}}{E}.
\par
Il est de plus requis que les particules reconstruites sélectionnées afin d'obtenir une paire de leptons taus, le \emph{dilepton}, correspondent aux particules ayant activé le chemin de déclenchement.
La correspondance est établie lorsque la particule reconstruite et l'objet ayant activé le chemin de déclenchement sont séparés de moins de \num{0.5} dans le plan $(\eta,\phi)$, \ie
\begin{equation}
\Delta R = \sqrt{(\Delta\eta)^2 + (\Delta\phi)^2} < \num{0.5}
\mend
\end{equation}