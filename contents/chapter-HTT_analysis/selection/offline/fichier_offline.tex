\subsection{Sélection finale}\label{chapter-HTT_analysis-section-offline_selection}
\begin{wraptable}{R}{.33\textwidth}
    \centering
    \begin{tabularx}{.9\linewidth}{cYY}
        \toprule
        Canal & $L_1$ & $L_2$ \\
        \midrule
        \tauh\tauh & \multicolumn{2}{c}{\tauh, $\pT^{(L_1)}>\pT^{(L_2)}$} \\
        \mu\tauh & \mu & \tauh \\
        \ele\tauh & \ele & \tauh \\
        \ele\mu & \mu & \ele \\
        \bottomrule
    \end{tabularx}
    \caption{Particules correspondant à $L_1$ et $L_2$ selon le canal.}
    \label{tab-chapter-HTT_analysis-section-offline_selection-L1L2}
\end{wraptable}
La phénoménologie des événements $\Higgs\to\tau\tau$ est décrite dans le chapitre~\ifref{chapter-MS-MSSM}{\ref{chapter-MS-MSSM}}{2}.
Les leptons tau peuvent se désintégrer hadroniquement en tau hadronique (\tauh) ou leptoniquement en électron (\ele) ou en muon (\mu), ces désintégrations s'accompagnent de l'émission de un (cas hadronique) ou deux (cas leptoniques) neutrinos.
Il existe ainsi six états finaux différents ou canaux pour ces événements dont seulement quatre sont considérés dans l'analyse:
le canal hadronique (\tauh\tauh),
les deux canaux semi-leptoniques (\mu\tauh, \ele\tauh)
et un canal leptonique (\ele\mu).
\par
Les produits de désintégration visibles des leptons tau sont notés $L_1$ et $L_2$ et correspondent, selon le canal, à un tau hadronique, un muon ou un électron comme exposé dans le tableau~\ref{tab-chapter-HTT_analysis-section-offline_selection-L1L2}.
Pour $L_1$ comme $L_2$, une liste de candidats est obtenue à partir des particules reconstruites auxquelles sont appliquées des coupures détaillées dans les sections qui suivent pour chacun des canaux.
À partir de ces deux listes de candidats, des paires $L_1L_2$ compatibles avec le cas de figure $\Higgs\to\tau\tau\to L_1L_2$ sont formées.
Une paire $L_1L_2$ est un \emph{dilepton}.
Un seul des \emph{dileptons} candidats est retenu selon la logique exposée dans la section~\ref{chapter-HTT_analysis-section-offline-dilepton}.
\par
Il est nécessaire de s'assurer, à cause de l'utilisation de canaux différents, qu'un événement donné ne peut être sélectionné dans le traitement de plusieurs canaux.
Pour cela, après avoir sélectionné le \emph{dilepton}, des vetos sur la présence de leptons supplémentaires à ceux du \emph{dilepton} sont appliqués.
Ces vetos sont explicités pour chaque canal dans les sections ci-après et sont au moins aussi lâches que les coupures de sélection les plus lâches des leptons de signal, \ie\ ceux utilisés pour les \emph{dileptons}, de tous les canaux.
Ainsi, si un lepton est sélectionné dans un canal pour former un \emph{dilepton}, alors il déclenche forcément le veto correspondant dans les autres canaux.

\subsubsection{Canal \tauh\tauh}\label{chapter-HTT_analysis-section-offline-tt}
\paragraph{Sélection des taus hadroniques}
\AllSatisfyingFollowing{tau hadronique}{$L_1$ ou $L_2$}:
\begin{itemize}
    \item $\pT^{\tauh} > \SI{40}{\GeV}$;
    \item $\abs{\eta^{\tauh}} < \num{2.1}$;
    \item \TauHdz;
    \item \NewDecayModeFinding[footnote];
    \item passer les points de fonctionnement:
        \begin{itemize}
            \item \PassDeepTau{très très relâché}{very very loose}{anti-electron},
            \item \PassDeepTau{très relâché}{very loose}{anti-muon},
            \item \PassDeepTau{moyen}{medium}{vs jet}.
        \end{itemize}
\end{itemize}
\paragraph{Sélection du \emph{dilepton}}
\AtLeastOneOSPair{\tauh\tauh}
\DeltaRPair{0.5}
\IfMoreOnePair
\paragraph{Correspondance du \emph{dilepton} avec les \HLTpaths}
Les deux taus hadroniques \FromPairMatchToHLTObjects{}
Trois \HLTregionsDefined:
\begin{itemize}
    \item bas \pT: $\pT^{\tauh 1} < \num{120} \, (\num{180}) \usp \SI{}{\GeV}$, $\pT^{\tauh 2} < \num{120} \, (\num{180}) \usp \SI{}{\GeV}$.
        Seuls les \HLTpaths\ \HLTDoubleTau{} sont considérés pour $L_1$ et $L_2$;
    \item moyen \pT: $\pT^{\tauh 1} > \num{120} \, (\num{180}) \usp \SI{}{\GeV}$, $\pT^{\tauh 2} < \num{120} \, (\num{180}) \usp \SI{}{\GeV}$.
        Une combinaison logique \og ou \fg{} des \HLTpaths\ \HLTSingleTau{} et \HLTDoubleTau{} est considérée pour $L_1$ et seulement les \HLTDoubleTau{} pour $L_2$;
    \item haut \pT: $\pT^{\tauh 1} > \num{120} \, (\num{180}) \usp \SI{}{\GeV}$, $\pT^{\tauh 2} > \num{120} \, (\num{180}) \usp \SI{}{\GeV}$.
        Une combinaison logique \og ou \fg{} des \HLTpaths\ \HLTSingleTau{} et \HLTDoubleTau{} est considérée pour $L_1$ et $L_2$.
\end{itemize}
Les taus hadroniques ne sont considérés comme correspondant aux objets HLT qu'à condition que leurs impulsions soient supérieure d'au moins \SI{5}{\GeV} au seuil d'activation du \HLTpath.
\paragraph{Vétos de leptons supplémentaires}
\LeptonVetoes
\begin{itemize}
    \item \LeptonVetoesExtraMuon;
    \item \LeptonVetoesExtraEle.
\end{itemize}

\subsubsection{Canal \mu\tauh}\label{chapter-HTT_analysis-section-offline-mt}
\paragraph{Sélection des muons}
\AllSatisfyingFollowing{muon}{$L_1$}:
\begin{itemize}
    \item $\pT^{\mu} > \num{23} \, (\num{25}) \usp \SI{}{\GeV}$ en 2016 (2017, 2018) et correspondre à l'objet HLT du \HLTpath\ \HLTSingleMu{} ou $\num{20} \, (\num{21}) < \pT^{\mu} \leq \num{23} \, (\num{25}) \usp \SI{}{\GeV}$ en 2016 (2017, 2018) et correspondre à l'objet HLT de type muon du \HLTpath\ \HLTMuTauCross{};
    \item $\abs{\eta^{\mu}} < \num{2.1}$;
    \item \Leptondzdxy;
    \item \RelIsoBelow{\mu}{0.15};
    \item passer le \MediumMuonID.
\end{itemize}
\paragraph{Sélection des taus hadroniques}
\AllSatisfyingFollowing{tau hadronique}{$L_2$}:
\begin{itemize}
    \item $\pT^{\tauh} > \SI{30}{\GeV}$;
    \item $\abs{\eta^{\tauh}} < \num{2.3}$;
    \item \TauHdz;
    \item \NewDecayModeFinding;
    \item passer les points de fonctionnement:
        \begin{itemize}
            \item \PassDeepTau{très très relâché}{very very loose}{anti-electron},
            \item \PassDeepTau{strict}{tight}{anti-muon},
            \item \PassDeepTau{moyen}{medium}{vs jet}.
        \end{itemize}
\end{itemize}
\paragraph{Sélection du \emph{dilepton}}
\AtLeastOneOSPair{\mu\tauh}
\DeltaRPair{0.5}
\IfMoreOnePair
\paragraph{Correspondance du \emph{dilepton} avec les \HLTpaths}
Le muon et le tau hadronique \FromPairMatchToHLTObjects{}
Deux \HLTregionsDefined:
\begin{itemize}
    \item bas \pT: $\pT^{\tauh} < \num{120} \, (\num{180}) \usp \SI{}{\GeV}$ en 2016 (2017, 2018).
        Une combinaison logique \og ou \fg{} des \HLTpaths\ \HLTSingleMu{} et \HLTMuTauCross{} est considérée;
    \item haut \pT: $\pT^{\tauh} > \num{120} \, (\num{180}) \usp \SI{}{\GeV}$ en 2016 (2017, 2018).
        Une combinaison logique \og ou \fg{} des \HLTpaths\ \HLTSingleMu{} et \HLTSingleTau{} est considérée.
\end{itemize}
Dans le cas de l'utilisation du \HLTpath\ \HLTMuTauCross{}, le tau hadronique doit de plus:
\begin{itemize}
    \item correspondre à l'objet HLT;
    \item $\pT^{\tauh} > \num{25} \, (\num{32}) \usp \SI{}{\GeV}$ en 2016 (2017, 2018).
\end{itemize}
\paragraph{Masse transverse du muon}
\TransverseMassWjets[footnote]{du muon}{\mu}{mu}
\paragraph{Vétos de leptons supplémentaires}
\LeptonVetoes
\begin{itemize}
    \item \LeptonVetoesSecondMuon;
    \item \LeptonVetoesExtraEle;
    \item \LeptonVetoesMuonPair.
\end{itemize}

\subsubsection{Canal \ele\tauh}\label{chapter-HTT_analysis-section-offline-et}
\paragraph{Sélection des électrons}
\AllSatisfyingFollowing{électron}{$L_1$}:
\begin{itemize}
    \item $\pT^{\ele} > \SI{26}{\GeV}$ en 2016, \num{28} en 2017 et \num{33} en 2018 et correspondre à l'objet HLT du \HLTpath\ \HLTSingleEle{} ou $\pT^{\ele}$ entre $\SI{25}{\GeV}$ et la valeur précédente et correspondre à l'objet HLT de type électron du \HLTpath\ \HLTEleTauCross{};
    \item $\abs{\eta^{\ele}} < \num{2.1}$;
    \item \Leptondzdxy;
    \item \RelIsoBelow{\ele}{0.15};
    \item passer le \NinetyNineEleMVA.
\end{itemize}
\paragraph{Sélection des taus hadroniques}
\AllSatisfyingFollowing{tau hadronique}{$L_2$}:
\begin{itemize}
    \item $\pT^{\tauh} > \SI{30}{\GeV}$,
    \item $\abs{\eta^{\tauh}} < \num{2.3}$,
    \item \TauHdz,
    \item \NewDecayModeFinding,
    \item passer les points de fonctionnement:
        \begin{itemize}
            \item \PassDeepTau{strict}{tight}{anti-electron},
            \item \PassDeepTau{très relâché}{very loose}{anti-muon},
            \item \PassDeepTau{moyen}{medium}{vs jet}.
        \end{itemize}
\end{itemize}
\paragraph{Sélection du \emph{dilepton}}
\AtLeastOneOSPair{\ele\tauh}
\DeltaRPair{0.5}
\IfMoreOnePair
\paragraph{Correspondance du \emph{dilepton} avec les \HLTpaths}
L'électron et le tau hadronique \FromPairMatchToHLTObjects{}
Deux \HLTregionsDefined:
\begin{itemize}
    \item bas \pT: $\pT^{\tauh} < \num{120} \, (\num{180}) \usp \SI{}{\GeV}$ en 2016 (2017, 2018).
        Une combinaison logique \og ou \fg{} des \HLTpaths\ \HLTSingleEle{} et \HLTEleTauCross{} est considérée;
    \item haut \pT: $\pT^{\tauh} > \num{120} \, (\num{180}) \usp \SI{}{\GeV}$ en 2016 (2017, 2018).
        Une combinaison logique \og ou \fg{} des \HLTpaths\ \HLTSingleEle{} et \HLTSingleTau{} est considérée.
\end{itemize}
Dans le cas de l'utilisation du \HLTpath\ \HLTEleTauCross{}, le tau hadronique doit de plus:
\begin{itemize}
    \item correspondre à l'objet HLT;
    \item vérifier $\pT^{\tauh} > \num{25} \, (\num{35}) \usp \SI{}{\GeV}{\GeV}$ en 2016 (2017, 2018);
    \item vérifier $\abs{\eta^{\tauh}} < \num{2.1}$.
\end{itemize}
\paragraph{Masse transverse de l'électron}
\TransverseMassWjets{de l'électron}{\ele}{ele}
\paragraph{Vétos de leptons supplémentaires}
\LeptonVetoes
\begin{itemize}
    \item \LeptonVetoesExtraMuon;
    \item \LeptonVetoesSecondEle;
    \item \LeptonVetoesElectronPair.
\end{itemize}

\subsubsection{Canal \ele\mu}\label{chapter-HTT_analysis-section-offline-em}
\paragraph{Sélection des muons}
\AllSatisfyingFollowing{muon}{$L_1$}:
\begin{itemize}
    \item $\pT^{\mu} > \SI{15}{\GeV}$;
    \item $\abs{\eta^{\mu}} < 2.4$;
    \item \Leptondzdxy;
    \item \RelIsoBelow{\mu}{0.2};
    \item passer le \MediumMuonID.
\end{itemize}
\paragraph{Sélection des électrons}
\AllSatisfyingFollowing{électron}{$L_2$}:
\begin{itemize}
    \item $\pT^{\ele} > \SI{15}{\GeV}$;
    \item $\abs{\eta^{\ele}} < 2.4$;
    \item \Leptondzdxy;
    \item \RelIsoBelow{\ele}{0.15};
    \item passer le \NinetyNineEleMVAnoIso;
    \item \LessTwoMissingHitsVertex;
    \item \PassConversionVeto.
\end{itemize}
\paragraph{Sélection du \emph{dilepton}}
\AtLeastOneOSPair{\mu\ele}
\DeltaRPair{0.3}
\IfMoreOnePair
\paragraph{Correspondance du \emph{dilepton} avec les \HLTpaths}
L'électron et le muon \FromPairMatchToHLTObjects{}
\subparagraph{Cas de la période 2016GH}
Pour les \emph{runs} G et H de l'année 2016, le filtre en $d_z$ des \HLTpaths\ utilisés pour le canal \ele\mu\ n'est pas modélisé dans les données simulées.
Il ne peut donc être appliqué que sur les données réelles.
Afin de prendre en compte ce filtre manquant, un filtre moins strict, sans le critère portant sur $d_z$, est appliqué sur les données simulées qui sont renormalisées selon l'efficacité du filtre sur $d_z$ manquant.
\par
La mesure sur un jeu de données simulées \ttbar\ avec un électron et un muon dans l'état final donne une efficacité de \SI{95.3}{\%}.
Aucune dépendance en \pT\ ou $\eta$ de l'électron ou du muon n'a été observée.
Pour les \emph{runs} B à F, ce filtre en $d_z$ n'est pas utilisé.
Ainsi, sur l'ensemble de l'année 2016, le facteur $SF(d_z)$ à appliquer aux événements simulés vaut
\begin{equation}
SF(d_z) = \frac{\lumi_\text{B-F}}{\lumi_\text{B-H}} + \num{0.953}\,\frac{\lumi_\text{G-H}}{\lumi_\text{B-H}} = \num{0.979}
\end{equation}
avec $\lumi_x$ la luminosité intégrée sur la période $x$, notion introduite dans le chapitre~\ifref{chapter-LHC}{\ref{chapter-LHC}}{3}.

%In 2016 data, the $d_z$ filter of the HLT paths employed for the \ele\mu channel in runs G and H is not modelled in simulation and can therefore only be applied for data.
%To account for the missing $d_z$ filter in simulation, the simulation is corrected by the efficiency of the filter in data, which amounts to 97.9~\%;
%the transverse momentum of either the electron or the muon matched to the high-\pT leg of the HLT path has to exceed 24~\GeV to ensure that the HLT path selection is fully efficient;
%
%
%- L 652ff: If I understand correctly, the issue is that the trigger in the MC is different from the one you use in the data. Apparently you decided against rerunning the correct trigger on the simulation, which is of course a huge effort. In principle it is fine to estimate a partial trigger efficiency from the data, but does it assume a factorization of the dz filter efficiency and is this assumption justified?
%
%%GREEN%
%The HLT paths in data contain the dz filter only for the runperiods G and H, not for B -- F. This means either way, a simulation of the HLT path with or without the dz filter does not map the (inhomogeneous) set HLT paths in data. To our understanding this can only be properly modelled by a Run-dependent MC mix.
%
%Given that, using a looser HLT path selection and applying the expected efficiency drop on top to account for the data runs G and H is the best we can do. We have studied the effect of the factorization of the dz filter (distance of closest approach between the e-mu candidates in z direction) and found this to be a valid assumption.
%
%The dz filter efficiency in Runs 2016G-H has been measured in a sample of ttbar events yielding the final state with electron and muon. The prescaled path HLT_Mu23_TrkIsoVVL_Ele12_CaloIdL_TrackIdL_IsoVL without dz filter requirement has been used as a monitor trigger. The efficiency has been measured inclusively and in bins of electron/muon pT and eta. The measured inclusive value of the dz filter efficiency is 95.3%. No apparent dependence of the dz filter efficiency on electron/muon pT and eta was observed. Given that the dz filter has not been applied in the simulated samples and that in RunsB-F we use a trigger w/o dz filter requirement, the lumi-weighted SF for the dz filter efficiency for the entire Run 2016 dataset is computed as
%<br> 
%SF(dz) = (1.0*lumi(B-F) + 0.953*lumi(G-H))/lumi(B-H) = 0.979.
%%ENDCOLOR%
%
%Alexei, Danny, could you please check and verify, how we get the number 97.9% for the dZ efficiency? Having a look at the “official” CMS measurements, this factor seems to be different:
%
%http://cms.cern.ch/iCMS/jsp/db_notes/notestable1.jsp?CMSNoteID=DP-2019/025
%
%Alexei : I don’t remember that we ever in any analysis obtained a value for the dZ filter efficiency close to 100% as quoted in the DP Note 2019/025.
%We have measured recently the dz filter efficiency for the same-sign dimuon trigger HLT_Mu17_Mu8_SameSign_DZ in 2016 dataset and obtained comparable number to what we have measured for the e-mu trigger in Runs 2016G-H (95-96%).
%Here is the link to the DP Note:
%https://cds.cern.ch/record/2721809/files/DP2020_029.pdf
%
%Also do you have slides demonstrating the factorisation?
%
%Alexei: Here is the plot showing dz filter efficiency measured in bins of electron/muon pT in Runs 2016G-H. 
%https://www.desy.de/~rasp/EMu_MSSM/DZFilter_pT.png
%No apparent dependence on pT is observed, implying factorisation of the dZ filter efficiency and efficiencies of electron and muon legs (Unfortunately I don’t have plot, showing the eta dependence, but I remember that we didn’t see systematic dependence of dz filter efficiency on eta either). 

\paragraph{Vétos de leptons supplémentaires}
\LeptonVetoes
\begin{itemize}
    \item \LeptonVetoesSecondMuon;
    \item \LeptonVetoesSecondEle, l'électron devant \PassConversionVeto\ et \LessTwoMissingHitsVertex.
\end{itemize}

\subsubsection{Sélection d'un unique \emph{dilepton}}\label{chapter-HTT_analysis-section-offline-dilepton}
Il est possible d'obtenir plusieurs candidats \emph{dilepton} après application des coupures précédemment exposées.
Dans ce cas, une seule et unique paire est conservée à l'aide des étapes de réjection suivantes:
\begin{enumerate}
    \item préférer une paire avec $L_1$ le plus isolé possible (le tau hadronique avec le plus haut \pT\ pour le canal \tauh\tauh, le muon pour les canaux \mu\tauh\ et \ele\mu, l'électron pour le canal \ele\tauh);
    \item en cas d'égalité, préférer une paire avec $L_1$ de plus haut \pT;
    \item en cas d'égalité, préférer une paire avec $L_1$ le plus isolé possible (le tau hadronique avec le plus bas \pT\ pour le canal \tauh\tauh, le tau hadronique pour les canaux \mu\tauh\ et \ele\tauh, l'électron pour le canal \ele\mu);
    \item en cas d'égalité, préférer une paire avec $L_2$ de plus haut \pT.
\end{enumerate}