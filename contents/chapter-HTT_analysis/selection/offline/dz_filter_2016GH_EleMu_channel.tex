\subparagraph{Cas de la période 2016GH}
Pour les \emph{runs} G et H de l'année 2016, le filtre en $d_z$ des \HLTpaths\ utilisés pour le canal \ele\mu\ n'est pas modélisé dans les données simulées.
Il ne peut donc être appliqué que sur les données réelles.
Afin de prendre en compte ce filtre manquant, un autre moins strict, sans le critère portant sur $d_z$, est appliqué sur les données simulées qui sont renormalisées selon l'efficacité du filtre sur $d_z$ manquant.
\par
La mesure sur un jeu de données simulées \ttbar\ avec un électron et un muon dans l'état final donne une efficacité de \SI{95.3}{\%}.
Aucune dépendance en \pT\ ou $\eta$ de l'électron ou du muon n'a été observée.
Pour les \emph{runs} B à F, ce filtre en $d_z$ n'est pas utilisé.
Ainsi, sur l'ensemble de l'année 2016, le facteur $SF(d_z)$ à appliquer aux événements simulés vaut
\begin{equation}
SF(d_z) = \frac{\lumi_\text{B-F}}{\lumi_\text{B-H}} + \num{0.953}\,\frac{\lumi_\text{G-H}}{\lumi_\text{B-H}} = \num{0.979}
\end{equation}
avec $\lumi_x$ la luminosité intégrée sur la période $x$, notion introduite dans le chapitre~\refChLHCCMS.

%In 2016 data, the $d_z$ filter of the HLT paths employed for the \ele\mu channel in runs G and H is not modelled in simulation and can therefore only be applied for data.
%To account for the missing $d_z$ filter in simulation, the simulation is corrected by the efficiency of the filter in data, which amounts to 97.9~\%;
%the transverse momentum of either the electron or the muon matched to the high-\pT leg of the HLT path has to exceed 24~\GeV to ensure that the HLT path selection is fully efficient;
%
%
%- L 652ff: If I understand correctly, the issue is that the trigger in the MC is different from the one you use in the data. Apparently you decided against rerunning the correct trigger on the simulation, which is of course a huge effort. In principle it is fine to estimate a partial trigger efficiency from the data, but does it assume a factorization of the dz filter efficiency and is this assumption justified?
%
%%GREEN%
%The HLT paths in data contain the dz filter only for the runperiods G and H, not for B -- F. This means either way, a simulation of the HLT path with or without the dz filter does not map the (inhomogeneous) set HLT paths in data. To our understanding this can only be properly modelled by a Run-dependent MC mix.
%
%Given that, using a looser HLT path selection and applying the expected efficiency drop on top to account for the data runs G and H is the best we can do. We have studied the effect of the factorization of the dz filter (distance of closest approach between the e-mu candidates in z direction) and found this to be a valid assumption.
%
%The dz filter efficiency in Runs 2016G-H has been measured in a sample of ttbar events yielding the final state with electron and muon. The prescaled path HLT_Mu23_TrkIsoVVL_Ele12_CaloIdL_TrackIdL_IsoVL without dz filter requirement has been used as a monitor trigger. The efficiency has been measured inclusively and in bins of electron/muon pT and eta. The measured inclusive value of the dz filter efficiency is 95.3%. No apparent dependence of the dz filter efficiency on electron/muon pT and eta was observed. Given that the dz filter has not been applied in the simulated samples and that in RunsB-F we use a trigger w/o dz filter requirement, the lumi-weighted SF for the dz filter efficiency for the entire Run 2016 dataset is computed as
%<br> 
%SF(dz) = (1.0*lumi(B-F) + 0.953*lumi(G-H))/lumi(B-H) = 0.979.
%%ENDCOLOR%
%
%Alexei, Danny, could you please check and verify, how we get the number 97.9% for the dZ efficiency? Having a look at the “official” CMS measurements, this factor seems to be different:
%
%http://cms.cern.ch/iCMS/jsp/db_notes/notestable1.jsp?CMSNoteID=DP-2019/025
%
%Alexei : I don't remember that we ever in any analysis obtained a value for the dZ filter efficiency close to 100% as quoted in the DP Note 2019/025.
%We have measured recently the dz filter efficiency for the same-sign dimuon trigger HLT_Mu17_Mu8_SameSign_DZ in 2016 dataset and obtained comparable number to what we have measured for the e-mu trigger in Runs 2016G-H (95-96%).
%Here is the link to the DP Note:
%https://cds.cern.ch/record/2721809/files/DP2020_029.pdf
%
%Also do you have slides demonstrating the factorisation?
%
%Alexei: Here is the plot showing dz filter efficiency measured in bins of electron/muon pT in Runs 2016G-H. 
%https://www.desy.de/~rasp/EMu_MSSM/DZFilter_pT.png
%No apparent dependence on pT is observed, implying factorisation of the dZ filter efficiency and efficiencies of electron and muon legs (Unfortunately I don't have plot, showing the eta dependence, but I remember that we didn't see systematic dependence of dz filter efficiency on eta either). 
