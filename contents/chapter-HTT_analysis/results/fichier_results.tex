\section{Résultats et interprétations}\label{chapter-HTT_analysis-section-results}
Les résultats de cette analyse comportent:
\begin{itemize}
\item des distributions des variables discriminantes dans les différentes catégories utilisées;
\item des limites d'exclusion obtenues indépendamment d'un modèle;
\item des contours d'exclusions dans le plan $(m_{\HiggsA},\tan\beta)$ pour des scénarios du MSSM.
\end{itemize}

\todo{see comments in tex file}
%- Fig. 34: If you treat the SM Higgs as BG, sigma(gg phi) is misleading because phi includes the SM-like Higgs boson
%
%How does this compare with earlier results?
%%GREEN%
%The variable $\phi$ does not refer to the SM Higgs boson in any way. It is a general expression for a not further specified scalar boson. Whether the SM Higgs boson is part of the search or not has to become clear from the limit plot or its caption. Check e.g.:
%
%http://cms-results.web.cern.ch/cms-results/public-results/publications/HIG-17-020/
%
%Figure 7 compared to Additional Figures 15 and 16.
%%ENDCOLOR%


%\begin{figure}[h]
%\centering
%
%\subcaptionbox{}[.475\textwidth]
%{\plotHTTModelIndepLimits{combined}{ggH}{cmb}{}}
%\hfill
%\subcaptionbox{}[.475\textwidth]
%{\plotHTTModelIndepLimits{combined}{bbH}{cmb}{}}
%
%\caption{}
%\end{figure}
%
%\begin{figure}[h]
%\centering
%
%\subcaptionbox{}[.475\textwidth]
%{\plotHTTModelDepLimits{mssm_vs_sm_h125}{}}
%\hfill
%\subcaptionbox{}[.475\textwidth]
%{\plotHTTModelDepLimits{mssm_vs_sm_h125}{_expOnly}}
%
%\subcaptionbox{}[.475\textwidth]
%{\plotHTTModelDepLimits{mssm_vs_sm_classic_h125}{}}
%\hfill
%\subcaptionbox{}[.475\textwidth]
%{\plotHTTModelDepLimits{mssm_vs_sm_classic}{}}
%
%\subcaptionbox{}[.475\textwidth]
%{\plotHTTModelDepLimits{mssm_vs_sm_CPV}{}}
%\hfill
%\subcaptionbox{}[.475\textwidth]
%{\plotHTTModelDepLimits{mssm_vs_sm_heavy}{}}
%
%\caption{}
%\end{figure}