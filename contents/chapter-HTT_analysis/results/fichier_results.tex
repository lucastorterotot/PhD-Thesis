\section{Résultats et interprétations}\label{chapter-HTT_analysis-section-results}
Les résultats présentés dans cette thèse,
obtenus dans le cadre de l'analyse collaborative MSSM \HAtoTauTau,
sont préliminaires
et
peuvent donc encore évoluer avant leur publication officielle.
Ils comportent:
\begin{itemize}
\item des distributions de variables de contrôle, données en annexe~\refApHTTctrlplots;
\item les distributions des variables discriminantes dans les différentes catégories utilisées, données en annexe~\refApHTTshapes;
\item des limites d'exclusion obtenues indépendamment d'un modèle;
\item les valeurs ajustées des paramètres de nuisance associées à ces limites, données en annexe~\refApHTTimpacts;
\item des contours d'exclusions
%dans les plans $(m_{\HiggsA},\tan\beta)$ ou $(m_{\Higgspm},\tan\beta)$
pour des scénarios du MSSM.
\end{itemize}
\subsection{Limites indépendantes du modèle}
Les limites d'exclusion indépendantes du modèle sont données en figure~\ref{fig-model_indep}
pour les processus
$\gluon\gluon\to\Phi\to\tau\tau$ (figure~\ref{subfig-model_indep-ggh})
et
$\gluon\gluon\to\quarkb\antiquarkb\Phi\to\tau\tau$ (figure~\ref{subfig-model_indep-bbh}).
Les précédents résultats obtenus par la collaboration CMS avec les données récoltées en 2016~\cite{CMS-PAS-HIG-17-020}
ainsi que les derniers résultats de la collaboration ATLAS sur l'intégralité du Run~II~\cite{ATLAS-MSSM-HTT_2020} y sont également affichés.
%Des points de masse supplémentaires seront proposés dans les résultats finaux de l'analyse MSSM \HAtoTauTau\ afin de couvrir une gamme allant de \SI{60}{\GeV} à \SI{3.5}{\TeV}.
\begin{figure}[h]
\centering

\subcaptionbox{Processus $\gluon\gluon\to\Phi\to\tau\tau$.\label{subfig-model_indep-ggh}}[.475\textwidth]
{\plotHTTModelIndepLimits{mssm_classic}{combined}{ggH}{_cmb_vs_CMS_ATLAS}}
\hfill
\subcaptionbox{Processus $\gluon\gluon\to\quarkb\antiquarkb\Phi\to\tau\tau$.\label{subfig-model_indep-bbh}}[.475\textwidth]
{\plotHTTModelIndepLimits{mssm_classic}{combined}{bbH}{_cmb_vs_CMS_ATLAS}}

% 110 120 130 140 160 180 200 250 300 350 400 450 500 600 700 800 900 1000 1200 1400 1600 1800 2000 2300 2600 2900 3200 3500

\caption[Limites d'exclusion indépendantes.]{Limites d'exclusion indépendantes du modèle physique obtenues avec l'intégralité des données du Run~II récoltées par CMS. Les précédents résultats~\cite{CMS-PAS-HIG-17-020} obtenus avec les données de l'année 2016 sont également donnés (CMS 2016) ainsi que ceux de la collaboration ATLAS sur l'intégralité du Run~II~\cite{ATLAS-MSSM-HTT_2020} (ATLAS Run~II).}
\label{fig-model_indep}
\end{figure}
\par
Les limites attendues (\emph{expected}) correspondent au cas dans lequel
les données observées correspondent exactement à la description des bruits de fond.
Ces limites permettent d'interpréter celles effectivement observées.
Lorsqu'elles sont égales,
comme c'est le cas pour $m_{\Phi}\gtrsim\SI{400}{\GeV}$ en figure~\ref{subfig-model_indep-bbh},
les observations sont en accord avec les prédictions du modèle.
Lorsque les limites observées sont plus basses que les limites attendues,
cela signifie que les données observées permettent de mieux exclure la présence d'un signal
que la modélisation du bruit de fond elle-même.
Ce cas de figure peut survenir lorsque cette modélisation est imparfaite.
Par exemple, la méthode des données encapsulées surestimait initialement la quantité d'événements
pour $\mTtot \gtrsim \SI{300}{\GeV}$ \cite{Gael_thesis} et donnait des limites d'exclusion observées plus basses que celles attendues, ce qui a été corrigé depuis.
Sur les figures~\ref{subfig-model_indep-ggh} et~\ref{subfig-model_indep-bbh},
les limites observées sont plus basses d'environ $2\sigma$ que celles attendues à basse masse ($m_\Phi \lesssim \SI{300}{\GeV}$).
La déviation de $2\sigma$ n'est toutefois pas statistiquement significative.
\par
En revanche, lorsque les limites observées sont plus hautes que les limites attendues,
cela signifie que les observations ne permettent pas d'exclure la présence d'un signal
aussi bien que les prédictions.
Ce peut être dû à un manque de statistiques dans la région correspondante
ou
à la présence effective d'un signal.
Sur la figure~\ref{subfig-model_indep-ggh},
la limite d'exclusion observée à $m_{\Phi}=\SI{1200}{\GeV}$ est de
\SI{8e-3}{\pico\barn}
contre
\SI{2e-3}{\pico\barn} attendus.
Cet écart est d'environ $4\sigma$.
Cependant, les derniers résultats de la collaboration ATLAS~\cite{ATLAS-MSSM-HTT_2020} excluent cette valeur.
La présence d'un signal pour expliquer cet excès est donc peu probable.
\subsection{Limites du scénario $M_{\higgs}^{125}$}
Les limites d'exclusion dans le plan $(m_{\HiggsA},\tan\beta)$ du scénario $M_{\higgs}^{125}$
sont données en figure~\ref{fig-model_dep}.
Sur la figure~\ref{subfig-model_dep-mssm_vs_sm_classic},
seules les catégories \CATbsm\ sont exploitées sans combinaison avec les catégories \CATsm,
à l'instar de ce qui a été fait lors de
l'analyse des données enregistrées en 2016~\cite{CMS-PAS-HIG-17-020}.
Dans le cas de la figure~\ref{subfig-model_dep-mssm_vs_sm_h125},
ces limites sont obtenues tel que décrit dans la section~\ref{chapter-HTT_analysis-section-signal_extraction-benchmarks-mh125},
\ie\ avec combinaison des catégories \CATbsm\ et \CATsm.
La zone hachurée en rouge ($m_{\higgsMSSM}\neq\num{125}\pm\SI{3}{\GeV}$) est forcément exclue,
car le boson \higgsMSSM\ n'y possède pas une masse compatible avec le boson découvert en 2012.
La zone bleue correspond à la région de l'espace des phases pour laquelle
l'hypothèse \hypSB\ correspondant au MSSM
est rejetée en faveur de
l'hypothèse \hypB\ correspondant au \SM\
d'après les données observées.
\begin{figure}[h]
\centering

\subcaptionbox{Limites d'exclusion obtenues avec la catégorisation classique \CATbsm\ décrite section~\ref{chapter-HTT_analysis-section-categorisation-BSM}.\label{subfig-model_dep-mssm_vs_sm_classic}}[.475\textwidth]
{\plotHTTModelDepLimits{mssm_vs_sm_classic}{_vs_CMS_ATLAS}}
\hfill
\subcaptionbox{Limites d'exclusion obtenues avec la catégorisation combinée $\CATsm + \CATbsm$ décrite section~\ref{chapter-HTT_analysis-section-categorisation-SM_and_BSM}.\label{subfig-model_dep-mssm_vs_sm_h125}}[.475\textwidth]
{\plotHTTModelDepLimits{mssm_vs_sm_h125}{_vs_CMS_ATLAS}}

\caption[Limites d'exclusion du scénario $M_{\higgs}^{125}$.]{Limites d'exclusion du scénario $M_{\higgs}^{125}$ obtenues avec l'intégralité des données du Run~II récoltées par CMS. Les précédents résultats~\cite{CMS-PAS-HIG-17-020} obtenus avec les données de l'année 2016 sont également donnés (CMS 2016) ainsi que ceux de la collaboration ATLAS sur l'intégralité du Run~II~\cite{ATLAS-MSSM-HTT_2020} (ATLAS Run~II).}
\label{fig-model_dep}
\end{figure}
\par
La comparaison des figures~\ref{subfig-model_dep-mssm_vs_sm_classic} et~\ref{subfig-model_dep-mssm_vs_sm_h125}
montre l'effet de la combinaison des catégories \CATbsm\ et \CATsm.
La conséquence majeure de cette combinaison est l'extension de la zone d'exclusion observée aux basses valeurs de $\tan\beta$.
Dans cette région de l'espace des phases, la masse de \higgsMSSM\ est en effet incompatible avec les données expérimentales.
Bien que cette région soit exclue par la zone hachurée,
cela montre que la prise en compte des propriétés de \higgsMSSM\
et leur comparaison à celles de \higgs\ observé permet d'obtenir de meilleurs résultats.
De plus, les limites observées à $\tan\beta\simeq\num{10}$ sont un peu plus étendues,
bien que toujours compatibles avec limites attendues à $2\sigma$
peu sensibles à la combinaison des catégories dans cette région de l'espace de phases.
%\par
Sur la figure~\ref{subfig-model_dep-mssm_vs_sm_h125},
%pour $\tan\beta<\num{10}$,
les valeurs de $m_{\HiggsA}$ inférieures à \SI{600}{\GeV} sont exclues.
Cette limite passe à \SI{1}{\TeV} pour $\tan\beta\gtrsim\num{10}$,
et \SI{2}{\TeV} pour $\tan\beta\gtrsim\num{50}$.
\subsection[Limites du scénario $M_{\Higgs_1}^{125}$ avec violation de $CP$]{Limites du scénario $M_{\Higgs_1}^{125}(\text{CPV})$}
\begin{wrapfigure}[16]{R}{.475\textwidth}
\vspace{-3\baselineskip}
\centering
\plotHTTModelDepLimits{mssm_vs_sm_CPV-new_categories}{}
\caption[Limites d'exclusion du scénario $M_{\Higgs_1}^{125}(\text{CPV})$.]{Limites d'exclusion du scénario $M_{\Higgs_1}^{125}$ $(\text{CPV})$ dans le plan $(m_{\Higgspm},\tan\beta)$ obtenues avec l'intégralité des données du Run~II récoltées par CMS.}
\label{fig-model_dep-CPV}
\end{wrapfigure}
Les limites d'exclusion du scénario $M_{\Higgs_1}^{125}(\text{CPV})$ dans le plan $(m_{\Higgspm},\tan\beta)$
sont données en figure \ref{fig-model_dep-CPV}.
Dans la région $\tan\beta\simeq\num{9}$ et $m_{\Higgspm}\simeq\SI{700}{\GeV}$,
elles sont affectées par les interférences entre $\Higgs_2$ et $\Higgs_3$.
Ces interférences destructives donnent un signal atténué, réduisant la sensibilité de l'analyse.
%Les données observées excluent toutefois toute la région $\tan\beta\gtrsim\num{8}$.
%Ainsi, le seul domaine de validité du scénario $M_{\Higgs_1}^{125}(\text{CPV})$
%se situe à $m_{\Higgspm}>\SI{600}{\GeV}$ et $\num{5.5}\lesssim\tan\beta\lesssim\num{8.5}$.
Les limites observées sont en bon accord avec les limites attendues.
Les valeurs inférieures à \SI{400}{\GeV} pour la masse $m_{\Higgspm}$ sont exclues.
Pour $\tan\beta>\num{10}$,
hormis la zone d'interférences à $m_{\Higgspm} \simeq \SI{650}{\GeV}$,
$m_{\Higgspm} < \SI{1}{\TeV}$ est rejeté.
Lorsque
$\tan\beta\simeq\num{20}$,
l'exclusion s'étend jusqu'à
$m_{\Higgspm} \simeq \SI{1.4}{\TeV}$.

%- Fig. 34: If you treat the SM Higgs as BG, sigma(gg phi) is misleading because phi includes the SM-like Higgs boson
%
%How does this compare with earlier results?
%%GREEN%
%The variable $\phi$ does not refer to the SM Higgs boson in any way. It is a general expression for a not further specified scalar boson. Whether the SM Higgs boson is part of the search or not has to become clear from the limit plot or its caption. Check e.g.:
%
%http://cms-results.web.cern.ch/cms-results/public-results/publications/HIG-17-020/
%
%Figure 7 compared to Additional Figures 15 and 16.
%%ENDCOLOR%
