\subsection{Région \og SM \fg{}}\label{chapter-HTT_analysis-section-categorisation-SM}
Les catégories SM, introduites dans les références~\cite{CMS-NOTE-2019-177,CMS-NOTE-2019-178}, sont construites dans le but d'étudier le boson de Higgs du modèle standard \higgs\ de masse \SI{125}{\GeV}.
Cette catégorisation est faite à l'aide d'un réseau de neurones dont l'objectif est de définir différentes catégories d'événements, chacune contenant un processus physique dominant.
Le principe des réseaux de neurones est abordé plus en détails dans le chapitre~\refChML.
Le réseau utilisé est ici décrit succinctement, plus de détails sont disponibles dans la référence~\cite{CMS-NOTE-2019-178}.
%AN 2019/177 and 178
\paragraph{Structure du réseau de neurones}
Les variables d'entrée du réseau les plus importantes sont:
\begin{itemize}
\item les impulsions transverses des éléments du \emph{dilepton};
\item la masse transverse du \emph{dilepton} dans le cas du canal \ele\mu\ ($\mT(\vpT^{(\ele)}+\vpT^{(\mu)},\vMET)$);
\item les impulsions transverses des deux principaux jets de l'événement;
\item le nombre de jets \Njets;
\item le nombre de jets de quarks~\quarkb\ \Nbjets;
\item la masse invariante du système des deux jets principaux \mjj;
\item la distance dans le plan $(\eta,\phi)$ entre les deux jets principaux \Detajj;
\item l'impulsion transverse totale des deux principaux jets de l'événement;
\item la masse du \emph{dilepton} estimée par \SVFIT, \msv;
\item la masse invariante du \emph{dilepton}, \mvis;
\item l'impulsion transverse du \emph{dilepton}, \pTvis.
\end{itemize}
Le réseau est constitué de deux couches cachées de 200 neurones chacune, complètement connectées.
Leur fonction d'activation est la tangente hyperbolique, fonction permettant d'obtenir des réseaux dont la tâche est de réaliser une catégorisation.
\par
Afin de permettre une catégorisation plus poussée qu'une simple discrimination signal ou bruit de fond, la couche de sortie du réseau contient autant de neurones que de catégories souhaitées.
Ce réseau fournit donc un vecteur et non un scalaire.
La fonction d'activation de ces neurones est la fonction exponentielle normalisée ou \emph{Softmax},
\begin{equation}
\mathit{Softmax}(\vec{y})_j = \frac{\exp(z_j)}{\sum_{k=1}^n \exp(z_k)}
\msep
j\in\set{1,\ldots,K}
\mend[,]
\end{equation}
chaque composante de ce vecteur correspond donc à la probabilité que l'événement appartienne à la catégorie correspondante.
\paragraph{Catégories obtenues}
Les principaux bruits de fond présents dans l'analyse sont présentés dans la section~\ref{chapter-HTT_analysis-section-bg_estimation}.
Pour chaque canal, en plus d'une catégorie \og signal \fg{} notée \CATxxh, une catégorie est définie pour chaque bruit de fond suffisamment présent.
Pour les bruits de fond ayant une faible contribution ou étant peu différentiables d'autres bruits de fond, une catégorie \CATmisc\ est également définie.
Les différentes catégories ainsi possibles sont listées dans le tableau~\ref{tab-chapter-HTT_analysis-section-categorisation-SM-cats_recap}.
La catégorie \CATemb\ doit correspondre aux données encapsulées décrites dans la section~\ref{chapter-HTT_analysis-section-bg_estimation-embedding}.
La catégorie \CATfake, quant à elle, doit contenir les événements décrits par la méthode des facteurs de faux présentée section~\ref{chapter-HTT_analysis-section-bg_estimation-FF_method}.
Le canal \tauh\tauh\ ne devant pas contenir d'électron ni de muons, les processus $\Zboson\to\ell\ell$ ($\ell\in\set{\ele,\mu}$), \ttbar\ et \Wjets\ contribuent peu au bruit de fond, c'est pourquoi il n'existe pas de catégories leur étant dédiées dans ce canal.
Ils sont donc associés à la catégorie \CATmisc\ pour le canal \tauh\tauh.
La catégorie \CATmisc\ couvre ainsi les processus
$\Zboson\to\ell\ell$, \ttbar, \Wjets\ et Diboson dans le canal \tauh\tauh;
Diboson dans les canaux \mu\tauh\ et \ele\tauh;
$\Zboson\to\ell\ell$ et \Wjets\ dans le canal \ele\mu.
Les processus EWK $\Zboson\to LL$ et EWK $\Zboson\to \nu\nu$, introduits dans l'annexe~\refApHTTdatasets, sont également associés à la catégorie \CATmisc\ dans les canaux \tauh\tauh, \mu\tauh\ et \ele\tauh.
\begin{table}[h]
\centering
\begin{tabular}{cccccccc}
\toprule
Canal & \multicolumn{7}{l}{Catégories possibles}\\
\midrule
\tauh\tauh & \CATxxh & \CATemb & & & & \CATfake & \CATmisc \\
\mu\tauh & \CATxxh & \CATemb & \CATzll & \CATttbar & & \CATfake & \CATmisc \\
\ele\tauh & \CATxxh & \CATemb & \CATzll & \CATttbar & & \CATfake & \CATmisc \\
\ele\mu & \CATxxh & \CATemb & & \CATttbar & \CATdib & \CATqcd & \CATmisc \\
\bottomrule
\end{tabular}
\caption[Catégories SM pour les quatre canaux considérés.]{Catégories SM pour les quatre canaux considérés.}
\label{tab-chapter-HTT_analysis-section-categorisation-SM-cats_recap}
\end{table}
%Minor background processes like electroweak Z production or di-boson events which provide
%only small event numbers are collected in one category for remaining processes called ”misc”.
%Individual categories would raise problems to the training due to the small event numbers. In
%this collection they are at least respected in the training of the neural network such that they
%can’t migrate to the signal categories without increasing the training loss. Since there are no
%leptons in the τ h τ h decay channel, the yields of the Z → ll, tt̄, W + jets processes are rather
%small as well which is the reason that there are no dedicated categories in this channel. Instead
%these processes are associated with the misc category as well. In summary, the misc category
%covers Z → ll and W + jets in the eμ channel, di-Boson, single top and EWKZ in the eτ h and
%μτ h channel and Z → ll, tt̄, W + jets, di-Boson, single top and EWKZ in the τ h τ h channel.
\subsubsection{Variable discriminante}
Le réseau de neurones utilisé a pour but de classer les événements selon leur nature dans les différentes catégories définies précédemment.
De l'entraînement de ce réseau résultent les frontières entre les différentes catégories.
Les régions frontalières, \ie\ les régions de l'espace des phases dans lesquelles les événements ont de proches probabilités d'appartenir à deux catégories ou plus, sont ainsi délibérément fixées lors de l'entraînement et en dépendent.
Modifier les paramètres du réseau ou de l'entraînement mène ainsi à des migrations d'événements frontaliers d'une catégorie à une autre.
Or, ces événements frontaliers sont ceux dont la probabilité d'appartenir à une catégorie ne peut être grande, car dans ce cas cet événement est très caractéristique de cette catégorie.
\par
Il est donc pertinent d'utiliser les valeurs de sortie du réseau pour définir une variable discriminante.
Le choix fait est d'utiliser la plus grande probabilité parmi celles données par le réseau, \ie\ la probabilité d'appartenir à la catégorie dans laquelle le réseau estime que cet événement fait partie.
Cette variable est dénommée \og score \fg{} et notée \NNscore.
%The task of the neural net is to define regions of the phase space that are dominated by certain
%processes. The resulting boundaries between the categories, meaning the region where events
%are assigned similar and therefore low probabilities for two categories, are actually rather de-
%liberate and depend very much on the training parameters e.g. additional class weights. A
%variation of the parameters would cause migrations of these events between the categories.
%Therefore, it is important to use the actual values of the output nodes for further discrimina-
%tion and to sort events by it. Here we use the value of the largest probability, which was used to
%assign the category, for further discrimination within the same category. We call this quantity
%exclusive probability or just the NN score.