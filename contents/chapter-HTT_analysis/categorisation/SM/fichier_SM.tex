\subsection{Région \og SM \fg{}}\label{chapter-HTT_analysis-section-categorisation-SM}
Les catégories SM, introduites dans la référence~\cite{CMS-PAS-HIG-19-010}, sont construites dans le but d'étudier le boson de Higgs du modèle standard, noté ici $\higgs$, avec une masse de \SI{125}{\GeV}.
\par %AN 2019/177 and 178
La catégorisation SM est basée sur un réseau de neurones.
Les réseaux de neurones sont abordés plus en détails dans le chapitre~\refChML.
Les caractéristiques du réseau utilisé ici sont données à titre informatif.
\par
Les variables d'entrée du réseau sont:
\begin{itemize}
\item les impulsions transverses des éléments du \emph{dilepton};
\item les impulsions transverses des deux principaux jets de l'événement;
\item le nombre de jets \Njets;
\item le nombre de jets de quarks~\quarkb\ \Nbjets;
\item la masse invariante du système des deux jets principaux \mjj;
\item la distance dans le plan $(\eta,\phi)$ entre les deux jets principaux \Detajj;
\item l'impulsion transverse totale des deux principaux jets de l'événement;
\item la masse du \emph{dilepton} estimée par \SVFIT, \msv;
\item la masse invariante du \emph{dilepton}, \mvis;
\item l'impulsion transverse du \emph{dilepton}, \pTvis.
\end{itemize}
\todo{mTdileptonMET for em + MELA output}
\par
Le réseau est constitué de deux couches cachées de 200 neurones chacune, complètement connectées.
Leur fonction d'activation est la tangente hyperbolique.
Enfin, la couche de sortie du réseau comporte plusieurs neurones.
Ce réseau fournit donc un vecteur, et non un scalaire, permettant une catégorisation plus poussée qu'une simple discrimination signal ou bruit de fond.

%The task of the neural net is to define regions of the phase space that are dominated by certain
%processes. The resulting boundaries between the categories, meaning the region where events
%are assigned similar and therefore low probabilities for two categories, are actually rather de-
%liberate and depend very much on the training parameters e.g. additional class weights. A
%variation of the parameters would cause migrations of these events between the categories.
%Therefore, it is important to use the actual values of the output nodes for further discrimina-
%tion and to sort events by it. Here we use the value of the largest probability, which was used to
%assign the category, for further discrimination within the same category. We call this quantity
%exclusive probability or just the NN score.
%
%Minor background processes like electroweak Z production or di-boson events which provide
%only small event numbers are collected in one category for remaining processes called ”misc”.
%Individual categories would raise problems to the training due to the small event numbers. In
%this collection they are at least respected in the training of the neural network such that they
%can’t migrate to the signal categories without increasing the training loss. Since there are no
%leptons in the τ h τ h decay channel, the yields of the Z → ll, tt̄, W + jets processes are rather
%small as well which is the reason that there are no dedicated categories in this channel. Instead
%these processes are associated with the misc category as well. In summary, the misc category
%covers Z → ll and W + jets in the eμ channel, di-Boson, single top and EWKZ in the eτ h and
%μτ h channel and Z → ll, tt̄, W + jets, di-Boson, single top and EWKZ in the τ h τ h channel.

et xxh
et tt
et zll
et misc
et emb
et ff

mt xxh
mt tt
mt zll
mt misc
mt emb
mt ff

tt xxh
tt emb
tt ff
tt misc

em xxh
em tt
em ss
em misc
em db
em emb