\subsection{Catégories BSM}\label{chapter-HTT_analysis-section-categorisation-BSM}
\subsubsection{Définition des catégories}
Les catégories BSM, introduites dans la référence~\cite{CMS-PAS-HIG-17-020}, sont construites dans le but de chercher une résonance correspondant à un boson de Higgs lourd.
\par
Une première catégorisation est basée sur la présence de jets issus de quarks~\quarkb.
Deux catégories sont ainsi définies:
\begin{itemize}
\item \CATnobtag: $\Nbtag =0$;
\item \CATbtag: $\Nbtag\geq1$.
\end{itemize}
Dans le cas des canaux \mu\tauh, \ele\tauh\ et \ele\mu, chacune de ces deux catégories est à nouveau subdivisée.
\paragraph{Canaux \mu\tauh\ et \ele\tauh}
Dans ces deux canaux, la masse transverse de $L_1$ (le muon ou l'électron, notés $\ell$) définie par
\begin{equation}
\mT^{(\ell)} = \sqrt{2 \, \pT^{(\ell)} \, \MET \, (1-\cos\Delta\phi)} \label{eq-mT_def-ell}
\end{equation}
avec $\Delta\phi = \phi^{(\ell)} - \phi^{(\MET)}$
est utilisée afin de définir deux catégories:
\begin{itemize}
\item \CATtightmt: $\mT^{(\ell)} < \SI{40}{\GeV}$;
\item \CATloosemt: $\SI{40}{\GeV} \leq \mT^{(\ell)} < \SI{70}{\GeV}$;
\end{itemize}
la limite haute sur \mT\ pour la catégorie \CATloosemt\ étant appliquée afin de s'assurer que la région de signal soit orthogonale à la région de détermination (DR) des facteurs de faux des événements $\Wboson+\text{jets}$.
Les facteurs de faux sont abordés dans la section~\ref{chapter-HTT_analysis-section-bg_estimation-FF_method}.
La majorité des événements de signal, en particulier pour \Higgs\ et \HiggsA\ de basse masse, se trouve dans la catégorie \CATtightmt.
La catégorie \CATloosemt\ permet quant à elle d'augmenter l'acceptance du signal pour $m_{\Higgs,\HiggsA} > \SI{700}{\GeV}$.
La figure~\ref{subfig-chapter-HTT_analysis-section-categorisation-BSM-subcats-mT} illustre ces coupures sur $\mT^{(\ell)}$ dans le cas du canal \ele\tauh\ pour l'année 2018.
\paragraph{Canal \ele\mu}
Trois catégories sont définies selon la valeur de \Dzeta\ définie page~\pageref{eq-Dzeta_def}:
\begin{itemize}
\item \CATlowdz: $-\SI{35}{\GeV} \leq \Dzeta < -\SI{10}{\GeV}$;
\item \CATmediumdz: $-\SI{10}{\GeV} \leq \Dzeta < \SI{30}{\GeV}$;
\item \CAThighdz: $\SI{30}{\GeV} \leq \Dzeta$;
\end{itemize}
la limite basse sur \Dzeta\ pour la catégorie \CATlowdz\ étant appliquée afin de s'assurer que la région de signal soit orthogonale à la région de contrôle (CR) du bruit de fond \ttbar.
Ces trois catégories permettent d'obtenir diverses pureté de signal et fractions de bruit de fond \ttbar.
La majorité des événements de signal se trouve dans la catégorie \CATmediumdz.
La figure~\ref{subfig-chapter-HTT_analysis-section-categorisation-BSM-subcats-Dz} illustre ces coupures sur \Dzeta.
\begin{figure}[h]
\centering

\subcaptionbox{Catégorisation basée sur $\mT^{(\ell)}$.\label{subfig-chapter-HTT_analysis-section-categorisation-BSM-subcats-mT}}[.475\textwidth]
{\plotHTTcontrolCATmt{2018}{emb_ff}{et}}
\hfill
\subcaptionbox{Catégorisation basée sur \Dzeta.\label{subfig-chapter-HTT_analysis-section-categorisation-BSM-subcats-Dz}}[.475\textwidth]
{\plotHTTcontrolCATdz{2018}{emb_ff}{em}}

\caption[Illustrations des catégorisations basées sur $\mT^{(\ell)}$ et \Dzeta]{Illustrations des catégorisations basées sur $\mT^{(\ell)}$ et \Dzeta, respectivement sur les événements des canaux \ele\tauh\ et \ele\mu\ de l'année 2018.}
\label{fig-chapter-HTT_analysis-section-categorisation-BSM-subcats}
\end{figure}
\paragraph{Catégories obtenues}
Les catégories BSM correspondant à la région de signal (SR), \ie\ en dehors des régions de détermination (DR) et de contrôle (CR), sont résumées sur la figure~\ref{fig-chapter-HTT_analysis-section-categorisation-BSM-cats_recap} pour les quatre canaux considérés.
\begin{figure}[h]
\centering
\begin{tikzpicture}
\foreach \dx in {0, 6}{
    
    \draw (\dx, -3) node {\small\CATlowdz\vphantom{Àq}};
    \draw (\dx+2, -3) node {\small\CATmediumdz\vphantom{Àq}};
    \draw (\dx+4, -3) node {\small\CAThighdz\vphantom{Àq}};
    
    \draw [thick] (\dx-.925,-3-.425) rectangle  + (1.85,.85); 
    \draw [thick] (\dx+2-.925,-3-.425) rectangle  + (1.85,.85); 
    \draw [thick] (\dx+4-.925,-3-.425) rectangle  + (1.85,.85); 
    
    \foreach \ddx/\cat in {0/\CATtightmt, 3/\CATloosemt}{
        \foreach \ddy in {-1, -2}{
            
            \draw (\dx+\ddx+.5, \ddy) node {\small\cat\vphantom{Àq}};
            \draw (\dx+\ddx+.5, \ddy) node {\small\cat\vphantom{Àq}};
            
            \draw [thick] (\dx+\ddx-.925,\ddy-.425) rectangle  + (2.85,.85); 
            \draw [thick] (\dx+\ddx-.925,\ddy-.425) rectangle  + (2.85,.85); 
            
        }
    }
    
    \draw [thick] (\dx-.925,-.425) rectangle  + (5.85,.85); 
}

\draw (-1, 0) node (a) [left] {$\Higgs\to\tau\tau\to\tauh\tauh$};
\draw (a.west) + (0, -1) node [right] {$\Higgs\to\tau\tau\to\mu\tauh$};
\draw (a.west) + (0, -2) node [right] {$\Higgs\to\tau\tau\to\ele\tauh$};
\draw (a.west) + (0, -3) node [right] {$\Higgs\to\tau\tau\to\ele\mu$};

\draw (2, 1) node {\CATnobtag\vphantom{Àq}};
\draw (8, 1) node {\CATbtag\vphantom{Àq}};

\draw [very thick] (-1.25, .5) -- (11.25, .5) ;
\draw [very thick] (5, 1.5) -- (5, -3.75) ;

\end{tikzpicture}
\caption{Catégories BSM pour les quatre canaux considérés.}
\label{fig-chapter-HTT_analysis-section-categorisation-BSM-cats_recap}
\end{figure}
\subsubsection{Variable discriminante}


in transverse plane,
with all neutrinos as one single particle,
with $m\ll E$ for final decays,
\begin{align}
m^2 &= E^2 - p^2
= \left( \sum_{i \in \set{L_1,L_2,\nu}} E_i \right)^2 - \left( \sum_{i \in \set{L_1,L_2,\nu}} \vpT^i \right)^2\\
&= \left( E_1 + E_2 + \MET \right)^2 - \left( \vpT^{(1)} + \vpT^{(2)} + \vMET \right)^2\\
&= E_1^2 + E_2^2 + \MET^2 + 2 \left( E_1E_2 + E_1\MET + E_2\MET \right)\\
&\hphantom{=} - \left( {\vpT^{(1)}}^2 + {\vpT^{(2)}}^2 + {\vMET}^2 \right) - 2 \left( \vpT^{(1)}\cdot\vpT^{(2)} + \vpT^{(1)}\cdot\vMET + \vpT^{(2)}\cdot\vMET \right)\\
&= 2 \left( \pT^{(1)}\pT^{(2)}(1-\cos\phi_{12}) + \pT^{(1)}\MET(1-\cos\phi_{1m}) + \pT^{(2)}\MET(1-\cos\phi_{2m}) \right)\\
&= {m_T^{\tau\tau}}^2 + {m_T^{(1)}}^2 + {m_T^{(2)}}^{2} = {m_T^{tot}}^2
\end{align}
