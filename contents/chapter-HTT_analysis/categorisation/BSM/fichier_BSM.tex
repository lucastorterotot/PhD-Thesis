\subsection{Catégories \og BSM \fg{}}\label{chapter-HTT_analysis-section-categorisation-BSM}
\subsubsection{Définition des catégories}
Les catégories \CATbsm, introduites dans la référence~\cite{CMS-PAS-HIG-17-020}, sont construites dans le but de chercher une résonance correspondant à un boson de Higgs lourd.
\par
Une première catégorisation est basée sur la présence de jets issus de quarks~\quarkb.
Deux catégories sont ainsi définies:
\begin{itemize}
\item \CATnobtag: $\Nbtag =0$;
\item \CATbtag: $\Nbtag\geq1$.
\end{itemize}
Dans le cas des canaux \mu\tauh, \ele\tauh\ et \ele\mu, chacune de ces deux catégories est à nouveau subdivisée.
\paragraph{Canaux \mu\tauh\ et \ele\tauh}
Dans ces deux canaux, la masse transverse de $L_1$ (le muon ou l'électron, notés $\ell$) définie par
\begin{equation}
\mT^{\ell} = \mT(\ell, \MET) = \sqrt{2 \, \pT^{\ell} \, \MET \, (1-\cos\Delta\phi)} \label{eq-mT_def-ell}
\end{equation}
avec $\Delta\phi = \phi^{\ell} - \phi^{\MET}$
est utilisée afin de définir deux catégories:
\begin{itemize}
\item \CATtightmt: $\mT^{\ell} < \SI{40}{\GeV}$;
\item \CATloosemt: $\SI{40}{\GeV} \leq \mT^{\ell} < \SI{70}{\GeV}$;
\end{itemize}
la limite haute sur \mT\ pour la catégorie \CATloosemt\ étant appliquée afin de s'assurer que la région de signal soit orthogonale à la région de détermination (DR) des facteurs de faux des événements $\Wboson+\text{jets}$.
Les facteurs de faux sont abordés dans la section~\ref{chapter-HTT_analysis-section-bg_estimation-FF_method}.
La majorité des événements de signal, en particulier pour \Higgs\ et \HiggsA\ de basse masse, se trouve dans la catégorie \CATtightmt.
La catégorie \CATloosemt\ permet quant à elle d'augmenter l'acceptation du signal pour $m_{\Higgs,\HiggsA} > \SI{700}{\GeV}$.
La figure~\ref{subfig-chapter-HTT_analysis-section-categorisation-BSM-subcats-mT} illustre ces coupures sur $\mT^{\ell}$ dans le cas du canal \ele\tauh\ pour l'année 2018.
\begin{figure}[h]
\centering

\subcaptionbox{Catégorisation basée sur $\mT^{\ell}$.\label{subfig-chapter-HTT_analysis-section-categorisation-BSM-subcats-mT}}[.475\textwidth]
{\plotHTTcontrolCATmt{2018}{emb_ff}{et}}
\hfill
\subcaptionbox{Catégorisation basée sur \Dzeta.\label{subfig-chapter-HTT_analysis-section-categorisation-BSM-subcats-Dz}}[.475\textwidth]
{\plotHTTcontrolCATdz{2018}{emb_ff}{em}}

\caption[Illustrations des catégorisations basées sur $\mT^{\ell}$ et \Dzeta]{Illustrations des catégorisations basées sur $\mT^{\ell}$ et \Dzeta, respectivement sur les événements des canaux \ele\tauh\ et \ele\mu\ de l'année 2018.}
\label{fig-chapter-HTT_analysis-section-categorisation-BSM-subcats}
\end{figure}
\paragraph{Canal \ele\mu}
Trois catégories sont définies selon la valeur de \Dzeta\ définie équation~\eqref{eq-Dzeta_def}:
\begin{itemize}
\item \CATlowdz: $-\SI{35}{\GeV} \leq \Dzeta < -\SI{10}{\GeV}$;
\item \CATmediumdz: $-\SI{10}{\GeV} \leq \Dzeta < \SI{30}{\GeV}$;
\item \CAThighdz: $\SI{30}{\GeV} \leq \Dzeta$;
\end{itemize}
la limite basse sur \Dzeta\ pour la catégorie \CATlowdz\ étant appliquée afin de s'assurer que la région de signal soit orthogonale à la région de contrôle (CR) du bruit de fond \ttbar.
Ces trois catégories permettent d'obtenir diverses pureté de signal et fractions de bruit de fond \ttbar.
La majorité des événements de signal se trouve dans la catégorie \CATmediumdz.
La figure~\ref{subfig-chapter-HTT_analysis-section-categorisation-BSM-subcats-Dz} illustre ces coupures sur \Dzeta.
\paragraph{Catégories obtenues}
Les catégories \CATbsm\ correspondant à la région de signal (SR), \ie\ en dehors des régions de détermination (DR) et de contrôle (CR), sont résumées sur la figure~\ref{fig-chapter-HTT_analysis-section-categorisation-BSM-cats_recap} pour les quatre canaux considérés.
\begin{figure}[h]
\centering
\begin{tikzpicture}
\foreach \dx in {0, 6}{
    
    \draw (\dx, -3) node {\small\CATlowdz\vphantom{Àq}};
    \draw (\dx+2, -3) node {\small\CATmediumdz\vphantom{Àq}};
    \draw (\dx+4, -3) node {\small\CAThighdz\vphantom{Àq}};
    
    \draw [thick] (\dx-.925,-3-.425) rectangle  + (1.85,.85); 
    \draw [thick] (\dx+2-.925,-3-.425) rectangle  + (1.85,.85); 
    \draw [thick] (\dx+4-.925,-3-.425) rectangle  + (1.85,.85); 
    
    \foreach \ddx/\cat in {0/\CATtightmt, 3/\CATloosemt}{
        \foreach \ddy in {-1, -2}{
            
            \draw (\dx+\ddx+.5, \ddy) node {\small\cat\vphantom{Àq}};
            \draw (\dx+\ddx+.5, \ddy) node {\small\cat\vphantom{Àq}};
            
            \draw [thick] (\dx+\ddx-.925,\ddy-.425) rectangle  + (2.85,.85); 
            \draw [thick] (\dx+\ddx-.925,\ddy-.425) rectangle  + (2.85,.85); 
            
        }
    }
    
    \draw [thick] (\dx-.925,-.425) rectangle  + (5.85,.85); 
}

\draw (-1, 0) node (a) [left] {$\Higgs\to\tau\tau\to\tauh\tauh$};
\draw (a.west) + (0, -1) node [right] {$\Higgs\to\tau\tau\to\mu\tauh$};
\draw (a.west) + (0, -2) node [right] {$\Higgs\to\tau\tau\to\ele\tauh$};
\draw (a.west) + (0, -3) node [right] {$\Higgs\to\tau\tau\to\ele\mu$};

\draw (2, 1) node {\CATnobtag\vphantom{Àq}};
\draw (8, 1) node {\CATbtag\vphantom{Àq}};

\draw [very thick] (-1.25, .5) -- (11.25, .5) ;
\draw [very thick] (5, 1.5) -- (5, -3.75) ;

\end{tikzpicture}
\caption{Catégories \CATbsm\ pour les quatre canaux considérés.}
\label{fig-chapter-HTT_analysis-section-categorisation-BSM-cats_recap}
\end{figure}
\subsubsection{Variable discriminante}
La masse invariante permet d'estimer mathématiquement, par un calcul de physique relativiste, la masse d'une particule à partir des propriétés cinématiques de chacun des ses produits de désintégration.
Cette observable est donc un choix pertinent de variable discriminante.
Elle est ainsi utilisée, par exemple, dans l'analyse $\higgs\to\Zboson\Zboson\to4\ell$~\cite{CMS-PAS-HIG-13-002}.
\par
Cependant, dans l'analyse $\Higgs\to\tau\tau$, l'état final comporte deux à quatre neutrinos issus des désintégrations des leptons tau.
La figure~\ref{subfig-chapter-HTT_analysis-section-categorisation-BSM-fgraph-H-tautau_mutau} illustre le cas du canal \mu\tauh\ dans lequel trois neutrinos sont ainsi présents.
Or, les neutrinos sont invisibles dans le détecteur CMS.
Il est donc impossible de déterminer la masse invariante.
\begin{figure}[h]
\centering
\vspace{\baselineskip}

\subcaptionbox{Événement réel.\label{subfig-chapter-HTT_analysis-section-categorisation-BSM-fgraph-H-tautau_mutau}}[.45\textwidth]
{\begin{fmffile}{H-tautau_mutau}\fmfstraight
\begin{fmfchar*}(50,40)
  \fmfleft{h}
  \fmfright{nu1,nub1,tau1,tau2,nub2,nu2}
  \fmf{dashes, label=$\Hs,, \Hn,, \Ha$, l.side=left, tension=5}{h,v}
  \fmf{fermion, label=$\antitau$, l.side=left, tension=4}{t1d,v}
  \fmf{fermion, label=$\leptau$, l.side=left, tension=4}{v,t2d}
  \fmf{fermion, tension=3}{nu1,t1d}
  \fmf{boson, label=$\Wbosonplus$, l.side=left, tension=2}{t1d,W1d}
  \fmf{fermion}{tau1,W1d,nub1}
  \fmf{fermion, tension=3}{t2d,nu2}
  \fmf{boson, label=$\Wbosonminus$, tension=2}{t2d,W2d}
  \fmf{fermion}{tau2,W2d,nub2}
  \fmflabel{$\antinutau$}{nu1}
  \fmflabel{$\nutau$}{nu2}
  \fmflabel{$\antinumu$}{nub2}
  \fmflabel{$\muon$}{tau2}
  \fmflabel{$\antiquark$}{tau1}
  \fmflabel{$\quark$}{nub1}
  \fmfdot{v,t1d,t2d,W1d,W2d}
\end{fmfchar*}
\end{fmffile}
}
\hfill
\subcaptionbox{Événement approximé.\label{subfig-chapter-HTT_analysis-section-categorisation-BSM-fgraph-H-tautau_mutau-for_mTtot}}[.45\textwidth]
{\begin{fmffile}{H-tautau_mutau-for_mTtot}\fmfstraight
\begin{fmfchar*}(50,30)
  \fmfleft{h}
  \fmfright{La,NU,Lb}
  \fmf{dashes, label=$\Hs,, \Hn,, \Ha$, l.side=left,tension=2}{h,v}
  \fmf{fermion, label=$\antitau$, l.side=left}{t1d,v}
  \fmf{fermion, label=$\leptau$, l.side=left}{v,t2d}
  \fmf{phantom}{t1d,La}
  \fmf{phantom}{t2d,Lb}
  \fmf{phantom, tension=2}{t2d,m,t1d}
  \fmffreeze
  \fmf{plain}{La,t1d}
  \fmf{plain}{t2d,Lb}
  \fmf{plain}{m,NU}
  \fmflabel{$L_2$}{La}
  \fmflabel{$L_1$}{Lb}
  \fmflabel{$\sum\neutrino$}{NU}
  \fmfdot{v}
  \fmfblob{.2w}{m}
\end{fmfchar*}
\end{fmffile}
}

\caption[Diagrammes de Feynman d'un événement $\Higgs\to\tau\tau\to\mu\tauh$.]{Diagrammes de Feynman d'un événement $\Higgs\to\tau\tau\to\mu\tauh$, avec trois neutrinos dans l'état final.}
\label{fig-chapter-HTT_analysis-section-categorisation-BSM-fgraph-H-tautau_mutau-for_mTtot}
\end{figure}
\par
Toutefois, l'énergie transverse manquante, introduite dans le chapitre~\refChJERC, correspond à la somme des impulsions transverses de toutes les particules invisibles de l'événement.
Alors, sauf défaut du détecteur menant à une mauvaise reconstruction des particules visibles,
\begin{equation}
\vMET = \sum_{\nu_i} \vpT^{\nu_i}
\end{equation}
où la somme se fait sur tous les neutrinos de l'événement dont ceux de l'état final du processus $\Higgs\to\tau\tau$.
\par
Ainsi, il est possible de considérer le système à trois corps suivant:
\begin{itemize}
\item $L_1$ la partie visible de la désintégration d'un des leptons tau, par exemple le muon de la figure~\ref{subfig-chapter-HTT_analysis-section-categorisation-BSM-fgraph-H-tautau_mutau};
\item $L_2$ la partie visible de la désintégration de l'autre lepton tau, par exemple le \tauh résultant de $\Wbosonplus\to\quarku\antiquarkd$ sur la figure~\ref{subfig-chapter-HTT_analysis-section-categorisation-BSM-fgraph-H-tautau_mutau};
\item $\sum\nu\simeq\MET$ l'ensemble des neutrinos issus des désintégrations des leptons \tau, eux-mêmes issus de la désintégration du boson de Higgs, dont une estimation des propriétés cinématiques dans le plan transverse est donnée par l'énergie transverse manquante \vMET;
\end{itemize}
comme illustré figure~\ref{subfig-chapter-HTT_analysis-section-categorisation-BSM-fgraph-H-tautau_mutau-for_mTtot}.
En se restreignant donc au plan transverse car \MET\ est utilisée,
pour des particules relativistes ($m\ll E$),
le calcul de la \og masse invariante \fg{} de ce système à trois corps donne
\begin{align}
m^2 &= E^2 - p^2
= \left( \sum_{i \in \set{L_1,L_2,\sum\nu}} E_i \right)^2 - \left( \sum_{i \in \set{L_1,L_2,\sum\nu}} \vpT^i \right)^2
\nonumber\\
&= \left( E_{L_1} + E_{L_2} + \MET \right)^2 - \left( \vpT^{L_1} + \vpT^{L_2} + \vMET \right)^2
\nonumber\\
&= E_{L_1}^2 + E_{L_2}^2 + \MET^2 + 2 \left( E_{L_1}E_{L_2} + E_{L_1}\MET + E_{L_2}\MET \right)
\nonumber\\
&\hphantom{=} - \left( {\vpT^{L_1}}^2 + {\vpT^{L_2}}^2 + {\vMET}^2 \right) - 2 \left( \vpT^{L_1}\cdot\vpT^{L_2} + \vpT^{L_1}\cdot\vMET + \vpT^{L_2}\cdot\vMET \right)
\nonumber\\
&= 2 \Big[ \pT^{L_1}\pT^{L_2}(1-\cos\Delta\phi^{(L_1,L_2)})
\nonumber\\
&\hphantom{=2(} \quad + \pT^{L_1}\MET(1-\cos\Delta\phi^{(L_1,\MET)}) + \pT^{L_2}\MET(1-\cos\Delta\phi^{(L_2,\MET)}) \Big]
\nonumber\\
&= \mT^2 (L_1,L_2) + \mT^2 (L_1,\MET) + \mT^2 (L_2,\MET) = {\mTtot}^2
\mend
\end{align}
\par
La variable discriminante utilisée dans les catégories \CATbsm\ est ainsi \mTtot, définie telle que
\begin{equation}
\mTtot = \sqrt{\mT^2 (L_1,L_2) + \mT^2 (L_1,\MET) + \mT^2 (L_2,\MET)}
\label{eq-def_mTtot}
\end{equation}
avec
\begin{equation}
\mT (A,B) = \sqrt{2 \, \pT^{A} \, \pT^{B} \, (1-\cos\Delta\phi^{(A,B)})}
\mend
\end{equation}
\par
À titre d'illustration, les distributions obtenues pour la catégorie \CATbtag\ \CATtightmt\ du canal \mu\tauh\ et \CATbtag\ \CATmediumdz\ du canal \ele\mu\ en 2017 sont représentées en figure~\ref{fig-mTtot_distribs_exemple}.
\begin{figure}[h]
\centering

\subcaptionbox{Canal \mu\tauh, catégorie \CATbtag\ \CATtightmt.}[.475\textwidth]
{\plotHTTshapes{mt_tot}{mssm_classic}{2017}{mt}{35}{prefit_linear_nosignal}}
\hfill
\subcaptionbox{Canal \ele\mu, catégorie \CATbtag\ \CATmediumdz.}[.475\textwidth]
{\plotHTTshapes{mt_tot}{mssm_classic}{2017}{em}{36}{prefit_linear_nosignal}}

\caption[Distributions de \mTtot\ en 2017 pour deux catégories et canaux.]{Distributions de \mTtot\ en 2017 pour deux catégories et canaux.}
\label{fig-mTtot_distribs_exemple}
\end{figure}