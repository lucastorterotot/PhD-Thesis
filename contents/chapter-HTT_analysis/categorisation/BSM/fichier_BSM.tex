\subsection{Région \og BSM \fg{}}\label{chapter-HTT_analysis-section-categorisation-BSM}
Les catégories BSM, introduites dans la référence~\cite{CMS-PAS-HIG-17-020}, sont construites dans le but de chercher une résonance correspondant à un boson de Higgs lourd.
Les coupures présentées dans cette section sont appliquées en plus des coupures permettant de séparer les régions SM et BSM.
\par
Une première catégorisation est basée sur la présence de jets issus de quarks~\quarkb.
Deux catégories sont ainsi définies:
\begin{itemize}
\item \CATnobtag: $\Nbtag =0$;
\item \CATbtag: $\Nbtag\geq1$.
\end{itemize}
Dans le cas des canaux \mu\tauh, \ele\tauh\ et \ele\mu, chacune de ces deux catégories est à nouveau subdivisée.
\paragraph{Canaux \mu\tauh\ et \ele\tauh}
Dans ces deux canaux, la masse transverse de $L_1$ (le muon ou l'électron, notés $\ell$) définie par
\begin{equation}
\mT^{(\ell)} = \sqrt{2 \, \pT^{(\ell)} \, \MET \, (1-\cos\Delta\phi)} \label{eq-mT_def-ell}
\end{equation}
avec $\Delta\phi = \phi^{(\ell)} - \phi^{(\MET)}$
est utilisée afin de définir deux catégories:
\begin{itemize}
\item \CATtightmt: $\mT^{(\ell)} < \SI{40}{\GeV}$;
\item \CATloosemt: $\SI{40}{\GeV} \leq \mT^{(\ell)} < \SI{70}{\GeV}$;
\end{itemize}
la limite haute sur \mT\ pour la catégorie \CATloosemt\ étant appliquée afin de s'assurer que la région de signal soit orthogonale à la région de détermination des facteurs de faux des événements $\Wboson+\text{jets}$. Les facteurs de faux sont abordés dans la section~\ref{chapter-HTT_analysis-section-bg_estimation-FF_method}.
\begin{wrapfigure}{R}{.45\textwidth}
\centering

\begin{tikzpicture}
%% base
\draw [->] (0,0)--(2,0) node [right] {$\bvec_x$};
\draw [->] (0,0)--(0,2) node [above] {$\bvec_y$};

\def\xMET{-1.5}
\def\yMET{-1.25}
\def\xELE{2}
\def\yELE{3}
\def\xMU{.5}
\def\yMU{-2}

\draw [thick, -latex, ltcolorred] (0,0) -- (\xMET,\yMET) coordinate (vMET);
\draw [ltcolorred] (vMET) node [below] {\vMET};
\draw [thick, -latex, ltcolorblue] (0,0) -- (\xELE,\yELE) coordinate (vE);
\draw [ltcolorblue] (vE) node [right] {$\vpT (\ele)$};
\draw [thick, -latex, ltcolorblue] (0,0) -- (\xMU,\yMU) coordinate (vM);
\draw [ltcolorblue] (vM) node [right] {$\vpT (\mu)$};

\draw [dashed, -latex] ({-\xELE+-\xMU}, {-\yELE+-\yMU}) -- ({1.25*(\xELE+\xMU)}, {1.25*(\yELE+\yMU)}) node [above] {$\vec{\zeta}$};

\draw [thick, ltcolorgreen, -latex] (0,0) -- ({\xELE+\xMU}, {\yELE+\yMU}) coordinate (vVIS);
\draw [ltcolorgreen] (vVIS) node [below right] {$p_\zeta^\text{vis}\hat{\zeta}$};

\draw [thick, -latex, ltcolorred4] (0,0) --+ ({180+acos((\xELE+\xMU)/(((\xELE+\xMU)*(\xELE+\xMU)+(\yELE+\yMU)*(\yELE+\yMU))^(0.5)))}:{(\xMET*\xMET+\yMET*\yMET)^(0.5)*cos(180-acos((\xELE+\xMU)/(((\xELE+\xMU)*(\xELE+\xMU)+(\yELE+\yMU)*(\yELE+\yMU))^(0.5)))-acos((\xMET)/((\xMET*\xMET+\yMET*\yMET)^(0.5))))})  coordinate (vMETzeta) ;
\draw [ltcolorred4] (vMETzeta) node [above] {$p_\zeta^\text{miss}\hat{\zeta}$};

\draw [dotted] (vMET) -- (vMETzeta);
\draw [dotted] (vE) -- (vVIS);
\draw [dotted] (vM) -- (vVIS);

\end{tikzpicture}
\caption{Illustration de la définition de $\hat{\zeta}$~\cite{Jang_thesis}. Le plan de ce schéma est le plan transverse.}
\label{fig-zeta_illustration}
\end{wrapfigure}
\paragraph{Canal \ele\mu}
La variable \Dzeta\ est définie selon
\begin{equation}
\Dzeta = p_\zeta^\text{miss} - \num{0.85} p_\zeta^\text{vis}
\label{eq-Dzeta_def}
\end{equation}
avec
\begin{equation}
p_\zeta^\text{miss} = \vMET \cdot \hat{\zeta}
\msep
p_\zeta^\text{vis} = \left( \vpT^{\ele} + \vpT^{\mu} \right) \cdot \hat{\zeta}
\end{equation}
où $\hat{\zeta}$ est la direction bisectionnelle entre l'électron et le muon dans le plan transverse~\cite{Jang_thesis}, comme illustré sur la figure~\ref{fig-zeta_illustration}.