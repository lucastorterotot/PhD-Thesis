\section{Catégorisation des événements et variables discriminantes}\label{chapter-HTT_analysis-section-categorisation}
Afin d'augmenter la sensibilité de l'analyse à un signal particulier, il est possible de définir des catégories.
Par exemple, comme exposé dans le chapitre~\refChMSSM, le mode de production dominant des bosons $\Higgs$ et $\HiggsA$ peut être celui en association avec des quarks~\quarkb.
Le signal correspondant, dans ce cas, se trouve dans les événements comportant des jets issus de quarks~\quarkb.
Il est donc pertinent de séparer les événements en deux groupes, avec et sans jets de quarks~\quarkb.
\par
Des catégories conçues pour la recherche de $\Higgs$ et $\HiggsA$ ont déjà été exploitées avec les données enregistrées en 2016 par l'expérience CMS~\cite{CMS-PAS-HIG-17-020}.
Ces catégories utilisées dans le cadre du MSSM, \ie\ au-delà du modèle standard (\emph{Beyond Standard Model}), sont notées \og BSM \fg{} et sont détaillées dans la section~\ref{chapter-HTT_analysis-section-categorisation-BSM} ci-après.
La recherche d'un signal supplémentaire, indépendamment de tout modèle, est réalisée avec ces catégories.
\par
En plus de la recherche d'un signal correspondant à de nouvelles particules, il est possible d'exploiter les signaux de particules déjà connues afin de tester la validité d'un modèle.
En effet, les modèles supersymétriques tels que le MSSM doivent en premier lieu être compatibles avec les propriétés des particules déjà connues, comme \higgs.
Pour obtenir une meilleure sensibilité au boson de Higgs du modèle standard \higgs, il est possible de combiner les catégories BSM avec un jeu de catégories issu de l'analyse des événements $\higgs\to\tau\tau$ dans le cadre du modèle standard~\cite{CMS-PAS-HIG-19-010,CMS-NOTE-2019-177,CMS-NOTE-2019-178}.
Il s'agit des catégories \og SM \fg, présentées dans la section~\ref{chapter-HTT_analysis-section-categorisation-SM}.
La combinaison des catégories SM et BSM est détaillée dans la section~\ref{chapter-HTT_analysis-section-categorisation-SM_and_BSM}.
\par
Afin de séparer signal et bruit de fond dans ces catégories, il est nécessaire de définir une variable discriminante.
Celle-ci peut être différente d'une catégorie à l'autre.
Les variables discriminantes utilisées dans les catégories BSM et SM sont définies dans les sections correspondantes.

\subsection{Catégories \og BSM \fg{}}\label{chapter-HTT_analysis-section-categorisation-BSM}
\subsubsection{Définition des catégories}
Les catégories \CATbsm, introduites dans la référence~\cite{CMS-PAS-HIG-17-020}, sont construites dans le but de chercher une résonance correspondant à un boson de Higgs lourd.
\par
Une première catégorisation est basée sur la présence de jets issus de quarks~\quarkb.
Deux catégories sont ainsi définies:
\begin{itemize}
\item \CATnobtag: $\Nbtag =0$;
\item \CATbtag: $\Nbtag\geq1$.
\end{itemize}
Dans le cas des canaux \mu\tauh, \ele\tauh\ et \ele\mu, chacune de ces deux catégories est à nouveau subdivisée.
\paragraph{Canaux \mu\tauh\ et \ele\tauh}
Dans ces deux canaux, la masse transverse de $L_1$ (le muon ou l'électron, notés $\ell$),
définie par
\begin{equation}
\mT^{\ell} = \mT(\ell, \MET) = \sqrt{2 \, \pT^{\ell} \, \MET \, (1-\cos\Delta\phi)} \label{eq-mT_def-ell}
\end{equation}
avec $\Delta\phi = \phi^{\ell} - \phi^{\MET}$,
est utilisée afin de définir deux catégories:
\begin{itemize}
\item \CATtightmt: $\mT^{\ell} < \SI{40}{\GeV}$;
\item \CATloosemt: $\SI{40}{\GeV} \leq \mT^{\ell} < \SI{70}{\GeV}$;
\end{itemize}
la limite haute sur \mT\ pour la catégorie \CATloosemt\ étant appliquée afin de s'assurer que la région de signal soit orthogonale à la région de détermination (DR) des facteurs de faux des événements $\Wboson+\text{jets}$.
%Les facteurs de faux sont abordés dans la section~\ref{chapter-HTT_analysis-section-bg_estimation-FF_method}.
La majorité des événements de signal, en particulier pour \Higgs\ et \HiggsA\ de basse masse, se trouve dans la catégorie \CATtightmt.
La catégorie \CATloosemt\ permet quant à elle d'augmenter l'acceptation du signal pour $m_{\Higgs,\HiggsA} > \SI{700}{\GeV}$.
La figure~\ref{subfig-chapter-HTT_analysis-section-categorisation-BSM-subcats-mT} illustre ces coupures sur $\mT^{\ell}$ dans le cas du canal \ele\tauh\ pour l'année 2016.
\begin{figure}[h]
\centering

\subcaptionbox{Catégorisation basée sur $\mT^{\ell}$.\label{subfig-chapter-HTT_analysis-section-categorisation-BSM-subcats-mT}}[.475\textwidth]
{\plotHTTcontrolCATmt{2016}{emb_ff}{et}}
\hfill
\subcaptionbox{Catégorisation basée sur \Dzeta.\label{subfig-chapter-HTT_analysis-section-categorisation-BSM-subcats-Dz}}[.475\textwidth]
{\plotHTTcontrolCATdz{2016}{emb_ff}{em}}

\caption[Illustrations des catégorisations basées sur $\mT^{\ell}$ et \Dzeta]{Illustrations des catégorisations basées sur $\mT^{\ell}$ et \Dzeta, respectivement sur les événements des canaux \ele\tauh\ et \ele\mu\ de l'année 2016.}
\label{fig-chapter-HTT_analysis-section-categorisation-BSM-subcats}
\end{figure}
\paragraph{Canal \ele\mu}
Trois catégories sont définies selon la valeur de \Dzeta\ définie équation~\eqref{eq-Dzeta_def}:
\begin{itemize}
\item \CATlowdz: $-\SI{35}{\GeV} \leq \Dzeta < -\SI{10}{\GeV}$;
\item \CATmediumdz: $-\SI{10}{\GeV} \leq \Dzeta < \SI{30}{\GeV}$;
\item \CAThighdz: $\SI{30}{\GeV} \leq \Dzeta$;
\end{itemize}
la limite basse sur \Dzeta\ pour la catégorie \CATlowdz\ étant appliquée afin de s'assurer que la région de signal soit orthogonale à la région de contrôle (CR) du bruit de fond \ttbar.
Ces trois catégories permettent d'obtenir diverses puretés de signal et fractions de bruit de fond \ttbar.
La majorité des événements de signal se trouve dans la catégorie \CATmediumdz.
La figure~\ref{subfig-chapter-HTT_analysis-section-categorisation-BSM-subcats-Dz} illustre ces coupures sur \Dzeta.
\paragraph{Catégories obtenues}
Les catégories \CATbsm\ correspondant à la région de signal (SR), \ie\ en dehors des régions de détermination (DR) et de contrôle (CR), sont résumées sur la figure~\ref{fig-chapter-HTT_analysis-section-categorisation-BSM-cats_recap} pour les quatre canaux considérés.
\begin{figure}[h]
\centering
\begin{tikzpicture}
\foreach \dx in {0, 6}{
    
    \draw (\dx, -3) node {\small\CATlowdz\vphantom{Àq}};
    \draw (\dx+2, -3) node {\small\CATmediumdz\vphantom{Àq}};
    \draw (\dx+4, -3) node {\small\CAThighdz\vphantom{Àq}};
    
    \draw [thick] (\dx-.925,-3-.425) rectangle  + (1.85,.85); 
    \draw [thick] (\dx+2-.925,-3-.425) rectangle  + (1.85,.85); 
    \draw [thick] (\dx+4-.925,-3-.425) rectangle  + (1.85,.85); 
    
    \foreach \ddx/\cat in {0/\CATtightmt, 3/\CATloosemt}{
        \foreach \ddy in {-1, -2}{
            
            \draw (\dx+\ddx+.5, \ddy) node {\small\cat\vphantom{Àq}};
            \draw (\dx+\ddx+.5, \ddy) node {\small\cat\vphantom{Àq}};
            
            \draw [thick] (\dx+\ddx-.925,\ddy-.425) rectangle  + (2.85,.85); 
            \draw [thick] (\dx+\ddx-.925,\ddy-.425) rectangle  + (2.85,.85); 
            
        }
    }
    
    \draw [thick] (\dx-.925,-.425) rectangle  + (5.85,.85); 
}

\draw (-1, 0) node (a) [left] {$\Higgs\to\tau\tau\to\tauh\tauh$};
\draw (a.west) + (0, -1) node [right] {$\Higgs\to\tau\tau\to\mu\tauh$};
\draw (a.west) + (0, -2) node [right] {$\Higgs\to\tau\tau\to\ele\tauh$};
\draw (a.west) + (0, -3) node [right] {$\Higgs\to\tau\tau\to\ele\mu$};

\draw (2, 1) node {\CATnobtag\vphantom{Àq}};
\draw (8, 1) node {\CATbtag\vphantom{Àq}};

\draw [very thick] (-1.25, .5) -- (11.25, .5) ;
\draw [very thick] (5, 1.5) -- (5, -3.75) ;

\end{tikzpicture}
\caption{Catégories \CATbsm\ pour les quatre canaux considérés.}
\label{fig-chapter-HTT_analysis-section-categorisation-BSM-cats_recap}
\end{figure}
\subsubsection{Variable discriminante}
La masse invariante d'une particule peut être obtenue par un calcul de relativité restreinte à partir des propriétés cinématiques de chacun des ses produits de désintégration.
Cette observable doit correspondre à la masse de cette particule et est donc un choix pertinent de variable discriminante.
Elle est ainsi utilisée, par exemple, dans l'analyse $\higgs\to\Zboson\Zboson\to4\ell$~\cite{CMS-PAS-HIG-13-002}.
\par
Cependant, dans l'analyse menée ici, l'état final comporte deux à quatre neutrinos issus des désintégrations des leptons~\tau.
La figure~\ref{subfig-chapter-HTT_analysis-section-categorisation-BSM-fgraph-H-tautau_mutau} illustre le cas du canal \mu\tauh\ dans lequel trois neutrinos sont ainsi présents.
Or, les neutrinos sont invisibles dans le détecteur CMS.
Il est donc impossible de déterminer la masse invariante totale de ce système.
\begin{figure}[h]
\centering
\vspace{\baselineskip}

\subcaptionbox{Événement réel.\label{subfig-chapter-HTT_analysis-section-categorisation-BSM-fgraph-H-tautau_mutau}}[.45\textwidth]
{\begin{fmffile}{H-tautau_mutau}\fmfstraight
\begin{fmfchar*}(50,40)
  \fmfleft{h}
  \fmfright{nu1,nub1,tau1,tau2,nub2,nu2}
  \fmf{dashes, label=$\Hs,, \Hn,, \Ha$, l.side=left, tension=5}{h,v}
  \fmf{fermion, label=$\antitau$, l.side=left, tension=4}{t1d,v}
  \fmf{fermion, label=$\leptau$, l.side=left, tension=4}{v,t2d}
  \fmf{fermion, tension=3}{nu1,t1d}
  \fmf{boson, label=$\Wbosonplus$, l.side=left, tension=2}{t1d,W1d}
  \fmf{fermion}{tau1,W1d,nub1}
  \fmf{fermion, tension=3}{t2d,nu2}
  \fmf{boson, label=$\Wbosonminus$, tension=2}{t2d,W2d}
  \fmf{fermion}{tau2,W2d,nub2}
  \fmflabel{$\antinutau$}{nu1}
  \fmflabel{$\nutau$}{nu2}
  \fmflabel{$\antinumu$}{nub2}
  \fmflabel{$\muon$}{tau2}
  \fmflabel{$\antiquark$}{tau1}
  \fmflabel{$\quark$}{nub1}
  \fmfdot{v,t1d,t2d,W1d,W2d}
\end{fmfchar*}
\end{fmffile}
\vspace{\baselineskip}}
\hfill
\subcaptionbox{Événement approximé.\label{subfig-chapter-HTT_analysis-section-categorisation-BSM-fgraph-H-tautau_mutau-for_mTtot}}[.45\textwidth]
{\begin{fmffile}{H-tautau_mutau-for_mTtot}\fmfstraight
\begin{fmfchar*}(50,30)
  \fmfleft{h}
  \fmfright{La,NU,Lb}
  \fmf{dashes, label=$\Hs,, \Hn,, \Ha$, l.side=left,tension=2}{h,v}
  \fmf{fermion, label=$\antitau$, l.side=left}{t1d,v}
  \fmf{fermion, label=$\leptau$, l.side=left}{v,t2d}
  \fmf{phantom}{t1d,La}
  \fmf{phantom}{t2d,Lb}
  \fmf{phantom, tension=2}{t2d,m,t1d}
  \fmffreeze
  \fmf{plain}{La,t1d}
  \fmf{plain}{t2d,Lb}
  \fmf{plain}{m,NU}
  \fmflabel{$L_2$}{La}
  \fmflabel{$L_1$}{Lb}
  \fmflabel{$\sum\neutrino$}{NU}
  \fmfdot{v}
  \fmfblob{.2w}{m}
\end{fmfchar*}
\end{fmffile}
\vspace{\baselineskip}}

\caption[Diagrammes de Feynman d'un événement $\Higgs\to\tau\tau\to\mu\tauh$.]{Diagrammes de Feynman d'un événement $\Higgs\to\tau\tau\to\mu\tauh$, avec trois neutrinos dans l'état final.}
\label{fig-chapter-HTT_analysis-section-categorisation-BSM-fgraph-H-tautau_mutau-for_mTtot}
\end{figure}
\par
L'énergie transverse manquante, introduite dans le chapitre~\refChLHCCMS, correspond à la somme des impulsions transverses de toutes les particules invisibles de l'événement.
Alors, pour un détecteur sans défaut dont l'acceptation spatiale est totale,
\ie\ avec une reconstruction parfaite de toutes les particules hormis les neutrinos,
\begin{equation}
\vMET = \sum_{\nu_i} \vpT^{\nu_i}
\end{equation}
où la somme se fait sur tous les neutrinos de l'événement dont ceux de l'état final du processus $\Higgs\to\tau\tau$.
\par
Ainsi, il est possible de considérer le système à trois corps suivant:
\begin{itemize}
\item $L_1$ la partie visible de la désintégration d'un des leptons~\tau, par exemple le muon de la figure~\ref{subfig-chapter-HTT_analysis-section-categorisation-BSM-fgraph-H-tautau_mutau};
\item $L_2$ la partie visible de la désintégration de l'autre lepton~\tau, par exemple le \tauh résultant de $\Wbosonplus\to\quarku\antiquarkd$ sur la figure~\ref{subfig-chapter-HTT_analysis-section-categorisation-BSM-fgraph-H-tautau_mutau};
\item $\sum\nu\simeq\MET$ l'ensemble des neutrinos issus des désintégrations des leptons~\tau, eux-mêmes issus de la désintégration du boson de Higgs, dont une estimation des propriétés cinématiques dans le plan transverse est donnée par l'énergie transverse manquante \vMET;
\end{itemize}
comme illustré figure~\ref{subfig-chapter-HTT_analysis-section-categorisation-BSM-fgraph-H-tautau_mutau-for_mTtot}.
En se restreignant donc au plan transverse car \MET\ est utilisée,
pour des particules relativistes ($m\ll E$),
le calcul de la \og masse invariante \fg{} de ce système à trois corps donne
\begin{align}
m^2 &= E^2 - p^2
= \left( \sum_{i \in \set{L_1,L_2,\sum\nu}} E_i \right)^2 - \left( \sum_{i \in \set{L_1,L_2,\sum\nu}} \vpT^i \right)^2
\nonumber\\
&= \left( E_{L_1} + E_{L_2} + \MET \right)^2 - \left( \vpT^{L_1} + \vpT^{L_2} + \vMET \right)^2
\nonumber\\
&= E_{L_1}^2 + E_{L_2}^2 + \MET^2 + 2 \left( E_{L_1}E_{L_2} + E_{L_1}\MET + E_{L_2}\MET \right)
\nonumber\\
&\hphantom{=} - \left( {\vpT^{L_1}}^2 + {\vpT^{L_2}}^2 + {\vMET}^2 \right) - 2 \left( \vpT^{L_1}\cdot\vpT^{L_2} + \vpT^{L_1}\cdot\vMET + \vpT^{L_2}\cdot\vMET \right)
\nonumber\\
&= 2 \Big[ \pT^{L_1}\pT^{L_2}(1-\cos\Delta\phi^{(L_1,L_2)})
\nonumber\\
&\hphantom{=2(} \quad + \pT^{L_1}\MET(1-\cos\Delta\phi^{(L_1,\MET)}) + \pT^{L_2}\MET(1-\cos\Delta\phi^{(L_2,\MET)}) \Big]
\nonumber\\
&= \mT^2 (L_1,L_2) + \mT^2 (L_1,\MET) + \mT^2 (L_2,\MET) = {\mTtot}^2
\mend
\end{align}
\par
La variable discriminante utilisée dans les catégories \CATbsm\ est ainsi \mTtot,
la masse transverse totale,
définie telle que
\begin{equation}
\mTtot = \sqrt{\mT^2 (L_1,L_2) + \mT^2 (L_1,\MET) + \mT^2 (L_2,\MET)}
\label{eq-def_mTtot}
\end{equation}
avec
\begin{equation}
\mT (A,B) = \sqrt{2 \, \pT^{A} \, \pT^{B} \, (1-\cos\Delta\phi^{(A,B)})}
\mend
\end{equation}
\par
À titre d'illustration, les distributions obtenues pour la catégorie \CATbtag\ \CATtightmt\ du canal \mu\tauh\ et \CATbtag\ \CATmediumdz\ du canal \ele\mu\ en 2017 sont représentées en figure~\ref{fig-mTtot_distribs_exemple}.
\begin{figure}[h]
\centering

\subcaptionbox{Canal \mu\tauh, catégorie \CATbtag\ \CATtightmt.}[.475\textwidth]
{\plotHTTshapes{mt_tot}{mssm_classic}{2017}{mt}{35}{prefit_linear_nosignal}}
\hfill
\subcaptionbox{Canal \ele\mu, catégorie \CATbtag\ \CATmediumdz.}[.475\textwidth]
{\plotHTTshapes{mt_tot}{mssm_classic}{2017}{em}{36}{prefit_linear_nosignal}}

\caption[Distributions de \mTtot\ en 2017 pour deux catégories et canaux.]{Distributions de \mTtot\ en 2017 pour deux catégories et canaux. Les données observées (points noirs) sont comparées à la modélisation des bruits de fond (histogrammes remplis en couleur et empilés). Les bandes grisées correspondent à l'incertitude totale (statistique et systématique) sur le bruit de fond avant ajustement des paramètres de nuisance par \COMBINE. Le rapport au bruit de fond est donné dans la partie inférieure des graphiques.}
\label{fig-mTtot_distribs_exemple}
\end{figure}
\subsection{Région \og SM \fg{}}\label{chapter-HTT_analysis-section-categorisation-SM}
Les catégories SM, introduites dans la référence~\cite{CMS-PAS-HIG-19-010}, sont construites dans le but d'étudier le boson de Higgs du modèle standard, noté ici $\higgs$, avec une masse de \SI{125}{\GeV}.
Les coupures présentées dans cette section sont appliquées en plus des coupures permettant de séparer les régions SM et BSM.
\par %AN 2019/177 and 178
La catégorisation SM est basée sur un réseau de neurones.
Les réseaux de neurones sont abordés plus en détails dans le chapitre~\ifref{chapter-ML}{\ref{chapter-ML}}{6}.
Les caractéristiques du réseau utilisé ici sont données à titre informatif.
\par
Les variables d'entrée du réseau sont:
\begin{itemize}
\item les impulsions transverses des éléments du \emph{dilepton};
\item les impulsions transverses des deux principaux jets de l'événement;
\item le nombre de jets \Njets;
\item le nombre de jets de quarks~\quarkb\ \Nbjets;
\item la masse invariante du système des deux jets principaux \mjj;
\item la distance dans le plan $(\eta,\phi)$ entre les deux jets principaux \Detajj;
\item l'impulsion transverse totale des deux principaux jets de l'événement;
\item la masse du \emph{dilepton} estimée par \SVFIT, \msv;
\item la masse invariante du \emph{dilepton}, \mvis;
\item l'impulsion transverse du \emph{dilepton}, \pTvis.
\end{itemize}
\todo{mTdileptonMET for em + MELA output}
\par
Le réseau est constitué de deux couches cachées de 200 neurones chacune, complètement connectées.
Leur fonction d'activation est la tangente hyperbolique.
Enfin, la couche de sortie du réseau comporte plusieurs neurones.
Ce réseau fournit donc un vecteur, et non un scalaire, permettant une catégorisation plus poussée qu'une simple discrimination signal ou bruit de fond.

%The task of the neural net is to define regions of the phase space that are dominated by certain
%processes. The resulting boundaries between the categories, meaning the region where events
%are assigned similar and therefore low probabilities for two categories, are actually rather de-
%liberate and depend very much on the training parameters e.g. additional class weights. A
%variation of the parameters would cause migrations of these events between the categories.
%Therefore, it is important to use the actual values of the output nodes for further discrimina-
%tion and to sort events by it. Here we use the value of the largest probability, which was used to
%assign the category, for further discrimination within the same category. We call this quantity
%exclusive probability or just the NN score.
%
%Minor background processes like electroweak Z production or di-boson events which provide
%only small event numbers are collected in one category for remaining processes called ”misc”.
%Individual categories would raise problems to the training due to the small event numbers. In
%this collection they are at least respected in the training of the neural network such that they
%can’t migrate to the signal categories without increasing the training loss. Since there are no
%leptons in the τ h τ h decay channel, the yields of the Z → ll, tt̄, W + jets processes are rather
%small as well which is the reason that there are no dedicated categories in this channel. Instead
%these processes are associated with the misc category as well. In summary, the misc category
%covers Z → ll and W + jets in the eμ channel, di-Boson, single top and EWKZ in the eτ h and
%μτ h channel and Z → ll, tt̄, W + jets, di-Boson, single top and EWKZ in the τ h τ h channel.

et xxh
et tt
et zll
et misc
et emb
et ff

mt xxh
mt tt
mt zll
mt misc
mt emb
mt ff

tt xxh
tt emb
tt ff
tt misc

em xxh
em tt
em ss
em misc
em db
em emb
\subsection{Combinaison des catégories SM et BSM}\label{chapter-HTT_analysis-section-categorisation-SM_and_BSM}
Les catégories \CATbsm\ sont définies pour être sensibles au signal de \Higgs\ et \HiggsA,
tandis que les \CATsm\ le sont pour celui de \higgs.
L'utilisation combinée des catégories \CATsm\ et \CATbsm\ rend alors l'analyse plus sensible aux propriétés du boson de Higgs du modèle standard \higgs\ par rapport à l'utilisation des catégories \CATbsm\ uniquement.
\begin{wrapfigure}{R}{7cm}
\centering
\begin{tikzpicture}
\def\gwidth{5}
\def\gheight{\gwidth}

\draw [thick, -latex] (0,0) --+ (0, \gheight + 0.5) node [right] {\mCutForCategories\ (\SI{}{\GeV})};

\fill [ltcolorblue1] (0,0) rectangle (\gwidth, \gheight);
\fill [ltcolorred1] (0,0) rectangle (\gwidth/2, \gheight/2);
\draw [ultra thick] (0,0) rectangle (\gwidth, \gheight);
\draw [thick] (\gwidth/2, 0) --+ (0, \gheight);
\draw [thick] (0, \gheight/2) --+ (\gwidth/2, 0);

\draw (\gwidth/4, 0) node [below] {$\Nbtag = 0$};
\draw ({3*\gwidth/4}, 0) node [below] {$\Nbtag \geq 1$};

\draw (0,0) node [left] {\num{0}};
\draw (0,\gheight/2) node [left] {\num{250}};
\draw (0,\gheight) node [left] {$\infty$};

\draw (\gwidth/4, \gheight/4) node {\og SM \fg{}};
\draw (\gwidth/4, {3*\gheight/4}) node {\og BSM \fg{}};
\draw ({3*\gwidth/4}, \gheight/2) node {\og BSM \fg{}};
\end{tikzpicture}
\caption{Définition des deux régions utilisant des catégories différentes.}
\label{fig-chapter-HTT_analysis-section-categorisation-SM_BSM_diagram}
\end{wrapfigure}
\par
Cette catégorisation combinée est une innovation importante et non triviale par rapport à la catégorisation classique \og \CATbsm\ uniquement \fg{} utilisée dans les précédentes analyses $\Higgs\to\tau\tau$ dans le cadre du MSSM~\cite{CMS-PAS-HIG-13-021,CMS-PAS-HIG-14-029,CMS-PAS-HIG-17-020}.
En effet, les propriétés de \higgs\ sont modifiées dans le MSSM par rapport au modèle standard, comme exposé dans le chapitre~\refChMSSM.
La modélisation de \higgs\ dans le cadre du MSSM est décrite dans la section~\ref{chapter-HTT_analysis-section-signal_extraction}.
La complémentarité de la recherche du signal de \Higgs\ et \HiggsA\ avec le test des propriétés de \higgs\ permet donc d'obtenir de plus fortes contraintes sur les modèles testés, comme cela a déjà été constaté dans des travaux récents~\cite{Artur_thesis}.
\par
Afin d'éviter tout recouvrement entre les catégories \CATsm\ et \CATbsm\ lors de leur utilisation combinée, deux régions sont définies, chacune utilisant les catégories correspondantes.
La région \CATsm\ concerne les événements ne comportant pas de jets issus de quark~\quarkb\ ($\Nbtag=0$) et tels que $\mCutForCategories < \SI{250}{\GeV}$ où
\mCutForCategoriesdef.
La région \CATbsm, quant à elle, concerne les événements contenant des jets issus de quark~\quarkb\ ($\Nbtag\geq1$) ou tels que $\mCutForCategories \geq \SI{250}{\GeV}$.
Les deux régions ainsi obtenues ne se recouvrent pas et peuvent se résumer selon le schéma de la figure~\ref{fig-chapter-HTT_analysis-section-categorisation-SM_BSM_diagram}.
%\par
%À titre d'illustration, les distributions obtenues avec et sans les coupures de séparation des catégories \CATsm\ pour la catégorie \CATnobtag\ du canal \tauh\tauh\ en 2017 sont représentées en figure~\ref{fig-mTtot_distribs_exemple-SM_sep}.
%\begin{figure}[h]
%\centering
%\subcaptionbox{Sans coupure sur $(\mCutForCategories, \Nbtag)$.}[.475\textwidth]
%{\plotHTTshapes{mt_tot}{mssm_classic}{2017}{tt}{32}{prefit_linear_nosignal}}
%\hfill
%\subcaptionbox{Avec coupure sur $(\mCutForCategories, \Nbtag)$.}[.475\textwidth]
%{\plotHTTshapes{mt_tot}{mssm_vs_sm_h125}{2017}{tt}{32}{prefit_linear_nosignal}}
%
%\caption[Distributions de \mTtot\ pour le canal \tauh\tauh\ en 2017 dans la catégorie \CATnobtag.]{Distributions de \mTtot\ pour le canal \tauh\tauh\ en 2017 dans la catégorie \CATnobtag. Afin de combiner les catégories \CATbsm\ avec les catégories \CATsm\ présentées section~\ref{chapter-HTT_analysis-section-categorisation-SM}, des coupures sont appliquées sur $(\mCutForCategories, \Nbtag)$ telles que $\mCutForCategories \geq \SI{250}{\GeV}$ ou $\Nbtag\geq1$. Ici, seule la sélection sur \mCutForCategories\ joue donc un rôle.}
%\label{fig-mTtot_distribs_exemple-SM_sep}
%\end{figure}