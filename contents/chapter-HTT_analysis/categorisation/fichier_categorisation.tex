%% TMP
\newpage
%% TMP

\section{Catégorisation des événements}\label{chapter-HTT_analysis-section-categorisation}
Afin d'augmenter la sensibilité de l'analyse à un signal particulier, il est possible de définir des catégories.
Par exemple, comme exposé dans le chapitre~\ifref{chapter-MS-MSSM}{\ref{chapter-MS-MSSM}}{2}, le mode de production dominant des bosons $\Higgs$ et $\HiggsA$ peut être celui en association avec des quarks~\quarkb.
Le signal correspondant, dans ce cas, se trouve dans les événements comportant des jets issus de quarks~\quarkb.
Il est donc pertinent de séparer les événements en deux groupes, avec et sans jets de quarks~\quarkb.
\par
La recherche d'un boson de Higgs supplémentaire de haute masse se désintégrant en paire de taus a été menée avec les données enregistrées en 2016 par l'expérience CMS~\cite{CMS-PAS-HIG-17-020}.
Les catégories y étant utilisées, définies dans la section~\ref{chapter-HTT_analysis-section-categorisation-BSM} ci-après, sont également exploitées dans la présente analyse.
Cette catégorisation utilisée dans le cadre du MSSM, \ie\ au-delà du modèle standard (\emph{Beyond Standard Model}), est notée \og BSM \fg{}.
Une autre catégorisation des événements est utilisée pour l'analyse des événements $\higgs\to\tau\tau$ dans le cadre du modèle standard~\cite{CMS-PAS-HIG-19-010}.
Elle est notée \og SM \fg{} et est présentée dans la section~\ref{chapter-HTT_analysis-section-categorisation-SM}.
\begin{wrapfigure}{R}{7cm}
\centering
\begin{tikzpicture}
\def\gwidth{5}
\def\gheight{\gwidth}

\draw [thick, -latex] (0,0) --+ (0, \gheight + 0.5) node [right] {\mCutForCategories\ (\SI{}{\GeV})};

\fill [ltcolorblue1] (0,0) rectangle (\gwidth, \gheight);
\fill [ltcolorred1] (0,0) rectangle (\gwidth/2, \gheight/2);
\draw [ultra thick] (0,0) rectangle (\gwidth, \gheight);
\draw [thick] (\gwidth/2, 0) --+ (0, \gheight);
\draw [thick] (0, \gheight/2) --+ (\gwidth/2, 0);

\draw (\gwidth/4, 0) node [below] {$\Nbtag = 0$};
\draw ({3*\gwidth/4}, 0) node [below] {$\Nbtag \geq 1$};

\draw (0,0) node [left] {\num{0}};
\draw (0,\gheight/2) node [left] {\num{250}};
\draw (0,\gheight) node [left] {$\infty$};

\draw (\gwidth/4, \gheight/4) node {\og SM \fg{}};
\draw (\gwidth/4, {3*\gheight/4}) node {\og BSM \fg{}};
\draw ({3*\gwidth/4}, \gheight/2) node {\og BSM \fg{}};
\end{tikzpicture}
\caption{Définition des deux régions utilisant des catégories différentes.}
\label{fig-chapter-HTT_analysis-section-categorisation-SM_BSM_diagram}
\end{wrapfigure}
\par
Deux régions sont ainsi définies, SM et BSM, chacune utilisant les catégories correspondantes.
La région SM concerne les événements ne comportant pas de jets issus de quark~\quarkb\ ($\Nbtag=0$) et tels que $\mCutForCategories < \SI{250}{\GeV}$ où
\mCutForCategoriesdef.
La région BSM, quant à elle, concerne les événements contenant des jets issus de quark~\quarkb\ ($\Nbtag\geq1$) ou tels que $\mCutForCategories \geq \SI{250}{\GeV}$.
Les deux régions ainsi obtenues ne se recouvrent pas et peuvent se résumer selon le schéma de la figure~\ref{fig-chapter-HTT_analysis-section-categorisation-SM_BSM_diagram}.
\par
Cette nouvelle catégorisation incluant les catégories SM est une innovation importante et non triviale par rapport à la catégorisation classique \og BSM uniquement \fg{} utilisée dans les précédentes analyses $\Higgs\to\tau\tau$ dans le cadre du MSSM~\cite{CMS-PAS-HIG-13-021,CMS-PAS-HIG-14-029,CMS-PAS-HIG-17-020}.
Elle rend l'analyse plus sensible aux propriétés du boson de Higgs du modèle standard \higgs, celles-ci pouvant être modifiées dans le MSSM par rapport au modèle standard, comme exposé dans le chapitre~\ifref{chapter-MS-MSSM}{\ref{chapter-MS-MSSM}}{2}.
En effet, les modèles supersymétriques tels que le MSSM doivent en premier lieu être compatibles avec les propriétés des particules déjà connues, comme \higgs.
La modélisation de \higgs\ dans le cadre du MSSM, décrite dans la section~\ref{chapter-HTT_analysis-section-signal_extraction}, ainsi que l'utilisation des catégories SM permettent de rendre compte de la compatibilité des propriétés de \higgs\ tel que décrit par le MSSM avec les observations.
\par
La complémentarité de la recherche du signal de \Higgs\ et \HiggsA\ avec ce test des propriétés de \higgs\ permet ainsi d'obtenir de plus fortes contraintes sur les modèles testés, comme cela a déjà été constaté dans des travaux récents~\cite{Artur_thesis}.

%There is no degree of improvisation in this strategy. The new categorization including the SM categories is an important and non-trivial novelty of this measurement compared to earlier versions of this analysis (like e.g. HIG-17-021, HIG-14-029 or HIG-13-021). In the AN we compare and validate this new strategy with the categorization as it has been used in the past. The paper will only contain the new categorization (i.e. BSM categories plus SM categories). We will add a corresponding statement in the AN if this is not clear enough, yet.   
%
%+ following with pots on https://docs.google.com/document/d/1PORzSvDwbtjWNuptZ99wx9nZLdJwSHWX_FTVam7jr-I/edit
%
%
%- L 695: Why do you choose here m_{T}^{tot} as an analysis variable, if you find a NN discriminator more beneficial elsewhere?
%
%%GREEN%
%The analysis combines the predefined SM classification of an analysis that is in the publication process right now (HIG-19-010), quasi on the datacard level, with the standard peak search for additional high mass Higgs bosons as traditionally followed up e.g. by HIG-17-020. This is a grown setup. A peak search is typically dominated by clear and well defined features, like to the tested mass of the corresponding particle. In the high mass tails such searches are highly statistics dominated and backgrounds are low. The statistical gain of an MVA classifier is not obvious. Also an MVA classifier comes with the need to control the potentially much higher dimensional input space to the discriminator. Several studies of using MVA methods in the high mass search have been made and are still under study. Since we want to publish in a finite time, the current setup is what we have. It is not excluded though that MVA methods might also be used also for the high mass search in future versions of this analysis.


\subsection{Région \og BSM \fg{}}\label{chapter-HTT_analysis-section-categorisation-BSM}
Les catégories BSM, introduites dans la référence~\cite{CMS-PAS-HIG-17-020}, sont construites dans le but de chercher une résonance correspondant à un boson de Higgs lourd.
Les coupures présentées dans cette section sont appliquées en plus des coupures permettant de séparer les régions SM et BSM.
\par
Une première catégorisation est basée sur la présence de jets issus de quarks~\quarkb.
Deux catégories sont ainsi définies:
\begin{itemize}
\item \CATnobtag: $\Nbtag =0$;
\item \CATbtag: $\Nbtag\geq1$.
\end{itemize}
Dans le cas des canaux \mu\tauh, \ele\tauh\ et \ele\mu, chacune de ces deux catégories est à nouveau subdivisée.
\paragraph{Canaux \mu\tauh\ et \ele\tauh}
Dans ces deux canaux, la masse transverse de $L_1$ (le muon ou l'électron, notés $\ell$) définie par
\begin{equation}
\mT^{(\ell)} = \sqrt{2 \, \pT^{(\ell)} \, \MET \, (1-\cos\Delta\phi)} \label{eq-mT_def-ell}
\end{equation}
avec $\Delta\phi = \phi^{(\ell)} - \phi^{(\MET)}$
est utilisée afin de définir deux catégories:
\begin{itemize}
\item \CATtightmt: $\mT^{(\ell)} < \SI{40}{\GeV}$;
\item \CATloosemt: $\SI{40}{\GeV} \leq \mT^{(\ell)} < \SI{70}{\GeV}$;
\end{itemize}
la limite haute sur \mT\ pour la catégorie \CATloosemt\ étant appliquée afin de s'assurer que la région de signal soit orthogonale à la région de détermination (DR) des facteurs de faux des événements $\Wboson+\text{jets}$.
Les facteurs de faux sont abordés dans la section~\ref{chapter-HTT_analysis-section-bg_estimation-FF_method}.
La majorité des événements de signal, en particulier pour \Higgs\ et \HiggsA\ de basse masse, se trouve dans la catégorie \CATtightmt.
La catégorie \CATloosemt\ permet quant à elle d'augmenter l'acceptance du signal pour $m_{\Higgs,\HiggsA} > \SI{700}{\GeV}$.
La figure~\ref{subfig-chapter-HTT_analysis-section-categorisation-BSM-subcats-mT} illustre ces coupures sur $\mT^{(\ell)}$ dans le cas du canal \ele\tauh\ pour l'année 2018.
\begin{wrapfigure}{R}{.45\textwidth}
\centering

\begin{tikzpicture}
%% base
\draw [->] (0,0)--(2,0) node [right] {$\bvec_x$};
\draw [->] (0,0)--(0,2) node [above] {$\bvec_y$};

\def\xMET{-1.5}
\def\yMET{-1.25}
\def\xELE{2}
\def\yELE{3}
\def\xMU{.5}
\def\yMU{-2}

\draw [thick, -latex, ltcolorred] (0,0) -- (\xMET,\yMET) coordinate (vMET);
\draw [ltcolorred] (vMET) node [below] {\vMET};
\draw [thick, -latex, ltcolorblue] (0,0) -- (\xELE,\yELE) coordinate (vE);
\draw [ltcolorblue] (vE) node [right] {$\vpT (\ele)$};
\draw [thick, -latex, ltcolorblue] (0,0) -- (\xMU,\yMU) coordinate (vM);
\draw [ltcolorblue] (vM) node [right] {$\vpT (\mu)$};

\draw [dashed, -latex] ({-\xELE+-\xMU}, {-\yELE+-\yMU}) -- ({1.25*(\xELE+\xMU)}, {1.25*(\yELE+\yMU)}) node [above] {$\vec{\zeta}$};

\draw [thick, ltcolorgreen, -latex] (0,0) -- ({\xELE+\xMU}, {\yELE+\yMU}) coordinate (vVIS);
\draw [ltcolorgreen] (vVIS) node [below right] {$p_\zeta^\text{vis}\hat{\zeta}$};

\draw [thick, -latex, ltcolorred4] (0,0) --+ ({180+acos((\xELE+\xMU)/(((\xELE+\xMU)*(\xELE+\xMU)+(\yELE+\yMU)*(\yELE+\yMU))^(0.5)))}:{(\xMET*\xMET+\yMET*\yMET)^(0.5)*cos(180-acos((\xELE+\xMU)/(((\xELE+\xMU)*(\xELE+\xMU)+(\yELE+\yMU)*(\yELE+\yMU))^(0.5)))-acos((\xMET)/((\xMET*\xMET+\yMET*\yMET)^(0.5))))})  coordinate (vMETzeta) ;
\draw [ltcolorred4] (vMETzeta) node [above] {$p_\zeta^\text{miss}\hat{\zeta}$};

\draw [dotted] (vMET) -- (vMETzeta);
\draw [dotted] (vE) -- (vVIS);
\draw [dotted] (vM) -- (vVIS);

\end{tikzpicture}
\caption{Illustration de la définition de $\hat{\zeta}$~\cite{Jang_thesis}. Le plan de ce schéma est le plan transverse.}
\label{fig-zeta_illustration}
\end{wrapfigure}
\paragraph{Canal \ele\mu}
La variable \Dzeta\ est définie selon
\begin{equation}
\Dzeta = p_\zeta^\text{miss} - \num{0.85} p_\zeta^\text{vis}
\label{eq-Dzeta_def}
\end{equation}
avec
\begin{equation}
p_\zeta^\text{miss} = \vMET \cdot \hat{\zeta}
\msep
p_\zeta^\text{vis} = \left( \vpT^{\ele} + \vpT^{\mu} \right) \cdot \hat{\zeta}
\end{equation}
où $\hat{\zeta}$ est la direction bisectionnelle entre l'électron et le muon dans le plan transverse~\cite{Jang_thesis}, comme illustré sur la figure~\ref{fig-zeta_illustration}.
Trois catégories sont alors définies:
\begin{itemize}
\item \CATlowdz: $-\SI{35}{\GeV} \leq \Dzeta < -\SI{10}{\GeV}$;
\item \CATmediumdz: $-\SI{10}{\GeV} \leq \Dzeta < \SI{30}{\GeV}$;
\item \CAThighdz: $\SI{30}{\GeV} \leq \Dzeta$;
\end{itemize}
la limite basse sur \Dzeta\ pour la catégorie \CATlowdz\ étant appliquée afin de s'assurer que la région de signal soit orthogonale à la région de contrôle (CR) du bruit de fond \ttbar.
Ces trois catégories permettent d'obtenir diverses pureté de signal et fractions de bruit de fond \ttbar.
La majorité des événements de signal se trouve dans la catégorie \CATmediumdz.
La figure~\ref{subfig-chapter-HTT_analysis-section-categorisation-BSM-subcats-Dz} illustre ces coupures sur \Dzeta.
\begin{figure}[h]
\centering

\subcaptionbox{Catégorisation basée sur $\mT^{(\ell)}$.\label{subfig-chapter-HTT_analysis-section-categorisation-BSM-subcats-mT}}[.475\textwidth]
{\plotHTTcontrolCATmt{2018}{fully_classic}{et}}
\hfill
\subcaptionbox{Catégorisation basée sur \Dzeta.\label{subfig-chapter-HTT_analysis-section-categorisation-BSM-subcats-Dz}}[.475\textwidth]
{\plotHTTcontrolCATdz{2018}{fully_classic}{em}}

\caption[Illustrations des catégorisations basées sur $\mT^{(\ell)}$ et \Dzeta]{Illustrations des catégorisations basées sur $\mT^{(\ell)}$ et \Dzeta, respectivement sur les événements des canaux \ele\tauh\ et \ele\mu\ de l'année 2018.}
\label{fig-chapter-HTT_analysis-section-categorisation-BSM-subcats}
\end{figure}
\paragraph{Récapitulatif}
Les catégories BSM correspondant à la région de signal (SR), \ie\ en dehors des régions de détermination (DR) et de contrôle (CR), sont résumées sur la figure~\ref{fig-chapter-HTT_analysis-section-categorisation-BSM-cats_recap} pour les quatre canaux considérés.
\begin{figure}[h]
\centering
\begin{tikzpicture}
\foreach \dx in {0, 6}{
    
    \draw (\dx, -3) node {\small\CATlowdz\vphantom{Àq}};
    \draw (\dx+2, -3) node {\small\CATmediumdz\vphantom{Àq}};
    \draw (\dx+4, -3) node {\small\CAThighdz\vphantom{Àq}};
    
    \draw [thick] (\dx-.925,-3-.425) rectangle  + (1.85,.85); 
    \draw [thick] (\dx+2-.925,-3-.425) rectangle  + (1.85,.85); 
    \draw [thick] (\dx+4-.925,-3-.425) rectangle  + (1.85,.85); 
    
    \foreach \ddx/\cat in {0/\CATtightmt, 3/\CATloosemt}{
        \foreach \ddy in {-1, -2}{
            
            \draw (\dx+\ddx+.5, \ddy) node {\small\cat\vphantom{Àq}};
            \draw (\dx+\ddx+.5, \ddy) node {\small\cat\vphantom{Àq}};
            
            \draw [thick] (\dx+\ddx-.925,\ddy-.425) rectangle  + (2.85,.85); 
            \draw [thick] (\dx+\ddx-.925,\ddy-.425) rectangle  + (2.85,.85); 
            
        }
    }
    
    \draw [thick] (\dx-.925,-.425) rectangle  + (5.85,.85); 
}

\draw (-1, 0) node (a) [left] {$\Higgs\to\tau\tau\to\tauh\tauh$};
\draw (a.west) + (0, -1) node [right] {$\Higgs\to\tau\tau\to\mu\tauh$};
\draw (a.west) + (0, -2) node [right] {$\Higgs\to\tau\tau\to\ele\tauh$};
\draw (a.west) + (0, -3) node [right] {$\Higgs\to\tau\tau\to\ele\mu$};

\draw (2, 1) node {\CATnobtag\vphantom{Àq}};
\draw (8, 1) node {\CATbtag\vphantom{Àq}};

\draw [very thick] (-1.25, .5) -- (11.25, .5) ;
\draw [very thick] (5, 1.5) -- (5, -3.75) ;

\end{tikzpicture}
\caption{Catégories BSM pour les quatre canaux considérés.}
\label{fig-chapter-HTT_analysis-section-categorisation-BSM-cats_recap}
\end{figure}
\subsection{Région \og SM \fg{}}\label{chapter-HTT_analysis-section-categorisation-SM}
Les catégories SM, introduites dans la référence~\cite{CMS-PAS-HIG-19-010}, sont construites dans le but d'étudier le boson de Higgs du modèle standard, noté ici $\higgs$, avec une masse de \SI{125}{\GeV}.
Les coupures présentées dans cette section sont appliquées en plus des coupures permettant de séparer les régions SM et BSM.
\par


%channel	N jets	delta R	pt	mjj	mT	Dzeta
%mt	2jet	lowdeltar		highmjj		
%mt	2jet	lowdeltar		lowmjj		
%mt	1jet	lowdeltar	mediumpt			
%mt	2jet	lowdeltar		mediummjj		
%mt	0jet				loosemt	
%mt	1jet	lowdeltar	lowpt			
%mt	0jet				tightmt	
%mt	geq1jet	highdeltar				
%mt	1jet	lowdeltar	highpt			
%et	2jet	lowdeltar		highmjj		
%et	2jet	lowdeltar		lowmjj		
%et	1jet	lowdeltar	mediumpt			
%et	2jet	lowdeltar		mediummjj		
%et	0jet				loosemt	
%et	1jet	lowdeltar	lowpt			
%et	0jet				tightmt	
%et	geq1jet	highdeltar				
%et	1jet	lowdeltar	highpt			
%tt	1jet	lowmediumdeltar	highpt			
%tt		highdeltar				
%tt	2jet	lowdeltar	lowmjj			
%tt	1jet	mediumdeltar	lowpt			
%tt	2jet	lowdeltar	highmjj	highjdeta		
%tt	1jet	lowdeltar	lowpt			
%tt	0jet	lowmediumdeltar				
%tt	2jet	lowdeltar	highmjj	lowjdeta		
%em	0jet		highpt			mediumdzeta
%em	0jet		lowpt			lowdzeta
%em	0jet		highpt			lowdzeta
%em	1jet		lowpt			
%em	1jet		highmediumpt			
%em	2jet			lowmjj		
%em	0jet		lowpt			mediumdzeta
%em	1jet		highpt			
%em	2jet			mediummjj		
%em	1jet		lowmediumpt			
