\subsection{Combinaison des catégories SM et BSM}\label{chapter-HTT_analysis-section-categorisation-SM_and_BSM}
Les catégories \CATbsm\ sont définies pour être sensibles au signal de \Higgs\ et \HiggsA,
tandis que les \CATsm\ le sont pour celui de \higgs.
L'utilisation combinée des catégories \CATsm\ et \CATbsm\ rend alors l'analyse plus sensible aux propriétés du boson de Higgs du \SM\ \higgs\ par rapport à l'utilisation des catégories \CATbsm\ uniquement.
\begin{wrapfigure}{R}{7cm}
\centering
\begin{tikzpicture}
\def\gwidth{5}
\def\gheight{\gwidth}

\draw [thick, -latex] (0,0) --+ (0, \gheight + 0.5) node [right] {\mCutForCategories\ (\SI{}{\GeV})};

\fill [ltcolorblue1] (0,0) rectangle (\gwidth, \gheight);
\fill [ltcolorred1] (0,0) rectangle (\gwidth/2, \gheight/2);
\draw [ultra thick] (0,0) rectangle (\gwidth, \gheight);
\draw [thick] (\gwidth/2, 0) --+ (0, \gheight);
\draw [thick] (0, \gheight/2) --+ (\gwidth/2, 0);

\draw (\gwidth/4, 0) node [below] {$\Nbtag = 0$};
\draw ({3*\gwidth/4}, 0) node [below] {$\Nbtag \geq 1$};

\draw (0,0) node [left] {\num{0}};
\draw (0,\gheight/2) node [left] {\num{250}};
\draw (0,\gheight) node [left] {$\infty$};

\draw (\gwidth/4, \gheight/4) node {\og SM \fg{}};
\draw (\gwidth/4, {3*\gheight/4}) node {\og BSM \fg{}};
\draw ({3*\gwidth/4}, \gheight/2) node {\og BSM \fg{}};
\end{tikzpicture}
\caption{Définition des deux régions utilisant des catégories différentes.}
\label{fig-chapter-HTT_analysis-section-categorisation-SM_BSM_diagram}
\end{wrapfigure}
\par
Cette catégorisation combinée est une innovation importante et non triviale par rapport à la catégorisation classique \og \CATbsm\ uniquement \fg{} utilisée dans les précédentes analyses $\Higgs\to\tau\tau$ dans le cadre du MSSM~\cite{CMS-PAS-HIG-13-021,CMS-PAS-HIG-14-029,CMS-PAS-HIG-17-020}.
En effet, les propriétés de \higgs\ sont modifiées dans le MSSM par rapport au \SM, comme exposé dans le chapitre~\refChMSSM.
La modélisation de \higgs\ dans le cadre du MSSM est décrite dans la section~\ref{chapter-HTT_analysis-section-signal_extraction}.
La complémentarité de la recherche du signal de \Higgs\ et \HiggsA\ avec le test des propriétés de \higgs\ permet donc d'obtenir de plus fortes contraintes sur les modèles testés, comme cela a déjà été constaté dans des travaux récents~\cite{Artur_thesis}.
\par
Afin d'éviter tout recouvrement entre les catégories \CATsm\ et \CATbsm\ lors de leur utilisation combinée, deux régions sont définies, chacune utilisant les catégories correspondantes.
La région \CATsm\ concerne les événements ne comportant pas de jets issus de quark~\quarkb\ ($\Nbtag=0$) et tels que $\mCutForCategories < \SI{250}{\GeV}$ où
\mCutForCategoriesdef.
La région \CATbsm, quant à elle, concerne les événements contenant des jets issus de quark~\quarkb\ ($\Nbtag\geq1$) ou tels que $\mCutForCategories \geq \SI{250}{\GeV}$.
Les deux régions ainsi obtenues ne se recouvrent pas et peuvent se résumer selon le schéma de la figure~\ref{fig-chapter-HTT_analysis-section-categorisation-SM_BSM_diagram}.
%\par
%À titre d'illustration, les distributions obtenues avec et sans les coupures de séparation des catégories \CATsm\ pour la catégorie \CATnobtag\ du canal \tauh\tauh\ en 2017 sont représentées en figure~\ref{fig-mTtot_distribs_exemple-SM_sep}.
%\begin{figure}[h]
%\centering
%\subcaptionbox{Sans coupure sur $(\mCutForCategories, \Nbtag)$.}[.475\textwidth]
%{\plotHTTshapes{mt_tot}{mssm_classic}{2017}{tt}{32}{prefit_linear_nosignal}}
%\hfill
%\subcaptionbox{Avec coupure sur $(\mCutForCategories, \Nbtag)$.}[.475\textwidth]
%{\plotHTTshapes{mt_tot}{mssm_vs_sm_h125}{2017}{tt}{32}{prefit_linear_nosignal}}
%
%\caption[Distributions de \mTtot\ pour le canal \tauh\tauh\ en 2017 dans la catégorie \CATnobtag.]{Distributions de \mTtot\ pour le canal \tauh\tauh\ en 2017 dans la catégorie \CATnobtag. Afin de combiner les catégories \CATbsm\ avec les catégories \CATsm\ présentées section~\ref{chapter-HTT_analysis-section-categorisation-SM}, des coupures sont appliquées sur $(\mCutForCategories, \Nbtag)$ telles que $\mCutForCategories \geq \SI{250}{\GeV}$ ou $\Nbtag\geq1$. Ici, seule la sélection sur \mCutForCategories\ joue donc un rôle.}
%\label{fig-mTtot_distribs_exemple-SM_sep}
%\end{figure}