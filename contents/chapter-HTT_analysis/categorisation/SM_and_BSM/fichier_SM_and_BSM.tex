\subsection{Combinaison des catégories SM et BSM}\label{chapter-HTT_analysis-section-categorisation-SM_and_BSM}
Les catégories \CATbsm\ introduites dans la section~\ref{chapter-HTT_analysis-section-categorisation-BSM} sont définies pour être sensibles au signal de \Higgs\ et \HiggsA.
De plus, les catégories \CATsm\ introduites dans la section~\ref{chapter-HTT_analysis-section-categorisation-SM} sont définies pour être sensibles au signal de \higgs.
L'utilisation combinée des catégories \CATsm\ et \CATbsm\ rend alors l'analyse plus sensible aux propriétés du boson de Higgs du modèle standard \higgs\ par rapport à l'utilisation des catégories \CATbsm\ uniquement.
\begin{wrapfigure}{R}{7cm}
\centering
\begin{tikzpicture}
\def\gwidth{5}
\def\gheight{\gwidth}

\draw [thick, -latex] (0,0) --+ (0, \gheight + 0.5) node [right] {\mCutForCategories\ (\SI{}{\GeV})};

\fill [ltcolorblue1] (0,0) rectangle (\gwidth, \gheight);
\fill [ltcolorred1] (0,0) rectangle (\gwidth/2, \gheight/2);
\draw [ultra thick] (0,0) rectangle (\gwidth, \gheight);
\draw [thick] (\gwidth/2, 0) --+ (0, \gheight);
\draw [thick] (0, \gheight/2) --+ (\gwidth/2, 0);

\draw (\gwidth/4, 0) node [below] {$\Nbtag = 0$};
\draw ({3*\gwidth/4}, 0) node [below] {$\Nbtag \geq 1$};

\draw (0,0) node [left] {\num{0}};
\draw (0,\gheight/2) node [left] {\num{250}};
\draw (0,\gheight) node [left] {$\infty$};

\draw (\gwidth/4, \gheight/4) node {\og SM \fg{}};
\draw (\gwidth/4, {3*\gheight/4}) node {\og BSM \fg{}};
\draw ({3*\gwidth/4}, \gheight/2) node {\og BSM \fg{}};
\end{tikzpicture}
\caption{Définition des deux régions utilisant des catégories différentes.}
\label{fig-chapter-HTT_analysis-section-categorisation-SM_BSM_diagram}
\end{wrapfigure}
\par
Cette catégorisation combinée est une innovation importante et non triviale par rapport à la catégorisation classique \og BSM uniquement \fg{} utilisée dans les précédentes analyses $\Higgs\to\tau\tau$ dans le cadre du MSSM~\cite{CMS-PAS-HIG-13-021,CMS-PAS-HIG-14-029,CMS-PAS-HIG-17-020}.
En effet, les propriétés de \higgs\ sont modifiées dans le MSSM par rapport au modèle standard, comme exposé dans le chapitre~\refChMSSM.
La modélisation de \higgs\ dans le cadre du MSSM est décrite dans la section~\ref{chapter-HTT_analysis-section-signal_extraction}.
La complémentarité de la recherche du signal de \Higgs\ et \HiggsA\ avec le test des propriétés de \higgs\ permet donc d'obtenir de plus fortes contraintes sur les modèles testés, comme cela a déjà été constaté dans des travaux récents~\cite{Artur_thesis}.
\par
Afin d'éviter tout recouvrement entre les catégories SM et BSM lors de leur utilisation combinée, deux régions sont définies, chacune utilisant les catégories correspondantes.
La région SM concerne les événements ne comportant pas de jets issus de quark~\quarkb\ ($\Nbtag=0$) et tels que $\mCutForCategories < \SI{250}{\GeV}$ où
\mCutForCategoriesdef.
La région BSM, quant à elle, concerne les événements contenant des jets issus de quark~\quarkb\ ($\Nbtag\geq1$) ou tels que $\mCutForCategories \geq \SI{250}{\GeV}$.
Les deux régions ainsi obtenues ne se recouvrent pas et peuvent se résumer selon le schéma de la figure~\ref{fig-chapter-HTT_analysis-section-categorisation-SM_BSM_diagram}.