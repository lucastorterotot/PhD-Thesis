\subsection{Modèle de vraisemblance}\label{chapter-HTT_analysis-section-signal_extraction-likelihood}
% AN 2013 171, p. 31
La fonction de vraisemblance \LKH\ à maximiser est définie par
le produit des probabilités poissoniennes $\Poisson (n_i | \nu_i(\mu,\theta))$ d'observer $n_i$ événements dans chaque segment $i$ de l'histogramme de la variable discriminante utilisée
selon
\begin{equation}
\LKH
=
\prod_i \Poisson(n_i | \nu_i(\mu,\theta))
\cdot
\prod_j \Constraint(\theta_j, \tilde{\theta}_j)
\msep
\Poisson(n_i | \nu_i(\mu,\theta))
=
\frac{\nu_i^{n_i}}{{n_i}!}\eexp{-\nu_i}
\end{equation}
où
\begin{itemize}
\item $\nu_i$ est le nombre d'événements attendus dans ce segment dans l'hypothèse \hypSB;
\item $\mu$ est le modificateur d'intensité du signal (\emph{signal strength modifier}).
Il représente la fréquence du signal, indéterminée, par rapport à une section efficace de référence, par exemple la section efficace de production du boson de Higgs \higgs;
\item $\theta$ est un paramètre de nuisance correspondant à une source d'incertitude présentée section~\ref{chapter-HTT_analysis-section-systematics}.
Les variations de ces paramètres changent la quantité d'événements de signal $s_i$ et de bruit de fond $b_i$ attendus dans le segment $i$;
\item $j$ est un indice courant sur les différentes contraintes \Constraint\ connues sur les paramètres de nuisance.
Chacune de ces contraintes représente la probabilité que ce paramètre prenne la valeur $\theta_j$, sachant que la meilleure estimation de ce dernier est $\tilde{\theta}_j$, obtenue par des mesures annexes.
\end{itemize}
La forme de la contrainte \Constraint\ dépend du type d'incertitude et est discutée ci-après.
\subsubsection{Incertitudes de normalisation}
%Les contraintes sur les incertitudes de normalisation sont modélisées par des fonctions de densité de probabilité log-normale ou Gamma.
%\par
Les contraintes sur les incertitudes correspondant à des facteurs multiplicatifs sur la quantité d'événements de signal ou de bruit de fond, par exemple les facteurs d'échelle, sont représentées par des fonctions de densité de probabilité log-normales,
\begin{equation}
\eval{\Constraint(\theta, \tilde{\theta})}_{\text{facteurs}}
=
\frac{1}{\sqrt{2\pi}\ln\kappa}\,\frac{1}{\theta}\,\exp(-\frac{(\ln(\theta/\tilde{\theta})^2}{2(\ln\kappa)^2})
\end{equation}
où $\kappa$ vaut $1+x$ avec $x$ l'incertitude relative sur l'observable contrainte.
Par exemple, pour une incertitude de \SI{10}{\%}, $\kappa=\num{1.10}$.
\par
Les contraintes sur les incertitudes d'origine statistique, par exemple les quantités d'événements observés dans les régions de contrôle, sont représentées par des fonctions de densité de probabilité Gamma,
\begin{equation}
\eval{\Constraint(\theta, \tilde{\theta})}_{\text{stat}}
=
\frac{1}{\kappa \, \tilde{\theta}!} \left(\frac{\theta}{\kappa}\right)^{\tilde{\theta}} \exp(-\frac{\theta}{\kappa})
\end{equation}
où $\kappa$ est le rapport attendu entre $\theta$ et $\tilde{\theta}$.
La valeur de $\kappa$ a sa propre incertitude, généralement traitée comme une contrainte log-normale supplémentaire.
\subsubsection{Incertitudes de forme}
Les incertitudes systématiques de forme sur les distributions des variables discriminantes du signal ainsi que du bruit de fond sont traitées par la technique du \og morphing vertical \fg.
Pour chaque source d'incertitude, une distribution centrale (ou nominale) ainsi que celles correspondant à des variations de $\pm1\sigma$ de l'incertitude sont déterminées.
Un paramètre de nuisance $\lambda$ est ajouté au modèle de vraisemblance afin d'interpoler entre ces différentes distributions.
\par
Les effets de plusieurs incertitudes de forme sont additifs.
Soient
$h_0$ la distribution centrale,
$h_j^+$ ($h_j^-$) la distribution correspondant à une variation de $+1\sigma$ ($-1\sigma$) de l'incertitude $j$ et
$\lambda_j$ le paramètre de nuisance ainsi obtenu.
Le modèle de distribution est donné par
\begin{equation}
h(\vec{\lambda}) = h_0 + \sum_j \left( a(\lambda_j)h_j^+ + b(\lambda_j) h_0 + c(\lambda_j)h_j^- \right)
\label{eq-h_interpolation}
\end{equation}
avec
\begin{equation}
a = \left\lbrace
\begin{aligned}
\lambda(\lambda+1)/2 &\msep \abs{\lambda}\leq1 \mend[,] \\
0 &\msep \lambda<-1 \mend[,] \\
\lambda &\msep \lambda>+1 \mend[,]
\end{aligned}
\right.
\qquad
b = \left\lbrace
\begin{aligned}
-\lambda^2 &\msep \abs{\lambda}\leq1 \mend[,] \\
-(\abs{\lambda}-1) &\msep \abs{\lambda}>1 \mend[,] 
\end{aligned}
\right.
\qquad
c = \left\lbrace
\begin{aligned}
\lambda(\lambda-1)/2 &\msep \abs{\lambda}\leq1 \mend[,] \\
\abs{\lambda} &\msep \lambda<-1 \mend[,] \\
0 &\msep \lambda>+1 \mend
\end{aligned}
\right.
\end{equation}
L'interpolation~\eqref{eq-h_interpolation} est réalisée lors de la maximisation de la fonction de vraisemblance.
\subsubsection{Incertitudes statistiques}
L'incertitude statistique dans les distributions des variables discriminantes et prise en compte par la méthode de Barlow-Beeston~\cite{BarlowBeeston,BarlowBeeston2}.
La quantité d'événements dans chaque segment peut varier dans l'incertitude statistique type, ce qui revient à créer une incertitude de forme.
\par
Afin de réduire la quantité de paramètres de nuisance, et donc le temps de calcul, la procédure suivante est suivie dans chaque segment:
\begin{enumerate}
\item Les processus $i$ contenant $x_i$ événements et une incertitude statistique $e_i$ tels que $e_i/x_i$ est supérieur à une valeur choisie sont sélectionnés.
\item L'incertitude totale $e_\text{tot}$ sur l'ensemble de ces processus est déterminée selon
\begin{equation}
e_\text{tot}^2 = \sum_{j\in\set{i}} e_j^2
\mend
\end{equation}
\item Les processus $i$ sont classés par valeur croissante de $e_i^2/e_\text{tot}^2$.
\item Dans l'ordre des processus obtenu, les incertitudes statistiques sont supprimées tant que la somme des carrés des incertitudes supprimées est inférieure à une valeur $r_\text{sub}$ choisie.
\item Les incertitudes restantes sont multipliées par un facteur
\begin{equation}
\sqrt{\frac{1}{1-r_\text{sub}/e_\text{tot}^2}}
\mend[,]
\end{equation}
ce qui permet de conserver une incertitude totale constante.
\end{enumerate}
Il s'agit donc de regrouper les incertitudes.
\par
Lors de ma thèse, j'ai observé que l'incertitude totale pouvait varier lors de cette procédure, comme cela est illustré sur la figure~\ref{fig-BBB_issue_2017_mt}.
Dans le dernier segment, il apparaît clairement sur le rapport données sur bruit de fond que l'incertitude totale sur le bruit de fond est modifiée par le regroupement.
Il s'agissait d'un bug que j'ai identifié et corrigé~\cite{BBB_PR} dans le code de \COMBINE.
En suivant ce code, les processus $i$ peuvent être classés dans quatre groupes:
\begin{enumerate}
\item dont l'incertitude statistique est supprimée et tels que $e_i < x_i$;
\item dont l'incertitude statistique n'est pas supprimée et tels que $e_i < x_i$;
\item dont l'incertitude statistique est supprimée et tels que $e_i \geq x_i$;
\item dont l'incertitude statistique n'est pas supprimée et tels que $e_i \geq x_i$.
\end{enumerate}
À l'étape 4, les incertitudes des processus des groupes 1 et 3 sont fixées à 0.
À l'étape 5, les incertitudes des processus des groupes 1 et 2 sont renormalisées.
Seul le groupe 2 est affecté en pratique, le groupe 1 ayant une incertitude de 0.
Or, la valeur de $e_\text{tot}^2$ est calculée à partir des groupes 1 et 2 (avant modification des incertitudes).

probleme groupe 3

fixé en les faisant tomber dans le groupe 4


\begin{figure}[h]
\centering

\subcaptionbox{Sans regroupement.}[.45\textwidth]
{\LARGE\includegraphics[width=.45\textwidth]{/home/torterotot/Documents/PhD-Thesis/plots_and_images/BBB_issue/prefit_plots_mt_inclusive-no_merging.tex}}
\hfill
\subcaptionbox{Avec regroupement.}[.45\textwidth]
{\LARGE\includegraphics[width=.45\textwidth]{/home/torterotot/Documents/PhD-Thesis/plots_and_images/BBB_issue/prefit_plots_mt_inclusive-do_merging.tex}}

\caption[Distributions de \mTtot\ avec et sans regroupement des incertitudes.]{Distributions de \mTtot\ avec et sans regroupement des incertitudes pour le canal \mu\tauh\ en 2017. Le tracé des données s'arrête à \SI{130}{\GeV}, avant la zone où le signal est attendu.}
\label{fig-BBB_issue_2017_mt}
\end{figure}


