\subsection{Interprétation dans les scénarios du MSSM}\label{chapter-HTT_analysis-section-signal_extraction-benchmarks}
Les résultats de l'analyse sont interprétés dans le cadre de scénarios de référence~\cite{Bagnaschi_2019}
respectant les limites fixées par les expériences du LEP, du Tevatron et du LHC.
En particulier, un des bosons de Higgs scalaire doit jouer le rôle du boson découvert en 2012 avec une masse de $\num{125}\pm\SI{3}{\GeV}$.
\subsubsection{Scénario $M_{\higgs}^{125}$}\label{chapter-HTT_analysis-section-signal_extraction-benchmarks-mh125}
Dans le scénario $M_{\higgs}^{125}$,
les masses des superpartenaires sont suffisamment élevées
pour que les processus physiques de production et de désintégration
des bosons de Higgs soient peu affectés par leur présence~\cite{Bagnaschi_2019}.
En particulier, les couplages de \higgs\ aux superpartenaires
sont faible par rapport à ceux aux particules du \SM\
et
\Higgs\ et \HiggsA, lorsque leurs masses sont inférieures à \SI{2}{\TeV},
se désintègrent uniquement en particules du \SM.
\par
Comme exposé dans le chapitre~\refChMSSM,
les deux paramètres permettant de décrire au premier ordre les bosons de Higgs dans le MSSM
sont $m_{\HiggsA}$ et $\tan\beta$,
pris comme paramètres libres.
Les autres sont fixés~\cite{Bagnaschi_2019}:
\begin{itemize}
\item les paramètres de masse des squarks de troisième génération $M_{Q_3}$, $M_{U_3}$ et $M_{D_3}$ pris à \SI{1.5}{\TeV};
\item les paramètres de masse des sleptons de troisième génération $M_{L_3}$ et $M_{E_3}$ pris à \SI{2}{\TeV};
\item le paramètre de masse des Higgsinos $\mu=\SI{1}{\TeV}$;
\item les paramètres de masse des \emph{gauginos} $M_1=\SI{1}{\TeV}$, $M_2=\SI{1}{\TeV}$ et $M_3=\SI{2.5}{\TeV}$;
\item le paramètre de mélange du stop, $X_t=\SI{2.8}{\TeV}$;
\item les constantes de couplage trilinéaire entre les Higgs et le stop, le sbottom et le stau, respectivement $A_t$, $A_b$ et $A_\tau$, prises égales entre elles.
\end{itemize}
\par
Le groupe en charge de la physique des bosons de Higgs au LHC
fourni en fonction de $m_{\HiggsA}$ et $\tan\beta$
les masses des bosons de Higgs, sections efficaces de production, rapports de branchement et incertitudes théoriques
associés de ce scénario~\cite{MSSMneutralHiggsTwiki}.
%- Fig. 36: Do you intend to give MSSM results also for the negative mu scenarios (hopefully yes)?
%
%%GREEN%
%Currently we aim to include the following two scenarios into the main paper:
%<br>
%$M_h^{125}$
%<br>
%$M_h^{125}(\tilde{\tau})$
%<br>
%All other scenarios can and will be added as additional material on short circuite with the Higgs PAG conveners. 
%This choice may also be changed in favour of the negative mu scenarios, to be decided later in the review process.
%%ENDCOLOR%
%The neutral MSSM Higgs boson production cross sections and the corresponding uncertainties
%are provided by the LHC Higgs Cross Section Group [66]. The program S US H I [67] has been
%used to calculate cross-sections for the gluon-fusion process and the 5 flavour b-associated
%production process. For the bbφ process, the four-flavor NLO QCD calculation [68, 69] and the
%five-flavor NNLO QCD calculation, as implemented in BBH @ NNLO [70] have been combined
%using the Santander matching scheme [71]. In all cross section programs used, the Higgs boson
%Yukawa couplings have been calculated with F EYN H IGGS [72–76]. The Higgs boson branching
%fraction to tau leptons in the different benchmark scenarios has been obtained with F EYN H IGGS
%and HDECAY [77].
%[66] LHC Higgs Cross Section Working Group Collaboration, “Handbook of LHC Higgs
%Cross Sections: 3. Higgs Properties”, doi:10.5170/CERN-2013-004,
%arXiv:1307.1347.
%[67] R. V. Harlander, S. Liebler, and H. Mantler, “SusHi: A program for the calculation of
%Higgs production in gluon fusion and bottom-quark annihilation in the Standard Model
%and the MSSM”, Comput. Phys. Commun. 184 (2013) 1605–1617,
%doi:10.1016/j.cpc.2013.02.006, arXiv:1212.3249.
%[68] S. Dittmaier, M. Kramer, 1, and M. Spira, “Higgs radiation off bottom quarks at the
%Tevatron and the CERN LHC”, Phys. Rev. D70 (2004) 074010,
%doi:10.1103/PhysRevD.70.074010, arXiv:hep-ph/0309204.
%[69] S. Dawson, C. B. Jackson, L. Reina, and D. Wackeroth, “Exclusive higgs boson production
%with bottom quarks at hadron colliders”, Phys. Rev. D 69 (Apr, 2004) 074027,
%doi:10.1103/PhysRevD.69.074027.
%[70] R. V. Harlander and W. B. Kilgore, “Higgs boson production in bottom quark fusion at
%next-to-next-to-leading order”, Phys. Rev. D 68 (Jul, 2003) 013001,
%doi:10.1103/PhysRevD.68.013001.
%[71] R. Harlander, M. Kramer, and M. Schumacher, “Bottom-quark associated Higgs-boson
%production: reconciling the four- and five-flavour scheme approach”,
%arXiv:1112.3478.
%[72] S. Heinemeyer, W. Hollik, and G. Weiglein, “FeynHiggs: A Program for the calculation of
%the masses of the neutral CP even Higgs bosons in the MSSM”, Comput. Phys. Commun.
%124 (2000) 76–89, doi:10.1016/S0010-4655(99)00364-1,
%arXiv:hep-ph/9812320.
%[73] S. Heinemeyer, W. Hollik, and G. Weiglein, “The Masses of the neutral CP - even Higgs
%bosons in the MSSM: Accurate analysis at the two loop level”, Eur. Phys. J. C9 (1999)
%343–366, doi:10.1007/s100529900006,10.1007/s100520050537,
%arXiv:hep-ph/9812472.
%[74] G. Degrassi et al., “Towards high precision predictions for the MSSM Higgs sector”, Eur.
%Phys. J. C28 (2003) 133–143, doi:10.1140/epjc/s2003-01152-2,
%arXiv:hep-ph/0212020.
%[75] M. Frank et al., “The Higgs Boson Masses and Mixings of the Complex MSSM in the
%Feynman-Diagrammatic Approach”, JHEP 02 (2007) 047,
%doi:10.1088/1126-6708/2007/02/047, arXiv:hep-ph/0611326.
%[76] T. Hahn et al., “High-Precision Predictions for the Light CP -Even Higgs Boson Mass of
%the Minimal Supersymmetric Standard Model”, Phys. Rev. Lett. 112 (2014), no. 14,
%141801, doi:10.1103/PhysRevLett.112.141801, arXiv:1312.4937.
%[77] A. Djouadi, J. Kalinowski, and M. Spira, “HDECAY: A Program for Higgs boson decays
%in the standard model and its supersymmetric extension”, Comput. Phys. Commun. 108
%(1998) 56–74, doi:10.1016/S0010-4655(97)00123-9, arXiv:hep-ph/9704448.
\par
Le signal du MSSM attendu (hypothèse \hypSB) est celui de $\higgsMSSM+\Higgs+\HiggsA$, \higgsMSSM\ étant le boson de Higgs léger du MSSM jouant le rôle du boson découvert en 2012 et interprété comme celui du \SM\ (\higgsSM).
En particulier, les propriétés de \higgsMSSM\ peuvent différer de celles de \higgsSM.
Pour \higgsMSSM, \Higgs\ et \HiggsA, la forme du signal est obtenue à partir des valeurs de $m_{\HiggsA}$ et $\tan\beta$.
Une combinaison linéaire donne le signal total $\Phi\in\set{\higgsMSSM,\Higgs,\HiggsA}\to\tau\tau$ attendu.
\par
Les modes de production considérés sont:
\begin{itemize}
\item $\gluon\gluon\higgs$, $\quarkb\antiquarkb\higgs$, VBF et VH pour \higgsMSSM;
\item $\gluon\gluon\Higgs$, $\quarkb\antiquarkb\Higgs$ pour \Higgs;
\item $\gluon\gluon\HiggsA$, $\quarkb\antiquarkb\HiggsA$ pour \HiggsA.
\end{itemize}
Dans le cas des modes de production de \higgsMSSM\ VBF et VH,
le signal attendu est pris comme celui du SM multiplié par
\begin{equation}
\sin^2(\beta-\alpha) \times \frac{\BR(\higgsMSSM\to\tau\tau)}{\BR(\higgsSM\to\tau\tau)}
\mend
\end{equation}
Dans la limite découplée, $\sin^2(\beta-\alpha)\simeq1$, seuls les rapport de branchement ont donc un effet significatif.
\par
L'hypothèse \hypB, \ie\ sans signal, correspond au cas où seul le boson de Higgs du SM est présent.
Ainsi, le paramètre $\nu_i(\mu,\vec{\theta})$ donnant le nombre d'événements attendus,
\begin{equation}
\nu_i(\mu,\vec{\theta}) = \mu \, s_i(\vec{\theta}) + b_i(\vec{\theta}) \mend[,]
\end{equation}
est réécrit sous la forme
\begin{equation}
\nu_i(\mu,\vec{\theta}) = \mu \, s_i^\text{MSSM}(\vec{\theta}) + (1-\mu)s_i^\text{SM}(\vec{\theta}) + b_i(\vec{\theta})
\end{equation}
avec
$s_i^\text{MSSM}$ le signal $\Phi\in\set{\higgsMSSM,\Higgs,\HiggsA}\to\tau\tau$ attendu selon le MSSM
dans le segment $i$ de l'histogramme de la variable discriminante
et
$s_i^\text{SM}$ cette même quantité pour le signal $\higgsSM\to\tau\tau$ du SM.
Le modificateur d'intensité du signal $\mu$ joue ainsi le rôle de distinction entre MSSM et SM.
En effet, l'existence simultanée de ces deux modèle n'est pas physique,
l'hypothèse du MSSM ($\mu=1$) doit donc être testée par rapport à celle du SM ($\mu=0$).
\par
Cependant, le profil du rapport de vraisemblance défini section~\ref{chapter-HTT_analysis-section-signal_extraction-CLS}
ne permet pas de tester $\mu=1$ par rapport à $\mu=0$,
contrairement à celui utilisé au Tevatron,
\begin{equation}
q_{\mu} = -2 \ln(\frac{\LKH(\text{données} |  \mu, \hat{\vec{\theta}}_{\mu})}{\LKH(\text{données} |  0, \hat{\vec{\theta}}_{0})})
\msep
0 \leq \mu
\mend
\end{equation}
C'est donc celui-ci qui est utilisé pour les limites dépendantes d'un modèle.
\par
Dans le MSSM, en plus de bosons de Higgs neutres supplémentaires, les propriétés du boson de Higgs correspondant à celui découvert en 2012 sont modifiées.
L'utilisation conjointe des catégories \CATsm\ et \CATbsm, introduite dans la section~\ref{chapter-HTT_analysis-section-categorisation-SM_and_BSM}, peut donc permettre d'obtenir des limites plus contraignantes sur le MSSM.
La présence attendue d'une contribution des différents modes de production considérés pour~\higgs, \Higgs\ et~\HiggsA\ au signal dans ces différentes catégories est donnée dans le tableau~\ref{tab-sign_in_cats_expected}.
L'utilisation des catégories \CATsm\ permet d'avoir une sensibilité spécifique à \higgs\ avec les modes $\gluon\gluon\higgs$, VBF et VH.
\begin{table}[h]
\centering
\begin{tabular}{lcccc}
\toprule
Catégorie & & $\gluon\gluon\higgs$, VBF, VH & $\quarkb\antiquarkb\higgs$ & \Higgs, \HiggsA\\
\midrule
\CATnobtag, $\mCutForCategories<\SI{250}{\GeV}$ & (\CATsm) & \OK & \KO & \KO\\
\CATnobtag, $\mCutForCategories\geq\SI{250}{\GeV}$ & (\CATbsm) & \KO & \KO & \OK\\
\CATbtag & (\CATbsm) & \OK & \OK & \OK\\
\bottomrule
\end{tabular}
\caption{Présences attendues des contributions au signal dans les catégories.}
\label{tab-sign_in_cats_expected}
\end{table}
\subsubsection{Scénario $M_{\Higgs_1}^{125}(\text{CPV})$}
Le boson de Higgs du \SM\ \higgsSM\ est prédit comme étant purement $CP$-pair,
ce qui est également le cas du boson de Higgs \higgsMSSM\ du MSSM sans violation de $CP$.
Dans ce cas, \Higgs\ est également $CP$-pair et \HiggsA\ est $CP$-impair,
tous ces bosons sont donc des états propres de $CP$.
\par
Cependant, comme exposé au chapitre~\refChMSSM,
une violation de $CP$ (CPV, \emph{CP Violation}) peut apparaître avec les bosons de Higgs.
Le scénario $M_{\Higgs_1}^{125}(\text{CPV})$ correspond à ce cas de figure.
Les paramètres fixes sont~\cite{Bagnaschi_2019}:
\begin{itemize}
\item les paramètres de masse des squarks de troisième génération $M_{Q_3}$, $M_{U_3}$ et $M_{D_3}$ pris à \SI{2}{\TeV};
\item les paramètres de masse des sleptons de troisième génération $M_{L_3}$ et $M_{E_3}$ pris à \SI{2}{\TeV};
\item le paramètre de masse des Higgsinos $\mu=\SI{1.65}{\TeV}$;
\item les paramètres de masse des \emph{gauginos} $M_1=\SI{1}{\TeV}$, $M_2=\SI{1}{\TeV}$ et $M_3=\SI{2.5}{\TeV}$;
\item les constantes de couplage trilinéaire entre les Higgs et le stop, le sbottom et le stau, respectivement $A_t$, $A_b$ et $A_\tau$, prises telles que
\begin{equation}
\abs{A_t} = \mu\cot\beta+\SI{2.8}{\TeV}
\msep
\phi_{A_t} = \frac{2\pi}{15}
\msep
A_b = A_\tau = \abs{A_t}
\mend
\end{equation}
\end{itemize}
La phase $\phi_{A_t}$ non nulle mène à des états propres de masse
pour les bosons de Higgs neutres différents des états propres de $CP$.
Les états propres de masse de ces bosons sont, par masses croissantes,
$\Higgs_1$, $\Higgs_2$ et $\Higgs_3$.
Il s'agit donc d'états $CP$ mixtes,
\ie\ avec une composante $CP$-paire et une $CP$-impaire.
Dans le cadre de la recherche de bosons supplémentaires de haute masse,
le rôle du boson de Higgs déjà observé \higgs\ est pris par $\Higgs_1$.
\par
La recherche d'une composante $CP$-impaire peut être réalisée directement sur \higgs.
Les événements avec une paire de leptons~\tau\ peuvent être étudiés à cette fin~\cite{Bourgatte_thesis}
à l'aide des méthodes du paramètres d'impact~\cite{BERGE2013488}, du plan de désintégration~\cite{Desch_2004} ou du vecteur polarimétrique~\cite{Cherepanov_cpv},
basées sur les propriétés cinématiques des particules de l'état final.
\par
Dans le cadre de l'analyse menée dans cette thèse,
le scénario $M_{\Higgs_1}^{125}(\text{CPV})$ est testé de manière similaire à $M_{\higgs}^{125}$.
Cependant,
le signal du MSSM considéré est
$\Phi\in\set{\Higgs_1,\Higgs_2,\Higgs_3}\to\tau\tau$.
De plus, comme \HiggsA\ n'est pas un état propre de masse,
le paramètre $m_{\HiggsA}$ est remplacé par la masse des bosons de Higgs chargés $m_{\Higgspm}$.
Enfin, des interférences entre $\Higgs_2$ et $\Higgs_3$ sont attendues dans une partie de l'espace des phases et peuvent mener à des limites d'exclusion plus faibles.
L'implémentation du traitement de ce scénario à partir des données fournies en fonction de $m_{\Higgspm}$ et $\tan\beta$~\cite{MSSMneutralHiggsTwiki}
est une de mes contributions à cette analyse.
%\cite{CMS-NOTE-2019-192,Goodsell_cpv,Arbey_cpv,Carena_cpv}
% https://twiki.cern.ch/twiki/bin/view/LHCPhysics/LHCHWGMSSMNeutral?redirectedfrom=LHCPhysics.LHCHXSWGMSSMNeutral#ROOT_histograms_2018_and_beyond