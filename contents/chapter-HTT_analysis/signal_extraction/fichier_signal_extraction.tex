\section{Extraction du signal}\label{chapter-HTT_analysis-section-signal_extraction}
% AN 2013 171
Afin de déterminer si un signal est présent ou non compte-tenu des observations,
un ajustement segmenté de maximum de vraisemblance (\emph{binned maximum likelihood fit}) est réalisé sur les catégories présentées section~\ref{chapter-HTT_analysis-section-categorisation} pour deux hypothèses:
\begin{itemize}
\item aucun signal, \ie\ uniquement des bruits de fond, notée \hypB;
\item présence d'un signal en plus des bruits de fond, notée \hypSB.
\end{itemize}
Le test statistique de ces deux hypothèse est fait par une approche fréquentiste modifiée connue sous le nom de méthode $CL_S$~\cite{Junk:1999kv,CLs_method,Read_2002}, implémentée dans \COMBINE, l'outil de combination statistique de la collaboration CMS basé sur \ROOSTATS~\cite{RooStats}.
\par
Le modèle de vraisemblance utilisé est détaillé dans la section~\ref{chapter-HTT_analysis-section-signal_extraction-likelihood}.
La méthode $CL_S$ est présentée dans la section~\ref{chapter-HTT_analysis-section-signal_extraction-CLS}.
L'application de cette méthode pour l'obtention de limites indépendantes d'un modèle est introduite section~\ref{chapter-HTT_analysis-section-signal_extraction-model_indep_and_likelihood}.
Enfin, la section~\ref{chapter-HTT_analysis-section-signal_extraction-benchmarks} expose l'interprétation de l'analyse dans le cas de scénarios spécifiques du MSSM.
\subsection{Modèle de vraisemblance}\label{chapter-HTT_analysis-section-signal_extraction-likelihood}
% AN 2013 171, p. 31
La fonction de vraisemblance \LKH\ à maximiser est définie par
le produit des probabilités poissoniennes $\Poisson (n_i | \nu_i(\mu,\theta))$ d'observer $n_i$ événements dans chaque segment $i$ de l'histogramme de la variable discriminante utilisée
selon
\begin{equation}
\LKH(n_i | \mu,\theta)
=
\prod_i \Poisson(n_i | \nu_i(\mu,\theta))
\cdot
\prod_j \Constraint(\theta_j, \tilde{\theta}_j)
\msep
\Poisson(n_i | \nu_i(\mu,\theta))
=
\frac{\nu_i^{n_i}}{{n_i}!}\eexp{-\nu_i}
\end{equation}
où
\begin{itemize}
\item $\nu_i$ est le nombre d'événements attendus dans ce segment dans l'hypothèse \hypSB, \ie
\begin{equation}
\nu_i(\mu,\theta) = \mu \, s(\theta) + b(\theta)
\end{equation}
avec
$s$ le nombre d'événements de signal
et
$b$ le nombre d'événements de bruit de fond.
Lorsque $\mu=0$, $\nu_i$ correspond donc au nombre d'événements attendus dans l'hypothèse \hypB;
\item $\mu$ est le modificateur d'intensité du signal (\emph{signal strength modifier}).
Il représente la fréquence du signal, indéterminée, par rapport à une section efficace de référence, par exemple la section efficace de production du boson de Higgs \higgs;
\item $\theta$ est un paramètre de nuisance correspondant à une source d'incertitude présentée section~\ref{chapter-HTT_analysis-section-systematics}.
Les variations de ces paramètres changent la quantité d'événements de signal $s_i$ et de bruit de fond $b_i$ attendus dans le segment $i$;
\item $j$ est un indice courant sur les différentes contraintes \Constraint\ connues sur les paramètres de nuisance.
Chacune de ces contraintes représente la probabilité que ce paramètre prenne la valeur $\theta_j$, sachant que la meilleure estimation de ce dernier est $\tilde{\theta}_j$, obtenue par des mesures annexes.
\end{itemize}
La forme de la contrainte \Constraint\ dépend du type d'incertitude et est discutée ci-après.
\subsubsection{Incertitudes de normalisation}
%Les contraintes sur les incertitudes de normalisation sont modélisées par des fonctions de densité de probabilité log-normale ou Gamma.
%\par
Les contraintes sur les incertitudes correspondant à des facteurs multiplicatifs sur la quantité d'événements de signal ou de bruit de fond, par exemple les facteurs d'échelle, sont représentées par des fonctions de densité de probabilité log-normales,
\begin{equation}
\eval{\Constraint(\theta, \tilde{\theta})}_{\text{facteurs}}
=
\frac{1}{\sqrt{2\pi}\ln\kappa}\,\frac{1}{\theta}\,\exp(-\frac{(\ln(\theta/\tilde{\theta})^2}{2(\ln\kappa)^2})
\end{equation}
où $\kappa$ vaut $1+x$ avec $x$ l'incertitude relative sur l'observable contrainte.
Par exemple, pour une incertitude de \SI{10}{\%}, $\kappa=\num{1.10}$.
\par
Les contraintes sur les incertitudes d'origine statistique, par exemple les quantités d'événements observés dans les régions de contrôle, sont représentées par des fonctions de densité de probabilité Gamma,
\begin{equation}
\eval{\Constraint(\theta, \tilde{\theta})}_{\text{stat}}
=
\frac{1}{\kappa \, \tilde{\theta}!} \left(\frac{\theta}{\kappa}\right)^{\tilde{\theta}} \exp(-\frac{\theta}{\kappa})
\end{equation}
où $\kappa$ est le rapport attendu entre $\theta$ et $\tilde{\theta}$.
La valeur de $\kappa$ a sa propre incertitude, généralement traitée comme une contrainte log-normale supplémentaire.
\subsubsection{Incertitudes de forme}
Les incertitudes systématiques de forme sur les distributions des variables discriminantes du signal ainsi que du bruit de fond sont traitées par la technique du \og morphing vertical \fg.
Pour chaque source d'incertitude, une distribution centrale (ou nominale) ainsi que celles correspondant à des variations de $\pm1\sigma$ de l'incertitude sont déterminées.
Un paramètre de nuisance $\lambda$ est ajouté au modèle de vraisemblance afin d'interpoler entre ces différentes distributions.
\par
Les effets de plusieurs incertitudes de forme sont additifs.
Soient
$h_0$ la distribution centrale,
$h_j^+$ ($h_j^-$) la distribution correspondant à une variation de $+1\sigma$ ($-1\sigma$) de l'incertitude $j$ et
$\lambda_j$ le paramètre de nuisance ainsi obtenu.
Le modèle de distribution est donné par
\begin{equation}
h(\vec{\lambda}) = h_0 + \sum_j \left( a(\lambda_j)h_j^+ + b(\lambda_j) h_0 + c(\lambda_j)h_j^- \right)
\label{eq-h_interpolation}
\end{equation}
avec
\begin{equation}
a = \left\lbrace
\begin{aligned}
\lambda(\lambda+1)/2 &\msep \abs{\lambda}\leq1 \mend[,] \\
0 &\msep \lambda<-1 \mend[,] \\
\lambda &\msep \lambda>+1 \mend[,]
\end{aligned}
\right.
\qquad
b = \left\lbrace
\begin{aligned}
-\lambda^2 &\msep \abs{\lambda}\leq1 \mend[,] \\
-\abs{\lambda} &\msep \abs{\lambda}>1 \mend[,]
\end{aligned}
\right.
\qquad
c = \left\lbrace
\begin{aligned}
\lambda(\lambda-1)/2 &\msep \abs{\lambda}\leq1 \mend[,] \\
\abs{\lambda} &\msep \lambda<-1 \mend[,] \\
0 &\msep \lambda>+1 \mend
\end{aligned}
\right.
\end{equation}
L'interpolation~\eqref{eq-h_interpolation} est réalisée lors de la maximisation de la fonction de vraisemblance.
\subsubsection{Incertitudes statistiques}
\paragraph{Principe}
L'incertitude statistique dans les distributions des variables discriminantes et prise en compte par la méthode de Barlow-Beeston~\cite{BarlowBeeston,BarlowBeeston2}.
La quantité d'événements dans chaque segment peut varier dans l'incertitude statistique type, ce qui revient à créer une incertitude de forme.
\par
Afin de réduire la quantité de paramètres de nuisance, et donc le temps de calcul, la procédure suivante est suivie dans chaque segment:
\begin{enumerate}
\item Les processus $i$ contenant $x_i$ événements et une incertitude statistique $e_i$ tels que $e_i/x_i$ est supérieur à une valeur \inlinecode{cpp}{AddThreshold} choisie sont sélectionnés.
\item L'incertitude totale $e_\text{tot}$ sur l'ensemble de ces processus est déterminée selon
\begin{equation}
e_\text{tot}^2 = \sum_{j\in\set{i}} e_j^2
\mend
\end{equation}
\item Les processus $i$ sont classés par valeur croissante de $e_i^2/e_\text{tot}^2$.
\item Dans l'ordre des processus obtenu, les incertitudes statistiques sont supprimées tant que la somme des carrés des incertitudes supprimées est inférieure à une fraction de l'incertitude totale au carré \inlinecode{cpp}{merge_threshold} choisie.
\item Les incertitudes restantes sont multipliées par un facteur permettant de conserver une incertitude totale constante.
\end{enumerate}
Il s'agit donc de regrouper les incertitudes.
\paragraph{Contribution personnelle}
Lors de ma thèse, j'ai observé que l'incertitude totale pouvait varier lors de cette procédure, comme cela est illustré sur la figure~\ref{fig-BBB_issue_2017_mt}.
Dans le dernier segment, il apparaît clairement sur le rapport données sur bruit de fond que l'incertitude totale sur le bruit de fond est modifiée par le regroupement.
Il s'agissait d'un bug que j'ai identifié et corrigé~\cite{BBB_PR} dans le code de \COMBINE.
\begin{figure}[h]
\centering

\subcaptionbox{Sans regroupement.}[.45\textwidth]
{\LARGE\includegraphics[width=.45\textwidth]{/home/torterotot/Documents/PhD-Thesis/plots_and_images/BBB_issue/prefit_plots_mt_inclusive-no_merging.tex}}
\hfill
\subcaptionbox{Avec regroupement.}[.45\textwidth]
{\LARGE\includegraphics[width=.45\textwidth]{/home/torterotot/Documents/PhD-Thesis/plots_and_images/BBB_issue/prefit_plots_mt_inclusive-do_merging.tex}}

\caption[Distributions de \mTtot\ avec et sans regroupement des incertitudes.]{Distributions de \mTtot\ avec et sans regroupement des incertitudes pour le canal \mu\tauh\ en 2017. Le tracé des données s'arrête à \SI{130}{\GeV}, avant la zone où le signal est attendu.}
\label{fig-BBB_issue_2017_mt}
\end{figure}
\par
Dans le code initial,
pour chaque segment des distributions dans chaque catégorie,
les processus $i$ peuvent être classés dans cinq groupes:
\begin{description}
\item[groupe Z] $x_i = 0$ et $e_i = 0$ (processus non présent dans le segment) ou $e_i/x_i$ inférieur à \inlinecode{cpp}{AddThreshold}, non traités par la procédure de regroupement;
\item[groupe A] incertitude à supprimer et $0 < e_i < x_i$;
\item[groupe B] incertitude à conserver et $0 < e_i < x_i$;
\item[groupe C] incertitude à supprimer et $0 < x_i \leq e_i$;
\item[groupe D] incertitude à conserver et $0 < x_i \leq e_i$.
\end{description}
Les processus tels que $0 < e_i < x_i$ (groupes A et B)
possèdent un attribut \inlinecode{cpp}{can_expand = true}
et sont ceux dont l'incertitude statistique est renormalisée (\emph{expand}) à l'étape 5 par un facteur
\begin{equation}
\text{\inlinecode{cpp}{expand = std::sqrt(1. / (1. - (removed / tot_bbb_added)))}}
\Leftrightarrow
E = \sqrt{\frac{1}{1-\frac{R}{T}}}
\end{equation}
avec
\begin{equation}
R = \text{\inlinecode{cpp}{removed}} = \sum_{i\in{\set{\text{A},\text{C}}}} e_i^2
\msep
T = \text{\inlinecode{cpp}{tot_bbb_added}} = \sum_{i\in{\set{\text{A},\text{B}}}} e_i^2
\mend
\end{equation}
\par
Ainsi, l'incertitude totale après regroupement s'exprime en fonction des incertitudes de chaque processus $i$ avant regroupement selon
\begin{align}
e_\text{tot,après} ^2
&=
\sum_{i\in{\set{\text{A}}}} (E\times 0\times e_i)^2
+
\sum_{i\in{\set{\text{B}}}} (E\times e_i)^2
+
\sum_{i\in{\set{\text{C}}}} (0\times e_i)^2
+
\sum_{i\in{\set{\text{D},\text{Z}}}} (e_i)^2
\nonumber\\&
=
E^2 \sum_{i\in{\set{\text{B}}}} e_i^2
+
\sum_{i\in{\set{\text{D},\text{Z}}}} e_i^2
\mend
\end{align}
Or,
\begin{equation}
E^2
=
\frac{1}{1-\frac{R}{T}}
=
\frac{T}{T-R}
=
\frac{\sum_{i\in{\set{\text{A},\text{B}}}} e_i^2}{\sum_{i\in{\set{\text{A},\text{B}}}} e_i^2-\sum_{i\in{\set{\text{A},\text{C}}}} e_i^2}
=
\frac{\sum_{i\in{\set{\text{A},\text{B}}}} e_i^2}{\sum_{i\in{\set{\text{B}}}} e_i^2-\sum_{i\in{\set{\text{C}}}} e_i^2}
\end{equation}
soit
\begin{equation}
e_\text{tot,après} ^2
=
\frac{\sum_{i\in{\set{\text{A},\text{B}}}} e_i^2}{\sum_{i\in{\set{\text{B}}}} e_i^2-\sum_{i\in{\set{\text{C}}}} e_i^2}
\times
\sum_{i\in{\set{\text{B}}}} e_i^2
+
\sum_{i\in{\set{\text{D},\text{Z}}}} e_i^2
\label{eq-e_tot_after_with_C}
\end{equation}
ce qui est différent de l'erreur initiale dans le cas général.
Cette formule a été testée numériquement, ce qui a permis de confirmer la bonne compréhension du code initial.
\par
Le problème vient du traitement du groupe C, \ie\ des processus dont l'incertitude est supprimée mais dont la quantité d'événement est inférieure à celle-ci.
En effet, ils ne sont pas pris en compte dans le calcul de $T$.
Le correctif proposé~\cite{BBB_PR} est de refuser le cas du groupe C et de rediriger ces processus dans le groupe D.
Alors, le groupe C étant forcément un ensemble vide, l'équation~\eqref{eq-e_tot_after_with_C} se réécrit
\begin{equation}
e_\text{tot,après} ^2
=
\frac{\sum_{i\in{\set{\text{A},\text{B}}}} e_i^2}{\sum_{i\in{\set{\text{B}}}} e_i^2}
\times
\sum_{i\in{\set{\text{B}}}} e_i^2
+
\sum_{i\in{\set{\text{D},\text{Z}}}} e_i^2
=
\sum_{i\in{\set{\text{A},\text{B}}}} e_i^2
+
\sum_{i\in{\set{\text{D},\text{Z}}}} e_i^2
=
e_\text{tot,avant} ^2
\mend[,]
\end{equation}
l'incertitude totale est donc bien conservée.
\par
Dans le cas d'étude de la figure~\ref{fig-BBB_issue_2017_mt}, ce correctif introduit six paramètres de nuisance supplémentaires, ce qui reste raisonnable en terme de charge computationnelle.
En effet, les processus devant entrer dans le groupe C sont peu nombreux.
\subsection{Méthode \CLS}\label{chapter-HTT_analysis-section-signal_extraction-CLS}
\subsubsection{Approche fréquentiste classique}
Afin de déterminer quantitativement quelle hypothèse, entre \hypB\ et \hypSB, est la plus compatible avec les résultats de l'analyse, il faut réaliser un test statistique.
Plusieurs tests existent, celui utilisé pour les expériences du LHC est le profil du rapport de vraisemblance (\emph{profile likelihood ratio}),
\begin{equation}
q_{\mu} = -2 \ln(\frac{\LKH(\text{données} |  \mu, \hat{\theta}_{\mu})}{\LKH(\text{données} |  \hat{\mu}, \hat{\theta}_{\hat{\mu}})})
\msep
0 \leq \hat{\mu} \leq \mu
\end{equation}
où
\og données \fg{} réfère aux quantités d'événements $n_i$ dans chaque segment $i$ des distributions des variables discriminantes dans chaque catégorie,
$\hat{\theta}_x$ est l'ensemble des paramètres de nuisance maximisant \LKH\ pour $\mu=x$.
L'ensemble $(\hat{\mu},\hat{\theta}_{\hat{\mu}})$ donne le maximum global de \LKH.
La contrainte $0 \leq \hat{\mu}$ impose une fréquence du signal positive, \ie\ que $\mu$ a une interprétation physique.
De plus, $\hat{\mu} \leq \mu$ interdit de rejeter $\mu$ plus petit que $\hat{\mu}$, valeur la plus probable du modificateur d'intensité du signal.
Lorsqu'une valeur de $\mu$ est rejetée, toutes les valeurs plus élevées le sont donc également.
\par
Les grandes valeurs de $q_{\mu}$ correspondent ainsi aux cas où la valeur de $\mu$ est incompatible avec les données.
À l'inverse, lorsque $q_{\mu}\simeq0$, les données sont compatibles avec $\mu$ dans le cadre de l'hypothèse \hypSB.
La probabilité d'obtenir une valeur de $q_{\mu}$ plus élevée que celle observée $q_{\mu}^\text{obs}$,
\ie\ de réaliser une observation moins compatible avec l'hypothèse \hypSB\ que celle effectivement réalisée,
est définie par
\begin{equation}
\CLSB = \int_{q_{\mu}^\text{obs}}^{+\infty} f (q_\mu | \mu, \hat{\theta}_\mu) \dd{q_\mu}
\end{equation}
où $f$ est la fonction de densité de probabilité pour $q_{\mu}$, obtenue en tirant au sort de nombreuses combinaisons des valeurs des paramètres de nuisance et de $\mu$.
\par
Une valeur de $\mu$ est ainsi considérée exclue avec un niveau de confiance $\alpha$ tel que
$\alpha = 1 - \CLSB$.
Un niveau de confiance de \SI{95}{\%} est généralement utilisé.
Toutefois, en prenant $\mu\simeq0$, cette approche mène statistiquement \SI{5}{\%} des analyses de physique à exclure la présence d'un signal.
Pour éviter ce cas de figure, une approche fréquentiste modifiée est utilisée.
\subsubsection{Approche fréquentiste modifiée}
Cette méthode est introduite pour traiter le cas d'un signal très faible par rapport au bruit de fond~\cite{Junk:1999kv,CLs_method,Read_2002}.
En plus de \CLSB\ définie précédemment,
la probabilité de réaliser une observation moins compatible avec l'hypothèse \hypB\ que celle effectivement réalisée, définie comme
\begin{equation}
\CLB = \int_{q_{\mu}^\text{obs}}^{+\infty} f (q_\mu | 0, \hat{\theta}_0) \dd{q_\mu}
\end{equation}
est déterminée.
%\par
La quantité \CLS\ est le rapport
\begin{equation}
\CLS = \frac{\CLSB}{\CLB}
\mend
\end{equation}
Dans l'approche fréquentiste modifiée, l'exclusion à \SI{95}{\%} de confiance est obtenue lorsque $\CLS \leq \num{0.05}$.
Plus de détails sur la méthode \CLS\ sont disponibles dans la référence~\cite{CMS-NOTE-2011-005}.
\subsection{Limites indépendantes du modèle et scans de \emph{likelihood}}\label{chapter-HTT_analysis-section-signal_extraction-model_indep_and_likelihood}


En l'absence de signal, des limites hautes sont déterminées sur la section efficace $\sigma$ de production des bosons de Higgs neutres du MSSM, multipliée par le rapport de branchement \BR\ à la désintégration en paire de leptons \tau.

In the absence of a signal, upper limits on the $\sigma\times\mathcal{B}^{\tau\tau}$ are set using the modified frequentist approach \citep{Junk:1999kv,Read_2002}.

Il s'agit de la valeur maximale de $\sigma\times\BR$ au-delà de laquelle un signal aurait été observé.
Pour cela, un ajustement segmenté de vraisemblance (\emph{binned Likelihood fit}) est réalisé sur les catégories présentées section~\ref{chapter-HTT_analysis-section-categorisation}.
\par
L'ajustement est réalisé à partir de la méthode $CL_S$~\cite{CLs_method} implémentée dans \COMBINE, l'outil de combination statistique de la collaboration CMS basé sur \ROOSTATS~\cite{RooStats}.


\subsection{Interprétation dans les scénarios du MSSM}\label{chapter-HTT_analysis-section-signal_extraction-benchmarks}
Les résultats de l'analyse sont interprétés dans le cadre de scénarios de référence~\cite{Bagnaschi_2019}
respectant les limites fixées par les expériences du LEP, du Tevatron et du LHC.
En particulier, un des bosons de Higgs scalaire doit jouer le rôle du boson découvert en 2012 avec une masse de $\num{125}\pm\SI{3}{\GeV}$.
\subsubsection{Scénario $M_{\higgs}^{125}$}\label{chapter-HTT_analysis-section-signal_extraction-benchmarks-mh125}
Dans le scénario $M_{\higgs}^{125}$,
les masses des superpartenaires sont suffisamment élevées
pour que les processus physiques de production et de désintégration
des bosons de Higgs soient peu affectés par leur présence~\cite{Bagnaschi_2019}.
En particulier, les couplages de \higgs\ aux superpartenaires
sont faible par rapport à ceux aux particules du \SM\
et
\Higgs\ et \HiggsA, lorsque leurs masses sont inférieures à \SI{2}{\TeV},
se désintègrent uniquement en particules du \SM.
\par
Comme exposé dans le chapitre~\refChMSSM,
les deux paramètres permettant de décrire au premier ordre les bosons de Higgs dans le MSSM
sont $m_{\HiggsA}$ et $\tan\beta$,
pris comme paramètres libres.
Les autres sont fixés~\cite{Bagnaschi_2019}:
\begin{itemize}
\item les paramètres de masse des squarks de troisième génération $M_{Q_3}$, $M_{U_3}$ et $M_{D_3}$ pris à \SI{1.5}{\TeV};
\item les paramètres de masse des sleptons de troisième génération $M_{L_3}$ et $M_{E_3}$ pris à \SI{2}{\TeV};
\item le paramètre de masse des Higgsinos $\mu=\SI{1}{\TeV}$;
\item les paramètres de masse des \emph{gauginos} $M_1=\SI{1}{\TeV}$, $M_2=\SI{1}{\TeV}$ et $M_3=\SI{2.5}{\TeV}$;
\item le paramètre de mélange du stop, $X_t=\SI{2.8}{\TeV}$;
\item les constantes de couplage trilinéaire entre les Higgs et le stop, le sbottom et le stau, respectivement $A_t$, $A_b$ et $A_\tau$, prises égales entre elles.
\end{itemize}
\par
Le groupe en charge de la physique des bosons de Higgs au LHC
fourni en fonction de $m_{\HiggsA}$ et $\tan\beta$
les masses des bosons de Higgs, sections efficaces de production, rapports de branchement et incertitudes théoriques
associés de ce scénario~\cite{MSSMneutralHiggsTwiki}.
%- Fig. 36: Do you intend to give MSSM results also for the negative mu scenarios (hopefully yes)?
%
%%GREEN%
%Currently we aim to include the following two scenarios into the main paper:
%<br>
%$M_h^{125}$
%<br>
%$M_h^{125}(\tilde{\tau})$
%<br>
%All other scenarios can and will be added as additional material on short circuite with the Higgs PAG conveners. 
%This choice may also be changed in favour of the negative mu scenarios, to be decided later in the review process.
%%ENDCOLOR%
%The neutral MSSM Higgs boson production cross sections and the corresponding uncertainties
%are provided by the LHC Higgs Cross Section Group [66]. The program S US H I [67] has been
%used to calculate cross-sections for the gluon-fusion process and the 5 flavour b-associated
%production process. For the bbφ process, the four-flavor NLO QCD calculation [68, 69] and the
%five-flavor NNLO QCD calculation, as implemented in BBH @ NNLO [70] have been combined
%using the Santander matching scheme [71]. In all cross section programs used, the Higgs boson
%Yukawa couplings have been calculated with F EYN H IGGS [72–76]. The Higgs boson branching
%fraction to tau leptons in the different benchmark scenarios has been obtained with F EYN H IGGS
%and HDECAY [77].
%[66] LHC Higgs Cross Section Working Group Collaboration, “Handbook of LHC Higgs
%Cross Sections: 3. Higgs Properties”, doi:10.5170/CERN-2013-004,
%arXiv:1307.1347.
%[67] R. V. Harlander, S. Liebler, and H. Mantler, “SusHi: A program for the calculation of
%Higgs production in gluon fusion and bottom-quark annihilation in the Standard Model
%and the MSSM”, Comput. Phys. Commun. 184 (2013) 1605–1617,
%doi:10.1016/j.cpc.2013.02.006, arXiv:1212.3249.
%[68] S. Dittmaier, M. Kramer, 1, and M. Spira, “Higgs radiation off bottom quarks at the
%Tevatron and the CERN LHC”, Phys. Rev. D70 (2004) 074010,
%doi:10.1103/PhysRevD.70.074010, arXiv:hep-ph/0309204.
%[69] S. Dawson, C. B. Jackson, L. Reina, and D. Wackeroth, “Exclusive higgs boson production
%with bottom quarks at hadron colliders”, Phys. Rev. D 69 (Apr, 2004) 074027,
%doi:10.1103/PhysRevD.69.074027.
%[70] R. V. Harlander and W. B. Kilgore, “Higgs boson production in bottom quark fusion at
%next-to-next-to-leading order”, Phys. Rev. D 68 (Jul, 2003) 013001,
%doi:10.1103/PhysRevD.68.013001.
%[71] R. Harlander, M. Kramer, and M. Schumacher, “Bottom-quark associated Higgs-boson
%production: reconciling the four- and five-flavour scheme approach”,
%arXiv:1112.3478.
%[72] S. Heinemeyer, W. Hollik, and G. Weiglein, “FeynHiggs: A Program for the calculation of
%the masses of the neutral CP even Higgs bosons in the MSSM”, Comput. Phys. Commun.
%124 (2000) 76–89, doi:10.1016/S0010-4655(99)00364-1,
%arXiv:hep-ph/9812320.
%[73] S. Heinemeyer, W. Hollik, and G. Weiglein, “The Masses of the neutral CP - even Higgs
%bosons in the MSSM: Accurate analysis at the two loop level”, Eur. Phys. J. C9 (1999)
%343–366, doi:10.1007/s100529900006,10.1007/s100520050537,
%arXiv:hep-ph/9812472.
%[74] G. Degrassi et al., “Towards high precision predictions for the MSSM Higgs sector”, Eur.
%Phys. J. C28 (2003) 133–143, doi:10.1140/epjc/s2003-01152-2,
%arXiv:hep-ph/0212020.
%[75] M. Frank et al., “The Higgs Boson Masses and Mixings of the Complex MSSM in the
%Feynman-Diagrammatic Approach”, JHEP 02 (2007) 047,
%doi:10.1088/1126-6708/2007/02/047, arXiv:hep-ph/0611326.
%[76] T. Hahn et al., “High-Precision Predictions for the Light CP -Even Higgs Boson Mass of
%the Minimal Supersymmetric Standard Model”, Phys. Rev. Lett. 112 (2014), no. 14,
%141801, doi:10.1103/PhysRevLett.112.141801, arXiv:1312.4937.
%[77] A. Djouadi, J. Kalinowski, and M. Spira, “HDECAY: A Program for Higgs boson decays
%in the standard model and its supersymmetric extension”, Comput. Phys. Commun. 108
%(1998) 56–74, doi:10.1016/S0010-4655(97)00123-9, arXiv:hep-ph/9704448.
\par
Le signal du MSSM attendu (hypothèse \hypSB) est celui de $\higgsMSSM+\Higgs+\HiggsA$, \higgsMSSM\ étant le boson de Higgs léger du MSSM jouant le rôle du boson découvert en 2012 et interprété comme celui du \SM\ (\higgsSM).
En particulier, les propriétés de \higgsMSSM\ peuvent différer de celles de \higgsSM.
Pour \higgsMSSM, \Higgs\ et \HiggsA, la forme du signal est obtenue à partir des valeurs de $m_{\HiggsA}$ et $\tan\beta$.
Une combinaison linéaire donne le signal total $\Phi\in\set{\higgsMSSM,\Higgs,\HiggsA}\to\tau\tau$ attendu.
\par
Les modes de production considérés sont:
\begin{itemize}
\item $\gluon\gluon\higgs$, $\quarkb\antiquarkb\higgs$, VBF et VH pour \higgsMSSM;
\item $\gluon\gluon\Higgs$, $\quarkb\antiquarkb\Higgs$ pour \Higgs;
\item $\gluon\gluon\HiggsA$, $\quarkb\antiquarkb\HiggsA$ pour \HiggsA.
\end{itemize}
Dans le cas des modes de production de \higgsMSSM\ VBF et VH,
le signal attendu est pris comme celui du SM multiplié par
\begin{equation}
\sin^2(\beta-\alpha) \times \frac{\BR(\higgsMSSM\to\tau\tau)}{\BR(\higgsSM\to\tau\tau)}
\mend
\end{equation}
Dans la limite découplée, $\sin^2(\beta-\alpha)\simeq1$, seuls les rapport de branchement ont donc un effet significatif.
\par
L'hypothèse \hypB, \ie\ sans signal, correspond au cas où seul le boson de Higgs du SM est présent.
Ainsi, le paramètre $\nu_i(\mu,\theta)$ donnant le nombre d'événements attendus,
\begin{equation}
\nu_i(\mu,\theta) = \mu \, s(\theta) + b(\theta) \mend[,]
\end{equation}
est réécrit sous la forme
\begin{equation}
\nu_i(\mu,\theta) = \mu \, s_\text{MSSM}(\theta) + (1-\mu)s_\text{SM}(\theta) + b(\theta)
\end{equation}
avec
$s_\text{MSSM}$ le signal $\Phi\in\set{\higgsMSSM,\Higgs,\HiggsA}\to\tau\tau$ attendu dans le cadre du MSSM
et
$s_\text{SM}$ le signal $\higgsSM\to\tau\tau$ du SM.
Le modificateur d'intensité du signal $\mu$ joue ainsi le rôle de distinction entre MSSM et SM.
En effet, l'existence simultanée de ces deux modèle n'est pas physique,
l'hypothèse du MSSM ($\mu=1$) doit donc être testée par rapport à celle du SM ($\mu=0$).
\par
Cependant, le profil du rapport de vraisemblance défini section~\ref{chapter-HTT_analysis-section-signal_extraction-CLS}
ne permet de tester $\mu=1$ par rapport à $\mu=0$.
En revanche, celui utilisé au Tevatron,
\begin{equation}
q_{\mu} = -2 \ln(\frac{\LKH(\text{données} |  \mu, \hat{\theta}_{\mu})}{\LKH(\text{données} |  0, \hat{\theta}_{0})})
\msep
0 \leq \mu
\end{equation}
le permet, c'est celui-ci qui est donc utilisé pour les limites dépendantes d'un modèle.
\par
Dans le MSSM, en plus de bosons de Higgs neutres supplémentaires, les propriétés du boson de Higgs correspondant à celui découvert en 2012 sont modifiées.
L'utilisation conjointe des catégories \CATsm\ et \CATbsm, introduite dans la section~\ref{chapter-HTT_analysis-section-categorisation-SM_and_BSM}, peut donc permettre d'obtenir des limites plus contraignantes sur le MSSM.
La présence attendue d'une contribution des différents modes de production considérés pour~\higgs, \Higgs\ et~\HiggsA\ au signal dans ces différentes catégories est donnée dans le tableau~\ref{tab-sign_in_cats_expected}.
L'utilisation des catégories \CATsm\ permet d'avoir une sensibilité spécifique à \higgs\ avec les modes $\gluon\gluon\higgs$, VBF et VH.
\begin{table}[h]
\centering
\begin{tabular}{lcccc}
\toprule
Catégorie & & $\gluon\gluon\higgs$, VBF, VH & $\quarkb\antiquarkb\higgs$ & \Higgs, \HiggsA\\
\midrule
\CATnobtag, $\mCutForCategories<\SI{250}{\GeV}$ & (\CATsm) & \OK & \KO & \KO\\
\CATnobtag, $\mCutForCategories\geq\SI{250}{\GeV}$ & (\CATbsm) & \KO & \KO & \OK\\
\CATbtag & (\CATbsm) & \OK & \OK & \OK\\
\bottomrule
\end{tabular}
\caption{Présences attendues des contributions au signal dans les catégories.}
\label{tab-sign_in_cats_expected}
\end{table}
\subsubsection{Scénario $M_{\Higgs_1}^{125}(\text{CPV})$}
Le boson de Higgs du \SM\ \higgsSM\ est prédit comme étant purement $CP$-pair,
ce qui est également le cas du boson de Higgs \higgsMSSM\ du MSSM sans violation de $CP$.
Dans ce cas, \Higgs\ est également $CP$-pair et \HiggsA\ est $CP$-impair,
tous ces bosons sont donc des états propres de $CP$.
\par
Cependant, comme exposé au chapitre~\refChMSSM,
une violation de $CP$ peut apparaître avec les bosons de Higgs.
Le scénario $M_{\Higgs_1}^{125}(\text{CPV})$ correspond à ce cas de figure.
Les paramètres fixes sont~\cite{Bagnaschi_2019}:
\begin{itemize}
\item les paramètres de masse des squarks de troisième génération $M_{Q_3}$, $M_{U_3}$ et $M_{D_3}$ pris à \SI{2}{\TeV};
\item les paramètres de masse des sleptons de troisième génération $M_{L_3}$ et $M_{E_3}$ pris à \SI{2}{\TeV};
\item le paramètre de masse des Higgsinos $\mu=\SI{1.65}{\TeV}$;
\item les paramètres de masse des \emph{gauginos} $M_1=\SI{1}{\TeV}$, $M_2=\SI{1}{\TeV}$ et $M_3=\SI{2.5}{\TeV}$;
\item les constantes de couplage trilinéaire entre les Higgs et le stop, le sbottom et le stau, respectivement $A_t$, $A_b$ et $A_\tau$, prises telles que
\begin{equation}
\abs{A_t} = \mu\cot\beta+\SI{2.8}{\TeV}
\msep
\phi_{A_t} = \frac{2\pi}{15}
\msep
A_b = A_\tau = \abs{A_t}
\mend
\end{equation}
\end{itemize}
La phase $\phi_{A_t}$ non nulle mène à des états propres de masse
pour les bosons de Higgs neutres différents des états propres de $CP$.
Les états propres de masse de ces bosons sont, par masses croissantes,
$\Higgs_1$, $\Higgs_2$ et $\Higgs_3$.
Il s'agit donc d'états $CP$ mixtes,
\ie\ avec une composante $CP$-paire et une $CP$-impaire.
Dans le cadre de la recherche de bosons supplémentaires de haute masse,
le rôle du boson de Higgs déjà observé \higgs\ est pris par $\Higgs_1$.
\par
La recherche d'une composante $CP$-impaire peut être réalisée directement sur \higgs.
Les événements avec une paire de leptons~\tau\ peuvent être étudiés à cette fin~\cite{Bourgatte_thesis}
à l'aide des méthodes du paramètres d'impact~\cite{BERGE2013488}, du plan de désintégration~\cite{Desch_2004} ou du vecteur polarimétrique~\cite{Cherepanov_cpv},
basées sur les propriétés cinématiques des particules de l'état final.
\par
Dans le cadre de l'analyse menée dans cette thèse,
le scénario $M_{\Higgs_1}^{125}(\text{CPV})$ est testé de manière similaire à $M_{\higgs}^{125}$.
Cependant,
le signal du MSSM considéré est
$\Phi\in\set{\Higgs_1,\Higgs_2,\Higgs_3}\to\tau\tau$.
De plus, comme \HiggsA\ n'est pas un état propre de masse,
le paramètre $m_{\HiggsA}$ est remplacé par la masse des bosons de Higgs chargés $m_{\Higgspm}$.
Enfin, des interférences entre $\Higgs_2$ et $\Higgs_3$ sont attendues dans une partie de l'espace des phases et peuvent mener à des limites d'exclusion plus faibles.
L'implémentation du traitement de ce scénario à partir des données fournies en fonction de $m_{\Higgspm}$ et $\tan\beta$~\cite{MSSMneutralHiggsTwiki}
est une de mes contributions à cette analyse.
%\cite{CMS-NOTE-2019-192,Goodsell_cpv,Arbey_cpv,Carena_cpv}
% https://twiki.cern.ch/twiki/bin/view/LHCPhysics/LHCHWGMSSMNeutral?redirectedfrom=LHCPhysics.LHCHXSWGMSSMNeutral#ROOT_histograms_2018_and_beyond

\todo{see comments in tex file}
%%GREEN%
%For the model-independent interpretation of the results the signal modelling is:
%<br>
%A single Higgs boson resonance phi:
%produced via gluon fusion “ggphi”,
%b-associated production “bbphi”,
%<br>  
%The main model-independent interpretation of the results will contain the SM Higgs boson as background, i.e. the data will be interpreted as a search for a Higgs boson _in addition_ to the observed Higgs boson. 
%<br>
%For the model-dependent interpretation of the results the chosen signal model is:
%<br>
%Light scalar h with mh=125 GeV:
%produced via gluon fusion “ggh”,
%b-associated production “bbh”,
%and vector boson fusion and associated production “(VBF + V(->qq))h = qqh”
%<br>
%Heavy scalar and pseudoscalar A/H:
%produced via gluon fusion “ggA/H”,
%b-associated production “bbA/H”,
%<br>
%Here the observed Higgs boson is interpreted as a SUSY particle (e.g. the h in the $M_h^{125}$ scenario). This interpretation of the data should take into account how compatible the predicted properties of the h are with the observation. This will become one of the most distinguishing signatures of MSSM models in near future. Present and future SUSY models should cope with the findings of our measurements of the observed Higgs boson.
%<br>
%For the qqh contribution, after consultation with theory colleagues, we use the SM prediction scaled by:
%<br>
%sin(beta - alpha)^2 * BR(h--->tautau) / BR(h_SM(125) ---> tautau)
%<br>
%In the decoupling limit, the factor sin(beta - alpha)^2 is essentially one, such that only the branching fraction difference plays a role for this production mode.
%<br>
%You can find some further elaboration to backup this strategy in the following: From
%<br>
%https://cms.cern.ch/iCMS/jsp/db_notes/noteInfo.jsp?cmsnoteid=CMS%20AN-2019/177
%<br>
%we expect a constraint of
%<br>
%mu = 1.0 +/- 0.12
%<br>
%for the observed h. This can be translated into the mA -- tanb plane, as done in HIG-19-005.
%
%I would just show the first plot to get the general message across that the MSSM modifies the Higgs XS’s and BRs). For the AN once we have implemented the SM categories in the MSSM analysis we can include clear plots showing how the SM categories change the picture (e.g limits with and without SM categories)

