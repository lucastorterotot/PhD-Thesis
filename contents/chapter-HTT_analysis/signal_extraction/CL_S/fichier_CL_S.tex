\subsection{Méthode \CLS}\label{chapter-HTT_analysis-section-signal_extraction-CLS}
\subsubsection{Approche fréquentiste classique}
Afin de déterminer quantitativement quelle hypothèse, entre \hypB\ et \hypSB, est la plus compatible avec les résultats de l'analyse, il faut réaliser un test statistique.
Plusieurs tests existent, celui utilisé pour les expériences du LHC est le profil du rapport de vraisemblance (\emph{profile likelihood ratio}),
\begin{equation}
q_{\mu} = -2 \ln(\frac{\LKH(\text{données} |  \mu, \hat{\theta}_{\mu})}{\LKH(\text{données} |  \hat{\mu}, \hat{\theta}_{\hat{\mu}})})
\msep
0 \leq \hat{\mu} \leq \mu
\end{equation}
où
\og données \fg{} réfère aux quantités d'événements $n_i$ dans chaque segment $i$ des distributions des variables discriminantes dans chaque catégorie,
$\hat{\theta}_x$ est l'ensemble des paramètres de nuisance maximisant \LKH\ pour $\mu=x$.
L'ensemble $(\hat{\mu},\hat{\theta}_{\hat{\mu}})$ donne le maximum global de \LKH.
La contrainte $0 \leq \hat{\mu}$ impose une fréquence du signal positive, \ie\ que $\mu$ a une interprétation physique.
De plus, $\hat{\mu} \leq \mu$ interdit de rejeter $\mu$ plus petit que $\hat{\mu}$, valeur la plus probable du modificateur d'intensité du signal.
Lorsqu'une valeur de $\mu$ est rejetée, toutes les valeurs plus élevées le sont donc également.
\par
Les grandes valeurs de $q_{\mu}$ correspondent ainsi aux cas où la valeur de $\mu$ est incompatible avec les données.
À l'inverse, lorsque $q_{\mu}\simeq0$, les données sont compatibles avec $\mu$ dans le cadre de l'hypothèse \hypSB.
La probabilité d'obtenir une valeur de $q_{\mu}$ plus élevée que celle observée $q_{\mu}^\text{obs}$,
\ie\ de réaliser une observation moins compatible avec l'hypothèse \hypSB\ que celle effectivement réalisée,
est définie par
\begin{equation}
\CLSB = \int_{q_{\mu}^\text{obs}}^{+\infty} f (q_\mu | \mu, \hat{\theta}_\mu) \dd{q_\mu}
\end{equation}
où $f$ est la fonction de densité de probabilité pour $q_{\mu}$, obtenue en tirant au sort de nombreuses combinaisons des valeurs des paramètres de nuisance et de $\mu$.
\par
Une valeur de $\mu$ est ainsi considérée exclue avec un niveau de confiance $\alpha$ tel que
$\alpha = 1 - \CLSB$.
Un niveau de confiance de \SI{95}{\%} est généralement utilisé.
Toutefois, en prenant $\mu\simeq0$, cette approche mène statistiquement \SI{5}{\%} des analyses de physique à exclure la présence d'un signal.
Pour éviter ce cas de figure, une approche fréquentiste modifiée est utilisée.
\subsubsection{Approche fréquentiste modifiée}
Cette méthode est introduite pour traiter le cas d'un signal très faible par rapport au bruit de fond~\cite{Junk:1999kv,CLs_method,Read_2002}.
En plus de \CLSB\ définie précédemment,
la probabilité de réaliser une observation moins compatible avec l'hypothèse \hypB\ que celle effectivement réalisée, définie comme
\begin{equation}
\CLB = \int_{q_{\mu}^\text{obs}}^{+\infty} f (q_\mu | 0, \hat{\theta}_0) \dd{q_\mu}
\end{equation}
est déterminée.
%\par
La quantité \CLS\ est le rapport
\begin{equation}
\CLS = \frac{\CLSB}{\CLB}
\mend
\end{equation}
Dans l'approche fréquentiste modifiée, l'exclusion à \SI{95}{\%} de confiance est obtenue lorsque $\CLS \leq \num{0.05}$.
Plus de détails sur la méthode \CLS\ sont disponibles dans la référence~\cite{CMS-NOTE-2011-005}.