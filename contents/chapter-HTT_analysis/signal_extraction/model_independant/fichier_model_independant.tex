\subsection{Limites indépendantes du modèle}\label{chapter-HTT_analysis-section-signal_extraction-model_indep_and_likelihood}
En l'absence de signal, des limites hautes sont déterminées sur la section efficace $\sigma$ de production des bosons de Higgs neutres du MSSM \Higgs\ et \HiggsA, multipliée par le rapport de branchement \BR\ à la désintégration en paire de leptons \tau.
La limite d'exclusion à \SI{95}{\%} de confiance sur $\sigma\times\BR$ est déterminée à partir de la valeur de $\mu$ telle que $CL_S = \num{0.05}$.
Il s'agit donc de la valeur maximale de $\sigma\times\BR$ au-delà de laquelle un signal aurait été considéré comme observé.
\par
La modélisation du signal consiste en un unique boson de Higgs $\Phi \Leftrightarrow (\Higgs + \HiggsA)$ avec pour modes de production:
\begin{itemize}
\item $\gluon\gluon\to\Phi\to\tau\tau$;
\item $\gluon\gluon\to\quarkb\antiquarkb\Phi\to\tau\tau$.
\end{itemize}
Lorsque les limites sont déterminées pour l'un de ces modes, aucune hypothèse n'est faite sur l'autre, en particulier sur sa normalisation.
\par
Dans le cas du processus $\gluon\gluon\to\Phi\to\tau\tau$, les contributions à la boucle fermionique des quarks~\quarkt, \quarkb\ et de leur interférence sont fixées à celles attendues dans le modèle standard.
Aucune modification due à $\tan\beta$ n'est donc considérée.
\par
En plus des bruits de fond usuels
$\Zboson\to\tau\tau$,
$\Zboson\to\ell\ell$,
\Wjets,
\ttbar,
Diboson, \emph{Single top}
et
QCD,
la contribution de boson de Higgs du modèle standard \higgs\ dans ses désintégrations
$\higgs\to\tau\tau$
et
$\higgs\to\Wboson\Wboson$
est également considérée comme un bruit de fond.
Le signal du MSSM est ainsi constitué des processus
$\gluon\gluon\to\Phi\to\tau\tau$
et
$\gluon\gluon\to\quarkb\antiquarkb\Phi\to\tau\tau$
avec les coupages du modèle standard aux quarks~\quarkt\ et~\quarkb\
et $m_\Phi\in[60, \num{3500}]\usp\SI{}{\GeV}$.
\par
Il s'agit donc de la recherche d'un boson de Higgs neutre en plus du boson déjà observé.
Les catégories \CATbsm\ introduites section~\ref{chapter-HTT_analysis-section-categorisation-BSM} sont utilisées, sans combinaison avec les catégories \CATsm.