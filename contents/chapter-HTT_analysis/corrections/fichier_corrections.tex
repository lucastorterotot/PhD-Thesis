\section{Corrections}\label{chapter-HTT_analysis-section-corrections}
Dans le but d'améliorer la description des données réelles par les données simulées et encapsulées, des corrections
obtenues à partir d'analyses annexes
leur sont appliquées.
Les corrections usuelles sont présentées dans le chapitre~\refChLHCCMS.
Des corrections spécifiques à cette analyse sont également appliquées.
Elles sont présentées ci-après.
\subsection{Efficacité des \HLTpaths}
\paragraph{Efficacité des \HLTpaths\ des \tauh\ (\emph{\tauh\ trigger scale factors})}
L'efficacité des \HLTpaths\ est mesurée à partir d'une méthode \og balise et sonde \fg{}  (\emph{tag and probe}).
La balise (\emph{tag}) est un \tauh\ respectant les critères de sélection utilisés dans l'analyse et correspondant au \tauh\ ayant activé le \HLTpath.
La sonde (\emph{probe}) est tout \tauh\ respectant les critères de sélection utilisés dans l'analyse, à l'exception du \tauh\ \emph{tag}.
L'efficacité $\epsilon$ du \HLTpath\ est alors
\begin{equation}
\epsilon = \frac{N_\text{pass}}{N_\text{total}}
\end{equation}
où
$N_\text{pass}$ est le nombre de \emph{probe} correspondant au \tauh\ ayant activé au moins un des \HLTpaths\ \HLTSingleTau\ utilisé
et
$N_\text{total}$ le nombre total de \emph{probe}.
Le facteur d'échelle correctif $SF$ à appliquer aux événements simulés est ainsi
\begin{equation}
SF = \frac{\epsilon(\text{données réelles})}{\epsilon(\text{données simulées})}
\mend
\end{equation}
Dans le cas des données encapsulées, le dénominateur est l'efficacité obtenue avec les \tauh\ simulés.
Les efficacités et le facteur d'échelle sont déterminés en fonction des propriétés cinématiques du \tauh\ ($\pT, \eta, \phi$).
L'efficacité des \HLTpaths\ \HLTDoubleTau\ est obtenue comme étant le produit des efficacité de chacun des deux \tauh.
\par
L'efficacité combinée des \HLTpaths\ \HLTSingleTau\ et \HLTSingleMu\ dans le canal \mu\tauh\ (\HLTSingleEle\ dans le canal \ele\tauh) est obtenue selon
\begin{equation}
\epsilon = \epsilon(1\tauh) + \epsilon(1\ell) - \epsilon(1\tauh) \times \epsilon(1\ell)
\end{equation}
où $\ell$ correspond au muon (à l'électron).
Dans le cas du canal \tauh\tauh, la présence de deux \tauh\ impose une formule plus complexe pour déterminer l'efficacité combinée des \HLTpaths\ \HLTSingleTau\ et \HLTDoubleTau,
\begin{align}
\epsilon &= \epsilon(2\tauh) + \epsilon(\tauh1) + \epsilon(\tauh2)
\nonumber\\ & \hphantom{=}
- \epsilon(2\tauh + \tauh1) - \epsilon(2\tauh + \tauh1) - \epsilon(\tauh1+\tauh2)
\nonumber\\ & \hphantom{=}
+ \epsilon(2\tauh + \tauh1 + \tauh2)
\mend[,]
\end{align}
où $2\tauh$ signifie \HLTDoubleTau,
$\tauh1$ \HLTSingleTau\ appliqué au \tauh\ de plus haut \pT\ et
$\tauh2$ \HLTSingleTau\ appliqué à l'autre \tauh.
\paragraph{Efficacité des \HLTpaths\ des muons et des électrons (\emph{lepton trigger scale factors})}
De manière similaire au cas des \tauh, l'efficacité des \HLTpaths\ des leptons (muons et électrons) est déterminée dans les données réelles et simulées en fonction de l'impulsion transverse et de la pseudo-rapidité du lepton.
Cette démarche est réalisée pour les \HLTpaths\ \HLTSingleMu, \HLTSingleEle, \HLTMuTauCross\ et \HLTEleTauCross.
\subsection{Impulsions des particules générées}
\paragraph{Repondération de l'impulsion transverse et de la masse du boson \Zboson\ (\emph{DY \pT-mass reweighting})}
Les impulsions transverses ainsi que la masse invariante des leptons issus de la désintégration du boson \Zboson\ sont corrigées
dans les événements simulés Drell-Yan.
Ces corrections sont déterminées dans une région de contrôle $\Zboson\to\mu\mu$ et n'introduisent pas de modification du nombre total d'événements.
\paragraph{Repondération de l'impulsion transverse du quark~\quarkt\ (\emph{top \pT\ reweighting})}
La modélisation du bruit de fond \ttbar\ est corrigée afin que les données simulées au NLO correspondent au NNLO.
Pour cela, la distribution en \pT\ des quarks~\quarkt\ est pondérée.
Le poids à appliquer à un quark~\quarkt, déterminé par le groupe \quarkt\antiquarkt\Higgs, s'exprime en fonction de l'impulsion transverse du quark~\quarkt\ en \SI{}{\GeV} selon
\begin{equation}
\omega = \exp(\num{0.088} - \num{8.7e-4}\times\pT + \num{9.2e-7}\times\pT^2)
\mend
\end{equation}
Le poids total à appliquer aux événements \ttbar\ contenant deux quarks~\quarkt\ est alors
\begin{equation}
\omega(\text{total}) = \sqrt{\omega(1) \times \omega(2)}
\mend
\end{equation}
\begin{wrapfigure}{R}{5cm}
\vspace{\baselineskip}
\centering
\begin{fmffile}{gg_loop_hHA}\fmfstraight
\begin{fmfchar*}(42,25)
  \fmfleft{g1,fi,g2}
  \fmfright{fo1,h,fo2}
  \fmf{gluon}{g1,g1loop}
  \fmf{gluon}{g2,g2loop}
  \fmf{phantom, tension=.6}{g1loop,fo1}
  \fmf{phantom, tension=.6}{g2loop,fo2}
  \fmffreeze
  \fmf{fermion}{g1loop,hloop,g2loop,g1loop}
  \fmf{fermion}{g2loop,g1loop}
  \fmf{dashes, label=$\Hs,, \Hn,, \Ha$, l.side=left, tension=1.75}{hloop,h}
  \fmfdot{g1loop,hloop,g2loop}
  \fmffreeze
  \fmf{phantom}{g1loop,fakev1}
  \fmf{phantom}{g2loop,fakev1}
  \fmffreeze
%  \fmf{phantom,tension=1.5}{hloop,fakev2}
%  \fmf{phantom, label=$t,,\bar{t}$, l.side=left}{fakev1,fakev2}
%  \fmf{phantom, label=$b,,\bar{b}$, l.side=right}{fakev1,fakev2}
  \fmflabel{\gluon}{g1}
  \fmflabel{\gluon}{g2}
\end{fmfchar*}
\end{fmffile}
\vspace{\baselineskip}
\caption[Production de boson de Higgs du MSSM par fusion de gluons.]{Diagramme de Feynman de production de boson de Higgs dans le cadre du MSSM par fusion de gluons (\gluon\gluon\Higgs).}
\label{fig-chapter-HTT_analysis-section-corrections-fgraph-gg_loop_hHA}
\end{wrapfigure}
\paragraph{Repondération de l'impulsion transverse du boson de Higgs}
Au premier ordre non nul (LO, \emph{Leading Order}), les propriétés cinématiques du signal ne dépendent que de la masse du boson de Higgs, ce qui est couvert par la variété des jeux de données utilisés, listés dans l'annexe~\refApHTTdatasets.
Cependant, à l'ordre supérieur (NLO, \emph{Next-to Leading Order}),
ce n'est plus vrai dans le cas du processus $\gluon\gluon\Higgs$,
illustré figure~\ref{fig-chapter-HTT_analysis-section-corrections-fgraph-gg_loop_hHA}.
\par
La boucle fermionique du diagramme comporte des contributions provenant des quarks,
les plus importantes étant celles des quarks~\quarkt\ et~\quarkb\ ainsi que leur interférence notée~\quarkt\quarkb.
Pour chacune d'entre elles, les distributions des impulsions transverses des bosons de Higgs (\higgs, \Higgs\ ou \HiggsA) sont indépendantes de $\tan\beta$, paramètre introduit dans le chapitre~\refChMSSM.
Toutefois, les proportions de ces contributions le sont.
Ainsi,
les propriétés cinématiques des bosons de Higgs
et par conséquent celles des leptons~\tau\
dépendent de $\tan\beta$.
\par
L'effet de $\tan\beta$ étant uniquement dû aux proportions des contributions des quarks à la boucle,
il est possible d'obtenir les distributions en \pT\ des bosons de Higgs au NLO
à partir des neuf contributions $(\higgs,\Higgs,\HiggsA)\times(\quarkt,\quarkb,\quarkt\quarkb)$
pour chaque point de masse utilisé
selon les méthodes introduites dans les références~\cite{Bagnaschi:2015qta,Bagnaschi:2015bop}.
\par
D'une part, une simulation de référence par point de masse est réalisée.
La génération de ces événements est faite au NLO à l'aide du module \texttt{gg\_H\_2HDM} de \POWHEG~\cite{Alioli:2010xd},
les gerbes partoniques, l'hadronisation et l'évenement sous-jacent sont simulés par \PYTHIA~\cite{pythia8.2} et
la modélisation du détecteur est traitée par \GEANTfour~\cite{geant4_2003,geant4_2006,geant4_2016}.
Cette simulation se fait dans le cadre d'un modèle général à deux doublets de Higgs (2HDM), dont le MSSM est un cas particulier comme exposé dans le chapitre~\refChMSSM.
\POWHEG\ permet alors d'obtenir les neuf contributions $(\higgs,\Higgs,\HiggsA)\times(\quarkt,\quarkb,\quarkt\quarkb)$.
En principe, toutes valeurs des paramètres $\alpha$ et $\tan\beta$ peuvent être utilisées.
En pratique, pour éviter d'obtenir un terme d'interférence presque nul menant à de faibles statistiques, ces paramètres sont fixés à $\alpha=\pi/4$ et $\tan\beta=15$.
\par
D'autre part, des simulations annexes sont réalisées pour différentes valeurs de $\tan\beta$, également avec le  module \texttt{gg\_H\_2HDM} de \POWHEG, mais uniquement au niveau générateur \ie\ sans propagation dans le détecteur.
Pour chaque valeur de $\tan\beta$, les distributions en \pT\ des neuf contributions considérées sont pondérées dans la simulation de référence de manière à correspondre à celles obtenues dans la simulation annexe correspondante.
\par
Le signal complet du MSSM est obtenu à partir des contributions individuelles et pondérées du 2HDM utilisé ($\alpha=\pi/4$, $\tan\beta=15$) selon
\begin{align}
\sigma_\text{MSSM} &=
\left(\frac{y_{\quarkt,\text{MSSM}}}{y_{\quarkt,\text{2HDM}}}\right)^2 \sigma_{\quarkt,\text{2HDM}}(Q_{\quarkt})
+
\left(\frac{y_{\quarkb,\text{MSSM}}}{y_{\quarkb,\text{2HDM}}}\right)^2 \sigma_{\quarkb,\text{2HDM}}(Q_{\quarkb})
\nonumber\\&\hphantom{=}
+
\left(\frac{y_{\quarkt,\text{MSSM}}}{y_{\quarkt,\text{2HDM}}}\frac{y_{\quarkb,\text{MSSM}}}{y_{\quarkb,\text{2HDM}}}\right)
\left[ \sigma_{\quarkt+\quarkb,\text{2HDM}}(Q_{\quarkt\quarkb}) - \sigma_{\quarkt,\text{2HDM}}(Q_{\quarkt\quarkb}) - \sigma_{\quarkb,\text{2HDM}}(Q_{\quarkt\quarkb}) \right]
\label{eq-2HDM_to_MSSM_xsec}
\end{align}
où $\sigma$ peut correspondre à la section efficace inclusive ou différentielle selon une variable donnée,
$y_{\quarkt}$ et $y_{\quarkb}$ sont les constantes de couplage de Yukawa pour les quarks~\quarkt\ et~\quarkb\ introduites dans le chapitre~\refChMSSM,
$Q$ l'échelle d'énergie~\cite{Bagnaschi:2015qta,Bagnaschi:2015bop}.
Les trois termes de cette formule correspondent aux contributions~\quarkt, \quarkb\ et~\quarkt\quarkb.
Les valeurs de $y_{\quarkt}$ et $y_{\quarkb}$ dépendent de $m_{\HiggsA}$ et $\tan\beta$ et sont définies pour chacun des bosons de Higgs (\higgs, \Higgs, \HiggsA).