\section{Corrections}\label{chapter-HTT_analysis-section-corrections}
Dans le but d'améliorer la description des données réelles par les données simulées et encapsulées, des corrections sont appliquées à ces dernières.
Ces corrections, présentées ci-après, sont obtenues à partir d'analyses annexes.
%\subsection{Pondérations} % weights and event scale factors
\paragraph{Pondération de l'empilement (\emph{Pileup reweighting})}
Les données simulées sont générées avec un réglage donné de luminosité instantanée, relié à la quantité d'empilement obtenu.
Or, la production de ces jeux de données est souvent faite avant la mesure de ces observables dans les données réelles.
Afin de corriger la différence sur le profil d'empilement obtenu, un poids est appliqué aux événements simulés afin que ce profil soit cohérent avec celui des données réelles.
\paragraph{Pondération du \emph{prefiring}}
En 2016 et 2017, le niveau L1 du système de déclenchement de CMS présentait un défaut.
Dans la partie à haute $\eta$ du ECAL, des objets physique responsables du déclenchement du L1 étaient associés à l'événement précédent.
Seul un événement sur trois consécutifs pouvant être enregistré, l'efficacité de la prise de données est moindre qu'attendue.
De plus, cette efficacité dépend de la topologie des événements.
En l'occurrence, les événements avec des jets de hautes valeurs de $\eta$ sont particulièrement touchés par cet effet.
Une pondération est alors appliquée afin de corriger cet effet.
Selon la topologie de l'événement, il peut être de \num{1.0} ou descendre à des valeurs de l'ordre de \num{0.95}.
\paragraph{Efficacité des \HLTpaths\ des \tauh\ (\emph{\tauh\ trigger scale factors})}
L'efficacité des \HLTpaths\ est mesurée à partir d'une méthode \og balise et sonde \fg{}  (\emph{tag and probe}).
La balise (\emph{tag}) est un \tauh\ respectant les critères de sélection utilisés dans l'analyse et correspondant au \tauh\ ayant activé le \HLTpath.
La sonde (\emph{probe}) est tout \tauh\ respectant les critères de sélection utilisés dans l'analyse, à l'exception du \tauh\ \emph{tag}.
L'efficacité $\epsilon$ du \HLTpath\ est alors
\begin{equation}
\epsilon = \frac{N_\text{pass}}{N_\text{total}}
\end{equation}
où
$N_\text{pass}$ est le nombre de \emph{probe} correspondant au \tauh\ ayant activé au moins un des \HLTpaths\ \HLTSingleTau\ utilisé
et
$N_\text{total}$ le nombre total de \emph{probe}.
Le facteur d'échelle correctif $SF$ à appliquer aux événements simulés est ainsi
\begin{equation}
SF = \frac{\epsilon(\text{données réelles})}{\epsilon(\text{données simulées})}
\mend
\end{equation}
Dans le cas des données encapsulé, le dénominateur est l'efficacité obtenue avec les \tauh\ simulés.
Les efficacités et le facteur d'échelle sont déterminés en fonction des propriétés cinématiques du \tauh\ ($\pT, \eta, \phi$).
L'efficacité des \HLTpaths\ \HLTDoubleTau\ est obtenue comme étant le produit des efficacité de chacun des deux \tauh.
\par
L'efficacité combinée des \HLTpaths\ \HLTSingleTau\ et \HLTSingleMu\ dans le canal \mu\tauh (\HLTSingleEle\ dans le canal \ele\tauh) est obtenue selon
\begin{equation}
\epsilon = \epsilon(1\tauh) + \epsilon(1\ell) - \epsilon(1\tauh) \times \epsilon(1\ell)
\end{equation}
où $\ell$ correspond au muon (à l'électron).
Dans le cas du canal \tauh\tauh, la présence de deux \tauh\ impose une formule plus complexe pour déterminer l'efficacité combinée des \HLTpaths\ \HLTSingleTau\ et \HLTDoubleTau\,
\begin{align}
\epsilon &= \epsilon(2\tauh) + \epsilon(\tauh1) + \epsilon(\tauh2)
\nonumber\\ & \hphantom{=}
- \epsilon(2\tauh + \tauh1) - \epsilon(2\tauh + \tauh1) - \epsilon(\tauh1+\tauh2)
\nonumber\\ & \hphantom{=}
+ \epsilon(2\tauh + \tauh1 + \tauh2)
\end{align}
où $2\tauh$ signifie \HLTDoubleTau,
$\tauh1$ \HLTSingleTau\ appliqué au \tauh\ de plus haut \pT\ et
$\tauh2$ \HLTSingleTau\ appliqué à l'autre \tauh.
\paragraph{Efficacité des \HLTpaths\ des muons et des électrons (\emph{lepton trigger scale factors})}
De manière similaire au cas des \tauh, l'efficacité des \HLTpaths\ des leptons (muons et électrons) est déterminée dans les données réelles et simulées en fonction de l'impulsion transverse et de la pseudorapidité du lepton.
Cette démarche est réalisée pour les \HLTpaths\ \HLTSingleMu, \HLTSingleEle, \HLTMuTauCross\ et \HLTEleTauCross.
\paragraph{Efficacité d'identification et isolation des \tauh\ (\emph{\tauh\ ID/iso scale factors})}
L'efficacité d'identification des \tauh\ n'est pas la même dans les données réelles et simulées~\cite{TauPOG}.
Des facteurs correctifs sont déterminés par le \POG\ tau à partir d'événements Drell-Yan dans le canal \mu\tauh.
Ils sont de plus donnés séparément pour les données simulées et encapsulées.
De même, la mesure de l'isolation des \tauh\ est ajustée dans les simulations.
\paragraph{Taux de mauvaise identification $\mu\to\tauh$ (\emph{$\mu\to\tauh$ fake rate})}
Il est possible que des muons soient identifiés à tort comme des \tauh.
Il s'agit alors de mauvais \tauh\ ou \og \ftauhs \fg{}.
L'efficacité de la réjection de ces \ftauhs\ diffère entre données réelles et simulées~\cite{TauPOG}.
Un facteur d'échelle à appliquer aux simulations est fourni par le \POG\ tau en fonction de la pseudorapidité du \ftauh\ comme exposé dans le tableau~\ref{tab-chapter-HTT_analysis-section-corrections-mu_to_tau_SF}.
\begin{table}[h]
\centering
\begin{tabular}{lcccc}
\toprule
Région du détecteur & WP & 2016 & 2017 & 2018 \\
\midrule
($\num{0}<\abs{\eta}<\num{0.4}$) & \emph{VLoose} & $\num{1.25}\pm\num{0.08}$ & $\num{1.12}\pm\num{0.09}$ & $\num{1.00}\pm\num{0.08}$ \\
 & \emph{Tight} & $\num{0.38}\pm\num{0.12}$ & $\num{0.92}\pm\num{0.17}$ & $\num{0.81}\pm\num{0.15}$ \\
($\num{0.4}<\abs{\eta}<\num{0.8}$) & \emph{VLoose} & $\num{0.96}\pm\num{0.15}$ & $\num{0.76}\pm\num{0.12}$ & $\num{1.08}\pm\num{0.14}$ \\
 & \emph{Tight} & $\num{0.72}\pm\num{0.30}$ & $\num{0.79}\pm\num{0.25}$ & $\num{1.02}\pm\num{0.35}$ \\
($\num{0.8}<\abs{\eta}<\num{1.2}$) & \emph{VLoose} & $\num{1.29}\pm\num{0.11}$ & $\num{0.99}\pm\num{0.10}$ & $\num{1.04}\pm\num{0.10}$ \\
 & \emph{Tight} & $\num{1.34}\pm\num{0.27}$ & $\num{0.67}\pm\num{0.26}$ & $\num{0.92}\pm\num{0.22}$ \\
($\num{1.2}<\abs{\eta}<\num{1.7}$) & \emph{VLoose} & $\num{0.92}\pm\num{0.20}$ & $\num{0.75}\pm\num{0.14}$ & $\num{0.95}\pm\num{0.16}$ \\
 & \emph{Tight} & $\num{1.03}\pm\num{0.65}$ & $\num{1.07}\pm\num{0.45}$ & $\num{0.83}\pm\num{0.47}$ \\
($\num{1.7}<\abs{\eta}<\num{2.3}$) & \emph{VLoose} & $\num{5.01}\pm\num{0.38}$ & $\num{4.44}\pm\num{0.30}$ & $\num{5.58}\pm\num{0.40}$ \\
 & \emph{Tight} & $\num{5.05}\pm\num{0.88}$ & $\num{4.08}\pm\num{0.85}$ & $\num{4.52}\pm\num{0.92}$ \\
\bottomrule
\end{tabular}
\caption[Corrections au taux d'identification des muons comme des \tauh.]{Corrections au taux d'identification des muons comme des \tauh\ en \SI{}{\%} avec incertitude pour les trois années du Run~II.}
\label{tab-chapter-HTT_analysis-section-corrections-mu_to_tau_SF}
\end{table}
\paragraph{Taux de mauvaise identification $\ele\to\tauh$ (\emph{$\ele\to\tauh$ fake rate})}
Il est également possible que des électrons soient identifiés à tort comme des \tauh.
À l'instar des muons, un facteur d'échelle à appliquer aux simulations est fourni par le \POG\ tau en fonction de la pseudorapidité du \ftauh\ comme exposé dans le tableau~\ref{tab-chapter-HTT_analysis-section-corrections-ele_to_tau_SF}.
\begin{table}[h]
\centering
\begin{tabular}{lcccc}
\toprule
Région du détecteur & WP & 2016 & 2017 & 2018 \\
\midrule
Barillet ($\abs{\eta}<\num{1.479}$) & \emph{VVLoose} & $\num{1.38}\pm\num{0.08}$ & $\num{1.11}\pm\num{0.09}$ & $\num{0.91}\pm\num{0.06}$ \\
 & \emph{Tight} & $\num{1.22}\pm\num{0.38}$ & $\num{1.22}\pm\num{0.32}$ & $\num{1.47}\pm\num{0.27}$ \\
Bouchons ($\abs{\eta}>\num{1.479}$) & \emph{VVLoose} & $\num{1.29}\pm\num{0.08}$ & $\num{1.03}\pm\num{0.09}$ & $\num{0.91}\pm\num{0.07}$ \\
 & \emph{Tight} & $\num{1.47}\pm\num{0.32}$ & $\num{0.93}\pm\num{0.38}$ & $\num{0.66}\pm\num{0.20}$ \\
\bottomrule
\end{tabular}
\caption[Corrections au taux d'identification des électrons comme des \tauh.]{Corrections au taux d'identification des électrons comme des \tauh\ en \SI{}{\%} avec incertitude pour les trois années du Run~II.}
\label{tab-chapter-HTT_analysis-section-corrections-ele_to_tau_SF}
\end{table}
\paragraph{Efficacité d'identification et isolation des muons et des électrons (\emph{muon/electron ID/iso efficiency})}
L'efficacité de l'identification des muons et des électrons ainsi que la mesure de leur isolation peut différer entre données réelles et simulées.
Des facteurs correctifs sont déterminés par le groupe \Higgs\tau\tau.
Ils sont appliqués individuellement à chaque muon et électron utilisé dans l'analyse et dépendent de l'année ainsi que de la nature des données, simulées ou encapsulées.
\paragraph{Efficacité du trajectographe (\emph{tracking efficiency})}
L'efficacité de la reconstruction des traces des particules n'est pas la même selon la nature des données, réelles ou simulées, comme l'ont constaté les \POG s EGamma et \emph{tracking} dans le cas des électrons et des muons.
Le cas des muons a une incidence sur les \tauh\ des données encapsulées.
Des facteurs d'échelle, que ces \POG s fournissent, sont appliqués afin de corriger cet effet.
\paragraph{Efficacité du \quarkb-\emph{tagging} (\emph{Btag efficiency})}
Le \POG\ BTV fournit des facteurs correctifs $SF$ à l'efficacité du \quarkb-\emph{tagging} en fonction de la saveur du jet au niveau généré, des propriété cinématiques du jet et du point de fonctionnement du discriminateur de \quarkb-\emph{tagging} utilisé.
Plus de détails sur l'obtention de ces facteurs sont disponibles dans la référence~\cite{Sirunyan_heavy_flavor_jets_2018}.
Le taux de mauvaise identification est également corrigé.
\par
Pour cela, une méthode de promotion-relégation (\emph{promote-demote}) est utilisée.
Une fraction des jets tagués~\quarkb, \ie\ identifiés comme issus d'un quark~\quarkb, est relégué à l'état de jet non tagué~\quarkb\ et
une fraction des jets non tagués~\quarkb\ est promue à l'état de jet tagué~\quarkb.
Un jet peut être promu si son facteur correctif $SF$ est supérieur à 1.
Sinon, il peut être relégué.
La probabilité d'être promu ou relégué s'exprime
\begin{equation}
P(\text{promu}) = \frac{SF-1}{\frac{1}{\epsilon}-1}, SF > 1
\msep
P(\text{relégué}) = 1 - SF, SF < 1
\mend[,]
\end{equation}
avec $\epsilon$ l'efficacité du \quarkb-\emph{tagging}.
%\subsection{Correction d'énergie}
\begin{wrapfigure}{R}{5cm}
\centering
\begin{fmffile}{gg_loop_hHA}\fmfstraight
\begin{fmfchar*}(42,25)
  \fmfleft{g1,fi,g2}
  \fmfright{fo1,h,fo2}
  \fmf{gluon}{g1,g1loop}
  \fmf{gluon}{g2,g2loop}
  \fmf{phantom, tension=.6}{g1loop,fo1}
  \fmf{phantom, tension=.6}{g2loop,fo2}
  \fmffreeze
  \fmf{fermion}{g1loop,hloop,g2loop,g1loop}
  \fmf{fermion}{g2loop,g1loop}
  \fmf{dashes, label=$\Hs,, \Hn,, \Ha$, l.side=left, tension=1.75}{hloop,h}
  \fmfdot{g1loop,hloop,g2loop}
  \fmffreeze
  \fmf{phantom}{g1loop,fakev1}
  \fmf{phantom}{g2loop,fakev1}
  \fmffreeze
%  \fmf{phantom,tension=1.5}{hloop,fakev2}
%  \fmf{phantom, label=$t,,\bar{t}$, l.side=left}{fakev1,fakev2}
%  \fmf{phantom, label=$b,,\bar{b}$, l.side=right}{fakev1,fakev2}
  \fmflabel{\gluon}{g1}
  \fmflabel{\gluon}{g2}
\end{fmfchar*}
\end{fmffile}
\vspace{\baselineskip}
\caption[Production de boson de Higgs du MSSM par fusion de gluons.]{Diagramme de Feynman de production de boson de Higgs dans le cadre du MSSM par fusion de gluons (\gluon\gluon\Higgs).}
\label{fig-chapter-HTT_analysis-section-corrections-fgraph-gg_loop_hHA}
\end{wrapfigure}
\paragraph{Repondération de l'impulsion transverse du boson de Higgs}
Cette correction concerne les événements où un boson de Higgs est produit par fusion de gluons, comme illustré figure~\ref{fig-chapter-HTT_analysis-section-corrections-fgraph-gg_loop_hHA}.
La boucle du diagramme, fermionique, comporte des contributions provenant des quarks.
Au premier ordre non nul (LO, \emph{Leading Order}), les propriétés cinématiques du signal ne dépendent que de la masse du boson de Higgs, ce qui est couvert par la variété des jeux de données utilisés, listés dans l'annexe~\refApHTTdatasets.
Cependant, à l'ordre supérieur (NLO, \emph{Next-to Leading Order}), l'impulsion du boson de Higgs dépend également du contenu en quarks de la boucle.
Dans le cadre du MSSM, les contributions dominantes sont celles des quarks~\quarkt\ et~\quarkb.
Or, les fractions relatives de ces contributions dépendent des valeurs de $m_{\HiggsA}$ et de $\tan\beta$, paramètres introduits dans le chapitre~\refChMSSM, dont de larges gammes sont utilisées dans cette analyse.
Afin de minimiser la quantité de données simulées à produire, il a été choisi d'utiliser une simulation de référence à corriger pour rendre compte de cette dépendance en $m_{\HiggsA}$ et de $\tan\beta$ selon les méthodes introduites dans les références~\cite{Bagnaschi:2015qta,Bagnaschi:2015bop}.
\par
D'une part, une simulation de référence par point de masse ($m_{\HiggsA}$) est réalisée.
La génération de ces événements est faite au NLO à l'aide du module \texttt{gg\_H\_2HDM} de \POWHEG~\cite{Alioli:2010xd},
les gerbes partoniques, l'hadronisation et l'évenement sous-jacent sont simulés par \PYTHIA~\cite{pythia8.2} et
la modélisation du détecteur est traitée par \GEANTfour~\cite{geant4_2003,geant4_2006,geant4_2016}.
Cette simulation se fait dans le cadre d'un modèle général à deux doublets de Higgs (2HDM), dont le MSSM est un cas particulier comme exposé dans le chapitre~\refChMSSM.
\POWHEG\ permet alors d'obtenir, pour chacun des trois bosons de Higgs considérés (\higgs, \Higgs, \HiggsA), les contributions des quarks~\quarkt et~\quarkb\ ainsi que de leur interférence.
Neuf contributions au signal sont donc considérées.
En principe, toutes valeurs des paramètres $\alpha$ et $\tan\beta$ peuvent être utilisées.
En pratique, pour éviter d'obtenir un terme d'interférence presque nul menant à de faibles statistiques, ces paramètres sont fixés à $\alpha=\pi/4$ et $\tan\beta=15$.
\par
D'autre part, des simulations annexes sont réalisées pour différentes valeurs de $m_{\HiggsA}$ et de $\tan\beta$, également avec le  module \texttt{gg\_H\_2HDM} de \POWHEG, mais uniquement au niveau générateur \ie\ sans propagation dans le détecteur.
Pour chaque valeurs de $m_{\HiggsA}$ et $\tan\beta$, les distributions en \pT\ des neuf contributions considérées sont pondérées dans la simulation de référence de manière à correspondre à celles obtenues dans la simulation annexe correspondante.
\par
Le signal complet du MSSM est obtenu à partir des contributions individuelles dans le 2HDM utilisé ($\alpha=\pi/4$, $\tan\beta=15$) selon
\begin{align}
\sigma_\text{MSSM} &=
\left(\frac{y_{\quarkt,\text{MSSM}}}{y_{\quarkt,\text{2HDM}}}\right)^2 \sigma_{\quarkt,\text{2HDM}}(Q_{\quarkt})
+
\left(\frac{y_{\quarkb,\text{MSSM}}}{y_{\quarkb,\text{2HDM}}}\right)^2 \sigma_{\quarkb,\text{2HDM}}(Q_{\quarkb})
\nonumber\\&\hphantom{=}
+
\left(\frac{y_{\quarkt,\text{MSSM}}}{y_{\quarkt,\text{2HDM}}}\frac{y_{\quarkb,\text{MSSM}}}{y_{\quarkb,\text{2HDM}}}\right)
\left[ \sigma_{\quarkt+\quarkb,\text{2HDM}}(Q_{\quarkt\quarkb}) - \sigma_{\quarkt,\text{2HDM}}(Q_{\quarkt\quarkb}) - \sigma_{\quarkb,\text{2HDM}}(Q_{\quarkt\quarkb}) \right]
\label{eq-2HDM_to_MSSM_xsec}
\end{align}
où $\sigma$ peut correspondre à la section efficace inclusive ou différentielle selon une variable donnée,
$y_{\quarkt}$ et $y_{\quarkb}$ sont les constantes de couplage de Yukawa pour les quarks~\quarkt\ et~\quarkb\ introduites dans le chapitre~\refChMSSM,
$Q$ l'échelle d'énergie~\cite{Bagnaschi:2015qta,Bagnaschi:2015bop}.
Les trois termes de cette formule correspondent aux contributions du quark~\quarkt, du quark~\quarkb\ et de leur interférence.
Les valeurs de $y_{\quarkt}$ et $y_{\quarkb}$ dépendent de $m_{\HiggsA}$ et $\tan\beta$ et sont définies pour chacun des bosons de Higgs (\higgs, \Higgs, \HiggsA).
\paragraph{Repondération de l'impulsion transverse et de la masse du boson \Zboson\ (\emph{DY \pT-mass reweighting})}
Les impulsions transverses ainsi que la masse invariante des leptons issus de la désintégration du boson \Zboson\ sont corrigées
dans les événements simulés Drell-Yan.
Ces corrections sont déterminées dans une région de contrôle $\Zboson\to\mu\mu$ et n'introduisent pas de modification du nombre total d'événements.
\todo{done by DESY group}
\paragraph{Repondération de l'impulsion transverse du quark~\quarkt\ (\emph{top \pT\ reweighting})}
La modélisation du bruit de fond \ttbar\ est corrigée afin que les données simulées au NLO correspondent au NNLO.
Pour cela, la distribution en \pT\ des quarks~\quarkt\ est pondérée.
La pondération à appliquer à un quark~\quarkt, déterminée par le groupe \quarkt\antiquarkt\Higgs\ s'exprime en fonction de l'impulsion transverse du quark~\quarkt\ en \SI{}{\GeV} selon
\begin{equation}
\omega = \exp(\num{0.088} - \num{8.7e-4}\times\pT + \num{9.2e-7}\times\pT^2)
\mend
\end{equation}
Le poids total à appliquer aux événements \ttbar\ contenant deux quarks~\quarkt\ est alors
\begin{equation}
\omega(\text{total}) = \sqrt{\omega(1) \times \omega(2)}
\mend
\end{equation}
\paragraph{Recul de \MET\ (\emph{MET recoil corrections})}
La modélisation de \MET\ dans les jeux de données simulées de production du boson de Higgs, de Drell-Yan (boson \Zboson) et de \Wjets\ ne correspond pas aux observations dans les données réelles.
Des corrections sur $\vec{U}$, défini comme la différence entre \MET\ et la comme des impulsions des neutrinos provenant de la désintégration du boson de Higgs, \Zboson\ ou \Wboson, \ie
\begin{equation}
\vec{U} = \vMET - \sum_{\nu_i \leftarrow \higgs,\Zboson,\Wboson} \vpT^{(\nu_i)}
\mend[,]
\end{equation}
sont appliquées pour corriger cet effet.
\par
Les composantes colinéaire $U_1$ et orthogonale $U_2$ du vecteur $\vec{U}$ à l'impulsion du boson sont déterminées dans des événements $\Zboson\to\mu\mu$ dans lesquels il n'y a pas de neutrino provenant de la désintégration du \Zboson, ce qui permet de mesurer précisément son impulsion.
L'écart à zéro de $U_1$ ainsi que la résolution sur $U_1$ et $U_2$ sont ainsi déterminés dans les données réelles et simulées.
Les données simulées sont alors corrigées afin de faire correspondre en moyenne ces valeurs à celles observées dans les données réelles.
Ces moyennes sont déterminées sur des intervalles d'impulsion du \Zboson\ ($[\num{0}, \num{10}[$, $[\num{10}, \num{20}[$, $[\num{20}, \num{30}[$, $[\num{30}, \num{50}[$ et $>\SI{50}{\GeV}$) et du nombre de jets ($\Njets\in\set{0, 1, \geq2}$).
\todo{done by DESY group}
\paragraph{Énergie des jets (\emph{jet energy calibration})}
L'énergie des jets est corrigée selon la procédure abordée en détails dans le chapitre~\refChJERC.
\paragraph{Énergie des \tauh\ (\emph{\tauh\ energy scale})}
L'énergie mesurée des \tauh\ peut différer entre les \tauh\ réels et simulés, ainsi que selon le DM du \tauh~\cite{TauPOG}.
Le \POG\ tau fournit les corrections à appliquer aux \tauh\ simulés, elles sont données dans le tableau~\ref{tab-chapter-HTT_analysis-section-corrections-tauES}.
Ces corrections sont obtenues à partir d'événements du canal \mu\tauh, par exploitation de la masse du \tauh\ et de la masse visible du système \mu\tauh.
Elles sont dépendantes de l'année, du DM et du type de données, simulées ou encapsulées.
Les données encapsulées sont présentées dans la section~\ref{chapter-HTT_analysis-section-bg_estimation-embedding}.
\begin{table}[h]
\centering
\subcaptionbox{Pour les données simulées.\label{tab-chapter-HTT_analysis-section-corrections-tauES-MC}}[0.45\textwidth]
{\begin{tabular}{lccc}
\toprule
DM & 2016 & 2017 & 2018\\
\midrule
0 & $\num{-0.6}\pm\num{1.0}$ & $\num{0.7}\pm\num{0.8}$ & $\num{-1.3}\pm\num{1.1}$ \\
1 & $\num{-0.5}\pm\num{0.9}$ & $\num{-0.2}\pm\num{0.8}$ & $\num{-0.5}\pm\num{0.9}$ \\
10 & $\num{0.0}\pm\num{1.1}$ & $\num{0.1}\pm\num{0.9}$ & $\num{-1.2}\pm\num{0.8}$ \\
11 & $\num{0.1}\pm\num{1.0}$ & $\num{-0.5}\pm\num{1.6}$ & $\num{0.1}\pm\num{1.0}$ \\
\bottomrule
\end{tabular}}
\qquad
\subcaptionbox{Pour les données encapsulées.\label{tab-chapter-HTT_analysis-section-corrections-tauES-EMB}}[0.45\textwidth]
{\begin{tabular}{lccc}
\toprule
DM & 2016 & 2017 & 2018\\
\midrule
0 & $\num{-0.2}\pm\num{0.5}$ & $\num{0.0}\pm\num{0.4}$ & $\num{-0.3}\pm\num{0.4}$ \\
1 & $\num{-0.2}\pm\num{0.3}$ & $\num{-1.2}\pm\num{0.5}$ & $\num{-0.6}\pm\num{0.4}$ \\
10 & $\num{-1.3}\pm\num{0.5}$ & $\num{-0.8}\pm\num{0.5}$ & $\num{-0.7}\pm\num{0.3}$ \\
11 & $\num{-1.3}\pm\num{0.5}$ & $\num{-0.8}\pm\num{0.5}$ & $\num{-0.7}\pm\num{0.3}$ \\
\bottomrule
\end{tabular}}
\caption[Corrections à l'énergie des taus hadroniques.]{Corrections à l'énergie des taus hadroniques en \SI{}{\%} avec incertitude pour les trois années du Run~II.}
\label{tab-chapter-HTT_analysis-section-corrections-tauES}
\end{table}
\paragraph{Énergie des muons identifiés comme \tauh\ (\emph{$\mu\to\tauh$ energy scale})}
Il est possible que des muons soient identifiés à tort comme des \tauh.
Il s'agit alors de mauvais \tauh\ ou \og \ftauhs \fg{}.
À l'instar des vrais \tauh\ discutés dans le paragraphe précédent, l'énergie mesurée de ces \ftauhs\ peut différer entre les données réelles et simulées.
Dans ce cas, le quadrivecteur du \ftauh\ est directement corrigé selon le DM du \tauh\ identifié.
Cette correction, généralement inférieure au pourcent, est appliquée uniquement aux DMs 0 et 1 et pour des \tauh\ correspondant au niveau généré à un muon.
La quantité de muons identifiés comme des \tauh\ avec un DM plus élevé, en particulier les DMs 10 et 11, est négligeable, c'est pourquoi aucune correction n'est prévu dans ce cas.
Les valeurs des corrections à appliquer aux données simulées sont données dans le tableau~\ref{tab-chapter-HTT_analysis-section-corrections-tauES-mu}.
\paragraph{Énergie des électrons identifiés comme \tauh\ (\emph{$\ele\to\tauh$ energy scale})}
Toute comme les muons, les électrons peuvent être identifiés à tort comme des \tauh.
La correction correspondante est similaire au cas des muons, mais peut être de l'ordre de \SI{5}{\%} selon le DM et la pseudorapidité.
Les valeurs des corrections à appliquer aux données simulées sont données dans les tableaux~\ref{tab-chapter-HTT_analysis-section-corrections-tauES-ele_barrel} et~\ref{tab-chapter-HTT_analysis-section-corrections-tauES-ele_endcap}.
\begin{table}[h]
\centering
\subcaptionbox{Muons.\label{tab-chapter-HTT_analysis-section-corrections-tauES-mu}}[0.3\textwidth]
{\begin{tabular}{lccc}
\toprule
DM & 2016 & 2017 & 2018\\
\midrule
0 & \num{0.0} & \num{-0.2} & \num{-0.2} \\
1 & \num{-0.5} & \num{-0.8} & \num{-1.0} \\
\bottomrule
\end{tabular}}
\hfill
\subcaptionbox{Électrons du barillet ($\abs{\eta}<\num{1.479}$).\label{tab-chapter-HTT_analysis-section-corrections-tauES-ele_barrel}}[0.3\textwidth]
{\begin{tabular}{lccc}
\toprule
DM & 2016 & 2017 & 2018\\
\midrule
0 & \num{0.7} & \num{0.9} & \num{1.4} \\
1 & \num{3.4} & \num{1.2} & \num{1.9} \\
\bottomrule
\end{tabular}}
\hfill
\subcaptionbox{Électrons des bouchons ($\abs{\eta}>\num{1.479}$).\label{tab-chapter-HTT_analysis-section-corrections-tauES-ele_endcap}}[0.3\textwidth]
{\begin{tabular}{lccc}
\toprule
DM & 2016 & 2017 & 2018\\
\midrule
0 & \num{-0.4} & \num{-2.6} & \num{-3.1} \\
1 & \num{5.0} & \num{1.5} & \num{-1.5} \\
\bottomrule
\end{tabular}}
\caption[Corrections à l'énergie des leptons identifiés comme des taus hadroniques.]{Corrections à l'énergie des électrons et des muons identifiés comme des taus hadroniques en \SI{}{\%} avec incertitude pour les trois années du Run~II.}
\label{tab-chapter-HTT_analysis-section-corrections-tauES-leptons}
\end{table}
\paragraph{Énergie des électrons (\emph{electron energy scale})} EMB
La mesure de l'énergie des électrons dans les données simulées est corrigée selon les recommandations du \POG\ EGamma (électrons et photons)~\cite{EGammaPOG}, résumées dans le tableau~\ref{tab-chapter-HTT_analysis-section-corrections-eleES}.
\begin{table}[h]
\centering
\begin{tabular}{lccc}
\toprule
Région du détecteur & 2016 & 2017 & 2018\\
\midrule
Barillet ($\abs{\eta}<\num{1.479}$) & $\num{-0.24}\pm\num{0.5}$ & $\num{-0.07}\pm\num{0.5}$ & $\num{-0.33}\pm\num{0.5}$ \\
Bouchons ($\abs{\eta}>\num{1.479}$) & $\num{-0.70}\pm\num{1.25}$ & $\num{-1.13}\pm\num{1.25}$ & $\num{-0.56}\pm\num{1.25}$ \\
\bottomrule
\end{tabular}
\caption[Corrections à l'énergie des électrons.]{Corrections à l'énergie des électrons en \SI{}{\%} avec incertitude pour les trois années du Run~II.}
\label{tab-chapter-HTT_analysis-section-corrections-eleES}
\end{table}