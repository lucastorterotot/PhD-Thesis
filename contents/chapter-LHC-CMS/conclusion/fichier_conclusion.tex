\section{Conclusion}\label{chapter-LHC-section-conclusion}
Le détecteur CMS, exploité par la collaboration du même nom, est le dispositif expérimental utilisé dans cette thèse.
Il s'agit d'une des quatre grandes expériences installées au LHC, le plus grand des collisionneurs de hadrons au monde à ce jour, qui se trouve au CERN.
\par L'acronyme \og CERN \fg{} signifie Conseil Européen pour la Recherche Nucléaire, mais le CERN correspond aujourd'hui à l'organisation européenne pour la recherche nucléaire.
Basé au Nord-Ouest de Genève, il s'étend des deux côtés de la frontière franco-suisse et comporte de nombreuses installations expérimentales de physique nucléaire et des particules, entre autres.
Des innovations majeures sont issues des recherches menées CERN, comme les écrans tactiles, le Web et la hadronthérapie.
\par Le LHC, le Grand Collisionneur de Hadrons, est un accélérateur circulaire de \SI{27}{\kilo\meter} de circonférence.
Il permet d'atteindre des énergies de collision dans le centre de masse de \SI{13}{\TeV} pour des protons et il est prévu de passer à \SI{14}{\TeV} dès cette année.
Les expériences ALICE, ATLAS, CMS, LHCb, LHCf, TOTEM et MoEDAL y sont installées.
\par La composition et le fonctionnement du détecteur CMS ont été détaillées.
Ce détecteur est de forme cylindrique et possède une structure en couches concentriques, chacune étant un sous-détecteur ayant un rôle précis.
Tout d'abord, le trajectographe permet d'obtenir les trajectoires des particules chargées.
Puis, le calorimètre électromagnétique stoppe les électrons et les photons en mesurant leurs énergies.
Le calorimètre hadronique fait de même avec les hadrons.
La couche suivante, le solénoïde, ne détecte pas les particules mais produit un champ magnétique de \SI{4}{\tesla} afin de courber les trajectoires des particules chargées.
Enfin, la couche externe est constituée d'une culasse d'acier pour le retour du champ magnétique dans laquelle sont insérées des chambres à muons, détectant le passage de ces particules.
\par Les 40 millions d'événements par seconde du LHC donnent une quantité de données bien trop importante pour toutes les stocker.
Un système de déclenchement à deux niveaux est utilisé afin de n'en conserver que 100 par seconde environ.
Le premier niveau se base sur les signaux bruts du détecteur tandis que le second procède à une reconstruction simple de l'événement pour l'analyser plus en détails.
\par
Un algorithme de reconstruction permet de déterminer quelles particules sont issues des collisions.
Cet algorithme se base sur les signaux du détecteur et leurs corrélations, en particulier spatiales, afin d'estimer la nature et les propriétés des particules présentes lors de l'événement.
Des objets physiques de haut niveau sont définis à partir des particules reconstruites.
Il s'agit de l'énergie transverse manquante, des jets et des taus hadroniques.
\par
Une simulation d'événements physiques et du détecteur lui-même permettent d'obtenir une estimation des observations attendues pour un modèle théorique donné.
Cette simulation est corrigée de divers effets, mesurés par des analyses dédiées.
Une de ces corrections porte sur l'énergie des jets
et est détaillée dans le chapitre~\refChJERC\
qui présente une analyse dédiée à son obtention.