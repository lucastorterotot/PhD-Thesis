\section{Introduction}\label{chapter-LHC-section-introduction}
L'étude des particules élémentaires présentées au chapitre~\refChMSSM\ nécessite des conditions expérimentales particulières.
En effet,
hormis le photon et les fermions de la première génération,
la plupart d'entre elles ont une durée de vie inférieure à la nanoseconde.
Il faut donc dans un premier temps les créer.
Dans certains cas, les conditions naturelles le permettent.
Par exemple,
les rayons cosmiques issus du Soleil produisent de nombreuses particules lors de leur interaction avec l'atmosphère.
Leur étude a ainsi permis la découverte des muons~\cite{muon_discovery}.
Toutefois,
les particules les plus massives nécessitent des gammes d'énergies bien plus élevées afin d'être produites
et
certains processus du modèle standard ont une faible section efficace, \ie\ une faible probabilité de survenir.
La caractérisation des particules de l'Univers ne peut donc se faire uniquement par des observations de processus naturels.
\par Le Grand Collisionneur de Hadrons (LHC, \emph{Large Hadron Collider})~\cite{LHC_paper1,LHC_paper2,LHC_paper3} de l'organisation européenne pour la recherche nucléaire ou CERN (Conseil Européen pour la Recherche Nucléaire)~\cite{CERN_website} permet de réaliser des collisions entre particules.
Cet environnement expérimental, contrôlé, permet d'atteindre des échelles d'énergies suffisantes pour produire des particules de haute masse et ce en très grand nombre.
Ces conditions sont nécessaires afin de d'observer et de caractériser certaines particules élémentaires comme le boson de Higgs~\cite{ATLAS_Higgs_discovery,CMS_Higgs_discovery,CMS_Higgs_discovery_2013,ATLAS-CMS-Higgs_combined_1,ATLAS-CMS-Higgs_combined_2}.
\par Le CERN, présenté dans la section~\ref{chapter-LHC-section-CERN},
possède plusieurs collisionneurs de particules dont
le plus grand et le plus puissant à ce jour, le LHC, est introduit dans la section~\ref{chapter-LHC-section-LHC}.
La section~\ref{chapter-LHC-section-CMS} présente le détecteur CMS, installé au LHC, dont les données expérimentales sont utilisées dans cette thèse.
Les particules ne sont pas observées en tant que telles dans le détecteur,
seuls des signaux de leur passage sont récoltés.
La reconstruction des particules physiques à partir de ces signaux est décrite dans la section~\ref{chapter-LHC-section-evt_reco}.
La compréhension et la validation des phénomènes fondamentaux liés au modèle théorique étudié nécessite quand à lui la comparaison des données récoltées expérimentalement avec des données simulées, ce qui est abordé dans la section~\ref{chapter-LHC-section-MC}.