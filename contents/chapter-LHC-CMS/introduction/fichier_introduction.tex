\section{Introduction}\label{chapter-LHC-section-introduction}
La matière \og du quotidien \fg{} est constituée des fermions de la première génération\footnote{Les particules du modèle standard sont présentées dans le chapitre~\ifref{chapter-MS-MSSM}{\ref{chapter-MS-MSSM}}{2}.}.
L'étude des autres particules fondamentales doit donc nécessairement se faire dans des conditions particulières.
Les rayons cosmiques issus du Soleil produisent de nombreuses particules lors de leur interaction avec l'atmosphère.
Leur étude a ainsi permis la découverte des muons~\cite{muon_discovery}.
Toutefois, les particules les plus massives nécessitent des gammes d'énergies bien plus élevées afin d'être produites.
De plus, certains processus du modèle standard ont une faible section efficace, \ie\ une faible probabilité de survenir.
Enfin, de nombreux paramètres sont difficiles voir impossibles à obtenir sans connaître les conditions expérimentales de tels événements naturels.
La caractérisation des particules de l'Univers ne peut donc se faire uniquement par ces observations.
\par Le Grand Collisionneur de Hadrons (LHC, \emph{Large Hadron Collider})~\cite{LHC_paper1,LHC_paper2,LHC_paper3} de l'organisation européenne pour la recherche nucléaire ou CERN (Conseil Européen pour la Recherche Nucléaire)~\cite{CERN_website} réalise des collisions entre particules.
Cet environnement expérimental, contrôlé, permet d'atteindre des échelles d'énergies suffisantes pour produire des particules de hautes masses et ce de nombreuses fois.
Ces conditions sont nécessaires afin de découvrir certaines particules fondamentales comme le boson de Higgs~\cite{ATLAS_Higgs_discovery,CMS_Higgs_discovery,CMS_Higgs_discovery_2013,ATLAS-CMS-Higgs_combined_1,ATLAS-CMS-Higgs_combined_2} et les caractériser.
\par Le CERN est présenté dans la section~\ref{chapter-LHC-section-CERN}.
Le plus grand et le plus puissant de ses collisionneurs de particules à ce jour, le LHC, est introduit dans la section~\ref{chapter-LHC-section-LHC}.
La section~\ref{chapter-LHC-section-CMS} présente l'expérience CMS, une des quatre grandes expériences du LHC, dont les données expérimentales sont utilisées dans cette thèse.
La comparaison des observations expérimentales aux prédictions théoriques peut se faire en simulant des collisions, ce qu'aborde la section~\ref{chapter-LHC-section-MC}.
Enfin, la section~\ref{chapter-LHC-section-evt_reco} explique comment les signaux détectés permettent de déterminer les particules présentes.