\section{Le LHC: \emph{Large Hadron Collider}}\label{chapter-LHC-section-LHC}

\subsection{Collisions de protons}\label{chapter-LHC-section-LHC-subsec-pp_collisions}

\subsection{Accélération de protons}\label{chapter-LHC-section-LHC-subsec-pp_acceleration}

\subsection{Luminosité et nombre d'événements}\label{chapter-LHC-section-LHC-subsec-lumi}

\subsection{L'empilement}\label{chapter-LHC-section-LHC-subsec-PU}

\subsection{Les expériences du LHC}\label{chapter-LHC-section-LHC-subsec-experiments}
Quatre grandes expériences sont présentes sur le LHC. Elles se situent chacune à un des points d'interaction de l'anneau afin d'étudier les collisions qui y sont produites.
\begin{description}
\item[ALICE]\cite{alice_paper}, \emph{A Large Ion Collider Experiment}, est une expérience conçue pour étudier le déconfinement des quarks et des gluons à l'aide de collisions d'ions lourds. Ces études permettent de mieux comprendre le fonctionnement de la chromodynamique quantique ou QCD.
\item[ATLAS]\cite{atlas_paper}, \emph{A Toroidal LHC ApparatuS}, est une expérience généraliste avec un éventail d'études très large, allant des mesures de précision des paramètres du modèle standard à la recherche de nouvelle physique.
\item[CMS]\cite{cms_paper}, \emph{Compact Muon Solenoid}, est également une expérience généraliste dont les objectifs sont similaires à ceux d'ATLAS. Les détecteurs d'ATLAS et de CMS étant conçus différemment, ces deux expériences peuvent valider leurs résultats de manière indépendante.
\item[LHCb]\cite{lhcb_paper}, \emph{Large Hadron Collider beauty}, se concentre sur l'étude de la violation de la symétrie CP avec le quark~\quarkb, qui lui donne son nom. Cette expérience réalise également des mesures de précision de certains paramètres du modèle standard.
\end{description}
