\section{Le LHC: \emph{Large Hadron Collider}}\label{chapter-LHC-section-LHC}
Le Grand Collisionneur de Hadrons~\cite{LHC_paper1,LHC_paper2,LHC_paper3} (LHC, \emph{Large Hadron Collider}) est le plus grand et le plus puissant accélérateur de particules au monde.
Son tracé ainsi que ceux du \emph{Booster}, du PS et du SPS sont illustrés sur la figure~\ref{fig-CERN_map}.
Le LHC est installé dans le même tunnel que le LEP, il s'agit donc d'un accélérateur circulaire de \SI{27}{\kilo\meter} de circonférence, situé entre \num{50} et \SI{100}{\meter} sous la frontière franco-suisse.
\par Le LHC permet de réaliser des collisions proton-proton, proton-ion lourd et ion lourd-ion lourd.
Les collisions d'ions lourds permettent de reproduire les conditions des premiers instants de l'Univers après le \emph{Big Bang} et sont principalement étudiée par l'expérience ALICE, une des quatre expériences du LHC présentées dans la section~\ref{chapter-LHC-section-LHC-subsec-experiments}.
Dans tous les cas, deux faisceaux de particules sont accélérés en sens inverses.
Seules les collisions de protons sont considérées dans cette thèse.
\subsection{Exploitation du LHC}\label{chapter-LHC-section-LHC-subsec-LHC_runs}
Le fonctionnement du LHC peut être divisé en plusieurs périodes ou \emph{Runs}.
Chaque \emph{Run} du LHC présente différentes caractéristiques, en particulier l'énergie dans le centre de masse des collisions.
Le tableau~\ref{tab-LHC_runs} résume les différents \emph{Runs} du LHC, passés et à venir.
Chaque \emph{Run} est lui-même divisé par année civile, des arrêts techniques étant faits en période hivernale.
Enfin, une année civile est subdivisée en plusieurs périodes (A, B, C, etc.) entre lesquelles les conditions expérimentales peuvent varier, comme la nature des particules entrant en collision.
\begin{table}[h]
\centering
\begin{tabular}{cccc}
\toprule
Run & Période & Énergie dans le centre de masse & Luminosité\\
\midrule
I & 2011-2012 & 7 à \SI{8}{\TeV} & \SI{30}{\femto\barn^{-1}} \\
II & 2016-2018 & \SI{13}{\TeV} & \SI{190}{\femto\barn^{-1}} \\
III & 2021-2024 & 13 à \SI{14}{\TeV} & \SI{350}{\femto\barn^{-1}}? \\
IV & 2027-2030 & \SI{14}{\TeV} & \multirow{2}{*}{\SI{3000}{\femto\barn^{-1}}?}\\
V & 2032-2034 & \SI{14}{\TeV} & \\
\bottomrule
\end{tabular}
\caption[Runs du LHC.]{Runs du LHC avec les énergies dans le centre de masse et les luminosités correspondantes~\cite{LHC_commissioning}. La luminosité est présentée dans la section~\ref{chapter-LHC-section-LHC-subsec-lumi}.}
\label{tab-LHC_runs}
\end{table}
\subsection{Accélération de protons}\label{chapter-LHC-section-LHC-subsec-pp_acceleration}
Les protons sont obtenus par ionisation de dihydrogène, directement issu d'une bouteille.
Les protons sont alors progressivement accélérés à travers différentes installations du CERN, illustrées sur la figure~\ref{fig-chapter-LHC-section-CERN-CERN_Accelerator-Complex}, menant les protons à des niveaux d'énergie de plus en plus hauts avant de pouvoir être injectés dans le LHC~\cite{LHC_paper3}:
\begin{itemize}
\item l'accélérateur linéaire 2 (LINAC~2)\footnote{Le LINAC~2 est remplacé pour le Run~III du LHC par le LINAC~4.} permet d'accélérer les protons à une énergie de \SI{50}{\MeV};
\item le \emph{Booster}, premier élément circulaire, amène les protons à \SI{1.4}{\GeV};
\item le PS permet d'atteindre \SI{25}{\GeV};
\item le SPS, dernier élément avant le LHC, accélère les protons jusqu'à \SI{450}{\GeV}.
\end{itemize}
Le LHC accélère alors les protons jusqu'à \SI{6.5}{\TeV} lors du Run~II et ira jusqu'à \SI{7}{\TeV} lors du Run~III, permettant de réaliser des collisions avec des énergies dans le centre de masse de \num{13} et \SI{14}{\TeV}, respectivement.
\par Le gain en énergie des particules, \ie\ l'accélération colinéaire aux faisceaux, se fait dans le LHC à l'aide de 16 cavités radiofréquences, 8 par faisceau.
Ces cavités créent un champ électrique oscillant.
Cette technique présente l'avantage d'accélérer les particules différemment selon leurs positions respectives, ce qui permet de les conserver en plusieurs paquets dont l'intérêt est développé dans la section suivante.
\par Afin de maintenir les particules dans le tube circulaire du LHC, leur trajectoire est courbée, \ie\ qu'une accélération orthogonale aux faisceaux est appliquée.
Cette courbure s'obtient grâce à un champ magnétique de \SI{8.33}{\tesla} généré par 1232 aimants dipolaires supraconducteurs répartis tout au long des \SI{27}{\kilo\meter} du LHC et refroidis à l'hélium superfluide à \SI{1.8}{\kelvin}.
\subsection{Collisions de protons}\label{chapter-LHC-section-LHC-subsec-pp_collisions}

\subsection{Luminosité et nombre d'événements}\label{chapter-LHC-section-LHC-subsec-lumi}

\subsection{L'empilement}\label{chapter-LHC-section-LHC-subsec-PU}

\subsection{Les expériences du LHC}\label{chapter-LHC-section-LHC-subsec-experiments}
Il existe sept expériences au LHC.
Parmi elles, quatre sont de \og grandes expériences \fg{} et se situent chacune aux points d'interactions de l'anneau afin d'étudier les collisions qui y sont produites.
\begin{description}
\item[ALICE]\cite{alice_paper}, \emph{A Large Ion Collider Experiment}, est une expérience conçue pour étudier le déconfinement des quarks et des gluons à l'aide de collisions d'ions lourds. Ces études permettent de mieux comprendre le fonctionnement de la chromodynamique quantique ou QCD. Elle est installée au point~2 et réutilise l'aimant octogonal rouge très caractéristique de l'expérience L3 du LEP.
\item[ATLAS]\cite{atlas_paper}, \emph{A Toroidal LHC ApparatuS}, est une expérience généraliste avec un éventail d'études très large, allant des mesures de précision des paramètres du modèle standard à la recherche de nouvelle physique. Ce détecteur se trouve au point~1 du LHC.
\item[CMS]\cite{cms_paper}, \emph{Compact Muon Solenoid}, est également une expérience généraliste dont les objectifs sont similaires à ceux d'ATLAS. Les détecteurs d'ATLAS et de CMS étant conçus différemment, ces deux expériences peuvent valider leurs résultats de manière indépendante. Le détecteur CMS est installé au point~5 du LHC, à l'exact opposé d'ATLAS.
\item[LHCb]\cite{lhcb_paper}, \emph{Large Hadron Collider beauty}, se concentre sur l'étude de la violation de la symétrie CP avec le quark~\quarkb, qui lui donne son nom. Cette expérience réalise également des mesures de précision de certains paramètres du modèle standard. L'expérience LHCb se situe au point~8.
\end{description}
\par Les trois autres expériences du LHC sont LHCf, TOTEM et MoEDAL.
L'expérience LHCf~\cite{lhcf_paper} (\emph{Large Hadron Collider forward}), installée à \SI{140}{\meter} de part et d'autre du détecteur ATLAS, observe les particules issues des collisions et presque alignées avec le faisceau du LHC afin de simuler des rayons cosmiques.
La plus \og longue \fg{} des expériences du CERN, TOTEM~\cite{totem_paper} (\emph{Total, elastic and diffractive cross-section measurement}), est quant à elle installée sur un demi kilomètre autour de CMS et étudie les protons grâce aux particules alignées avec le faisceau.
Enfin, MoEDAL~\cite{moedal_paper} (\emph{Monopole and Exotics Detector At the LHC}) cherche à détecter l'existence de monopoles magnétiques et de particules ionisantes massives prédites par certains modèles au-delà du modèle standard grâce à des détecteurs installés près de LHCb.