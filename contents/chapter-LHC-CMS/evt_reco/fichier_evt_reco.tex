\section{Reconstruction des événements}\label{chapter-LHC-section-evt_reco}
L'analyse des événements à partir des signaux bruts issus du détecteur n'est pas aisée, qu'il s'agisse de données réelles ou de simulées.
Une interprétation de ces signaux en termes de particules physiques donne un point de départ beaucoup plus accessible.
%En effet, ces signaux sont comme des \og symptômes \fg{} du passage des particules dans le détecteur.
%La reconstruction des événements consiste donc en un \og diagnostic \fg{} de l'événement.
Pour y parvenir, un algorithme de reconstruction est utilisé.
Son rôle est de déterminer quelles particules sont issues de la collision étant donnés les signaux dans le détecteur.
Cette reconstruction est réalisée de manière identique dans les données réelles et simulées.
\par
Les signaux caractéristiques des différents types de particules dans le plan transverse du détecteur CMS sont illustrées sur la figure~\ref{fig-chapter-LHC-section-evt_reco-cms_slice}.
Dans toute la suite de ce manuscrit, les schémas des événements dans le détecteur utilisent la même présentation que sur la figure~\ref{fig-chapter-LHC-section-evt_reco-cms_slice}, en particulier pour les couleurs des sous-détecteurs et des particules.
Il s'agit de la conséquence directe de la description du détecteur de la section~\ref{chapter-LHC-section-CMS}.
La plupart des particules laissent des signaux dans plusieurs sous-détecteurs.
Pour une particule donnée, ces signaux doivent donc présenter une corrélation.
\par La reconstruction des événements se fait ainsi en combinant les informations issues des différents sous-détecteurs.
Un algorithme spécialement développé afin d'optimiser cette combinaison, l'algorithme de reconstruction du flux de particules (\PF, \emph{Particle Flow})~\cite{particle-flow,Dordevic_particle_flow}, a pour rôle de réaliser cette reconstruction.
Dans un premier temps, l'algorithme \PF\ reconstruit les éléments d'identification des particules à partir des signaux de chaque sous-détecteur.
Dans un second temps, ces éléments d'identification sont combinés afin de reconstruire les particules de l'événement.
Les sections~\ref{chapter-LHC-section-evt_reco-subsec-PF_elements} et~\ref{chapter-LHC-section-evt_reco-subsec-ptc_ID} présentent ces deux étapes.
\begin{figure}[h]
\includegraphics[width=\textwidth]{\PhDthesisdir/plots_and_images/CMS_slices/own/all_ptcs.tex}
\caption[Coupe transverse schématique du détecteur CMS.]{Coupe transverse schématique du détecteur CMS et signaux caractéristiques laissés par les particules. Les hadrons forment un dépôt dans le HCAL. Les hadrons chargés présentent de plus une trace dans le trajectographe dont l'extrapolation doit passer par ce dépôt. De même, les photons et électrons forment un dépôt dans le ECAL et les électrons présentent une trace dans le trajectographe dont l'extrapolation doit passer par ce dépôt. Enfin, les muons se propagent à travers tout le détecteur et laissent une trace dans les chambres à muons dont l'extrapolation doit correspondre à une trace du trajectographe. Réalisé à l'aide de CMSTransverseTikZ~\cite{CMSTransverseTikZ}.}
\label{fig-chapter-LHC-section-evt_reco-cms_slice}
\end{figure}
\subsection{Éléments d'identification du \emph{Particle Flow}}\label{chapter-LHC-section-evt_reco-subsec-PF_elements}
\subsubsection{Traces des particules chargées et vertex}
Les particules chargées laissent des traces de leur passage dans le trajectographe et, dans le cas des muons, dans les chambres à muons.
La structure du trajectographe présentée en section~\ref{chapter-LHC-section-CMS-subsec-tracker} ne donne pas de traces continues mais des points de passage de ces particules.
\par Les traces des particules sont alors reconstruites à partir de ces points de passage à l'aide d'une méthode itérative~\cite{track_reco}.
Les traces présentant au moins huit points de passage avec au plus un point manquant le long de l'extrapolation obtenue sont considérées pour la suite de la reconstruction~\cite{particle-flow}.
Il est également requis que les traces proviennent d'un cylindre de quelques millimètres de rayon autour du faisceau et qu'elles correspondent à une impulsion transverse minimale de \SI{0.9}{\GeV}.
Les performances de cette méthode sont discutées dans la référence~\cite{CMS_TDR_1}.
\par Les particules peuvent être déviées par interaction avec la matière du trajectographe~\cite{moliere_scat_1,moliere_scat_2}.
Un algorithme dédié à ce phénomène est utilisé et permet de déterminer les lieux de ces interactions, représentés sur la figure~\ref{fig-chapter-LHC-section-evt_reco-subsec-PF_elements-CMS-self-radio}.
La structure du trajectographe se retrouve et le taux de mauvaise reconstruction est réduit par un traitement spécifique des traces présentant de telles déviations.
\begin{figure}[h]
\centering
\subcaptionbox{Dans le plan longitudinal \plane{r}{z}.\label{subfig-chapter-LHC-section-evt_reco-subsec-PF_elements-particle-flow-Figure_005-a}}[.45\textwidth]
{\includegraphics[width=.45\textwidth]{\PhDthesisdir/plots_and_images/from_particle-flow/Figure_005-a.png}}
\hfill
\subcaptionbox{Dans le plan transverse \plane{x}{y}.\label{subfig-chapter-LHC-section-evt_reco-subsec-PF_elements-particle-flow-Figure_005-b}}[.45\textwidth]
{\includegraphics[width=.45\textwidth]{\PhDthesisdir/plots_and_images/from_particle-flow/Figure_005-b.png}}
\caption[Points d'interactions entre particules des événements et matière composant le détecteur.]{Carte des points d'interactions entre particules des événements et matière composant le détecteur~\cite{particle-flow} à partir de données prises en 2011 à $\sqrt{s}=\SI{13}{\TeV}$.}
\label{fig-chapter-LHC-section-evt_reco-subsec-PF_elements-CMS-self-radio}
\end{figure}
\par Les traces des électrons et des muons présentent des spécificités particulières.
Les électrons émettent avant de parvenir au ECAL une fraction non négligeable de leur énergie par \emph{bremsstrahlung}, \ie\ sous forme de photons.
Les performances de reconstruction des électrons dépendent ainsi fortement de la capacité à identifier ces photons et mesurer leurs énergies.
Dans le cas des muons, les signaux de leur passage dans le trajectographe et dans les chambres à muons permettent de définir trois types de traces pour les éléments de reconstruction:
\begin{itemize}
\item les muons seuls (\emph{standalone muons}), reconstruits uniquement à partir des signaux des chambres à muons;
\item les muons globaux (\emph{global muons}), obtenus par la correspondance d'une trace dans le trajectographe avec l'extrapolation de la trace d'un muon seul;
\item les muons du trajectographe (\emph{tracker muons}) sont les traces du trajectographe d'impulsion transverse supérieure à \SI{0.5}{\GeV} dont l'extrapolation passe par une des chambres à muons ayant détecté le passage d'une particule.
\end{itemize}
Plus de détails sont disponibles dans les sections~3.2 et~3.3 de la référence~\cite{particle-flow}.
\subsubsection{Dépôts dans les calorimètres}
Les dépôts dans les calorimètres sont regroupés de proche en proche en agglomérats (\emph{clusters})~\cite{particle-flow}, indépendamment pour chaque sous-partie des calorimètres.
Plusieurs raisons existent à cette agglomération:
\begin{itemize}
\item détecter et mesurer les énergies et directions des particules neutres stables comme les photons et les hadrons neutres;
\item séparer les dépôts des particules neutres de ceux des particules chargées;
\item reconstruire et identifier les électrons et les photons issus du \emph{bremsstrahlung} correspondant;
\item améliorer la mesure de l'énergie des hadrons chargés dont les traces sont imprécises.
\end{itemize}
\par La construction de ces agglomérats commence par l'identification des cellules des calorimètres mesurant une énergie supérieure à un seuil, défini pour chaque sous-partie des calorimètres.
Les cellules adjacentes sont ajoutées à l'agglomérat.
Puis, toute cellule avec au moins un coin en commun avec une cellule déjà dans l'agglomérat et mesurant une énergie supérieure à deux fois le niveau moyen du bruit est ajoutée à l'agglomérat.
\par Les photons et les hadrons neutres ne peuvent être reconstruits qu'à l'aide de leurs dépôts dans les calorimètres.
Des dépôts isolés vis-à-vis des traces de particules chargées sont ainsi une signature claire des particules neutres.
Cependant, un dépôt de particule neutre situé au même endroit qu'un dépôt de particule chargée est identifié comme un excès d'énergie par rapport à l'énergie déterminée à l'aide du trajectographe.
Les agglomérats obtenus sont alors calibrés afin de pouvoir identifier les dépôts de particules neutres chevauchant un dépôt de particule chargée.
Plus de détails sont disponibles dans la section~3.5 de la référence~\cite{particle-flow}.
\subsection{Identification et reconstruction des particules}\label{chapter-LHC-section-evt_reco-subsec-ptc_ID}
Des éléments d'identification du \PF\ dans différents sous-détecteurs sont généralement dus à une même particule.
La reconstruction des particules se fait alors par association de ces éléments.
L'association des éléments dus à une particule est uniquement limitée par la granularité des sous-détecteurs et par le nombre de particules par unité d'angle solide~\cite{particle-flow}.
De même, l'association de tous les éléments dus à une seule particule est limitée par la quantité de matière traversée en amont des calorimètres ou, le cas échéant, des chambres à muons, pouvant dévier la particule~\cite{moliere_scat_1,moliere_scat_2}.
\par Un algorithme teste les paires d'éléments de reconstruction possibles.
Afin de limiter les temps de calcul, seules les paires d'éléments les plus proches entre eux dans le plan $(\eta,\phi)$ sont considérées.
Des conditions supplémentaires sont requises afin d'associer deux éléments et sont détaillées dans les sections suivantes.
Lorsque deux éléments sont associés, une distance est définie par l'algorithme afin de quantifier la qualité de cette association.
Des \og blocs \fg{} du \PF\ sont ainsi obtenus par association des éléments de reconstruction.
Selon le contenu de ce bloc, un type de  particule est reconstruit.
Ces différents types de particules sont détaillés dans les sections qui suivent.
\subsubsection{Muons}
Les muons sont reconstruits à partir des éléments d'identifications que sont les muons globaux, seuls et du trajectographe définis dans la section~\ref{chapter-LHC-section-evt_reco-subsec-PF_elements-subsubsec-tracks}.
\par Tout d'abord, les muons globaux isolés, \ie\ sans autre activité dans le voisinage de la trajectoire correspondante, sont sélectionnés~\cite{particle-flow}.
Les traces additionnelles et les dépôts d'énergie dans les calorimètres se situant dans un cône de rayon $\Delta R$ inférieur à \num{0.3} dans le plan $(\eta,\phi)$, où
\begin{equation}
\Delta R_{ij}^2 = (\eta_i-\eta_j)^2 + (\phi_i-\phi_j)^2
\mend[,]
\end{equation}
sont également associés au muon global.
Il est requis que la somme des impulsions transverses et des énergies de ces traces et dépôts n'excède pas \SI{10}{\%} de l'impulsion transverse du muon global.
Ce critère est suffisant pour rejeter les hadrons réussissant à traverser le HCAL.
Ensuite, les muons globaux non isolés sont sélectionnés à l'aide d'un critère d'identification strict (\emph{Tight} \muonID)~\cite{CMS-MUO-16-001} auquel la présence d'au moins trois segments de trace compatibles est requise.
\par Les muons non identifiés à ce stade peuvent l'être en utilisant les muons seuls et les muons du trajectographe.
Les muons seuls présentant un grand nombre de signaux dans les chambres à muons, au moins 23 dans les DT (pour un maximum possible de 32) ou 15 dans les CSC (pour un maximum possible de 24), et dont l'ajustement de la trace à ces signaux est de bonne qualité sont ainsi retenus.
Les muons du trajectographe sont également retenus s'ils contiennent au moins 13 points de passage dans le trajectographe et que les agglomérats dans les calorimètres sont compatibles avec la traduction de la trace correspondante en tant que muon.
\par La résolution sur l'impulsion transverse des muons reconstruits est de \SI{1}{\%} dans le \CMSbarrel\ et \SI{3}{\%} dans les \CMSendcaps\ pour les muons d'impulsion transverse inférieure à \SI{100}{\GeV}
et inférieure à \SI{7}{\%} dans le \CMSbarrel\ jusqu'à $\pT=\SI{1}{\TeV}$~\cite{CMS-MUO-16-001}.
L'efficacité de reconstruction est de \SI{95}{\%} et le taux d'identification de hadrons en tant que muons inférieur à \SI{1}{\%}.
Les éléments d'identification du \PF\ utilisés pour reconstruire les muons sont retirés dans la suite du processus de reconstruction de l'événement.
\subsubsection{Électrons et photons isolés}
L'identification des électrons et des photons isolés se base sur les éléments d'identification du \PF\ provenant du trajectographe et du ECAL.
De par la présence de la matière du trajectographe, les électrons émettent des photons par \emph{bremsstrahlung} et les photons se convertissent en paires $\antielectron\electron$, ces électrons étant également sujets au \emph{bremsstrahlung}, etc.
C'est pour cela qu'électrons et photons isolés sont traités de manières similaires pour leur reconstruction.
\par Un candidat électron est défini lorsqu'une trace du trajectographe, extrapolée jusqu'au ECAL, est associée à un dépôt d'énergie, si ce dépôt n'est pas lui-même relié à trois autres traces ou plus.
Les candidats photons isolés correspondent aux dépôts du ECAL avec une énergie transverse supérieure à \SI{10}{\GeV} n'étant pas associés à une trace.
Pour tous ces candidats, la somme des énergies mesurées dans les cellules du HCAL se situant dans un cône de rayon $\Delta R$ inférieur à \num{0.15} dans le plan $(\eta,\phi)$ ne doit pas correspondre à plus de \SI{10}{\%} de l'énergie du dépôt du ECAL.
Les traces identifiées comme celles de conversions de photons et les dépôts du ECAL associés sont de plus rattachées au candidat initial.
\par Les électrons et photons isolés sont alors obtenus en soumettant aux candidats définis précédemment des critères d'identification, prenant en compte jusqu'à 14 variables~\cite{particle-flow}.
Les définitions exactes de ces critères varient d'une année à l'autre en fonction des performances du détecteur et plusieurs niveaux d'exigence existent.
Les éléments d'identification du \PF\ utilisés pour reconstruire les électrons et photons isolés sont retirés dans la suite du processus de reconstruction de l'événement.
\subsubsection{Hadrons et photons non isolés}
Les muons, électrons et photons isolés ayant été reconstruits, seuls les hadrons et les photons non isolés issus de la formation des jets et de l'hadronisation restent à être reconstruits.
La formation des jets ainsi que l'hadronisation sont détaillées dans le chapitre~\refChJERC.
Ces particules sont généralement détectées comme des hadrons chargés (\pionpm, \Kaonpm, protons), des hadrons neutres (\Kaonlong, neutrons), des photons non isolés (désintégrations des \pionnull) et plus rarement comme des muons (désintégrations de hadrons lourds).
\par Dans la région d'acceptance du trajectographe ($\abs{\eta}<\num{2.5}$), les photons non isolés et les hadrons neutres sont reconstruits respectivement à partir des dépôts d'énergie dans les ECAL et HCAL non associés à une trace.
Une priorité est donnée aux photons dans la mesure où \SI{25}{\%} de l'énergie des jets est portée par ces particules alors que seulement \SI{3}{\%} de l'énergie des jets est déposée dans le ECAL par les hadrons neutres.
Au-delà de l'acceptance du trajectographe, il n'est pas possible de faire la distinction entre hadrons neutres et chargés.
Près de \SI{25}{\%} de l'énergie des jets est ainsi déposée dans le ECAL et les agglomérats du ECAL se situant dans la même région qu'un agglomérat du HCAL sont considérés comme dus à la même gerbe hadronique, \ie\ au même hadron.
Les autres dépôts du ECAL sont considérés comme dus à des photons.
\par Les hadrons chargés sont identifiés à partir des agglomérats restant dans le HCAL, associés aux traces dans le trajectographe non utilisées pour l'identification des particules précédentes.
Ces traces peuvent elles-mêmes être reliées à un agglomérat résiduel du ECAL.
Pour chaque bloc du \PF\ ainsi construit, l'énergie dans les calorimètres est comparée à la somme des moments des traces.
Si un excès est observé avec les calorimètres, il est interprété comme la présence d'une particule neutre supplémentaire.
Si cet excès est inférieur à l'énergie dans le ECAL et plus grand que \SI{500}{\MeV}, alors la particule neutre est considérée comme étant un photon d'énergie égale à cet excès.
Sinon, l'énergie dans le ECAL donne un photon et si la partie de l'excès dans le HCAL est supérieure à \SI{1}{\GeV}, un hadron neutre est également considéré.
Puis, à partir de l'énergie calorimétrique restante, chaque trace du bloc du \PF\ donne un hadron chargé.
