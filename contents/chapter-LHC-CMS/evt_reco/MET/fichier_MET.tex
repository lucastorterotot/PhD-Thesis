\subsection{Énergie transverse manquante}\label{chapter-CMS-section-MET}
Des neutrinos peuvent être produits lors des collisions.
Or, ces particules se propagent sans laisser de signal dans le détecteur, elles sont donc invisibles.
Toutefois, lorsque de telles particules sont produites en association avec des particules détectées, leur présence peut être déduite du déséquilibre dans l'impulsion totale des particules de l'événement~\cite{CMS-PAS-JME-17-001}.
\par
Lors d'une collision de protons, dans l'état initial,
la composante longitudinale de l'impulsion est inconnue
et
l'impulsion totale dans le plan transverse est nulle
comme discuté dans la section~\ref{chapter-LHC-section-LHC-subsec-pp_collisions}.
Par conservation, l'impulsion totale dans le plan transverse est nulle dans l'état final.
Les neutrinos n'étant pas détectés, leurs impulsions transverses sont manquantes dans le bilan de l'état final.
L'observable définie afin de quantifier ce manque est
l'énergie transverse manquante (MET, \emph{Missing Transverse Energy}).
Bien que son nom mentionne une énergie, il s'agit bien d'une impulsion.
\par
La MET peut être déterminée à partir de l'algorithme de \PF\ (PF MET)
ou
par l'algorithme \PUPPI\ (\PUPPI\ MET).
Il s'agit de la MET brute, à laquelle la calibration en énergie des jets détaillée au chapitre~\refChJERC\ doit être propagée.
\paragraph{MET brute issue de l'algorithme de \PF}
La somme des impulsions transverses des particules invisibles doit compenser celle des particules reconstruites, \ie
\begin{equation}
\sum_{\substack{\text{toutes les}\\\text{particules}}} \vpT = \vec{0}
\Leftrightarrow
\sum_{\substack{\text{particules}\\\text{invisibles}}} \vpT + \sum_{\substack{\text{particules}\\\text{reconstruites}}} \vpT = \vec{0}
\Leftrightarrow
\sum_{\substack{\text{particules}\\\text{invisibles}}} \vpT = -\sum_{\substack{\text{particules}\\\text{reconstruites}}} \vpT \mend
\end{equation}
\par La MET brute issue de l'algorithme de \PF\ représente ainsi l'impulsion transverse totale des particules invisibles et est définie comme
\begin{equation}
\vMET(\text{\PF}) = -\sum_{i\in\set{\text{particules}}} \vpT^{i}
\mend[,]
\end{equation}
où les particules sont celles reconstruites par l'algorithme de \PF.
Cette définition, simple, est toutefois sensible aux particules issues de l'empilement.
Afin de réduire l'effet de l'empilement, l'algorithme \PUPPI\ a été développé.
\paragraph{MET brute issue de l'algorithme \PUPPI}
La MET peut également être estimée par l'algorithme \PUPPI\ (\emph{PileUp Per Particle Identification}) \cite{PUPPI}.
La \og \PUPPI MET \fg{} obtenue est moins sensible à l'empilement (\emph{pileup}) que la MET issue de l'algorithme de \PF\ (PFMET).
L'algorithme \PUPPI\ exploite en effet des informations sur:
\begin{itemize}
\item l'environnement de chaque particule identifiée par l'algorithme de \PF\;
\item les propriétés de l'empilement dans l'événement;
\item les données issues du trajectographe;
\end{itemize}
afin d'associer un poids $w_i$ à chaque particule $i$, lié à la probabilité que celle-ci provienne de l'empilement au lieu du vertex primaire principal.
Ce poids varie entre
\num{0} pour des particules issues de l'empilement
et
\num{1} pour des particules provenant du vertex primaire principal.
Plus de détails dans la détermination de $w_i$ sont disponibles dans les références~\cite{CMS-PAS-JME-17-001,PUPPI}.
\par La MET issue de l'algorithme \PUPPI\ est définie comme
\begin{equation}
\vMET(\text{\PUPPI}) = -\sum_{i\in\set{\text{particules}}} w_i\,\vpT^{i}
\mend
\end{equation}
\paragraph{Propagation de la calibration en énergie des jets à \MET}
La calibration en énergie des jets, abordée dans le chapitre~\refChJERC, doit être propagée à \MET\ afin de conserver une description cohérente des événements.
Il s'agit de la correction dite de \og type~I \fg{} \cite{CMS_WorkBookMetAnalysis},
réalisée selon
\begin{equation}
\vMET(\text{Type~I}) = \vMET(\text{brute}) - \sum_{i\in\set{\text{jets}}} \left( \vpT^i_\cali) - \vpT^i_\reco) \right)
\end{equation}
où
\og \reco \fg{} correspond aux observables avant calibration
et
\og \cali \fg{} après.