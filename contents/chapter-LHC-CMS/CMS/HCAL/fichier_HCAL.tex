\subsection{Le calorimètre hadronique ou HCAL}\label{chapter-LHC-section-CMS-subsec-HCAL}
%\cite{cms_paper,CERN-LHCC-97-031,CMS-DP-2016-071,CMS-DP-2017-016,CMS-DP-2017-017,CMS-DP-2017-033,CMS-DP-2017-034,CMS-DP-2017-042,CMS-DP-2018-018,CMS-DP-2018-019}
Le calorimètre hadronique (HCAL)~\cite{cms_paper,CERN-LHCC-97-031} permet de mesurer l'énergie des hadrons par un processus destructif.
Situé à l'intérieur du solénoïde de CMS, les particules déposant leur énergie dans le HCAL ne sont donc pas perturbées par une traversée du solénoïde.
La figure~\ref{fig-chapter-LHC-section-CMS-subsec-HCAL-cms_paper-fig_5-1} présente l'agencement du HCAL.
\begin{figure}[h]
\centering
\includegraphics[width=0.8\textwidth]{\PhDthesisdir/plots_and_images/from_cms_paper/fig_5-1.png}
\caption[Schéma du calorimètre hadronique de CMS.]{Schéma d'un cadrant du détecteur CMS~\cite{cms_paper} montrant la localisation des calorimètres hadroniques du barillet (HB), externe (HO), du bouchon (HB) et avancé (HF). Certaines valeurs de $\eta$ et les directions associées sont indiquées.}
\label{fig-chapter-LHC-section-CMS-subsec-HCAL-cms_paper-fig_5-1}
\end{figure}
\par Tout comme le ECAL, il comporte une partie barillet (HB) couvrant la région $\abs{\eta}<\num{1.3}$ et deux bouchons (HE) couvrant $\num{1.3}<\abs{\eta}<\num{3}$.
Ces parties du HCAL sont composées de couches alternées d'absorbeur et de scintillateur.
L'absorbeur, du laiton, permet d'initier la gerbe hadronique.
Le laiton est paramagnétique, il ne subit donc pas d'effet néfaste de la part du champ magnétique.
Les muons interagissent peu avec ce matériau, leur mesure par les chambres à muons situées au-delà du HCAL n'est donc pas perturbée par leur traversée du HCAL.
Le scintillateur est fait en plastique.
Des fibres optiques permettent de recueillir la lumière émise par les gerbes hadroniques.
La mesure de ce signal lumineux donne une mesure de l'énergie des hadrons.
\par Cependant, le nombre de longueurs de radiations combinées des ECAL et HCAL dans le barillet, de l'ordre de dix, est insuffisant pour contenir toutes les gerbes hadroniques~\cite{cms_paper}.
Le HB est ainsi complété par un calorimètre hadronique externe (HO) installé sur la face interne de la culasse, \ie\ de l'autre côté du solénoïde et avant les chambres à muons.
\par Enfin, une couverture plus large en $\eta$ est assurée par le calorimètre hadronique avancé (HF) couvrant $\num{2.9}<\abs{\eta}<\num{5.2}$.
Les deux HF, un à chaque extrémité de CMS, sont des détecteurs cylindriques ayant des absorbeurs en acier auxquels sont connectés des fibres optiques de quartz.
Les particules incidentes émettent de la lumière Cherenkov lors de leur passage dans l'acier.
Cette lumière est alors recueillie par les fibres optiques.
\par La réponse relative du HCAL, \ie\ l'énergie reconstruite dans le HCAL par rapport à l'énergie effective du dépôt, est représentée en fonction de l'énergie simulée du dépôt sur la figure~\ref{fig-chapter-LHC-section-CMS-subsec-HCAL-CMS-DP-2016-071-hcal_method2_3_performance}.
Elle ne dévie pas de plus de \SI{5}{\%}~\cite{CMS-DP-2016-071} une fois l'empilement asynchrone\footnote{L'empilement asynchrone est défini dans la section~\ref{chapter-LHC-section-LHC-subsec-PU}.} retiré au-delà de \SI{10}{\GeV}.
La résolution $\sigma$ obtenue sur l'énergie des hadrons, par combinaison avec les signaux du ECAL, a été déterminée à l'aide d'un faisceau test de pions comme étant
\begin{equation}
\frac{\sigma}{E} = \frac{\num{1.1}}{\sqrt{E}} + \frac{9}{100}
\end{equation}
où $E$ est l'énergie mesurée en \SI{}{\GeV}.
\begin{figure}[h]
\centering
\includegraphics[width=0.45\textwidth]{\PhDthesisdir/plots_and_images/from_CMS-DP-2016-071/hcal_method2_3_performance.pdf}
\caption[Réponse relative du calorimètre hadronique de CMS.]{Réponse relative du calorimètre hadronique de CMS~\cite{CMS-DP-2016-071} en fonction de l'énergie simulée du dépôt. En noir, sans correction de l'empilement asynchrone (OOTPU). En bleu, avec des correction en ligne, \ie\ un ajustement des amplitudes et temps d'arrivée des signaux en prenant en compte jusqu'à trois signaux avant et après le signal d'intérêt. En rouge, avec l'ensemble des corrections.}
\label{fig-chapter-LHC-section-CMS-subsec-HCAL-CMS-DP-2016-071-hcal_method2_3_performance}
\end{figure}