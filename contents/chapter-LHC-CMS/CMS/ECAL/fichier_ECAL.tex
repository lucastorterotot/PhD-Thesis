\subsection{Le calorimètre électromagnétique ou ECAL}\label{chapter-LHC-section-CMS-subsec-ECAL}
% role, structure, composition, perfiormances
\cite{cms_paper,CERN-LHCC-97-033}
%\cite{cms_paper,CERN-LHCC-97-033,CMS-CR-1999-006,CMS-EGM-11-001,CMS-CR-2014-410,CMS-DP-2016-031,CMS-CR-2018-162,CMS-DP-2019-005,CMS-DP-2020-021}
%/home/torterotot/Documents/PhD-Thesis/plots_and_images/from_CMS-EGM-11-001/figures_calorimeter.png also in cms_paper

% /home/torterotot/Documents/PhD-Thesis/plots_and_images/from_CMS-DP-2019-005/histories_2011-2012-2015-2016-2017-2018_190208.png

La résolution $\sigma$ du ECAL est paramétrisée selon
\begin{equation}
\frac{\sigma}{E}
=
\frac{S}{\sqrt{E}}
\oplus
\frac{N}{E}
\oplus
C
\end{equation}
où $\oplus$ désigne une somme quadratique, \ie\ $(a\oplus b)^2 = a^2 + b^2$,
$S$ un terme stochastique prenant en compte la largeur latérale de la gerbe électronique,
$N$ le terme de bruit des composants électroniques et
$C$ une constante rendant compte des erreurs de calibration.
% /home/torterotot/Documents/PhD-Thesis/plots_and_images/from_CMS-DP-2020-021/final_Resolution_RunII_Inclusive.pdf