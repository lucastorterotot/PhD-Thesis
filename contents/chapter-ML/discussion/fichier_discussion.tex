\section{Discussions}\label{chapter-ML-section-discussion}

\subsection{Effets de l'intervalle de masse}

\subsection{Effets de l'empilement}
also show PU effect (see fig 2 and 3 from report 2020-11-20, update with new models and samples)

\subsection{Effets de la reconstruction}
show trained/tested on gen tau, gen tau decays, reco tau decays (=real), see fig 3 from report 2021-01-11

the model understand the physics, now it has to deal with the reco resolution and fakes.

\subsection{Effets des faux taus hadroniques}

\subsection{Effets de la séparation des canaux}
not relevant (fig3 report  2021-01-21)

\subsection{Effets de bord}
use the custom loss with boundaries cuts (basically all the report 2021-02-04)

Follow report from 2021-02-04 but for section 3 : We saw that predictions come out too low, which already is a motivation to put larger weights on higher masses, i.e. to weight by truth. Choosing sqrt(truth) is of course just a guess then

extend up to 1TeV using the tails


\subsection{Modèle final}

\DEEPTAU

1 TeV

all inputs

activation softplus

loss mapesqrt\_b

opti Adam

glorot uniform

3 layers of 1000 neurons

show reponses and 2d histo
