\section{Conclusion}\label{chapter-ML-section-conclusion}
Le \emph{Machine Learning} (ML) est une branche de l'Intelligence Artificielle (IA)
permettant d'obtenir des modèles pouvant réaliser des classifications ou des régressions.
De tels modèles sont déjà exploités en physique des particules.
\par
Nous avons étudié la possibilité de prédire la masse d'une résonance se désintégrant en paire de leptons tau grâce au ML.
En effet, la phénoménologie de ces événements ne permet pas d'obtenir la masse invariante totale du système dans l'état final.
Des travaux réalisés par \todo{citation and discussion here}
À partir d'événements simulés par nos soins à l'aide de \FASTSIM,
nous avons donc construit et entraîné
des arbres de décision améliorés à l'aide de la librairie \XGBOOST\
et
des réseaux de neurones profonds à l'aide des librairies \KERAS\ et \TENSORFLOW.
Le principe et l'entraînement de ces types de modèle ont été présentés.
\par
De nombreuses combinaisons d'hyper-paramètres,
propriétés des modèles régissant par exemple leur structure interne,
ont été étudiées et comparées.
Il en ressort que certaines variables d'entrée sont des informations pertinentes pour les modèles
afin d'estimer plus fidèlement la masse de la résonance.
Les réseaux de neurones
proposent de meilleures performances que
les arbres de décision améliorés
d'après les métriques d'évaluation que nous avons utilisé,
en particulier pour les valeurs de masse de la résonance correspondant aux bosons~\Zboson\ et \higgs\ du modèle standard.
Les performances des réseaux de neurones dépendent également fortement de l'algorithme d'optimisation utilisé lors de l'entraînement.
\par
Une fois une combinaison performante d'hyper-paramètres trouvée,
divers effets sur les prédictions du modèle B correspondant ont été étudiés.
L'empilement ... \todo{continue here}




pas que ggH pour entraînement

L'utilisation de \mml\ en tant que telle dans l'analyse introduite au chapitre~\refChHTT\ n'est donc pas pertinente.
Elle le peut toutefois pour d'autres analyses comme celles axées sur le modèle standard par exemple.


Une note d'analyse (CMS AN 2021/054) \cite{CMS-NOTE-2021-054} est en cours de rédaction.
Les scripts utilisés pour la génération des événements peuvent être consultés~\cite{fastsim_ece},
tout comme ceux permettant d'entraîner les modèles étudiés~\cite{DL_for_HTT_mass}.
Enfin, le modèle issu de ces travaux peut être récupéré~\cite{DiTau_ML_mass} et utilisé dans d'autres analyses.