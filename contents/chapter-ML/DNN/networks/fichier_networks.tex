\subsection{Réseaux de neurones}\label{chapter-ML-section-DNN-networks}
Un NN est obtenu par l'interconnexion de plusieurs neurones entre eux.
Ces connexions peuvent se faire selon diverses architectures~\cite{Sarle1994NeuralNA,DNN}.
Nous utilisons ici,
comme dans les travaux de \citeauthor{BARTSCHI201929}~\cite{BARTSCHI201929},
une architecture
normale profonde à propagation avant complètement connectée (\emph{normal deep feedforward fully-connected}),
représentée sur la figure~\ref{fig-neural_network_fr},
\ie\ avec:
\begin{itemize}
\item des neurones répartis en couches (normale);
\item plusieurs couches \og cachées \fg, situées entre les couches d'entrée et de sortie (profonde);
\item toutes les sorties de la couche $k-1$ utilisées comme entrées de chacun des neurones de la couche $k$ (à propagation avant complètement connectée).
\end{itemize}
Le nombre de neurones par couche cachée est noté \NNeurons,
le nombre de couches cachées \NLayers.
Le NN ayant une structure profonde, il s'agit d'un DNN (\emph{Deep Neural Network}).
\begin{figure}[h]
\centering
\def\drawN{}
\renewcommand{\drawN}[2][c]{
\node [draw, circle, minimum size = .66 cm] (#1) at (#2) {};
}
\def\linkN#1#2{
\draw [-latex] (#1) -- (#2);
}
\def\yscale{1.25/1.75}
\begin{tikzpicture}[scale=1.75]

\foreach \y in {0,1,2,-2}{
\foreach \x in {1,2,...,5}{
\drawN[\x\y]{\x,{\y*\yscale}}
}
}

\foreach \yi/\N in {1.5/1,0.5/2,-1.5/n}{
\fill (-.5,{\yi*\yscale}) circle(2pt) node [left] {$x_{\N}$};
\foreach \y in {0,1,2,-2}{
\draw [-latex] (-.5,{\yi*\yscale}) -- (1\y);
}
}

\drawN[No]{6.5,0}

\draw [-latex] (No) --+ (.5,0) node [right] {$y=F(\vec{x})$};

\foreach \ya in {0,1,2,-2}{
\foreach \yb in {0,1,2,-2}{
\linkN{5\ya}{No}
\foreach \xa/\xb in {1/2,2/3,3/4,4/5}{
\linkN{\xa\ya}{\xb\yb}
}
}
}

\fill[white] (2.33,{-2.5*\yscale}) rectangle (4.66,{2.5*\yscale});

\foreach \x/\y in {-.5/-.5,1/-1,2/-1,5/-1}{
\fill (\x,{\y*\yscale}) circle (1pt);
\fill (\x,{(\y+.2)*\yscale}) circle (1pt);
\fill (\x,{(\y-.2)*\yscale}) circle (1pt);
}

\foreach \y in {0,1,2,-2}{
\fill (3.5,{\y*\yscale}) circle (1pt);
\fill (3.5+.2,{\y*\yscale}) circle (1pt);
\fill (3.5-.2,{\y*\yscale}) circle (1pt);
}

\foreach \x in {.25,5.75}{
\draw [thick, dotted, ltcolorblue] (\x,{-2.5*\yscale}) -- (\x,{2.5*\yscale});
}

\draw [ltcolorblue, left] (.25, {2.5*\yscale}) node {Couche d'entrée\vphantom{Àq}};
\draw [ltcolorblue] (3, {2.5*\yscale}) node {Couches cachées\vphantom{Àq}};
\draw [ltcolorblue, right] (5.75, {2.5*\yscale}) node {Couche de sortie\vphantom{Àq}};

\draw [thick, ltcolorred, latex-latex] (.25,{-2.5*\yscale}) -- (5.75,{-2.5*\yscale});
\draw [ltcolorred] (3, {-2.25*\yscale}) node {$N_L$\vphantom{Àq}};

\draw [thick, ltcolorred, latex-latex] (4.5,{-2.125*\yscale}) -- (4.5,{2.125*\yscale});
\draw [ltcolorred] (4.5, {0*\yscale}) node [left] {$N_N$\vphantom{Àq}};

\end{tikzpicture}

\caption[Structure d'un réseau de neurones.]{Structure normale profonde à propagation avant complètement connectée d'un réseau de neurones. Une couche d'entrée comporte autant de neurones que de variables $x_i$. La couche de sortie en comporte autant que de valeurs à donner, \ie\ une. Les fonctions d'activation de ces deux couches sont linéaires. Entre elles se trouvent \NLayers\ couches cachées, chacune contenant \NNeurons\ neurones. Diverses fonctions d'activation peuvent être utilisées dans les couches cachées.}
\label{fig-neural_network_fr}
\end{figure}
\par
La tâche du réseau est une régression vers une seule grandeur, $m_{\higgsML}$, à partir de $n$ variables d'entrée $x_j$, $j\in\set{1,\ldots,n}$.
La couche de sortie est donc composée d'un seul neurone dont la fonction d'activation est l'identité.
La couche d'entrée comporte $n$ neurones, chacun se contentant de transmettre la variable d'entrée correspondante.
Il s'agit donc d'une couche d'adaptation entre le nombre d'entrées $n_\text{in}$ et le nombre de neurones dans la couche suivante \NNeurons.
Tous les neurones des couches cachées ont la même fonction d'activation.
Plusieurs fonctions d'activation sont testées dans la section~\ref{chapter-ML-section-hyperparameters}.