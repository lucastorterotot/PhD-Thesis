\chapter{Reconstruction de la masse d'une résonance grâce au \emph{Machine Learning}}\label{chapter-ML}


\remarque{Citations incontournables:
\begin{itemize}
\item \DELPHES~3.4.2~\cite{Delphes,Delphes_additions}?
\item CMS Fast Simulation (\FASTSIM)~\cite{FastSim_2011,FastSim_2014,FastSim_2017_1,FastSim_2017_2}
\item \PYTHIA~8.235~\cite{pythia8.2}
\item \FASTJET~\cite{Cacciari:2011ma,Cacciari:2006} % Fast Jet
\item \KERAS~\cite{keras}
\item \TENSORFLOW~\cite{tensorflow}
\item \XGB~\cite{xgboost}
\item \cite{jet_flavor_deep_nn} for an example of nn use in HEP
\item \cite{Sarle1994NeuralNA}
\item \cite{BARTSCHI201929}
\item \SVFIT~\cite{SVFit_Bianchini_2014}
\end{itemize}}

Citer également la thèse de Gaël:\\\fullcite{Gael_thesis}

\begin{itemize}
\item type of samples/events
\item preselection (small HTT analysis)
\item inputs
\item performances: métrique?
\item mass range + plots
\item METcov + plots
\item PU + plots
\end{itemize}

\section*{Étapes des choix}
\subsection*{Inputs variables}
phéno, tau1 tau2 MET pT eta phi + mT 1 2 tt tot.
\subsection*{Inputs events}
sélection des événements, target flattening. pas de PU

80-800 GeV (aller plus loin que \cite{BARTSCHI201929})
\subsection*{DNN}
Comme dans~\cite{BARTSCHI201929}, table 1.

output activation function to linear instead of relu to not cut?

changement de la structure: 3 couches de 1000 neurones?

loss mse

optimizer adadelta

w_init_mode uniform

early stopping

Changement du mass range (biais aux bords) -> down to 50 GeV

\subsection*{XGB}
avoid overfitting, max\_depth

learning rate (eta)

num\_round, early stopping (choose value of 5)

loss

\section{Introduction}\label{chapter-HTT_analysis-section-introduction}
Dans le chapitre~\refChMSSM,
il a été montré que le modèle standard (SM, \emph{Standard Model}) souffre de lacunes quant à l'explication à apporter à certaines observations.
Certaines peuvent être comblées par des modèles allant au-delà (BSM, \emph{Beyond Standard Model}) comme l'extension supersymétrique minimale du modèle standard ou \og MSSM \fg.
Une des conséquences du MSSM est l'existence de cinq bosons de Higgs, dont trois neutres, \higgs, \Higgs\ et \HiggsA.
L'un d'entre-eux doit correspondre au boson découvert en 2012 et interprété comme étant le boson de Higgs du \SM~\cite{ATLAS_Higgs_discovery,CMS_Higgs_discovery,CMS_Higgs_discovery_2013,ATLAS-CMS-Higgs_combined_1,ATLAS-CMS-Higgs_combined_2}.
L'existence des deux bosons de Higgs neutres supplémentaires peut être testée expérimentalement avec des accélérateurs de particules,
comme cela a été fait au LEP~\cite{Schael:2006cr}.
Ces bosons se désintègrent préférentiellement en paire de quarks~\quarkb\ ou de leptons~\tau.
Bien que le rapport de branchement (\BR) de ces bosons aux \quarkb\ soit 5 à 10 fois supérieur que celui aux \tau,
ces derniers proposent une meilleure accessibilité expérimentale dans les collisionneurs hadroniques comme le Tevatron,
ou ces désintégrations en \tau\ ont été étudiées~\cite{Aaltonen:2009vf,Abazov:2011jh}.
\par
L'expérience CMS installée au LHC et présentée dans le chapitre~\refChLHCCMS\ permet elle aussi de tester expérimentalement le MSSM, dans des conditions de collision inédites.
La recherche de bosons de Higgs supplémentaires se désintégrant en paire de \tau\ a été menée dans les collisions de protons avec une énergie dans le centre de masse de $\sqrt{s}=\num{7}$ et $\SI{8}{\TeV}$ (Run~I) \cite{Chatrchyan:2012vp,CMS-MSSM-HTT_2014,CMS-PAS-HIG-13-021,CMS-PAS-HIG-14-029} ainsi qu'avec les données récoltées en 2016 avec une énergie de $\sqrt{s}=\SI{13}{\TeV}$ \cite{CMS-PAS-HIG-17-020}.
Plusieurs thèses portent sur l'analyse des événements où un boson de Higgs se désintègre en paire de \tau~\cite{Gael_thesis,Artur_thesis}.
La désintégration en paire de \quarkb\ est également exploitée~\cite{Chatrchyan:2013qga,Khachatryan:2015tra},
ainsi que celle en paire de muons~\cite{CMS:2015ooa}.
L'expérience ATLAS mène des recherches similaires~\cite{Aad:2012cfr,ATLAS-MSSM-HTT_2018,ATLAS-MSSM-HTT_2020}.
\par
Ce chapitre présente la
recherche de bosons de Higgs supplémentaires de haute masse se désintégrant en paire de \tau\
avec les données récoltées par l'expérience CMS
lors du Run~II du LHC (années 2016, 2017 et 2018),
correspondant à une luminosité intégrée de \SI{137}{\femto\barn^{-1}} ($\num{35.9}+\num{41.5}+\SI{59.7}{\femto\barn^{-1}}$)
à une énergie dans le centre de masse de $\sqrt{s}=\SI{13}{\TeV}$.
Sur les six canaux de désintégration de la paire de leptons \tau\ introduits dans le chapitre~\refChMSSM,
les quatre présentant les plus grands \BR\ sont considérés dans l'analyse.
Il s'agit des canaux
hadronique (\tauh\tauh),
semi-leptoniques (\mu\tauh, \ele\tauh)
et
leptonique asymétrique (\ele\mu).
Les canaux leptoniques symétriques (\mu\mu, \ele\ele) ne sont pas exploités.
\par
Dans les données réelles, les particules doivent forcément être reconstruites à partir des signaux qu'elles produisent dans le détecteur.
Dans le cas des données simulées, la réponse du détecteur aux particules est modélisée.
À partir des signaux réels comme simulés, les particules individuelles sont reconstruits comme exposé dans le chapitre~\refChLHCCMS.
Elles permettent d'obtenir les objets physique de haut niveau, introduits chapitre~\refChJERC, que sont l'énergie transverse manquante (MET), les jets et les taus hadroniques (\tauh).
Les simulations n'étant pas exemptées de défauts, des corrections déterminées à l'aide d'analyses annexes leurs sont appliquées.
Les corrections spécifiques à la présente analyse sont présentées dans la section~\ref{chapter-HTT_analysis-section-corrections}.
Les autres corrections sont introduites dans le chapitre~\refChLHCCMS.
Les objets reconstruits et corrigés permettent de sélectionner les événements d'intérêt pour l'analyse selon la procédure explicitée en section~\ref{chapter-HTT_analysis-section-selection}.
Des processus physiques différents de ceux du signal recherché passent cette sélection et constituent le bruit de fond.
Afin d'interpréter les observations, il est nécessaire de modéliser ce bruit de fond.
Cette modélisation est présentée section~\ref{chapter-HTT_analysis-section-bg_estimation}.
En plus de l'utilisation de données simulées, des techniques basées sur les données réelles sont exploitées.
Des données dites \og encapsulées \fg{} (\emph{embedded}) sont ainsi produites selon la procédure exposée section~\ref{chapter-HTT_analysis-section-bg_estimation-embedding} et décrivent les événements contenant une vraie paire de leptons \tau.
Une estimation du bruit de fond dû aux jets identifiés à tort comme des taus hadroniques (\ftauhs) est quant à elle obtenue grâce à la méthode des facteurs de faux (\emph{fake factors}) introduite section~\ref{chapter-HTT_analysis-section-bg_estimation-FF_method}.
Les événements sont par la suite catégorisés afin d'augmenter la sensibilité de l'analyse.
Les catégories utilisées sont présentées en section~\ref{chapter-HTT_analysis-section-categorisation}.
Les sources d'incertitudes systématiques sont données section~\ref{chapter-HTT_analysis-section-systematics}.
Leur prise en compte dans l'extraction du signal ainsi que la modélisation de celui-ci sont exposée dans la section~\ref{chapter-HTT_analysis-section-signal_extraction}.
Enfin, les résultats obtenus sont disponibles section~\ref{chapter-HTT_analysis-section-results}.
Certains sont indépendants de tout modèle, d'autres sont obtenus dans le cadre de scénarios spécifiques au MSSM~\cite{Bagnaschi_2019}.
\par
Une note d'analyse (CMS AN 2020/218) \cite{CMS-NOTE-2020-218} est déjà disponible pour les membres de la collaboration et un article est en préparation~\cite{HIG-21-001}.
Ces travaux sont réalisés au sein d'une équipe regroupant:
\begin{itemize}
\item l'Institut de Physique des 2 Infinis (IP2I) de l'Université Claude Bernard de Lyon, mon laboratoire de rattachement;
\item l'\emph{Institut für Experimentelle Teilchenphysik} (ETP) du \emph{Karlsruher Institut für Technologie} (KIT) de Karlsruhe;
\item le \emph{Deutsches Elektronen-Synchrotron} (DESY) de Hambourg;
\item l'\emph{Imperial College} de Londres;
\item l'\emph{Institut für Hochenergiephysik} (HEPHY)
% de l'\emph{Österreichische Akademie der Wissenschaften}
 de Vienne;
\item le \emph{Tata Institute of Fundamental Research} de Bombay.
\end{itemize}
\par
En début de thèse, j'ai travaillé sur les données de l'année 2017 en équipe avec Gaël \textsc{Touquet} qui a exploité le canal \tauh\tauh\ dans sa thèse~\cite{Gael_thesis}.
Je me suis concentré sur les  canaux semi-leptoniques et plus particulièrement le canal \mu\tauh.
La présence de \tauh\ dans nos canaux respectifs nous a mené à de travailler en étroite collaboration.
Les événements étaient analysés à l'aide d'un code basé sur \HEPPY~\cite{heppy},
indépendant de celui utilisé par les autres instituts listés précédemment,
ce qui a permis à l'ensemble des acteurs de cette analyse de valider la bonne implémentation des différentes corrections et sélections détaillées dans ce chapitre.
À cette occasion, j'ai découvert une erreur dans le code de \COMBINE.
Cette erreur a été comprise et corrigée.
Le correctif~\cite{BBB_PR} a été transmis à la collaboration CMS qui l'a pris en compte.
\par
J'ai par la suite travaillé directement avec le groupe de Karlsruhe dans le cadre de l'analyse du Run~II.
J'ai implémenté le traitement du scénario avec violation de la symétrie $CP$.
J'ai de plus participé au traitement des jeux de données utilisés, listés dans l'annexe~\refApHTTdatasets.
Il s'agissait de s'assurer du bon déroulement de plusieurs milliers de tâches informatiques et du regroupement de leurs résultats.
Enfin, j'ai activement participé à la rédaction de la note d'analyse CMS correspondante~\cite{CMS-NOTE-2020-218}.
%Homework concerning the bbH and ggH samples
%1) Monitor the production ---> ask in case there are invalid samples
%2) Process the new samples
%3) Rederive the ggH weights based on input POWHEG samples
%4) Update the signal modelling for new samples in CombineHarvester
%And on the experimental side I recall:
%newes FF's
%MET tail correction and uncertainty
%Are these the only items left before wrapping up or am I missing anything in addition?

\section{Le \emph{Machine Learning}}
\subsection{Généralités}
\subsection{Le \emph{Gradient Boosting}}
\subsection{Le \emph{Deep Learning}}

\section{Application du \emph{Machine Learning} aux événements $\Higgs\to\tau\tau$}
\subsection{Génération des événements}
\subsection{Variables d'entrées}
\subsection{Performances sur les événements de test}
\subsection{Performances sur les événements de l'analyse CMS}

\section{Prise en compte de l'empilement}
\subsection{Génération des événements}
\subsection{Performances} (sur ces nouveaux événements)
\subsection{Variables d'entrées supplémentaires}
\subsection{Performances} (avec les nouvelles variables)

\section{Effets sur les résultats de l'analyse MSSM HTT}
(remplacement de mttot par les prédictions du meilleur modèle, nouveaux plots d'exclusion, comparaison)

\section{Conclusion}\label{chapter-ML-section-conclusion}
Le \emph{machine learning} (ML) est une branche de l'intelligence artificielle
permettant d'obtenir des modèles pouvant réaliser des classifications ou des régressions.
De tels modèles sont déjà exploités en physique des particules afin de réaliser diverses tâches,
comme l'identification des jets issus de quarks~\quarkb\ par exemple.
\par
Nous avons étudié la possibilité de prédire la masse d'une résonance se désintégrant en paire de leptons~\tau\ grâce au ML.
En effet, la phénoménologie de ces événements ne permet pas d'obtenir la masse invariante totale du système dans l'état final.
Les travaux réalisés par \citeauthor{BARTSCHI201929} sur ce sujet donnent des résultats prometteurs,
mais l'empilement n'y est pas pris en compte et la modélisation du détecteur CMS approximée.
À partir d'événements simulés par nos soins à l'aide de \FASTSIM,
nous avons pris en compte l'empilement et modélisé le détecteur plus précisément.
\par
Nous avons construit et entraîné
des arbres de décision améliorés
à l'aide de la librairie \XGBOOST\
et
des réseaux de neurones profonds
à l'aide des librairies \KERAS\ et \TENSORFLOW.
Le principe et l'entraînement de ces types de modèle ont été présentés.
De nombreuses combinaisons d'hyper-paramètres,
propriétés des modèles régissant par exemple leur structure interne,
ont été étudiées et comparées.
Il en ressort que certaines variables d'entrée sont des informations pertinentes pour les modèles
afin d'estimer plus fidèlement la masse de la résonance.
Les réseaux de neurones
proposent de meilleures performances que
les arbres de décision améliorés
d'après les métriques d'évaluation que nous avons utilisé,
en particulier pour les valeurs de masse de la résonance correspondant aux bosons~\Zboson\ et \higgs\ du modèle standard.
Les performances des réseaux de neurones dépendent également fortement de l'algorithme d'optimisation utilisé lors de l'entraînement.
\par
Nous avons alors déterminé une combinaison performante d'hyper-paramètres correspondant au modèle B.
Divers effets sur ses prédictions ont été étudiés.
Ainsi, l'empilement doit être pris en compte lors de l'entraînement.
\par
Un effet majeur sur la précision des prédictions est lié à la reconstruction des particules.
Dans le cas d'une reconstruction parfaite des particules,
\ie\ en utilisant les objets générés correspondants au lieu des objets reconstruits,
la résolution relative sur la masse de la résonance est de \SI{3}{\%}
contre \num{20} à \SI{25}{\%} sinon.
De plus, les \ftauhs\ perturbent les prédictions des modèles à basse masse,
en particulier dans la région des bosons~\Zboson\ et~\higgs.
Cependant, l'entraînement de modèles spécifiques aux différents canaux ou aux différentes phénoménologies de canaux n'apporte pas de gain en termes de précision des prédictions.
L'utilisation de la PFMET au lieu de la \PUPPI MET a un effet négligeable sur les prédictions du modèle face à sa résolution.
\par
La gamme de masse explorée lors de l'entraînement définit la zone utile du modèle, ses prédictions étant en bonne approximation restreintes à cet intervalle.
Cependant, il n'est pas possible d'étendre cet intervalle à l'infini
et des effets de bord apparaissent sur les prédictions du modèle.
Nous avons modifié la fonction de coût afin de rejeter dynamiquement certains événements de l'entraînement pour réduire cet effet de bord avec succès.
%La nouvelle fonction de coût ne respecte pas la condition d'être nulle uniquement si la prédiction \ypred\ est égale à la valeur vraie \ytrue,
%toutefois le nouveau modèle B' obtenu donne des prédictions plus proches de \ytrue.
L'exploitation de la queue à hautes valeurs de la distribution de la masse de la résonance, objectif des prédictions des modèles,
a permis d'améliorer encore les prédictions moyennes obtenues avec le modèle B".
Ce modèle permet de reconstruire avec succès la masse d'une résonance se désintégrant en paire de leptons~\tau\
entre \SI{50}{\GeV} et \SI{800}{\GeV} avec une précision de \num{20} à \SI{25}{\%}.
\par
La prédiction du modèle B", \mml, a été utilisée en tant que variable discriminante à la place de \mTtot\ pour obtenir les limites d'exclusion indépendantes du modèle de l'analyse présentée dans le chapitre~\refChHTT\ sur l'année 2017.
Malgré des valeurs plus proches de la vraie masse de la résonance
ainsi qu'une meilleure résolution que \mTtot, \mml\
ne permet pas de repousser les limites d'exclusion obtenues.
Ceci est dû aux processus physiques tels que les \ftauhs\ ne correspondant pas à une résonance se désintégrant en paire de leptons~\tau\
mais passant les critères de sélection des événements appliqués.
L'utilisation de \mml\ en tant que variable discriminante pour la recherche de bosons de Higgs supplémentaires de haute masse n'est donc pas pertinente.
D'autres analyses peuvent en revanche bénéficier de ce projet.
Son utilisation dans un autre but que d'obtenir une variable discriminante est également envisageable,
par exemple pour la sélection des événements.
\par
Notre modèle permet en effet de mieux séparer les événements $\Zboson\to\tau\tau$ des \ftauhs\ que \mTtot.
De plus, une comparaison des valeurs de \mml\ à celles de \msv, obtenues par l'algorithme \SVFIT\ déjà utilisé par la collaboration CMS,
a permis de mettre en lumière une meilleure description du boson~\Zboson\ par notre modèle.
En effet, certains événements $\Zboson\to\tau\tau$ sont prédits au-delà de \SI{250}{\GeV} par \SVFIT,
alors que la masse du \Zboson\ est de \SI{91.2}{\GeV}.
Parmi ces événements, notre modèle en prédit aux alentours de \SI{100}{\GeV}, ce qui est plus proche de la valeur vraie.
L'effet inverse,
\ie\ des événements tels que $\mml>\SI{250}{\GeV}$ alors que $\msv\simeq\SI{100}{\GeV}$ pour le boson~\Zboson,
n'est pas observé.
La sensibilité au boson de Higgs du modèle standard~\higgs\ est similaire avec \msv\ ou \mml\ dans certaines topologies d'événements.
Pour d'autres, le signal de \higgs\ est plus étendu avec \mml\ qu'avec \msv, donnant une sensibilité moindre.
L'utilisation de processus physiques plus variés pour entraîner les modèles pourrait améliorer leurs prédictions sur de telles topologies.
Des performances en termes de résolution similaires voire meilleures que celles de \SVFIT\ sont donc envisageables avec des réseaux de neurones.
\par
Le temps nécessaire pour obtenir les prédictions de masse est 60 fois plus court avec notre modèle qu'avec \SVFIT.
Les futures analyses de la collaboration CMS seront basées sur de plus grandes quantité d'événements,
l'utilisation des réseaux de neurones au lieu de \SVFIT\ présente donc un intérêt certain afin de minimiser leur coût computationnel.
\par
Le modèle B" développé au cours de ma thèse~\cite{DL_for_HTT_mass} peut être récupéré~\cite{DiTau_ML_mass} et utilisé dans d'autres analyses.
Le groupe en charge de
l'analyse des événements avec une paire de leptons~\tau\ dans le cadre du NMSSM (\emph{Next to MSSM}),
modèle contenant sept bosons de Higgs contre cinq dans le MSSM introduit au chapitre~\refChMSSM,
a déjà manifesté un intérêt pour notre modèle.
De même,
l'utilisation de B" pour l'analyse des événements $\higgs\higgs\to\quarkb\antiquarkb\tau\tau$,
\ie\
avec deux bosons de Higgs dont l'un se désintègre en paire de quarks~\quarkb\
et l'autre en paire de leptons~\tau,
est déjà étudiée.
La topologie de ces événements, différente des événements contenant uniquement $\higgs\to\tau\tau$, permet de tester notre modèle dans des situations inédites.