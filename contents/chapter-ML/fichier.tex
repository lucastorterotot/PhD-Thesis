\chapter{Reconstruction de la masse d'une résonance grâce au \emph{Machine Learning}}\label{chapter-ML}


\remarque{Citations incontournables:
\begin{itemize}
\item \DELPHES~3.4.2~\cite{Delphes,Delphes_additions}?
\item CMS Fast Simulation (\FASTSIM)~\cite{FastSim_2011,FastSim_2014,FastSim_2017_1,FastSim_2017_2}
\item \PYTHIA~8.235~\cite{pythia8.2}
\item \FASTJET~\cite{Cacciari:2011ma,Cacciari:2006} % Fast Jet
\item \KERAS~\cite{keras}
\item \TENSORFLOW~\cite{tensorflow}
\item \XGB~\cite{xgboost}
\item \cite{jet_flavor_deep_nn} for an example of nn use in HEP
\item \cite{Sarle1994NeuralNA}
\item \cite{BARTSCHI201929}
\item \SVFIT~\cite{SVFit_Bianchini_2014}
\end{itemize}}

Citer également la thèse de Gaël:\\\fullcite{Gael_thesis}

\begin{itemize}
\item type of samples/events
\item preselection (small HTT analysis)
\item inputs
\item performances: métrique?
\item mass range + plots
\item METcov + plots
\item PU + plots
\end{itemize}

\section*{Étapes des choix}
\subsection*{Inputs variables}
phéno, tau1 tau2 MET pT eta phi + mT 1 2 tt tot.
\subsection*{Inputs events}
sélection des événements, target flattening. pas de PU

80-800 GeV (aller plus loin que \cite{BARTSCHI201929})
\subsection*{DNN}
Comme dans~\cite{BARTSCHI201929}, table 1.

output activation function to linear instead of relu to not cut?

changement de la structure: 3 couches de 1000 neurones?

loss mse

optimizer adadelta

w\_init\_mode uniform

early stopping

Changement du mass range (biais aux bords) -> down to 50 GeV

\subsection*{XGB}
avoid overfitting, max\_depth

learning rate (eta)

num\_round, early stopping (choose value of 5)

loss

\section{Introduction}\label{chapter-HTT_analysis-section-introduction}

Citer \fullcite{CMS-PAS-HIG-17-020}

et aussi nouvelle version full runII si possible

+ HIG-14-029, HIG-13-021


Citer la thèse de Gaël:\\\fullcite{Gael_thesis}

Citer également la thèse d'Artur?\\\fullcite{Artur_thesis}


Études déjà menées au LEP~\cite{Schael:2006cr} et au Tevatron~\cite{Aaltonen:2009vf,Abazov:2011jh}

LHC: \cite{CMS-PAS-HIG-13-021,CMS-PAS-HIG-14-029,CMS-PAS-HIG-17-020}

aussi avec \quarkb\antiquarkb~\cite{Chatrchyan:2013qga,Khachatryan:2015tra}

ATLAS \mu\mu\ et \tau\tau~\cite{Aad:2012cfr,ATLAS-MSSM-HTT_2018,ATLAS-MSSM-HTT_2020}

CMS \mu\mu~\cite{CMS:2015ooa} \tau\tau~\cite{Chatrchyan:2012vp,CMS-MSSM-HTT_2014,CMS-PAS-HIG-17-020}


données réelles, simulées et encapsulées  -> appendix
seulement quatre canaux, pas six
$L_1$, $L_2$

what were my tasks?

	
Artur Il Darovic Gottmann
12:26
Homework concerning the bbH and ggH samples?
12:28




1) Monitor the production ---> ask in case there are invalid samples
2) Process the new samples
3) Rederive the ggH weights based on input POWHEG samples
4) Update the signal modelling for new samples in CombineHarvester
rwolf profile image	
Roger Wolf
12:32
And on the experimental side I recall:
newes FF's
MET tail correction and uncertainty
Are these the only items left before wrapping up or am I missing anything in addition?

\section{Le \emph{Machine Learning}}
\subsection{Généralités}
\subsection{Le \emph{Gradient Boosting}}
\subsection{Le \emph{Deep Learning}}

\section{Application du \emph{Machine Learning} aux événements $\Higgs\to\tau\tau$}
\subsection{Génération des événements}
\subsection{Variables d'entrées}
\subsection{Performances sur les événements de test}
\subsection{Performances sur les événements de l'analyse CMS}

\section{Prise en compte de l'empilement}
\subsection{Génération des événements}
\subsection{Performances} (sur ces nouveaux événements)
\subsection{Variables d'entrées supplémentaires}
\subsection{Performances} (avec les nouvelles variables)

\section{Effets sur les résultats de l'analyse MSSM HTT}
(remplacement de mttot par les prédictions du meilleur modèle, nouveaux plots d'exclusion, comparaison)

\section{Conclusion}\label{chapter-JERC-section-conclusion}
