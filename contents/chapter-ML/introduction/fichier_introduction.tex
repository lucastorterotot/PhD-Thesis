\section{Introduction}\label{chapter-ML-section-intro}
L'utilisation de l'intelligence artificielle (IA) s'est grandement développée au cours des dernières années.
L'IA est la capacité qu'ont des programmes à prendre des décisions, selon les informations qui leurs sont données par exemple sur leur environnement, de manière à maximiser leurs chances de réussite.
L'entreprise Google DeepMind a par exemple développé AlphaGo~\cite{alphago},
un programme destiné à jouer au jeu de Go,
qui a battu en 2016 le champion du monde de la discipline 4 à 1.
\par
Le \emph{Machine Learning} (ML) est une branche de l'IA
dans laquelle un modèle (algorithme ou programme) s'améliore à réaliser une tâche par
accumulation d'expérience sur des jeux de données d'entraînement,
sans pour autant être programmé explicitement pour réaliser cette tâche.
Pour y parvenir,
les jeux de données d'entraînement comprennent les informations $\vec{x}$ à donner au modèle
ainsi que les \og bonnes réponses \fg\ \ytrue\ qu'il doit fournir en sortie.
L'objectif du modèle est donc de donner une fonction $F$ approximant celle reliant l'entrée $\vec{x}$ à la cible \ytrue.
Il peut alors donner des prédictions $\ypred=F(\vec{x})$ sur d'autres jeux de données.
La tâche du modèle est:
\begin{description}
\item[une classification] lorsque $y$ est discrète, par exemple lorsqu'il s'agit de déterminer si une image représente un chat ou un chien~\cite{datafrog_img_reco};
\item[une régression] lorsque $y$ est continue, par exemple estimer le prix d'un bien immobilier~\cite{house_prices_regression}.
\end{description}
\par
Les applications du ML à la physique des particules sont variées et proposent de nombreux sujets d'étude~\cite{Gael_thesis,scham_moritz_2020_21993,kopf_tanja_2019_21500,Baldi_2015}.
Dans les chapitres précédents, le ML est déjà activement utilisé pour diverses tâches:
\begin{itemize}
\item identification des jets issus de quarks~\quarkb\ (\quarkb-\emph{tagging}) avec \DeepCSV~\cite{jet_flavor_deep_nn,Sirunyan_heavy_flavor_jets_2018,DeepJet};
\item identification des taus hadroniques avec \DEEPTAU~\cite{CMS-DP-2019-033};
\item catégorisation des événements comme exposé dans le chapitre~\refChHTT~\cite{CMS-NOTE-2019-177,CMS-NOTE-2019-178}.
\end{itemize}
\par
Dans les événements $\Higgs\to\tau\tau$ présentés au chapitre~\refChMSSM,
et plus généralement lors de tout processus physique $X\to\tau\tau$ où une particule $X$ se désintègre en paire de leptons tau,
des neutrinos sont émis lors des désintégrations des taus.
Or, ils sont invisibles dans les détecteurs tels que CMS ou ATLAS.
Il est donc impossible de déterminer la masse invariante totale du système $\tau\tau$ issu de $X$.
Plusieurs méthodes ont été développées afin de reconstruire la masse du système $\tau\tau$~\cite{ELAGIN2011481,Barr_2011,Gripaios_2013}.
Dans le cadre des analyses $\Higgs\to\tau\tau$, la collaboration CMS utilise \SVFIT~\cite{SVFit_Bianchini_2014}.
\par
La reconstruction la masse de la particule $X$, ou résonance, se désintégrant en paire de leptons tau grâce au \emph{Machine Learning}
a été étudiée par \author{BARTSCHI201929}~\cite{BARTSCHI201929} dans le cas où $X$ est un boson de Higgs avec une masse entre \num{80} et \SI{300}{\GeV}.
Ils ont obtenu une résolution de \SI{8.4}{\%} sur la masse du Higgs, contre \SI{17}{\%} avec \SVFIT.
Le temps de calcul nécessaire à l'obtention de la masse est de plus bien plus court avec le ML.
L'utilisation du ML est donc très prometteuse.
Cependant,
ces travaux utilisent des événements générés
avec une simulation grossière du détecteur CMS basée sur
\DELPHES~\cite{Delphes,Delphes_additions}
et sans empilement, notion introduite dans le chapitre~\refChLHCCMS.
\par
Les travaux présentés dans ce chapitre vont plus loin.
La génération des événements, introduite dans la section~\ref{chapter-ML-section-evt_gen},
utilise \FASTSIM~\cite{FastSim_2011,FastSim_2014,FastSim_2017_1,FastSim_2017_2}
pour modéliser le détecteur CMS.
Bien qu'il ne s'agisse pas de la simulation complète basée sur \GEANTfour~\cite{geant4_2003,geant4_2006,geant4_2016},
\FASTSIM\ est bien plus proche de la réalité que \DELPHES.
De plus, l'empilement est pris en compte.
Les modèles obtenus sont ainsi directement utilisables dans de réelles analyses, telles que celle présentée dans le chapitre~\refChHTT.
\par
Deux types de modèle sont étudiés:
\begin{itemize}
\item des arbres de décision basés sur \XGB~\cite{xgboost}, introduits section~\ref{chapter-ML-section-XGB};
\item des réseaux de neurones profonds, introduits section~\ref{chapter-ML-section-DNN}.
\end{itemize}
La comparaison des modèles obtenus et la sélection de l'un d'entre eux est présentée section~\ref{chapter-ML-section-hyperparameters}.
Dans la section~\ref{chapter-ML-section-discussion},
divers effets sur les performances des modèles sont discutés,
en particulier la prise en compte de l'empilement.
Enfin, l'utilisation en conditions réelles du modèle issu de ces travaux dans des analyses de physique est présentée dans la section~\ref{chapter-ML-section-use_HTT}.