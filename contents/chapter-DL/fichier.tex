\chapter{Reconstruction de la masse d'une résonance grâce au \emph{Machine Learning}}\label{chapter-DL}


\remarque{Citations incontournables:
\begin{itemize}
\item \DELPHES~3.4.2~\cite{Delphes,Delphes_additions}?
\item CMS Fast Simulation (\FASTSIM)~\cite{FastSim_2011,FastSim_2014,FastSim_2017_1,FastSim_2017_2}
\item \PYTHIA~8.235~\cite{pythia8.2}
\item \FASTJET~\cite{Cacciari:2011ma,Cacciari:2006} % Fast Jet
\item \KERAS~\cite{keras}
\item \TENSORFLOW~\cite{tensorflow}
\item \XGB~\cite{xgboost}
\item \cite{jet_flavor_deep_nn} for an example of nn use in HEP
\item \cite{Sarle1994NeuralNA}
\item \cite{BARTSCHI201929}
\end{itemize}}

Citer également la thèse de Gaël:\\\fullcite{Gael_thesis}

\begin{itemize}
\item type of samples/events
\item preselection (small HTT analysis)
\item inputs
\item performances: métrique?
\item mass range + plots
\item METcov + plots
\item PU + plots
\end{itemize}


\section{Introduction}

\section{Le \emph{Machine Learning}}
\subsection{Généralités}
\subsection{Le \emph{Gradient Boosting}}
\subsection{Le \emph{Deep Learning}}

\section{Application du \emph{Machine Learning} aux événements $\Higgs\to\tau\tau$}
\subsection{Génération des événements}
\subsection{Variables d'entrées}
\subsection{Performances sur les événements de test}
\subsection{Performances sur les événements de l'analyse CMS}

\section{Prise en compte de l'empilement}
\subsection{Génération des événements}
\subsection{Performances} (sur ces nouveaux événements)
\subsection{Variables d'entrées supplémentaires}
\subsection{Performances} (avec les nouvelles variables)

\section{Effets sur les résultats de l'analyse MSSM HTT}
(remplacement de mttot par les prédictions du meilleur modèle, nouveaux plots d'exclusion, comparaison)

\section{Conclusion}