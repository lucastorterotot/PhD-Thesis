\chapter*{Introduction}\label{chapter-introduction}
\addcontentsline{toc}{chapter}{Introduction}

La physique est une science dont le but est de
comprendre et modéliser les phénomènes régissant l'Univers.
Les propriétés des objets dépendant de celles de leurs constituants,
la description la plus fondamentale de l'Univers s'appuie sur la bonne compréhension
du comportement des constituants de la matière et de leurs interactions.
\par
Les avancées scientifiques, théoriques et expérimentales,
ont dévoilé la nature de la matière à des échelles de plus en plus fines.
Tout objet est un arrangement d'atomes, dont la taille est de l'ordre de
\SI{1}{\angstrom}, soit \SI{e-10}{\meter}.
Ils sont eux-mêmes constitués d'un noyau atomique d'environ \SI{e-15}{\meter},
autour duquel se trouve un nuage d'électrons.
Ce noyau est lui-même constitué de protons et de neutrons,
eux-même constitués de quarks~\quarku\ et~\quarkd.
À ce jour, il n'a pas été observé de sous-structure aux quarks ou à l'électron.
De telles particules sont dites élémentaires.
\par
La physique des particules, aussi nommée physique des hautes énergies, étudie les constituants élémentaires de la matière.
Plus de 50 ans d'échanges entre prédictions théoriques et résultats expérimentaux
ont permis de mettre en place dès les années 1960 le modèle standard,
%qui est une théorie quantique des champs basée sur l'invariance de jauge locale.
présenté dans le chapitre~\refChMSSM.
Ce modèle a par exemple prédit l'existence
du quark~\quarkt,
des gluons,
des bosons~\Wboson\ et~\Zboson\
ou encore
du boson de Higgs,
parfois des dizaines d'années avant leur observation.
\par
Cependant,
malgré plusieurs décennies de prédictions expérimentalement vérifiées,
ce modèle n'explique pas toutes les observations réalisées.
Par exemple,
il ne prédit pas l'existence de la matière noire,
dont des preuves de sa présence dans l'Univers
sont issues d'études cosmologiques.
Les neutrinos du modèle standard sont sans masse,
or 
leurs oscillations sont une preuve qu'ils en possèdent une.
D'autres lacunes sont encore inhérentes à ce modèle
et
de nombreuses extensions sont proposées afin de les combler.
L'une d'entre elles est l'extension supersymétrique minimale du modèle standard, ou MSSM,
également présentée dans le chapitre~\refChMSSM.
\par
Afin de tester la validité du modèle standard et des nouvelles théories allant au-delà,
divers dispositifs expérimentaux ont été mis au point
tels que
le Grand Collisionneur de Hadrons (LHC, \emph{Large Hadron Collider})
du CERN.
Il permet, entre autres, de faire entrer en collision des protons.
Lors de cet événement,
l'énergie cinétique des protons
peut être convertie en masse,
celle-ci correspondant alors à d'autres particules,
par exemple le boson de Higgs.
\par
Les particules issues des collisions sont pour la plupart instables.
Elles se désintègrent alors en d'autres particules,
qu'il est possible d'observer à l'aide de détecteurs tels que
le Solénoïde Compact à Muons (CMS, \emph{Compact Muon Solenoid}),
présenté dans le chapitre~\refChLHCCMS.
À partir de l'observation de leurs produits de désintégration,
il est possible d'étudier la présence et les propriétés des particules issues des collisions.
Le détecteur CMS a ainsi participé à la découverte du boson de Higgs en 2012.
\par
La bonne caractérisation des objets physiques mesurés par CMS est un enjeu majeur
pour réaliser des analyses de qualité.
Le chapitre~\refChJERC\ présente la calibration en énergie des jets.
Ces objets physiques complexes, constitués de nombreuses particules,
sont une manifestation des quarks et des gluons.
Les incertitudes systématiques liées aux jets sont dominantes dans plusieurs analyses,
leur bonne calibration est donc cruciale.
Lors de ma thèse, j'ai analysé des événements \Gjets\
où l'énergie du photon (\photon) permet de déterminer celle du jet.
Les résultats obtenus
concernent les données enregistrées en 2018 et en 2017-UL,
\ie\ dans un second traitement de l'année 2017.
Ils sont utilisés dans la calibration officielle de CMS
et donc dans toutes les analyses publiées par la collaboration
exploitant les données récoltées ces années-là.
\par
Le MSSM prédit cinq bosons de Higgs.
L'un d'entre eux, \higgs, doit correspondre au boson scalaire découvert en 2012.
Parmi les quatre supplémentaires, deux sont chargés, \Higgsplus\ et \Higgsminus.
Les deux autres sont neutres,
l'un scalaire, \Higgs,
l'autre pseudo-scalaire, \HiggsA.
L'observation de ces deux 
bosons supplémentaires
serait une preuve directe d'une nouvelle physique au-delà du modèle standard.
Leurs masses respectives sont plus élevées que celle de \higgs\
et vraisemblablement proches,
c'est pourquoi cette thèse porte sur la
recherche de bosons de Higgs supplémentaires de haute masse.
Leur canal de désintégration le plus prometteur,
vu la phénoménologie de ces bosons au LHC,
est la paire de leptons~\tau.
Dans ce contexte,
l'analyse des événements enregistrés par la collaboration CMS de 2016 à 2018
est présentée dans le chapitre~\refChHTT.
\par
Toutefois,
les leptons~\tau\ ne sont pas stables
et se désintègrent avant d'entrer dans le détecteur CMS.
Lors de ce processus,
des neutrinos sont produits.
Or, ces derniers sont invisibles pour le détecteur,
il est donc impossible de mesurer toutes les particules de l'état final.
Les développements récents de l'intelligence artificielle
et plus particulièrement du \emph{machine learning}
proposent de nouveaux outils tels que les réseaux de neurones
afin de réaliser des tâches complexes.
Le chapitre~\refChML\ étudie ainsi
la reconstruction de la masse d'une particule se désintégrant en paire de leptons~\tau.
L'application du modèle obtenu lors de cette thèse
dans l'analyse du chapitre~\refChHTT\ est discutée.
Une comparaison avec
à l'algorithme actuellement utilisé par la collaboration CMS
est également réalisée.