Cette thèse n'aurait pas pu être ce qu'elle est
ni menée à son terme
sans la présence, l'accompagnement et le soutien
de mes collègues, amis et membres de ma famille.
Je souhaite ici leur exprimer ma reconnaissance,
tant professionnelle que personnelle.
\par
Merci à Colin, mon \og papa de thèse \fg,
d'avoir accepté de m'encadrer sur ce sujet.
Tu m'as amené à être encore plus rigoureux
et
tu as su m'orienter sur des travaux réalisables
malgré les aléas et contraintes de cette deuxième moitié de thèse.
Tu m'as dans le même temps toujours laissé le choix et je t'en suis reconnaissant.
Je te souhaite le meilleur dans ta nouvelle vie professionnelle !
\par
Merci à Gaël, mon \og grand frère de thèse \fg,
de m'avoir fait découvrir tous les outils que j'ai utilisé dans ma thèse.
Que ce soit le terminal et \emph{bash}, Git ou même \HEPPY,
tu m'as donné les clefs de leur maîtrise !
Et ce, toujours dans la bonne humeur même quand tu étais sous pression.
Je ne connais que peu de personnes avec une telle patience.
\par
Merci à Ece, ma \og maman de thèse \fg,
pour ton soutien sur les événements \Gjets,
pour ton aide et tes conseils précieux sur le projet de \emph{Machine Learning}
et tes retours sur mes travaux.
Nous avons formé une très bonne équipe
et
je ne serai pas allé si loin sans toi.
Je te souhaite bonne chance pour ton prochain poste,
et surtout qu'il soit moins mouvementé !
\par
Merci à Hugues, mon \og cousin de thèse \fg,
d'avoir pris le temps nécessaire à la transmission
de ta responsabilité sur l'analyse des événements \Gjets.
Le C++ m'était inconnu,
mais tu m'as donné les informations nécessaires pour comprendre
le code associé et fournir les résultats à la collaboration.
\par
Merci à
Yi,
Giuseppe,
Juska
et
Mikko
pour leur écoute et leurs instructions claires
quant aux suites attendues à l'analyse des événements \Gjets.
Mention spéciale à Juska pour tes réponses à mes questions pas forcément évidentes,
et pour ton invitation à venir avec mes futurs élèves au point~5.
Ce ne sera pas pour tout de suite visiblement, mais ce n'est que partie remise !
\par
\emph{Vielen Dank} aux membres de l'\emph{Institut für Experimentelle Teilchenphysik}
du \emph{Karlsruher Institut für Technologie}
de m'avoir accueilli dans votre groupe, même à distance,
afin de mener ensemble l'analyse des événements $\Higgs\to\tau\tau$ du Run~II.
Merci à Günter pour ta bonne humeur et l'attention portée à ma bonne santé en ces temps difficiles.
Merci à Roger d'avoir accepté cette collaboration.
Merci à Artur pour ton aide et tes nombreux conseils, lors de l'implémentation du scénario CPV mais aussi tout au long de l'analyse.
Merci à Sebastian B. et Maximilian pour vos indications sur le fonctionnement des ressources informatiques au KIT,
sans lesquelles je n'aurai pas pu les prendre en main aussi rapidement.
Merci à Sebastian W. pour notre discussion fructueuse autour de la fonction de coût utilisée pour obtenir notre estimateur de la masse d'une paire de~\tau.
Merci à Felix pour l'intérêt que tu portes à cet estimateur et tes retours sur son utilisation dans une nouvelle analyse.
\par
Merci aux membres du groupe CMS MSSM HTT pour notre collaboration autour de cette analyse.
En plus des membres de KIT,
je pense à
Aleksei,
Christopher,
Daniel,
David,
Georges,
Janek,
Janik,
Mareike,
Oleg,
Suman,
et tous ceux que je n'ai pas mentionné ici.
Vos retours sur mes contributions à la note d'analyse m'ont aussi aidé pour ce manuscrit.
Leurs qualités n'en sont que meilleures.
\par
Merci à Davide pour l'interface de notre réseau de neurones en C++.
Pour sûr, cela ouvre la porte à une utilisation sur de nouvelles analyses à CMS.
Merci aussi pour tes retours sur son application aux événements $\higgs\higgs\to\quarkb\antiquarkb\tau\tau$,
dont la topologie différente de ceux que j'ai étudié nous montre son comportement dans cette situation inédite.
\par
Merci aux doctorants de l'IP2I pour
toutes les discussions, scientifiques, techniques ou plus personnelles,
les pauses café,
les parties de billard, de ping-pong, de pétanque,
les soirées et autres sorties.
Je pense en particulier à
Antoine L., Grégoire, Jean-François
et surtout Aurélien,
dont je souhaite saluer l'honnêteté et la franchise.
À Corentin et Martin aussi,
le voyage des nouveaux doctorants au CERN n'aurait pas eu lieu sans votre aide.
À tous, je vous souhaite bon courage pour la fin de vos thèse et pour la suite.
\par
Merci à Antoine C., pour ton parrainage, les discussions que nous avons eu
et les opportunités de diffusion scientifique auprès du jeune public.
\par
Merci aux membres de mon jury,
Corinne \textsc{Augier},
Lucia \textsc{Di Ciaccio},
Anne-Catherine \textsc{Le Bihan},
Steve \textsc{Muanza},
David \textsc{Rousseau} et
Roger \textsc{Wolf},
d'avoir accepté d'en faire partie et d'évaluer rigoureusement mes travaux.
%Merci en particulier à mes rapporteurs,
%Anne-Catherine \textsc{Le Bihan} et
%David \textsc{Rousseau},
%d'avoir épluché ce manuscrit
%et pour vos commentaires qui l'ont enrichi.
\par
Merci aux membres de la direction de l'IP2I, ou IPNL initialement,
pour le cadre dans lequel vous m'avez permis de réaliser ma thèse.
Vous avez toujours été à l'écoute et présents, même lors des démarches administratives les plus complexes.
\par
Merci aux membres du groupe CMS de l'IP2I et du CERN
pour votre accueil, vos conseils et la découverte du monde de la recherche.
Je le quitte aujourd'hui pour celui de l'enseignement,
mais par préférence et non par rejet.
Vous avez été de formidables collègues.
\par
Merci aux guides de la collaboration CMS pour votre accueil
et en particulier à Jacob pour ton énergie et ton enthousiasme
à organiser les visites au point~5.
J'espère que cette activité pourra reprendre bientôt
pour faire découvrir ce fantastique lieu de sciences
au public.
\par
Merci à vous deux, Papa et Maman,
d'avoir été des parents qui m'ont permis d'en arriver là.
Vous m'avez laissé le choix de suivre cette voie sans fermer les autres
et
soutenu à chaque étape.
Vous m'avez transmis bien plus que la myopie familiale
et ces quelques mots sont loin d'exprimer toute ma reconnaissance.
\par
Merci à Sarah,
pour tout ce que nous vivons à deux.
Merci pour le rythme de vie plus sain que celui que je me serai fixé sans toi.
Merci pour ces moments de décompression tout au long de nos thèses.
Merci pour le temps que tu as passé à relire ce manuscrit pour y relever mes maladresses.
\par
Et à tous ceux que j'aurais oublié ici,
merci.